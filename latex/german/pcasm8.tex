%-*- latex -*-
\chapter{Dynamic Link Libraries}

\section[Benutzung der Windows API und DLLs]{Benutzung der Windows
API und Dynamic Link Libraries}

UNIX Systeme stellen ein einfaches, C-basiertes Application
Programming Interface (API) \index{API} bereit. Im Gegensatz dazu
packt Microsoft Windows\texttrademark \hspace{0.2em} sein API in
Dynamic Link Libraries, das sich vor seiner Benutzung in den
Adressraum jeden Prozesses l\"{a}dt. Aus der Perspektive eines C/C++
Programmierers erscheint es als ein normaler C Funktionsaufruf;
jedoch unterscheidet es sich auf der Ebene des Assemblers.

\subsection{Die Standard Call Aufruf\/konvention}
Die Windows API benutzt die \emph{Standard Call}
\index{Aufrufkonvention!standard call} Aufruf\/konvention. Wie
fr\"{u}her ausgef\"{u}hrt, legt diese Konvention die Argumente in
umgekehrter Reihenfolge auf den Stack, genauso wie die Standard C
Konvention. Jedoch gibt es zwei wichtige Unterschiede. Erstens ist
die Subroutine f\"{u}r die Entfernung der Parameter vom Stack
verantwortlich. Zweitens wird das Label der Funktion anders
generiert. Wie zuvor wird dem Namen ein Unterstrich vorangestellt,
aber zus\"{a}tzlich wird das \emph{@} Zeichen am Ende des
Funktionsnamens angeh\"{a}ngt, zusammen mit einer Zahl (in dezimal), die
gleich der Anzahl Bytes ist, die die Parameter der Funktion auf dem
Stack belegen.

\index{CloseHandle|(} Betrachten wir zum Beispiel die Windows API
Funktion {\code CloseHandle}. Ihr Prototyp sieht so aus:
\begin{lstlisting}[stepnumber=0]{}
 BOOL WINAPI CloseHandle( HANDLE hObject );
\end{lstlisting}
Da ein {\code HANDLE} in 32-bit Windows ein Doppelwort ist, w\"{u}rde
das Label f\"{u}r diese Funktion {\code \_CloseHandle@4} sein. Das
{\code WINAPI} \index{WINAPI} im Prototyp ist ein C Makro, das als
{\code \_\_stdcall} definiert ist. Unten ist ein Beispielaufruf der
Funktion unter der Annahme, dass der Wert von {\code hObject} in
{\code EBX} ist.
\begin{AsmCodeListing}[frame=single, numbers=left, commandchars=\\\{\}]
     push  ebx                ; bringe hObject auf den Stack
     call  _CloseHandle@4
     mov   esi, eax           ; sichere R\"{u}ckgabewert in ESI
\end{AsmCodeListing}
Der Stack muss nach dem Funktionsaufruf nicht bereinigt werden, der
Stack wird von {\code CloseHandle} automatisch in Ordnung gebracht.
\index{CloseHandle|)}

\subsection{Statische und shared Bibliotheken}

\index{statische Bibliothek|(} Eine \emph{statische Bibliothek} ist
eine Sammlung von Objektdateien, die zu einer ausf\"{u}hrbaren Datei
hinzugef\"{u}gt werden kann, wenn diese erstellt wird. Der Objektcode
wird direkt in die ausf\"{u}hrbare Datei eingef\"{u}gt, genau wie eine
gew\"{o}hnliche Objektdatei. Die Bibliotheksdatei dient nur der
Bequemlichkeit. Es erm\"{o}glicht einer einzigen Datei, im Linkschritt
des Erstellungsprozesses eingebunden zu werden. Wahrscheinlich wird
nicht der gesamte Objektcode in der endg\"{u}ltigen ausf\"{u}hrbaren Datei
gespeichert. Der Linker sieht nach, welche Objektmodule ben\"{o}tigt
werden und f\"{u}gt nur die erforderlichen ein. Statische Bib"-liotheken
werden mit Hilfe des Programms {\code LIB} unter Windows oder {\code
ar} unter UNIX erstellt. Windows Bibliotheken enden mit einer {\code
.lib} Erweiterung und UNIX Bibliotheken enden mit einer {\code .a}
Erweiterung. \index{statische Bibliothek|)}

\index{shared Bibliothek|(} Eine shared Bibliothek (oder DLL in
Windows) teilt, wie ihr Name impliziert, ihren Code mit Programmen.
Wenn das Programm gestartet wird, findet das OS all die shared
Bibliotheken, die es ben\"{o}tigt und l\"{a}dt diese in den
Prozess-Speicher, damit der Prozess den Code da\-rin verwenden kann.
Durch die Verwendung der Abbildungsm\"{o}glichkeiten des virtuellen
Speichers auf einem protected mode Betriebssystem, wird der Code von
shared Bibliotheken, die von zwei oder mehr gleichzeitig laufenden
Prozessen benutzt wird, nur einmal in den physikalischen Speicher
geladen.

Shared Bibliotheken haben Vor- und Nachteile. Der erste Vorteil ist,
dass die Gr\"{o}{\ss}en von Programmdateien stark reduziert werden k\"{o}nnen.
Wenn eine gro{\ss}e Bibliothek durch viele Programme benutzt wird, ist
nur eine Kopie des Codes (in der shared Bibliothek) auf dem System
(und im physikalischen Speicher). Wenn die Bibliothek statisch
eingeschlossen wurde, w\"{u}rde jedes Programm eine Kopie des Codes
beinhalten.

Ein weiterer Vorteil ist, dass wenn ein Fehler in der shared
Bibliothek gefunden wird, dann kann die Bibliothek durch eine
ersetzt werden, in der der Fehler behoben ist. Die Programme werden
automatisch den neuen, bereinigten Code verwenden, ohne die
Programme neu zu kompilieren. Jedoch funktioniert dies nur, wenn die
Schnittstelle der Funktionen in der shared Bibliothek unver\"{a}ndert
bleibt.

Shared Bibliotheken gestatten es auch, dass Code, der in
verschiedenen Sprachen geschrieben ist, miteinander arbeitet. Zum
Beispiel kann C++ unter Windows durch die Benutzung von shared
Bibliotheken zusammen mit Visual Basic, \index{Visual Basic} DotNet
\index{DotNet} und Java \index{Java} benutzt werden.

Der Nachteil der shared Bibliotheken ist, dass sie komplizierter zu
warten sind. Eines der h\"{a}ufigsten Probleme ist, dass eine neue
Version der Bibliothek \"{a}lteren Code unbrauchbar macht, weil sie sich
etwas unterschiedlich verh\"{a}lt. Dann brauchen einige Programme eine
Version und andere ben\"{o}tigen eine davon verschiedene. In Windows ist
dieser Zustand als \emph{DLL H\"{o}lle} bekannt. \index{DLL!H\"{o}lle}
\index{shared Bibliothek|)}

\index{DLL|(}
\subsection{Windows DLLs}

Eine Windows DLL wird ganz so wie ein ausf\"{u}hrbares Programm
zusammengebaut. Objektdateien werden zusammen gebunden, um eine DLL
Datei zu bilden. Anders als ein ausf\"{u}hrbares Programm kann eine DLL
viele Eingangspunkte haben. Ein Eingangspunkt ist gerade eine
Funktion, die von au{\ss}erhalb der DLL aus aufgerufen werden kann. Nur
Funktionen, die \emph{exportiert} \index{export} wurden, k\"{o}nnen
extern aufgerufen werden.

Eine Funktion (oder globale Variable) kann exportiert werden, indem
entweder ihr Name in die Definitionsdatei
\index{DLL!Definitionsdatei} der DLL aufgenommen wird oder durch die
Verwendung von Microsoft-spezifischer Schl\"{u}sselw\"{o}rter.

\begin{figure}[ht]
\begin{Verbatim}[frame=single, numbers=left, commandchars=\\\{\}]
 LIBRARY \textit{Basisname der Bibliothek}
 DESCRIPTION '\textit{kurzer Beschreibungstext}'
 EXPORTS
 \textit{Liste der exportierten Funktionen (eine pro Zeile)}
\end{Verbatim}
\caption{Windows DLL Definitionsdatei\label{fig:DefFile}}
\end{figure}

\subsubsection{Definitionsdatei}
Das ist eine Textdatei mit einer {\code .def} Erweiterung, die all
die Funktionen auf\/listet, die die DLL exportiert. Diese Datei wird
w\"{a}hrend des Linkschrittes bei der Erstellung der DLL verwendet.
Abbildung~\ref{fig:DefFile} zeigt das allgemeine Format einer
Definitionsdatei.

\index{DLL|)}
