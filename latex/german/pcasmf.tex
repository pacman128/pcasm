% -*-LaTex-*-
%front matter of pcasm book

\chapter{Vorwort}

\section*{Ziel}

Das Ziel dieses Buches besteht darin, dem Leser ein besseres
Verst\"{a}ndnis dar\"{u}ber zu geben, wie Computer auf einem niedrigeren
Level als in Programmiersprachen wie Pascal wirklich arbeiten. Durch
den Erwerb eines tieferen Verst\"{a}ndnisses wie Computer arbeiten, kann
der Lesern oft sehr viel produktiver Software in Hochsprachen wie C
und C++ entwickeln. Ein ausgezeichneter Weg, um dieses Ziel zu
erreichen, ist, in Assembler programmieren zu lernen. Andere
PC-Assemblerb\"{u}cher lehren immer noch den 8086 Prozessor zu
programmieren, den der originale PC 1981 benutzte! Die 8086
Prozessoren unterst\"{u}tzten nur den \emph{real} Modus. In diesem Modus
kann jedes Programm alle Speicherstellen oder Ger\"{a}te im Computer
ansprechen. Dieser Modus ist f\"{u}r ein sicheres Betriebssystem mit
Multitasking nicht geeignet. Dagegen behandelt dieses Buch wie der
80386 und sp\"{a}tere Prozessoren im \emph{protected} Modus programmiert
werden (dem Modus, in dem Windows und Linux laufen). Dieser Modus
unterst\"{u}tzt die Merkmale, die moderne Betriebssysteme erwarten, wie
virtuellen Speicher und gesch\"{u}tzten Speicher. Es gibt verschiedene
Gr\"{u}nde den protected Mode zu verwenden:
\begin{enumerate}
\parskip=-0.50em %reduce the spacing <<<<<<<<<<<<<<<<<<<<<<<<<<<<<<<<<<<<<<<<<<
\item Es ist einfacher im protected Mode zu programmieren als im 8086
real Mode, den andere B\"{u}cher verwenden.
\item Alle modernen PC-Betriebssysteme laufen im protected Mode.
\item Es gibt freie Software, die in diesem Modus l\"{a}uft.
\end{enumerate}
Das Fehlen von Lehrb\"{u}chern f\"{u}r die Assemblerprogrammierung des PC im
protected Mode ist der Hauptgrund, dass der Autor dieses Buch
schrieb.

Wie oben angedeutet, macht dieses Buch Gebrauch von Free/Open Source
Software: n\"{a}mlich dem NASM Assembler und dem DJGPP C/C++ Compiler.
Beide stehen zum Download im Internet zur Verf\"{u}gung. Der Text
bespricht au{\ss}erdem, wie der Assemblercode von NASM unter Linux und
mit den C/C++ Compilern von Borland und Microsoft unter Windows
verwendet werden kann. Beispiele f\"{u}r alle diese Plattformen k\"{o}nnen
auf meiner Webseite gefunden werden: {\code
http://www.drpaulcarter.com/pcasm}. Sie \emph{m\"{u}ssen} den
Beispielcode he"-runterladen, wenn Sie viele der Beispiele in diesem
Tutorial assemblieren und laufen lassen m\"{o}chten.

Sind Sie sich dar\"{u}ber bewusst, dass dieser Text nicht versucht,
jeden Aspekt der Assemblerprogrammierung abzudecken. Der Autor hat
versucht, die wichtigsten Themen abzudecken, mit denen \emph{alle}
Programmierer bekannt sein sollten.

\section*{Danksagungen}

Der Autor m\"{o}chte den vielen Programmierern auf der Welt danken, die
zur Free/Open Source Bewegung beigetragen haben. All die Programme
und sogar dieses Buch selbst wurden unter Verwendung freier Software
produziert. Besonders m\"{o}chte der Autor John~S.~Fine, Simon~Tatham,
Julian~Hall und anderen f\"{u}r die Entwicklung des NASM Assemblers
danken, auf dem alle Beispiele in diesem Buch basieren; DJ Delorie
f\"{u}r die Entwicklung des verwendeten DJGPP C/C++ Compilers; den
zahlreichen Personen, die zum GNU gcc Compiler beigetragen haben,
auf dem DJGPP beruht; Donald E.\ Knuth und anderen f\"{u}r die
Entwicklung der \TeX\ and \LaTeXe\ Satzsprachen, die benutzt wurden,
um diesen Buch zu produzieren; Richard Stallman (Gr\"{u}nder der Free
Software Foundation), Linus Torvalds (Sch\"{o}pfer des Linux Kernels)
und anderen, die die zugrunde liegende Software produziert haben,
die der Autor benutzte, um dieses Werk zu produzieren.

Dank geb\"{u}hrt den folgenden Personen f\"{u}r Korrekturen:

\begin{itemize}
\parskip=-0.50em %reduce the spacing <<< is stretched by Latex <<<<<<<<<<<<<<<<
\item John S.\ Fine
\item Marcelo Henrique Pinto de Almeida
\item Sam Hopkins
\item Nick D'Imperio
\item Jeremiah Lawrence
\item Ed Beroset
\item Jerry Gembarowski
\item Ziqiang Peng
\item Eno Compton
\item Josh I Cates
\item Mik Mifflin
\item Luke Wallis
\item Gaku Ueda
\item Brian Heward
\item Chad Gorshing
\item F.\ Gotti
\item Bob Wilkinson
\item Markus Koegel
\item Louis Taber
\item Dave Kiddell
\item Eduardo Horowitz
\item S\'{e}bastien Le Ray
\item Nehal Mistry
\end{itemize}

\pagebreak[4] % <<< table _must_ be on a right (odd) page to fit on paper <<<<
\section*{Quellen im Internet\footnote{Stand 2006-12-15 [Anm.\ d.\ \"{U}\@.]}}

\begin{center}
\begin{tabular}{|ll|}
 \hline
 Die Seite des Authors  & {\code http://www.drpaulcarter.com/} \\
 %NASM                   & {\code http://nasm.2y.net/} \\
 NASM SourceForge Seite & {\code http://sourceforge.net/projects/nasm/} \\
 DJGPP                  & {\code http://www.delorie.com/djgpp} \\
 Linux Assembly         & {\code http://www.linuxassembly.org/} \\
 The Art of Assembly    & {\code http://webster.cs.ucr.edu/} \\
 USENET                 & {\code comp.lang.asm.x86 } \\
 Intel Dokumentation    & {\code http://developer.intel.com/design/Pentium4/documentation.htm} \\
 %update to /Pentium5/ ?? not yet 2006-12-06 -ubi
 \hline
\end{tabular}
\end{center}

\section*{Feedback}

Der Autor begr\"{u}{\ss}t jedes Feedback \"{u}ber dieses Werk.
\begin{center}
\begin{tabular}{ll}
\textbf{E-mail:} & {\code pacman128@gmail.com} \\
\textbf{WWW:}    & {\code http://www.drpaulcarter.com/pcasm} \\
\end{tabular}
\end{center}
