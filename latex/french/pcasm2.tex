\chapter{Bases du Langage Assembleur}

\section{Travailler avec les Entiers\index{entiers|(}}

\subsection{Repr\'{e}sentation des entiers\index{entiers!repr\'{e}sentation|(}}

\index{entiers!non sign\'{e}s|(}
Les entiers se d\'{e}composent en deux cat\'{e}gories : sign\'{e}s et non sign\'{e}s. Les
entiers non sign\'{e}s (qui sont positifs) sont repr\'{e}sent\'{e}s d'une mani\`{e}re binaire
tr\`{e}s intuitive. Les nombre 200 en tant qu'entier non sign\'{e} sur un octet serait
repr\'{e}sent\'{e} par 11001000 (ou C8 en hexa).
\index{entiers!non sign\'{e}s|)}

\index{entiers!sign\'{e}s|(}
Les entiers sign\'{e}s (qui peuvent \^{e}tre positifs ou n\'{e}gatifs) sont repr\'{e}sent\'{e}s
d'une fa�on plus compliqu\'{e}e. Par exemple, consid\'{e}rons $-56$. $+56$ serait repr\'{e}sent\'{e}
par l'octet 00111000. Sur papier, on peut repr\'{e}senter $-56$ comme $-111000$,
mais comment cela serait-il repr\'{e}sent\'{e} dans un octet en m\'{e}moire de l'ordinateur.
Comment serait stock\'{e} le signe moins ?

Il y a trois techniques principales qui ont \'{e}t\'{e} utilis\'{e}es pour repr\'{e}senter les
entiers sign\'{e}s dans la m\'{e}moire de l'ordinateur. Toutes ces m\'{e}thodes utilisent
le bit le plus significatif de l'entier comme \emph{bit de signe}. \index{entiers!bit de signe}
Ce bit vaut 0 si le nombre est positif et 1 s'il est n\'{e}gatif.

\subsubsection{Grandeur sign\'{e}e \index{entiers!repr\'{e}sentation!grandeur sign\'{e}e}}

La premi\`{e}re m\'{e}thode est la plus simple, elle est appel\'{e}e \emph{grandeur sign\'{e}e}.
Elle repr\'{e}sente l'entier en deux parties. La premi\`{e}re partie est le bit de
signe et la seconde est la grandeur enti\`{e}re. Donc 56 serait repr\'{e}sent\'{e}
par l'octet $\underline{0}0111000$ (le bit de signe est soulign\'{e}) et $-56$
serait $\underline{1}0111000$. La valeur d'octet la plus grande serait
$\underline{0}1111111$ soit $+127$ et la plus petite valeur sur un octet serait
$\underline{1}1111111$ soit $-127$. Pour obtenir l'oppos\'{e} d'une valeur, on inverse
le bit de signe. Cette m\'{e}thode est intuitive mais a ses inconv\'{e}nients.
Tout d'abord, il y a deux valeurs possibles pour 0 : $+0$ ($\underline{0}0000000$)
et $-0$ ($\underline{1}0000000$). Comme z\'{e}ro n'est ni positif ni n\'{e}gatif,
ces deux repr\'{e}sentations devraient se comporter de la m\^{e}me fa�on. Cela complique
la logique de l'arithm\'{e}tique pour le processeur. De plus, l'arithm\'{e}tique g\'{e}n\'{e}rale
est \'{e}galement compliqu\'{e}e. Si l'on ajoute 10 \`{a} $-56$, cela doit \^{e}tre transform\'{e} en
10 moins 56. L\`{a} encore, cela complique la logique du processeur.

\subsubsection{Compl\'{e}ment \`{a} 1\index{entiers!representation!one's complement}}
La seconde m\'{e}thode est appel\'{e}e repr\'{e}sentation en \emph{compl\'{e}ment \`{a} un}.
Le compl\'{e}ment \`{a} un d'un nombre est trouv\'{e} en inversant chaque bit du nombre
(une autre fa�on de l'obtenir est que la valeur du nouveau bit est
$1 - \mathrm{ancienne valeur du bit}$). Par exemple, le compl\'{e}ment \`{a} un de
$\underline{0}0111000$ ($+56$) est $\underline{1}1000111$. Dans la notation en
compl\'{e}ment \`{a} un, calculer le compl\'{e}ment \`{a} un est \'{e}quivalent \`{a} la n\'{e}gation. Donc
$\underline{1}1000111$ est la repr\'{e}sentation de $-56$. Notez que le bit de signe
a \'{e}t\'{e} automatiquement chang\'{e} par le compl\'{e}ment \`{a} un et que, comme l'on s'y attendait,
appliquer le compl\'{e}ment \`{a} un deux fois redonne le nombre de d\'{e}part. Comme pour la
premi\`{e}re m\'{e}thode, il y a deux repr\'{e}sentation pour 0 : 
$\underline{0}0000000$ ($+0$) et $\underline{1}1111111$ ($-0$). L'arithm\'{e}tique
des nombres en compl\'{e}ment \`{a} un est compliqu\'{e}e.

Voici une astuce utile pour trouver le compl\'{e}ment \`{a} un d'un nombre en hexad\'{e}cimal
sans repasser en binaire. L'astuce est d'\^{o}ter le chiffre hexa de F (ou 15 en d\'{e}cimal).
Cette m\'{e}thode suppose que le nombre de bits dans le nombre est multiple de 4. Voici
un exemple : $+56$ est repr\'{e}sent\'{e} par 38 en hexa. pour trouver le compl\'{e}ment \`{a} un,
\^{o}tez chaque chiffre de F pour obtenir C7 en hexa. Cela concorde avec le r\'{e}sultat
ci-dessus.

\subsubsection{Compl\'{e}ment \`{a} deux \index{entiers!repr\'{e}sentation!compl\'{e}ment \`{a} deux|(}
               \index{compl\'{e}ment \`{a} deux|(}}

Les deux premi\`{e}res m\'{e}thodes d\'{e}crites ont \'{e}t\'{e} utilis\'{e}es sur les premiers ordinateurs.
Les ordinateurs modernes utilisent une troisi\`{e}me m\'{e}thode appel\'{e}e repr\'{e}sentation en
\emph{compl\'{e}ment \`{a} deux}. On trouve le e compl\'{e}ment \`{a} deux d'un nombre en effectuant
les deux op\'{e}rations suivantes :
\begin{enumerate}
\item Trouver le compl\'{e}ment \`{a} un du nombre
\item Ajouter un au r\'{e}sultat de l'\'{e}tape 1
\end{enumerate}
Voici un exemple en utilisant $\underline{0}0111000$ (56). Tout d'abord, on calcule
le compl\'{e}ment \`{a} un : $\underline{1}1000111$. Puis, on ajoute un :
\[
\begin{array}{rr}
 & \underline{1}1000111 \\
+&                    1 \\ \hline
 & \underline{1}1001000
\end{array}
\]

Dans la notation en compl\'{e}ment \`{a} deux, calculer le compl\'{e}ment \`{a} deux est \'{e}quivalent
\`{a} trouver l'oppos\'{e} d'un nombre. Donc, $\underline{1}1001000$ est la repr\'{e}sentation
en compl\'{e}ment \`{a} deux de $-56$. Deux n\'{e}gations devraient reproduire le nombre original.
Curieusement le compl\'{e}ment \`{a} deux ne rempli pas cette condition. Prenez le compl\'{e}ment
\`{a} deux de $\underline{1}1001000$ en ajoutant un au compl\'{e}ment \`{a} un.
\[
\begin{array}{rr}
 & \underline{0}0110111 \\
+&                    1 \\ \hline
 & \underline{0}0111000
\end{array}
\]

Lorsque l'on effectue l'addition dans l'op\'{e}ration de compl\'{e}mentation \`{a} deux, l'addition
du bit le plus \`{a} gauche peut produire une retenue. Cette retenue n'est \emph{pas} utilis\'{e}e.
Souvenez vous que toutes les donn\'{e}es sur un ordinateur sont d'une taille fixe (en terme de
nombre de bits). Ajouter deux octets produit toujours un r\'{e}sultat sur un octet (comme
ajouter deux mots donne un mot, {\em etc\/}.) Cette propri\'{e}t\'{e} est importante pour la
notation en compl\'{e}ment \`{a} deux. Par exemple, consid\'{e}rons z\'{e}ro comme un nombre en compl\'{e}ment
\`{a} deux sur un octet ($\underline{0}0000000$). Calculer son compl\'{e}ment \`{a} deux produit la somme :
\[
\begin{array}{rr}
 & \underline{1}1111111 \\
+&                    1 \\ \hline
c& \underline{0}0000000
\end{array}
\]
o\`{u} $c$ repr\'{e}sente une retenue (Plus tard, nous montrerons comment d\'{e}tecter cette retenue,
mais elle n'est pas stock\'{e}e dans le r\'{e}sultat). Donc, en notation en compl\'{e}ment \`{a} deux,
il n'y a qu'un z\'{e}ro. Cela rend l'arithm\'{e}tique en compl\'{e}ment \`{a} deux plus simple que les m\'{e}thodes
pr\'{e}c\'{e}dentes.

\begin{table}
\centering
\begin{tabular}{||c|c||}
\hline
Nombre & Repr\'{e}sentation Hexa \\
\hline
0 & 00 \\
1 & 01 \\
127 & 7F \\
-128 & 80 \\
-127 & 81 \\
-2 & FE \\
-1 & FF \\
\hline
\end{tabular}
\caption{Two's Complement Representation \label{tab:twocomp}}
\end{table}

En utilisant la notation en compl\'{e}ment \`{a} deux, un octet sign\'{e} peut \^{e}tre utilis\'{e} pour
repr\'{e}senter les nombres $-128$ \`{a} $+127$. Le Tableau~\ref{tab:twocomp} montre quelques
valeurs choisies. Si 16 bits sont utilis\'{e}s, les nombres sign\'{e}s $-32,768$ \`{a} 
$+32,767$ peuvent \^{e}tre repr\'{e}sent\'{e}s. $+32,767$ est repr\'{e}sent\'{e} par 7FFF, $-32,768$ 
par 8000, -128 par FF80 et -1 par FFFF. Les nombres en compl\'{e}ment \`{a} deux sur 32 bits
vont de $-2$ milliards \`{a} $+2$ milliards environ. 


Le processeur n'a aucune id\'{e}e de ce qu'un octet en particulier (ou un mot ou un double
mot) est suppos\'{e} repr\'{e}sent\'{e}. L'assembleur n'a pas le concept de types qu'un langage de
plus haut niveau peut avoir. La fa�on dont les donn\'{e}es sont interpr\'{e}t\'{e}es d\'{e}pendent de
l'instruction dans laquelle on les utilise. Que la valeur FF soit consid\'{e}r\'{e}e comme
repr\'{e}sentant un $-1$ sign\'{e} ou un $+255$ non sign\'{e} d\'{e}pend du programmeur. Le langage
C d\'{e}finit des types entiers sign\'{e}s et non sign\'{e}s. Cela permet au compilateur C de
d\'{e}terminer les instructions correctes \`{a} utiliser avec les donn\'{e}es.

\index{compl\'{e}ment \`{a} deux|)}
\index{entiers!repr\'{e}sentation!compl\'{e}ment \`{a} deux|)}
\index{entiers!sign\'{e}s|)}

\subsection{Extension de signe\index{entiers!extension de signe|(}}

En assembleur, toutes les donn\'{e}es ont une taille bien sp\'{e}cifi\'{e}e. Il n'est pas
rare de devoir changer la taille d'une donn\'{e}e pour l'utiliser avec d'autres.
R\'{e}duire la taille est le plus simple.

\subsubsection{R\'{e}duire la taille des donn\'{e}es}

Pour r\'{e}duire la taille d'une donn\'{e}e, il suffit d'en retirer les bits les plus
significatifs. Voici un exemple trivial :
\begin{AsmCodeListing}[numbers=none,frame=none]
      mov    ax, 0034h      ; ax = 52 (stock\'{e} sur 16 bits)
      mov    cl, al         ; cl = les 8 bits de poids faible de ax
\end{AsmCodeListing}

Bien s\^{u}r, si le nombre ne peut pas \^{e}tre repr\'{e}sent\'{e} correctement dans
une plus petite taille, r\'{e}duire la taille ne fonctionne pas. Par exemple,
si {\code AX}  valait 0134h (ou 308 en d\'{e}cimal) alors le code ci-dessus
mettrait quand m\^{e}me {\code CL} \`{a} 34h. Cette m\'{e}thode fonctionne \`{a} la fois avec
les nombres sign\'{e}s et non sign\'{e}s. Consid\'{e}rons les nombres sign\'{e}s, si
{\code AX} valait FFFFh ($-1$ sur un mot), alors {\code CL} vaudrait FFh 
($-1$ sur un octet). Cependant, notez que si la valeur dans {\code AX}) \'{e}tait sign\'{e},
elle aurait \'{e}t\'{e} tronqu\'{e} et le r\'{e}sultat aurait \'{e}t\'{e} faux !

La r\`{e}gle pour les nombres non sign\'{e}s est que tous les bits retir\'{e}s soient \`{a}
0 afin que la convertion soit correcte. La r\`{e}gle pour les nombres sign\'{e}s est
que les bits retir\'{e}s doivent soit tous \^{e}tre des 1 soit tous des 0. De plus,
Le premier bit \`{a} ne pas \^{e}tre retir\'{e} doit valoir la m\^{e}me chose que ceux qui l'ont
\'{e}t\'{e}. Ce bit sera le bit de signe pour la valeur plus petite. Il est important
qu'il ait la m\^{e}me valeur que le bit de signe original !

\subsubsection{Augmenter la taille des donn\'{e}es}

Augmenter la taille des donn\'{e}es est plus compliqu\'{e} que la r\'{e}duire.
Consid\'{e}rons l'octet hexa FF. S'il est \'{e}tendu \`{a} un mot, quelle valeur
aura-t-il ? Cela d\'{e}pend de la fa�on dont FF est interpr\'{e}t\'{e}. Si FF
est un octet non sign\'{e} (255 en d\'{e}cimal), alors le mot doit \^{e}tre 00FF ;
cependant, s'il s'agit d'un octet sign\'{e} ($-1$ en d\'{e}cimal), alors le
mot doit \^{e}tre FFFF.

En g\'{e}n\'{e}ral, pour \'{e}tendre un nombre non sign\'{e}, on met tous les bits suppl\'{e}mentaires
du nouveau nombre \`{a} 0. Donc, FF devient 00FF. Cependant, pour \'{e}tendre un nombre
sign\'{e}, on doit \emph{\'{e}tendre} le bit de signe. \index{entiers!bit de signe} Cela signifie
que les nouveaux bits deviennent des copies du bit de signe. Comme le bit de signe
de FF est \`{a} 1, les nouveaux bits doivent \'{e}galement \^{e}tre des uns, pour donner FFFF. Si le
nombre sign\'{e} 5A (90 en d\'{e}cimal) \'{e}tait \'{e}tendu, le r\'{e}sultat serait 005A.

Il y a plusieurs instructions fournies par le 80386 pour \'{e}tendre les nombres. Souvenez
vous que l'ordinateur ne sait pas si un nombre est sign\'{e} ou non. C'est au programmeur
d'utiliser l'instruction ad\'{e}quate.

Pour les nombres non sign\'{e}s, on peut placer des 0 dans les bits de poids fort de fa�on simple
en utilisant l'instruction {\code MOV}. Par axemple pour \'{e}tendre l'octet de AL en un mot
non sign\'{e} dans AX :
\begin{AsmCodeListing}[numbers=none,frame=none]
      mov    ah, 0   ; met les 8 bits de poids for \`{a} 0
\end{AsmCodeListing}
Cependant, il n'est pas possible d'utiliser une instruction {\code MOV} pour
convertir le mot non sign\'{e} de AX en un double mot non sign\'{e} dans EAX. Pourquoi pas ?
Il n'y a aucune fa�on d'acc\'{e}der aux 16 bits de poids fort de EAX dans un {\code
MOV}. Le 80386 r\'{e}soud ce probl\`{e}me en fournissant une nouvelle instruction :
{\code MOVZX}. \index{MOVZX} Cette instruction a deux op\'{e}randes. La destination
(premier op\'{e}rande) doit \^{e}tre un registre de 16 ou 32~bits. La source (deuxi\`{e}me
op\'{e}rande) doit \^{e}tre un registre 8 ou 16~bits. L'autre restriction est que la destination
doit \^{e}tre plus grande que la source (la plupart des instructions recqui\`{e}rent que la source
soit de la m\^{e}me taille que la destination). Voici quelques exemples :
\begin{AsmCodeListing}[numbers=none,frame=none]
      movzx  eax, ax      ; \'{e}tend ax \`{a} eax
      movzx  eax, al      ; \'{e}tend al \`{a} eax
      movzx  ax, al       ; \'{e}tend al \`{a} ax
      movzx  ebx, ax      ; \'{e}tend ax \`{a} ebx
\end{AsmCodeListing}

Pour les nombres sign\'{e}s, il n'y a pas de fa�on simple d'utiliser l'instruction
{\code MOV} quelque soit le cas. Le 8086 fournit plusieurs instructions pour \'{e}tendre
les nombres sign\'{e}s. L'instruction {\code CBW} \index{CBW} (Convert Byte to
Word, Convertir un Octet en Mot) \'{e}tend le signe du registre AL dans AX. Les
op\'{e}randes sont implicites. L'instruction {\code CWD} \index{CWD} (Convert Word to Double
word, Convertir un Mot en Double mot) \'{e}tend le signe de AX dans DX:AX. La notation
DX:AX signifie qu'il faut consid\'{e}rer les registres DX et AX comme un seul registre 32~bits
avec les 16 bits de poids forts dans DX et les bits de poids faible dans AX (souvenez
vous que le 8086 n'avait pas de registre 32~bits du tout !). Le 80386 a apport\'{e}
plusieurs nouvelles instructions. L'instruction {\code CWDE} \index{CWDE} (Convert Word to Double
word Extended, convertir un mot en double mot \'{e}tendue) \'{e}tend le signe de AX dans EAX.
L'instruction {\code CDQ}\index{CDQ} (Convert Double word to Quad word, Convertir un Double
mot en Quadruple mot) \'{e}tend le signe de EAX dans EDX:EAX\index{registre!EDX:EAX} (64~bits !)
Enfin, l'instruction {\code MOVSX} \index{MOVSX} fonctionne comme {\code MOVZX} except\'{e}
qu'elle utilise les r\`{e}gles des nombres sign\'{e}s.

\subsubsection{Application \`{a} la programmation en C}

L'extension \MarginNote{Le C ANSI ne d\'{e}finit pas si le type {\code char}
est sign\'{e} ou non, c'est \`{a} chaque compilateur de le d\'{e}cider. C'est pourquoi
on d\'{e}finit le type explicitement dans la Figure~\ref{fig:charExt}.} d'entiers
sign\'{e}s et non sign\'{e}s a \'{e}galement lieu en C. Les variables en C peuvent \^{e}tre
d\'{e}clar\'{e}es soit comme sign\'{e}es soit comme non sign\'{e}es (les {\code int} sont
sign\'{e}s). Consid\'{e}rons le code de la Figure~\ref{fig:charExt}.  A la ligne~3,
la variable {\code a} est \'{e}tendue en utilisant les r\`{e}gles pour les valeurs
non sign\'{e}es (avec {\code MOVZX}), mais \`{a} la ligne~4, les r\`{e}gles sign\'{e}es
sont utilis\'{e}es pour {\code b} (avec {\code MOVSX}).

\begin{figure}[t]
\begin{lstlisting}[frame=tlrb]{}
unsigned char uchar = 0xFF;
signed char   schar = 0xFF;
int a = (int) uchar;     /* a = 255 (0x000000FF) */
int b = (int) schar;     /* b = -1  (0xFFFFFFFF) */
\end{lstlisting}
\caption{Extension de signe en C}
\label{fig:charExt}
\end{figure}

Il existe un bug de programmation courant en C qui est directement
li\'{e} \`{a} notre sujet. Consid\'{e}rons le code de la Figure~\ref{fig:IObug}.
La prototype de {\code fgetc()}{\samepage est :
\begin{CodeQuote}
int fgetc( FILE * );
\end{CodeQuote}
On peut se demander }pourquoi est-ce que la fonction renvoie un  {\code int}
puisqu'elle lit des caract\`{e}res ? La raison est qu'elle renvoie normalement
un {\code char} (\'{e}tendu \`{a} une valeur {\code int} en utilisant l'extension
de z\'{e}ro). Cependant, il y a une valeur qu'elle peut retourner qui n'est pas un
caract\`{e}re, {\code EOF}. C'est une macro habituellement d\'{e}finie comme valant
$-1$. Donc, {\code fgetc()} retourne soit un {\code char} \'{e}tendu \`{a} une valeur
{\code int} (qui ressemble \`{a} {\code 000000{\em xx}} en hexa) soit {\code EOF} 
(qui ressemble \`{a} {\code FFFFFFFF} en hexa).

\begin{figure}[t]
\begin{lstlisting}[stepnumber=0,frame=tlrb]{}
char ch;
while( (ch = fgetc(fp)) != EOF ) {
  /* faire qqch avec ch */
}
\end{lstlisting}
\caption{Utilisation de la fonction fgetc}
\label{fig:IObug}
\end{figure}

Le probl\`{e}me avec le programme de la Figure~\ref{fig:IObug} est que {\code fgetc()}
renvoie un {\code int}, mais cette valeur est stock\'{e}e dans un {\code char}.
Le C tronquera alors les bits de poids fort pour faire tenir la valeur de
l'{\code int} dans le {\code char}. Le seul probl\`{e}me est que les nombres
(en hexa) {\code 000000FF} et {\code FFFFFFFF} seront tous deux tronqu\'{e}
pour donner l'octet {\code FF}. Donc, la boucle while ne peut pas distinguer
la lecture de l'octet {\code FF} dans le fichier et la fin de ce fichier.

Ce que fait exactement le code dans ce cas change selon que le {\code char}
est sign\'{e} ou non. Pourquoi ? Parce~que ligne~2, {\code ch} est compar\'{e} avec
{\code EOF}. Comme {\code EOF} est un {\code int}\footnote{Il est 
courant de penser que les fichiers ont un carct\`{e}re EOF comme dernier caract\`{e}re.
Ce n'est \emph{pas} vrai !}, {\code ch} sera \'{e}tendu \`{a} un {\code int} afin
que les deux valeurs compar\'{e}es soient de la m\^{e}me taille\footnote{La raison de
cette n\'{e}cessit\'{e} sera expos\'{e}e plus tard.}.
Comme la Figure~\ref{fig:charExt} le montrait, le fait qu'une variable soit
sign\'{e}e ou non est tr\`{e}s important.

Si le {\code char} n'est pas sign\'{e}, {\code FF} est \'{e}tendu et donne {\code
000000FF}. Cette valeur est compar\'{e}e \`{a} {\code EOF} ({\code FFFFFFFF}) et
est diff\'{e}rente. Donc la boucle ne finit jamais !

So le {\code char} est sign\'{e}, {\code FF} est \'{e}tendu et donne {\code
FFFFFFFF}. Les deux valeurs sont alors \'{e}gale et la boucle se termine.
Cependant, comme l'octet {\code FF} peut avoir \'{e}t\'{e} lu depuis le fichier,
la boucle peut se terminer pr\'{e}matur\'{e}ment.

La solution \`{a} ce probl\`{e}me est de d\'{e}finir la variable {\code ch} comme un
{\code int}, pas un {\code char}. Dans ce cas, aucune troncation ou extension
n'est effectu\'{e}e \`{a} la ligne~2. Dans la boucle, il est sans danger de tronquer
la valeur puisque {\code ch} \emph{doit} alors \^{e}tre r\'{e}ellement un octet.

\index{entiers!extension de signe|)}
\index{entiers!repr\'{e}sentation|)}

\subsection{Arithm\'{e}tique en compl\'{e}ment \`{a} deux\index{compl\'{e}ment \`{a} deux!arithm\'{e}tique|(}}

Comme nous l'avons vu plus t\^{o}t, l'instruction {\code add} effectue une
addition et l'instruction {\code sub} effectue une soustraction. Deux
des bits du registre EFLAGS que ces instructions positionnent sont
\emph{overflow} (d\'{e}passement de capacit\'{e}) et \emph{carry flag} (retenue).
Le drapeau d'overflow est \`{a} 1 si le r\'{e}sultat r\'{e}el de l'op\'{e}ration est trop
grand et ne tient pas dans la destination pour l'arithm\'{e}tique
sign\'{e}e. Le drapeau de retenue est \`{a} 1 s'il y a une  retenue dans le
bit le plus significatif d'une addition d'une soustraction. Donc,
il peut \^{e}tre utilis\'{e} pour d\'{e}tecter un d\'{e}passement de capacit\'{e} en
arithm\'{e}tique non sign\'{e}e. L'utilisation du drapeau de retenue pour 
l'arithm\'{e}tique sign\'{e}e va \^{e}tre vu sous peu. Un des grands avantages du
compl\'{e}ment \`{a}~2 est qu les r\`{e}gles pour l'addition et la soustraction sont 
exactement les m\^{e}mes que pour les entiers non sign\'{e}s. Donc,
{\code add} et {\code sub} peuvent \^{e}tre utilis\'{e}es pour les entiers sign\'{e}s
ou non.
\[
\begin{array}{rrcrr}
 & 002\mathrm{C} & & & 44\\
+& \mathrm{FFFF} & &+&(-1)\\ \cline{1-2} \cline{4-5}
 & 002\mathrm{B} & & & 43
\end{array}
\]
Il y a une retenue de g\'{e}n\'{e}r\'{e}e mais elle ne fait pas partie de la r\'{e}ponse.

\index{entiers!multiplication|(}
\index{MUL|(}
\index{IMUL|(}
Il y a deux instructions de multiplication et de division diff\'{e}rentes.
Tout d'abord, pour multiplier, utilisez les instructions {\code MUL} 
ou {\code IMUL}. L'instruction {\code MUL} est utilis\'{e}e pour multiplier
les nombres non sign\'{e}s et {\code IMUL} est utilis\'{e}e pour multiplier les 
entiers sign\'{e}s. Pourquoi deux instructions diff\'{e}rentes sont n\'{e}cessaires ?
Les r\`{e}gles pour la multiplication sont diff\'{e}rentes pour les nombres non
sign\'{e}s et les nombres sign\'{e}s en compl\'{e}ment \`{a}~2. Pourquoi cela ? Consid\'{e}rons
la multiplication de l'octet FF avec lui-m\^{e}me donnant un r\'{e}sultat sur un
mot. En utilisant la multiplication non sign\'{e}e, il s'agit de 255 fois 255
soit 65025 (ou FE01 en hexa). En utilisant la multiplication sign\'{e}e, il
s'agit de $-1$ fois $-1$ soit 1 (ou 0001 en hexa).

Il y a plusieurs formes pour les instructions de multiplication. La
plus ancienne ressemble \`{a} cela :
\begin{AsmCodeListing}[numbers=none,frame=none]
      mul   source
\end{AsmCodeListing}
La \emph{source} est soit un registre soit une r\'{e}f\'{e}rence m\'{e}moire. Cela ne peut
pas \^{e}tre une valeur imm\'{e}diate. Le type exact de la multiplication d\'{e}pend de
la taille de l'op\'{e}rande source. Si l'op\'{e}rande est un octet, elle est multipli\'{e}e
par l'octet situ\'{e} dans le registre AL et le r\'{e}sultat est stock\'{e} dans les
16 bits de AX. Si la source fait 16~bits, elle est multipli\'{e}e par le mot situ\'{e}
dans AX et le r\'{e}sultat 32 bits est stock\'{e} dans DX:AX. Si la source fait 32 bits,
elle est multipli\'{e}e par EAX et le r\'{e}sultat 64 bits est stock\'{e} dans EDX:EAX\index{registre!EDX:EAX}.
\index{MUL|)}

\begin{table}[t]
\centering
\begin{tabular}{|c|c|c|l|}
\hline
{ \bf dest} & { \bf source1 } & {\bf source2} & \multicolumn{1}{c|}{\bf Action} \\ \hline
            & reg/mem8        &               & AX = AL*source1 \\
            & reg/mem16       &               & DX:AX = AX*source1 \\
            & reg/mem32       &               & EDX:EAX = EAX*source1 \\
reg16       & reg/mem16       &               & dest *= source1 \\
reg32       & reg/mem32       &               & dest *= source1 \\
reg16       & immed8          &               & dest *= immed8 \\
reg32       & immed8          &               & dest *= immed8 \\
reg16       & immed16         &               & dest *= immed16 \\
reg32       & immed32         &               & dest *= immed32 \\
reg16       & reg/mem16       & immed8        & dest = source1*source2 \\
reg32       & reg/mem32       & immed8        & dest = source1*source2 \\
reg16       & reg/mem16       & immed16       & dest = source1*source2 \\
reg32       & reg/mem32       & immed32       & dest = source1*source2 \\
\hline
\end{tabular}
\caption{{\code Instructions imul}\label{tab:imul}}
\end{table}

L'instruction {\code IMUL} a les m\^{e}mes formats que {\code MUL}, mais ajoute \'{e}galement
d'autres formats d'instruction. Il y a des formats \`{a} deux et \`{a} trois op\'{e}randes :
\begin{AsmCodeListing}[numbers=none,frame=none]
      imul   dest, source1
      imul   dest, source1, source2
\end{AsmCodeListing}
Le Tableau~\ref{tab:imul} montre les combinaisons possibles.
\index{IMUL|)}
\index{entiers!multiplication|)}

\index{entiers!division|(}
\index{DIV}
Les deux op\'{e}rateurs de division sont {\code DIV} et {\code IDIV}. Il effectuent
respectivement une division enti\`{e}re non sign\'{e}e et une division enti\`{e}re sign\'{e}e.
Le format g\'{e}n\'{e}ral est :
\begin{AsmCodeListing}[numbers=none,frame=none]
      div   source
\end{AsmCodeListing}
Si la source est sur 8~bits, alors AX est divis\'{e} par l'op\'{e}rande. Le quotient
est stock\'{e} dans AL et le reste dans AH. Si la source est sur 16~bits,
alors DX:AX est divis\'{e} par l'op\'{e}rande. Le quotient est stock\'{e} dans AX et le reste
dans DX. Si la source est sur 32~bits, alors EDX:EAX\index{registre!EDX:EAX} est
divis\'{e} par l'op\'{e}rande, le quotient est stock\'{e} dans EAX et le reste dans EDX.
L'instruction {\code IDIV}\index{IDIV} fonctionne de la m\^{e}me fa�on. Il n'y a pas
d'instructions {\code IDIV} sp\'{e}ciales comme pour {\code IMUL}. Si le quotient est
trop grand pour tenir dans son registre ou que le diviseur vaut 0, le programme
et interrompu et se termine. Une erreur courante est d'oublier d'initialiser DX
ou EDX avant la division.
\index{entiers!division|)}

L'instruction {\code NEG} \index{NEG} inverse son op\'{e}rande en calculant son
compl\'{e}ment \`{a} deux. L'op\'{e}rande peut \^{e}tre n'importe quel registre ou emplacement
m\'{e}moire sur 8~bits, 16~bits, or 32~bits.

\subsection{Programme exemple}
\index{math.asm|(}
\begin{AsmCodeListing}[label=math.asm]
%include "asm_io.inc"
segment .data         ; Chaines affichees
prompt          db    "Entrez un nombre : ", 0
square_msg      db    "Le carre de l'entree vaut ", 0
cube_msg        db    "Le cube de l'entree vaut ", 0
cube25_msg      db    "Le cube de l'entree fois 25 vaut ", 0
quot_msg        db    "Le quotient de cube/100 vaut ", 0
rem_msg         db    "Le reste de cube/100 vaut ", 0
neg_msg         db    "La n\'{e}gation du reste vaut ", 0

segment .bss
input   resd 1

segment .text
        global  _asm_main
_asm_main:
        enter   0,0               ; routine d'initialisation
	pusha

        mov     eax, prompt
        call    print_string

        call    read_int
        mov     [input], eax

        imul    eax               ; edx:eax = eax * eax
        mov     ebx, eax          ; sauvegarde le r\'{e}sultat dans ebx
        mov     eax, square_msg
        call    print_string
        mov     eax, ebx
        call    print_int
        call    print_nl

        mov     ebx, eax
        imul    ebx, [input]      ; ebx *= [entree]
        mov     eax, cube_msg
        call    print_string
        mov     eax, ebx
        call    print_int
        call    print_nl

        imul    ecx, ebx, 25      ; ecx = ebx*25
        mov     eax, cube25_msg
        call    print_string
        mov     eax, ecx
        call    print_int
        call    print_nl

        mov     eax, ebx
        cdq                       ; initialise edx avec extension de signe
        mov     ecx, 100          ; on ne peut pas diviser par une valeur imm\'{e}diate
        idiv    ecx               ; edx:eax / ecx
        mov     ecx, eax          ; sauvegarde le r\'{e}sultat dans ecx
        mov     eax, quot_msg
        call    print_string
        mov     eax, ecx
        call    print_int
        call    print_nl
        mov     eax, rem_msg
        call    print_string
        mov     eax, edx
        call    print_int
        call    print_nl
        
        neg     edx               ; inverse le reste
        mov     eax, neg_msg
        call    print_string
        mov     eax, edx
        call    print_int
        call    print_nl

        popa
        mov     eax, 0            ; retour au C
        leave                     
        ret
\end{AsmCodeListing}
\index{math.asm|)}

\subsection{Arithm\'{e}tique en pr\'{e}cision \'{e}tendue\label{sec:ExtPrecArith} \index{entiers!pr\'{e}cision \'{e}tendue|(}}}

Le langage assembleur fournit \'{e}galement des instructions qui permettent
d'effectuer des additions et des soustractions sur des nombres plus grands
que des doubles mots. Ces instructions utilisent le drapeau de retenue.
Comme nous l'avons dit plus haut, les instructions {\code ADD}
\index{ADD} et {\code SUB} \index{SUB} modifient le drapeau de retenue
si une retenue est g\'{e}n\'{e}r\'{e}e. Cette information stock\'{e}e dans le drapeau
de retenue peut \^{e}tre utilis\'{e}e pour additionner ou soustraire de grands
nombres en morcelant l'op\'{e}ration en doubles mots (ou plus petit).

Les instructions {\code ADC} \index{ADC} et {\code SBB} \index{SBB} 
utilisent les informations donn\'{e}es par le drapeau de retenue.
L'instruction {\code ADC} effectue l'op\'{e}ration suivante :
\begin{center}
{\code \emph{op\'{e}rande1} = \emph{op\'{e}rande1} + drapeau de retenue + \emph{op\'{e}rande2} }
\end{center}
L'instruction {\code SBB} effectue :
\begin{center}
{\code \emph{op\'{e}rande1} = \emph{op\'{e}rande1} - drapeau de retenue - \emph{op\'{e}rande2} }
\end{center}
Comment sont elles utilis\'{e}es ? Consid\'{e}rons la somme d'entiers 64~bits dans 
EDX:EAX\index{registre!EDX:EAX} et EBX:ECX. Le code suivant stockerait la somme
dans EDX:EAX :
\begin{AsmCodeListing}[frame=none]
      add    eax, ecx       ; additionne les 32 bits de poids faible
      adc    edx, ebx       ; additionne les 32 bits de poids fort et la retenue
\end{AsmCodeListing}
La soustraction est tr\`{e}s similaire. Le code suivant soustrait EBX:ECX de 
EDX:EAX :
\begin{AsmCodeListing}[frame=none]
      sub    eax, ecx       ; soustrait les 32 bits de poids faible
      sbb    edx, ebx       ; soustrait les 32 bits de poids fort et la retenue
\end{AsmCodeListing}

Pour les nombres \emph{vraiment} grands, une boucle peut \^{e}tre utilis\'{e}e (voir
Section~\ref{sec:control}). Pour une boucle de somme, il serait pratique
d'utiliser l'instruction {\code ADC} pour toutes les it\'{e}rations (au lieu de toutes
sauf la premi\`{e}re). Cela peut \^{e}tre fait en utilisant l'instruction {\code CLC} \index{CLC}
(CLear Carry) juste avant que la boucle ne commence pour initialiser le drapeau de
retenue \`{a} 0. Si le drapeau de retenue vaut 0, il n'y a pas de diff\'{e}rence entre
les instructions {\code ADD} et {\code ADC}. La m\^{e}me id\'{e}e peut \^{e}tre utilis\'{e}e pour la
soustraction.
\index{entiers!pr\'{e}cision \'{e}tendue|)}
\index{compl\'{e}ment \`{a} deux!arithm\'{e}tique|)}

\section{Structures de Contr\^{o}le}
\label{sec:control}
Les langages de haut niveau fournissent des structures de contr\^{o}le de haut niveau
(\emph{p.e.}, les instructions \emph{if} et \emph{while}) qui contr\^{o}lent le flot
d'ex\'{e}cution. Le langage assembleur ne fournit pas de structures de contr\^{o}le aussi
complexes. Il utilise \`{a} la place l'inf\^{a}me \emph{goto} et l'utiliser de mani\`{e}re
inappropri\'{e}e peut conduire \`{a} du code spaghetti ! Cependant, il \emph{est} possible
d'\'{e}crire des programmes structur\'{e}s en assembleur. La proc\'{e}dure de base est de
concevoir la logique du programme en utilisant les structures de contr\^{o}le de haut
niveau habituelles et de traduire cette conception en langage assembleur (comme
le ferait un compilateur).

\subsection{Comparaison \index{entiers!comparaisons|(} \index{CMP|(}}
%TODO: Make a table of all the FLAG bits

\index{registre!FLAGS|(}
Les structures de contr\^{o}le d\'{e}cident ce qu'il faut faire en comparant des donn\'{e}es.
En assembleur, le r\'{e}sultat d'une comparaison est stock\'{e} dans le registre
FLAGS pour \^{e}tre utilis\'{e} plus tard. Le 80x86 fournit l'instruction 
{\code CMP} pour effectuer des comparaisons. Le registre FLAGS est positionn\'{e} selon
la diff\'{e}rence entre les deux op\'{e}randes de l'instruction {\code CMP}. Les op\'{e}randes
sont soustraites et le registre FLAGS est positionn\'{e} selon le r\'{e}sultat, mais ce
r\'{e}sultat n'est stock\'{e} \emph{nulle part}. Si vous avez besoin du r\'{e}sultat, utilisez
l'instruction {\code SUB} au lieu de {\code CMP}.

\index{entiers!non sign\'{e}s|(}
Pour les entiers non sign\'{e}s, il y a deux drapeaux (bits dans le registre FLAGS)
qui sont importants : les drapeaux z\'{e}ro (ZF) \index{registre!FLAGS!ZF} et
retenue (CF) \index{registre!FLAGS!CF}. Le drapeau zero est allum\'{e}~(1)
si le r\'{e}sultat de la diff\'{e}rence serait 0. Consid\'{e}rons une comparaison comme :
\begin{AsmCodeListing}[frame=none, numbers=none]
      cmp    vleft, vright
\end{AsmCodeListing}
La diff\'{e}rence {\code vleft~-~vright} est calcul\'{e}e et les drapeaux sont positionn\'{e}s
en fonction. Si la diff\'{e}rence calcul\'{e}e par {\code CMP} vaut 0, {\code 
vleft~=~vright}, alors ZF est allum\'{e}(\emph{i.e.} 1) et CF est \'{e}teint
(\emph{i.e.} 0). Si {\code vleft~>~vright}, ZF est \'{e}teint et CF \'{e}galement.
Si {\code vleft~<~vright}, alors ZF est \'{e}teint et CF est allum\'{e} (retenue).
\index{entiers!non sign\'{e}s|)}

\index{entiers!sign\'{e}s|(} 
Pour les entiers sign\'{e}s, il y a trois drapeaux importants : le drapeau
z\'{e}ro \index{registre!FLAGS!ZF} (ZF), le drapeau d'overflow \index{registre!FLAGS!OF}(OF)
et le drapeau de signe\index{register!FLAGS!SF} (SF). \MarginNote{Pourquoi SF~=~OF
si {\code vleft~>~vright} ? S'il n'y a pas d'overflow, alors la diff\'{e}rence aura la
valeur correcte et doit \^{e}tre positive ou nulle. Donc, SF~=~OF~=~0.
Par contre, s'il y a un overflow, la diff\'{e}rence n'aura pas la valeur correcte
(et sera en fait n\'{e}gative). Donc, SF~=~OF~=~1.}Le drapeau d'overflow est
allum\'{e} si le r\'{e}sultat d'une op\'{e}ration d\'{e}passe la capacit\'{e} (de trop ou de trop peu).
Le drapeau de signe est allum\'{e} si le r\'{e}sultat d'une op\'{e}ration est n\'{e}gatif.
Si {\code vleft~=~vright}, ZF est allum\'{e} (comme pour les entiers non sign\'{e}s).
Si {\code vleft~>~vright}, ZF est \'{e}teint et SF~=~OF.  Si {\code vleft~<~vright}, 
ZF est \'{e}teint et SF~$\neq$~OF.
\index{entiers!sign\'{e}s|)}

N'oubliez pas que d'autres instructions peuvent aussi changer le registre
FLAGS, pas seulement {\code CMP}.
\index{CMP|)}
\index{entiers!comparaisons|)}
\index{registre!FLAGS|)}
\index{entiers|)}

\subsection{Instructions de branchement}

Les instructions de branchement peuvent transf\'{e}rer l'ex\'{e}cution \`{a} n'importe quel
point du programme. En d'autres termes, elles agissent comme un \emph{goto}. Il
y a deux types de branchements : conditionnel et inconditionnel. Un branchement
inconditionnel est exactement comme un goto, il effectue toujours le branchement.
Un branchement conditionnel peut effectuer le branchement ou non selon les 
drapeaux du registre FLAGS. Si un branchement conditionnel n'effecute pas le
branchement, le contr\^{o}le passe \`{a} l'instruction suivante.

\index{JMP|(}
L'instruction {\code JMP} (abbr\'{e}viation de \emph{jump}) effectue les
branchements inconditionnels. Son seul argument est habituellement
l'\emph{\'{e}tiquette de code} de l'instruction \`{a} laquelle se brancher.
L'assembleur ou l'\'{e}diteur de liens remplacera l'\'{e}tiquette par l'adresse 
correcte de l'instruction. C'est une autre des op\'{e}rations p\'{e}nibles
que l'assembleur fait pour rendre la vie du programmeur plus simple.
Il est important de r\'{e}aliser que l'instruction imm\'{e}diatement apr\`{e}s
l'instruction {\code JMP} ne sera jamais ex\'{e}cut\'{e}e \`{a} moins qu'une autre
instruction ne se branche dessus !

Il y a plusieurs variantes de l'instruction de saut :
\begin{description}

\item[SHORT] Ce saut est tr\`{e}s limit\'{e} en port\'{e}e. Il ne peut bouger que
de plus ou moins 128~octets en m\'{e}moire. L'avantage de ce type de saut
est qu'il utilise moins de m\'{e}moire que les autres. Il utilise un seul
octet sign\'{e} pour stocker le \emph{d\'{e}placement} du saut. Le d\'{e}placement
est le nombre d'octets \`{a} sauter vers l'avant ou vers l'arri\`{e}re (le
d\'{e}placement est ajout\'{e} \`{a} EIP). Pour sp\'{e}cifier un saut court, utilisez
le mot cl\'{e} {\code SHORT} imm\'{e}diatement avant l'\'{e}tiquette dans 
l'instruction {\code JMP}.

\item[NEAR] Ce saut est le saut par d\'{e}faut, que ce soit pour les branchements
inconditionnels ou conditionnels, il peut \^{e}tre utilis\'{e} pour sauter vers
n'importe quel emplacement dans un segment. En fait, le 80386 supporte
deux types de sauts proches. L'un utilise 2 ocets pour le d\'{e}placement.
Cela permet de se d\'{e}placer en avant ou en arri\`{e}re d'environ 32~000~octets.
L'autre type utilise quatre octets pour le d\'{e}placement, ce qui, bien s\^{u}r,
permet de se d\'{e}placer n'importe o\`{u} dans le segment. Le type de saut \`{a}
deux octets peut \^{e}tre sp\'{e}cifi\'{e} en pla�ant le mot cl\'{e} {\code WORD}
avant l'\'{e}tiquette dans l'instruction {\code JMP}.

\item[FAR] Ce saut permet de passer le contr\^{o}le \`{a} un autre segment de code.
C'est une chose tr\`{e}s rare en mode prot\'{e}g\'{e} 386.
\end{description}

Les \'{e}tiquettes de code valides suivent les m\^{e}mes r\`{e}gles que les \'{e}tiquettes de
donn\'{e}es. Les \'{e}tiquettes de code sont d\'{e}finies en les pla�ant dans le segment
de code au d\'{e}but de l'instruction qu'ils \'{e}tiquettent. Deux~points sont plac\'{e}s
\`{a} la fin de l'\'{e}tiquette lors de sa d\'{e}finition. Ces deux~points ne font \emph{pas}
partie du nom.
\index{JMP|)}

\index{branchement conditionnel|(}
\begin{table}[t]
\center
\begin{tabular}{|ll|}
\hline
JZ  & saute uniquement si ZF est allum\'{e} \\
JNZ & saute uniquement si ZF est \'{e}teint \\
JO  & saute uniquement si OF est allum\'{e}\\
JNO & saute uniquement si OF est \'{e}teint \\
JS  & saute uniquement si SF est allum\'{e} \\
JNS & saute uniquement si SF est \'{e}teint \\
JC  & saute uniquement si CF est allum\'{e} \\
JNC & saute uniquement si CF est \'{e}teint \\
JP  & saute uniquement si PF est allum\'{e} \\
JNP & saute uniquement si PF est \'{e}teint \\
\hline
\end{tabular}
\caption{Branchements Conditionnels Simples\label{tab:SimpBran} \index{JZ} \index{JNZ}
        \index{JO} \index{JNO} \index{JS} \index{JNS} \index{JC} \index{JNC}
        \index{JP} \index{JNP}}
\end{table}

Il y a beaucoup d'instructions de branchement conditionnel diff\'{e}rentes. Elles
prennent toutes une \'{e}tiquette de code comme seule op\'{e}rande. Les plus simples
se contentent de regarder un drapeau dans le registre FLAGS pour d\'{e}terminer
si elles doivent sauter ou non. Voyez le Tableau~\ref{tab:SimpBran} pour une
liste de ces instructions (PF est le \emph{drapeau de parit\'{e}} \index{registre!FLAGS!PF}
qui indique si le nombre de bits dans les 8~bits de poids faible du r\'{e}sultat est pair ou
impair).

Le pseudo-code suivant :
\begin{Verbatim}
if ( EAX == 0 )
  EBX = 1;
else
  EBX = 2;
\end{Verbatim}
peut \^{e}tre \'{e}crit de cette fa�on en assembleur :
\begin{AsmCodeListing}[frame=none]
      cmp    eax, 0            ; positionne les drapeaux (ZF allum\'{e} si eax - 0 = 0)
      jz     thenblock         ; si ZF est allum\'{e}, branchement vers thenblock
      mov    ebx, 2            ; partie ELSE du IF
      jmp    next              ; saute par dessus la partie THEN du IF
thenblock:
      mov    ebx, 1            ; partie THEN du IF
next:
\end{AsmCodeListing}

Les autres comparaisons ne sont pas si simples si l'on se contente des branchements
conditionnels du Tableau~\ref{tab:SimpBran}. Pour illustrer, consid\'{e}rons le
pseudo-code suivant :
\begin{Verbatim}
if ( EAX >= 5 )
  EBX = 1;
else
  EBX = 2;
\end{Verbatim}
Si EAX est plus grand ou \'{e}gal \`{a} 5, ZF peut \^{e}tre allum\'{e} ou non et SF
sera \'{e}gal \`{a} OF. Voici le code assembleur qui teste ces conditions
(en supposant que EAX est sign\'{e}) :
\begin{AsmCodeListing}[frame=none]
      cmp    eax, 5
      js     signon            ; goto signon si SF = 1
      jo     elseblock         ; goto elseblock si OF = 1 et SF = 0
      jmp    thenblock         ; goto thenblock si SF = 0 et OF = 0
signon:
      jo     thenblock         ; goto thenblock si SF = 1 et OF = 1
elseblock:
      mov    ebx, 2
      jmp    next
thenblock:
      mov    ebx, 1
next:
\end{AsmCodeListing}

\begin{table}
\center
\begin{tabular}{|ll|ll|}
\hline
\multicolumn{2}{|c|}{\textbf{Sign\'{e}}} & \multicolumn{2}{c|}{\textbf{Non Sign\'{e}}} \\
\hline
JE & saute si {\code vleft = vright} & JE & saute si {\code vleft = vright} \\
JNE & saute si {\code vleft $\neq$ vright} & JNE & saute si {\code vleft $\neq$ vright} \\
JL, JNGE & saute si {\code vleft < vright} & JB, JNAE & saute si {\code vleft < vright} \\
JLE, JNG & saute si {\code vleft $\leq$ vright} & JBE, JNA & saute si {\code vleft $\leq$ vright} \\
JG, JNLE & saute si {\code vleft > vright} & JA, JNBE & saute si {\code vleft > vright} \\
JGE, JNL & saute si {\code vleft $\geq$ vright} & JAE, JNB & saute si {\code vleft $\geq$ vright} \\
\hline
\end{tabular}
\caption{Instructions de Comparaison Sign\'{e}es et Non Sign\'{e}es\label{tab:CompBran} \index{JE} \index{JNE}
         \index{JL} \index{JNGE} \index{JLE} \index{JNG} \index{JG} \index{JNLE} \index{JGE}
         \index{JNL}}
\end{table}

Le code ci-dessus est tr\`{e}s confus. Heureusement, le 80x86 fournit des instructions
de branchement additionnelles pour rendre ce type de test \emph{beaucoup} plus simple.
Il y a des versions sign\'{e}es et non sign\'{e}es de chacune. Le Tableau~\ref{tab:CompBran} 
montre ces instructions. Les branchements \'{e}gal et non \'{e}gal (JE et JNE) sont les m\^{e}mes
pour les entiers sign\'{e}s et non sign\'{e}s (en fait, JE et JNE sont r\'{e}ellement identiques \`{a}
JZ et JNZ, respectivement). Chacune des autres instructions de branchement a deux
synonymes. Par exemple, regardez JL (jump less than, saut si inf\'{e}rieur \`{a}) et
JNGE (jump not greater than or equal to, saut si non plus grand ou \'{e}gal). Ce sont les
m\^{e}mes instructions car :
\[ x < y \Longrightarrow \mathbf{not}( x \geq y ) \]
Les branchements non sign\'{e}s utilisent A pour \emph{above} (au dessus) et B
pour \emph{below} (en dessous) \`{a} la place de L et G.

En utilisant ces nouvelles instructions de branchement, le pseudo-code ci-dessus
peut \^{e}tre traduit en assembleur plus facilement.
\begin{AsmCodeListing}[frame=none]
      cmp    eax, 5
      jge    thenblock
      mov    ebx, 2
      jmp    next
thenblock:
      mov    ebx, 1
next:
\end{AsmCodeListing}
\index{branchement conditionnel|)}

\subsection{Les instructions de boucle}

Le 80x86 fournit plusieurs instructions destin\'{e}es \`{a} impl\'{e}menter des boucles de
style \emph{for}. Chacune de ces instructions prend une \'{e}tiquette de code comme
seule op\'{e}rande.
\begin{description}
\item[LOOP] 
\index{LOOP}
Decr\'{e}mente ECX, si ECX $\neq$ 0, saute vers l'\'{e}tiquette
\item[LOOPE, LOOPZ]
\index{LOOPE} \index{LOOPZ}
D\'{e}cremente ECX (le registre FLAGS n'est pas modifi\'{e}), si
                    ECX $\neq$ 0 et ZF = 1, saute
\item[LOOPNE, LOOPNZ]
\index{LOOPNE} \index{LOOPNZ}
D\'{e}cremente ECX (FLAGS inchang\'{e}), si ECX $\neq$ 0
                      et ZF = 0, saute
\end{description}

Les deux derni\`{e}res instructions de boucle sont utiles pour les boucles de recherche
s\'{e}quentielle. Le pseudo-code suivant :
\begin{lstlisting}[stepnumber=0]{}
sum = 0;
for( i=10; i >0; i-- )
  sum += i;
\end{lstlisting}
\noindent peut \^{e}tre traduit en assembleur de la fa�on suivante :
\begin{AsmCodeListing}[frame=none]
      mov    eax, 0          ; eax est sum
      mov    ecx, 10         ; ecx est i
loop_start:
      add    eax, ecx
      loop   loop_start
\end{AsmCodeListing}

\section{Traduire les Structures de Contr\^{o}le Standards}

Cette section d\'{e}crit comment les structures de contr\^{o}le standards des langages
de haut niveau peuvent \^{e}tre impl\'{e}ment\'{e}es en langage assembleur.

\subsection{Instructions if\index{instruction if|(}}
Le pseudo-code suivant :
\begin{lstlisting}[stepnumber=0]{}
if ( condition )
  then_block;
else
  else_block;
\end{lstlisting}
\noindent peut \^{e}tre impl\'{e}ment\'{e} de cette fa�on :
\begin{AsmCodeListing}[frame=none]
      ; code pour positionner FLAGS
      jxx    else_block    ; choisir xx afin de sauter si la condition est fausse
      ; code pour le bloc then
      jmp    endif
else_block:
      ; code pour le bloc else
endif:
\end{AsmCodeListing}

S'il n'y a pas de else, le branchement {\code else\_block} peut \^{e}tre remplac\'{e} par un
branchement \`{a}  {\code endif}.
\begin{AsmCodeListing}[frame=none]
      ; code pour positionner FLAGS
      jxx    endif          ; choisir xx afin de sauter si la condition est fausse
      ; code pour le bloc then
endif:
\end{AsmCodeListing}
\index{instruction if|)}

\subsection{Boucles while\index{boucle while|(}}
La boucle \emph{while} est une boucle avec un test au d\'{e}but :
\begin{lstlisting}[stepnumber=0]{}
while( condition ) {
  corps de la boucle;
}
\end{lstlisting}
\noindent Cela peut \^{e}tre traduit en :
\begin{AsmCodeListing}[frame=none]
while:
      ; code pour positionner FLAGS selon la condition
      jxx    endwhile       ; choisir xx pour sauter si la condition est fausse
      ; body of loop
      jmp    while
endwhile:
\end{AsmCodeListing}
\index{boucle while|)}

\subsection{Boucles do while\index{boucle do while|(}}
La boucle \emph{do while} est une boucle avec un test \`{a} la fin :
\begin{lstlisting}[stepnumber=0]{}
do {
  corps de la boucle;
} while( condition );
\end{lstlisting}
\noindent Cela peut \^{e}tre traduit en :
\begin{AsmCodeListing}[frame=none]
do:
      ; corps de la boucle
      ; code pour positionner FLAGS selon la condition
      jxx    do          ; choisir xx pour se brancher si la condition est vraie
\end{AsmCodeListing}
\index{boucle do while|)}


\begin{figure}[t]
\begin{lstlisting}[frame=tlrb,escapechar=\$]{}
  unsigned guess;   /* $\slshape candidat courant pour \^{e}tre premier $ */
  unsigned factor;  /* diviseur possible de guess */
  unsigned limit;   /* $\slshape rechercher les nombres premiers jusqu'\`{a} cette valeur $*/

  printf("Rechercher les nombres premier jusqu'a : ");
  scanf("%u", &limit);
  printf("2\n");    /* traite les deux premier nombres premiers comme  */
  printf("3\n");    /* $\slshape des cas sp\'{e}ciaux$  */
  guess = 5;        /* candidat initial */
  while ( guess <= limit ) {
    /* recherche un diviseur du candidat */
    factor = 3;
    while ( factor*factor < guess &&
            guess % factor != 0 )
     factor += 2;
    if ( guess % factor != 0 )
      printf("%d\n", guess);
    guess += 2;    /* ne regarder que les nombres impairs */
  }
\end{lstlisting}
\caption{}\label{fig:primec}
\end{figure}

\section{Exemple : Trouver des Nombres Premiers}
Cette section d\'{e}taille un programme qui trouve des nombres premiers. Rappelez
vous que les nombres premiers ne sont divisibles que par 1 et par eux-m\^{e}mes.
Il n'y a pas de formule pour les rechercher. La m\'{e}thode de base qu'utilise 
ce programme est de rechercher les diviseurs de tous les nombres impairs
\footnote{2 est le seul nombre premier pair.} sous une limite donn\'{e}e. Si aucun
facteur ne peut \^{e}tre trouv\'{e} pour un nombre impair, il est premier. La
Figure~\ref{fig:primec} montre l'algorithme de base \'{e}crit en C.

Voici la version assembleur :
\index{prime.asm|(}
\begin{AsmCodeListing}[label=prime.asm]
%include "asm_io.inc"
segment .data
Message         db      "Rechercher les nombres premiers jusqu'a : ", 0

segment .bss
Limit           resd    1               ;\small rechercher les nombres premiers jusqu'\`{a} cette limite
Guess           resd    1               ; candidat courant pour \^{e}tre premier

segment .text
        global  _asm_main
_asm_main:
        enter   0,0               ; routine d'intialisation
        pusha

        mov     eax, Message
        call    print_string
        call    read_int             ; scanf("%u", &limit);
        mov     [Limit], eax

        mov     eax, 2               ; printf("2\(\backslash\)n");
        call    print_int
        call    print_nl
        mov     eax, 3               ; printf("3\(\backslash\)n");
        call    print_int
        call    print_nl

        mov     dword [Guess], 5     ; Guess = 5;
while_limit:                         ; while ( Guess <= Limit )
        mov     eax,[Guess]
        cmp     eax, [Limit]
        jnbe    end_while_limit      ; on utilise jnbe car les nombres ne sont pas signes

        mov     ebx, 3               ; ebx est factor = 3;
while_factor:
        mov     eax,ebx
        mul     eax                  ; edx:eax = eax*eax
        jo      end_while_factor     ; si la r\'{e}ponse ne tient pas dans eax seul
        cmp     eax, [Guess]
        jnb     end_while_factor     ; si !(factor*factor < guess)
        mov     eax,[Guess]
        mov     edx,0
        div     ebx                  ; edx = edx:eax % ebx
        cmp     edx, 0
        je      end_while_factor     ; si !(guess % factor != 0)

        add     ebx,2                ; factor += 2;
        jmp     while_factor
end_while_factor:
        je      end_if               ; si !(guess % factor != 0)
        mov     eax,[Guess]          ; printf("%u\(\backslash\)n")
        call    print_int
        call    print_nl
end_if:
        add     dword [Guess], 2     ; guess += 2
        jmp     while_limit
end_while_limit:

        popa
        mov     eax, 0            ; retour au C
        leave                     
        ret
\end{AsmCodeListing}
\index{prime.asm|)}
