% -*-LaTex-*-
%front matter of pcasm book

\chapter{Pr\'{e}face}

\section*{Objectif}

L'objectif de ce livre est de permettre au lecteur de mieux comprendre
comment les ordinateurs fonctionnent r\'{e}ellement \`{a} un niveau plus bas que
les langages de programmation comme Pascal. En ayant une compr\'{e}hension
plus profonde de la fa�on dont fonctionnent les ordinateurs, le lecteur
peu devenir plus productif dans le d\'{e}veloppement de logiciel dans des langages
de plus haut niveau comme le C et le C++. Apprendre \`{a} programmer en
assembleur est un excellent moyen d'atteindre ce but. Les autres
livres d'assembleur pour PC apprennent toujours \`{a} programmer le 
processeur 8086 qu'utilisait le PC originel de 1981 ! Le processeur
8086 ne supportait que le mode \emph{r\'{e}el}. Dans ce mode, tout
programme peu adresser n'importe quel endroit de la m\'{e}moire ou n'importe
quel p\'{e}riph\'{e}rique de l'ordinateur. Ce mode n'est pas utilisable pour un
syst\`{e}me d'exploitation s\'{e}curis\'{e} et multit\^{a}che. Ce livre parle plut\^{o}t
de la fa�on de programmer les processeurs 80386 et plus r\'{e}cents en mode
\emph{prot\'{e}g\'{e}} (le mode dans lequel fonctionnent Windows et Linux).
Ce mode supporte les fonctionnalit\'{e}s que les syst\`{e}mes d'exploitation
modernes offrent, comme la m\'{e}moire virtuelle et la protection m\'{e}moire.
Il y a plusieurs raisons d'utiliser le mode prot\'{e}g\'{e} :
\begin{enumerate}
\item Il est plus facile de programmer en mode prot\'{e}g\'{e} qu'en mode r\'{e}el 8086
	que les autres livres utilisent.
\item Tous les syst\`{e}mes d'exploitation PC modernes fonctionnent en mode prot\'{e}g\'{e}.
\item Il y a des logiciels libres disponibles qui fonctionnent dans ce mode.
\end{enumerate}
Le manque de livres sur la programmation en assembleur PC en mode prot\'{e}g\'{e} est la
principale raison qui a conduit l'auteur \`{a} \'{e}crire ce livre.

Comme nous y avons fait allusion ci-dessus, ce dexte utilise des logiciels
Libres/Open~Source : l'assembleur NASM et le compilateur C/C++ DJGPP.
Les deux sont disponibles en t\'{e}l\'{e}chargement sur Internet. Ce texte parle \'{e}galement
de l'utilisation de code assembleur NASM sous Linux et avec les compilateurs
C/C++ de Borland et Microsoft sous Windows. Les exemples pour toutes ces
plateformes sont disponibles sur mon site Web : 
{\code http://www.drpaulcarter.com/pcasm}.
Vous \emph{devez} t\'{e}l\'{e}charger le code exemple si vous voulez assembler et
ex\'{e}cuter la plupart des exemples de ce tutorial.

Soyez conscient que ce texte ne tente pas de couvrir tous les aspects de la
programmation assembleur. L'auteur a essay\'{e} de couvrir les sujets les plus
importants avec lesquels \emph{tous} les programmeurs devraient \^{e}tre familiers.

\section*{Remerciements}

L'auteur voutrait remercier les nombreux programmeurs qui ont contribu\'{e}
au mouvement Libre/Open~Source. Tous les programmes et m\^{e}me ce livre
lui-m\^{e}me ont \'{e}t\'{e} cr\'{e}\'{e}s avec des logiciels gratuits. L'auteur voudrait
remercier en particulier John~S.~Fine, Simon~Tatham, Julian~Hall et les
autres d\'{e}veloppeurs de l'assembleur NASM sur lequel tous les exemples de
ce livre sont bas\'{e}s ; DJ Delorie pour le d\'{e}veloppement du compilateur C/C++
DJGPP utilis\'{e} ; les nombreuses personnes qui ont contribu\'{e} au compilateur
GNU gcc sur lequel DJGPP est bas\'{e} ; Donald Knuth et les autres pour avoir
d\'{e}velopp\'{e} les langages composition \TeX\ et \LaTeXe\ qui ont \'{e}t\'{e} utilis\'{e}s
pour produire le livre ; Richard Stallman (fondateur de la Free Software
Foundation), Linus Torvalds (cr\'{e}ateur du noyau Linux) et les autres qui ont
cr\'{e}\'{e} les logiciels sous-jacents utilis\'{e}s pour ce travail.

Merci aux personnes suivantes pour leurs corrections :
\begin{itemize}
\item John S. Fine
\item Marcelo Henrique Pinto de Almeida
\item Sam Hopkins
\item Nick D'Imperio
\item Jeremiah Lawrence
\item Ed Beroset
\item Jerry Gembarowski
\item Ziqiang Peng
\item Eno Compton
\item Josh I Cates
\item Mik Mifflin
\item Luke Wallis
\item Gaku Ueda
\item Brian Heward
\item Chad Gorshing
\item F. Gotti
\item Bob Wilkinson
\item Markus Koegel
\item Louis Taber
\end{itemize}


\section*{Ressources sur Internet}
\begin{center}
\begin{tabular}{|ll|}
\hline
Page de l'auteur & {\code http://www.drpaulcarter.com/} \\
%NASM   & {\code http://nasm.2y.net/} \\
Page NASM sur SourceForge & {\code http://sourceforge.net/projects/nasm/} \\
DJGPP  & {\code http://www.delorie.com/djgpp} \\
Assembleur Linux & {\code http://www.linuxassembly.org/} \\
The Art of Assembly & {\code http://webster.cs.ucr.edu/} \\
USENET & {\code comp.lang.asm.x86 } \\
Documentation Intel & {\code http://developer.intel.com/design/Pentium4/documentation.htm} \\
\hline
\end{tabular}
\end{center}


\section*{R\'{e}actions}

L'auteur accepte toute r\'{e}action sur ce travailThe author welcomes any feedback on this work.
\begin{center}
\begin{tabular}{ll}
\textbf{E-mail:} & {\code pacman128@gmail.com} \\
\textbf{WWW:}    & {\code http://www.drpaulcarter.com/pcasm} \\
\end{tabular}
\end{center}



