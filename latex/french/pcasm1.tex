% -*-latex-*-
\chapter{Introduction}
\section{Systèmes Numériques}

La mémoire d'un ordinateur est constituée de nombres. Cette mémoire ne
stocke pas ces nombres en décimal (base 10). Comme cela simplifie
grandement le matériel, les ordinateurs stockent toutes les informations
au format binaire (base 2). Tout d'abord, revoyons ensemble le système
décimal.

\subsection{Décimal\index{décimal}}

Les nombres en base 10 sont constitués de 10 chiffres possibles (0-9).
Chaque chiffre d'un nombre est associé à une puissance de 10 selon sa
position dans le nombre. Par exemple~:
\begin{displaymath}
234 = 2 \times 10^2 + 3 \times 10^1 + 4 \times 10^0
\end{displaymath}

\subsection{Binaire\index{binaire|(}}

Les nombres en base 2 sont composés de deux chiffres possibles (0 et 1).
Chaque chiffre est associé à une puissance de 2 selon sa position dans
le nombre (un chiffre binaire isolé est appelé bit). Par exemple~:
\begin{eqnarray*}
11001_2 & = & 1 \times 2^4 + 1 \times 2^3 + 0 \times 2^2 + 0 \times 2^1 
              + 1 \times 2^0 \\
 & = & 16 + 8 + 1 \\
 & = & 25 
\end{eqnarray*}

Cet exemple montre comment passer du binaire au décimal. Le
tableau~\ref{tab:dec-bin} montre la représentation binaire des premiers
nombres.
\begin{table}[t]
\begin{center}
\begin{tabular}{||c|c||cc||c|c||}
\hline
Decimal & Binary & & & Decimal & Binary \\
\hline
0       & 0000   & & & 8       & 1000 \\
\hline
1       & 0001   & & & 9       & 1001 \\
\hline
2       & 0010   & & & 10      & 1010 \\
\hline
3       & 0011   & & & 11      & 1011 \\
\hline
4       & 0100   & & & 12      & 1100 \\
\hline
5       & 0101   & & & 13      & 1101 \\
\hline
6       & 0110   & & & 14      & 1110 \\
\hline
7       & 0111   & & & 15      & 1111 \\
\hline
\end{tabular}
\caption{Décimaux 0 à 15 en Binaire\label{tab:dec-bin}}
\end{center}
\end{table}


\begin{figure}[h]
\begin{center}
\begin{tabular}{|rrrrrrrrp{.1cm}|p{.1cm}rrrrrrrr|}
\hline
& \multicolumn{7}{c}{No previous carry} & & & \multicolumn{7}{c}{Previous carry} & \\
\hline
&  0 & &  0 & &  1 & &  1 & & &  0 & &  0 & &  1 & & 1  & \\
& +0 & & +1 & & +0 & & +1 & & & +0 & & +1 & & +0 & & +1 &  \\
\cline{2-2} \cline{4-4} \cline{6-6} \cline{8-8} \cline{11-11} \cline{13-13} \cline{15-15} \cline{17-17}
& 0  & & 1  & & 1  & & 0  & & & 1  & & 0  & & 0  & & 1 & \\
&    & &    & &    & & c  & & &    & & c  & & c  & & c & \\
\hline
\end{tabular}

\caption{Addition binaire (c signifie \emph{carry}, retenue)\label{fig:bin-add}}
%TODO: Change this so that it is clear that single bits are being added,
%not 4-bit numbers. Ideas: Table or do sums horizontally.
\index{binaire!addition}
\end{center}
\end{figure}

La figure~\ref{fig:bin-add} montre comment des chiffres binaires individuels ({\em
i.e.}, des bits) sont additionnés. Voici un exemple~:

\begin{tabular}{r}
 $11011_2$ \\
+$10001_2$ \\
\hline
$101100_2$ \\
\end{tabular}

Si l'on considère la division décimale suivante~:
\[ 1234 \div 10 = 123\; r\; 4 \]
on peut voir que cette division sépare le chiffre le plus à droite du nombre
et décale les autres chiffres d'une position vers la droite. Diviser par deux
effectue une opération similaire, mais pour les chiffres binaires du nombre.
Considérons la division binaire suivante\footnote{L'indice 2 est utilisé
pour indiquer que le nombre est représenté en binaire, pas en décimal}~:
\[ 1101_2 \div 10_2 = 110_2\; r\; 1 \]
Cette propriété peut être utilisée pour convertir un nombre décimal en son
équivalent binaire comme le montre la Figure~\ref{fig:dec-convert}. Cette
méthode trouve le chiffre binaire le plus à droite en premier, ce chiffre
est appelée le \emph{bit le moins significatif} (lsb, least significant bit). 
Le chiffre le plus à gauche est appelé le \emph{bit le plus significatif} (msb, most
significant bit). L'unité de base de la mémoire consiste en un jeu de 8 bits
appelé \emph{octet} (byte).
\index{binaire|)}

\begin{figure}[t]
\centering
\fbox{\parbox{\textwidth}{
\begin{eqnarray*}
\mathrm{Decimal} & \mathrm{Binary} \\
25 \div 2 = 12\;r\;1 & 11001 \div 10 = 1100\;r\;1 \\
12 \div 2 = 6\;r\;0  & 1100 \div 10 = 110\;r\;0 \\
6 \div 2 = 3\;r\;0   & 110 \div 10 = 11\;r\;0 \\
3 \div 2 = 1\;r\;1   & 11 \div 10 = 1\;r\;1 \\
1 \div 2 = 0\;r\;1   & 1 \div 10 = 0\;r\;1 \\
\end{eqnarray*}

\centering
Donc $25_{10} = 11001_{2}$
}}
\caption{Conversion décimale \label{fig:dec-convert}}
\end{figure}

\subsection{Hexadecimal\index{hexadécimal|(}}

Les nombres hexadécimaux utilisent la base 16. L'hexadécimal
(ou \emph{hexa} en abrégé) peut être utilisé comme notation pour les
nombres binaires. L'hexa a 16 chiffres possibles. Cela pose un problème car
il n'y a pas de symbole à utiliser pour les chiffres supplémentaires après 9.
Par convention, on utilise des lettres pour les représenter.
Les 16 chiffres de l'hexa sont 0-9 puis A, B, C, D, E et F. Le chiffre A équivaut
10 en décimal, B à 11, etc. Chaque chiffre d'un nombre en hexa est associé à une
puissance de 16. Par exemple~:\begin{eqnarray*}
\rm
2BD_{16} & = & 2 \times 16^2 + 11 \times 16^1 + 13 \times 16^0 \\
         & = & 512 + 176 + 13 \\
         & = & 701 \\
\end{eqnarray*}
Pour convertir depuis le décimal vers l'hexa, utilisez la même idée que celle
utilisée pour la conversion binaire en divisant par 16. Voyez la Figure~\ref{fig:hex-conv}
pour un exemple.

\begin{figure}[t]
\centering
\fbox{\parbox{\textwidth}{
\begin{eqnarray*}
589 \div 16 & = & 36\;r\;13 \\
36 \div 16 & = & 2\;r\;4 \\
2 \div 16 & = & 0\;r\;2 \\
\end{eqnarray*}

\centering
Donc $589 = 24\mathrm{D}_{16}$
}}
\caption{\label{fig:hex-conv}}
\end{figure}

La raison pour laquelle l'hexa est utile est que la conversion entre l'hexa
et le binaire est très simple. Les nombres binaires deviennent grands et
incompréhensibles rapidement. L'hexa fournit une façon beaucoup plus compacte
pour représenter le binaire.

Pour convertir un nombre hexa en binaire, convertissez simplement chaque chiffre
hexa en un nombre binaire de 4 bits. Par exemple, $\mathrm{24D}_{16}$ est converti
en \mbox{$0010\;0100\; 1101_2$}. Notez que les 0 de tête des 4 bits sont importants~!
Si le 0 de tête pour le chiffre du milieu de $\mathrm{24D}_{16}$ n'est pas utilisé,
le résultat est faux. Le conversion du binaire vers l'hexa est aussi simple. On effectue
la même chose, dans l'autre sens. Convertissez chaque segment de 4 bits du binaire vers
l'hexa. Commencez à droite, pas à gauche, du nombre binaire. Cela permet de
s'assurer que le procédé utilise les segments de 4 bits corrects\footnote{S'il n'est
pas clair que le point de départ est important, essayez de convertir l'exemple en
partant de la gauche}. Exemple~:\newline

\begin{tabular}{cccccc}
$110$ & $0000$ & $0101$ & $1010$ & $0111$ & $1110_2$ \\
  $6$ & $0$    &   $5$  &   A  &  $7$   &    $\mathrm{E}_{16}$ \\
\end{tabular}\newline

Un nombre de 4 bits est appelé \emph{quadruplet}\index{quadruplet}(nibble). Donc, chaque chiffre hexa
correspond à un quadruplet. Deux quadruplets forment un octet et donc un octet peut
être représenté par un nombre hexa à deux chiffres. La valeur d'un bit va
de 0 à 11111111 en binaire, 0 à FF en hexa et 0 à 255 en décimal.
\index{hexadécimal|)}

\section{Organisation de l'Ordinateur}

\subsection{Mémoire\index{mémoire|(}}

L'unité mémoire de base est l'octet. \index{octet} \MarginNote{La mémoire est mesurée
en kilo octets (~$2^{10} = 1 024$ octets), mega~octets ($2^{20} =
1 048 576$ octets) et giga octets ($2^{30} = 1 073 741 824$ octets).}Un
ordinateur avec 32 mega octets de mémoire peut stocker jusqu'à environ
32 millions d'octets d'informations. Chaque octet en mémoire est étiqueté
par un nombre unique appelé son adresse comme le montre la Figure~\ref{fig:memory}.

\begin{figure}[ht]
\begin{center}
\begin{tabular}{rcccccccc}
Adresse & 0 & 1 & 2 & 3 & 4 & 5 & 6 & 7 \\
\cline{2-9}
Mémoire & \multicolumn{1}{|c}{2A}  & \multicolumn{1}{|c}{45}  
       & \multicolumn{1}{|c}{B8} & \multicolumn{1}{|c}{20} 
       & \multicolumn{1}{|c}{8F} & \multicolumn{1}{|c}{CD} 
       & \multicolumn{1}{|c}{12} & \multicolumn{1}{|c|}{2E} \\
\cline{2-9}
\end{tabular}
\caption{ Adresses Mémoire \label{fig:memory} }
\end{center}
\end{figure}

\begin{table}
\begin{center}
\begin{tabular}{|l|l|}
\hline
mot & 2 octets \\ \hline
double mot & 4 octets \\ \hline
quadruple mot & 8 octets \\ \hline
paragraphe & 16 octets \\ \hline
\end{tabular}
\caption{ Unités Mémoire \label{tab:mem_units} }
\end{center}
\end{table}

Souvent, la mémoire est utilisée par bouts plus grand que des octets isolées.
Sur l'architecture PC, on a donné des noms à ces portions de mémoire plus
grandes, comme le montre le Tableau ~\ref{tab:mem_units}.

Toutes les données en mémoire sont numériques. Les caractères sont stockés en
utilisant un \emph{code caractère} qui fait correspondre des nombres aux caractères.
Un des codes caractère les plus connus est appelé \emph{ASCII} (American
Standard Code for Information Interchange, Code Américain Standard pour l'Echange
d'Informations). Un nouveau code, plus complet, qui supplante l'ASCII est l'Unicode.
Une des différences clés entre les deux codes est que l'ASCII utilise un octet
pour encoder un caractère alors que l'Unicode en utilise deux (ou un \emph{mot}).
Par exemple, l'ASCII fait correspondre l'octet $41_{16}$ ($65_{10}$) au caractère
majuscule \emph{A}~; l'Unicode y fait correspondre le mot $0041_{16}$. Comme
l'ASCII n'utilise qu'un octet, il est limité à 256 caractères diffrérents au 
maximum\footnote{En fait, l'ASCII n'utilise que les 7 bits les plus faible et
donc n'a que 128 valeurs à utiliser.}. L'Unicode étend les valeurs ASCII
à des mots et permet de représenter beaucoup plus de caractères. C'est important
afin de représenter les caractères de tous les langages du monde.
\index{mémoire|)}

\subsection{Le CPU (processeur)\index{processeur|(}}

Le processeur (CPU, Central Processing Unit) est le dispositif physique qui
exécute les instructions. Les instructions que les processeurs peuvent exécuter
sont généralement très simples. Elles peuvent nécessiter que les
données sur lesquelles elles agissent soient présentes dans des emplacements
de stockage spécifiques dans le processeur lui-même appelés \emph{registres}. \index{registre}
Le processeur peut accéder aux données dans les registres plus rapidement qu'aux
données en mémoire. Cependant, le nombre de registres d'un processeur est limité,
donc le programmeur doit faire attention à n'y conserver que les données
actuellement utilisées.

Les instructions que peut exécuter un type de processeur constituent le
\emph{langage machine}. \index{langage machine}Les programmes machine ont une structure beaucoup
plus basique que les langages de plus haut niveau. Les instructions du langage
machine sont encodées en nombres bruts, pas en format texte lisible. Un
processeur doit être capable de décoder les instructions très
rapidement pour fonctionner efficacement. Le langage machine est conçu
avec ce but en tête, pas pour être facilement déchiffrable par les humains.
Les programmes écrits dans d'autres langages doivent être convertis dans le
langage machine natif du processeur pour s'exécuter sur l'ordinateur.
Un \emph{compilateur} \index{compilateur} est un programme qui traduit les programmes écrits
dans un langage de programmation en langage machine d'une architecture
d'ordinateur particulière. En général, chaque type de processeur a son
propre langage machine unique. C'est une des raisons pour lesquelles un
programme écrit pour Mac ne peut pas être exécuté sur un PC.

Les ordinateurs utilisent une \emph{horloge} \index{horloge} pour synchroniser l'exécution des
\MarginNote{\emph{GHz} signifie gigahertz ou un milliards de cycles par seconde.
Un processeur à 1,5GHz a 1,5 milliards d'impulsions horloge par seconde.} instructions.
% La MarginNote est mal placée...
L'horloge tourne à une fréquence fixée (appelée \emph{vitesse d'horloge}).
Lorsque vous achetez un ordinateur à 1,5GHz, 1,5GHz est la fréquence de
cette horloge. L'horloge ne décompte pas les minutes et les secondes. Elle
bat simplement à un rythme constant. Les composants électroniques du processeur
utilisent les pulsations pour effectuer leurs opérations correctement,
comme le battement d'un métronome aide à jouer de la musique à un rythme correct.
Le nombre de battements (ou, comme on les appelle couramment \emph{cycles}) que
recquiert une instruction dépend du modèle et de la génération du processeur.
Le nombre de cycles dépend de l'instruction.


\subsection{La famille des processeurs 80x86\index{processeur!80x86}}

Les PC contiennent un processeur de la famille des Intel 80x86 (ou un clone).
Les processeurs de cette famille ont tous des fonctionnalités en commun,
y compris un langage machine de base. Cependant, les membre les plus
récents améliorent grandement les fonctionnalités.
\begin{description}

\item[8088,8086:] Ces processeurs, du point de vue de la programmation
sont identiques. Ils étaient les processeurs utilisés dans les tous premiers
PC. Il offrent plusieurs registres 16 bits, BX, CX, DX, SI, DI, BP, SP,
CS, DS, SS, ES, IP, FLAGS. Ils ne supportent que jusqu'à 1Mo de mémoire
et n'opèrent qu'en \emph{mode réel}.  Dans ce mode, un programme peut
accéder à n'importe quelle adresse mémoire, même la mémoire des autres
programmes~! Cela rend le débogage et la sécurité très difficiles~! De plus,
la mémoire du programme doit être divisée en \emph{segments}. Chaque segment
ne peut pas dépasser les 64Ko.

\item[80286:] Ce processeur était utilisé dans les PC de type AT. Il apporte
quelques nouvelles instructions au langage machine de base des 8088/86. 
Cependant, sa principale nouvelle fonctionnalité est le 
\emph{mode protégé 16 bits}. Dans ce mode, il peut accéder jusqu'à 16Mo de mémoire
et empêcher les programmes d'accéder à la mémoire des uns et des autres.
Cependant, les programmes sont toujours divisés en segments qui ne peuvent pas
dépasser les 64Ko.

\item[80386:] Ce processeur a grandement amélioré le 80286. Tout d'abord,
il étend la plupart des registres à 32 bits (EAX, EBX, ECX, EDX, ESI, EDI, EBP, ESP,
EIP) et ajoute deux nouveaux registres 16 bits~: FS et GS. Il ajoute également un
nouveau \emph{mode protégé 32 bits}. Dans ce mode, il peut accéder jusqu'à 4Go
de mémoire. Les programmes sont encore divisés en segments mais maintenant,
chaque segment peut également faire jusqu'à 4Go~!

\item[80486/Pentium/Pentium Pro:] Ces membres de la famille 80x86 apportent
très peu de nouvelles fonctionnalités. Il accélèrent principalement l'exécution
des instructions.

\item[Pentium MMX:] Ce processeur ajoute les instructions MMX (MultiMedia eXentions)
au Pentium. Ces instructions peuvent accélérer des opérations graphiques courantes.

\item[Pentium II:] C'est un processeur Pentium Pro avec les instructions MMX
(Le Pentium III est grossièrement un Pentium II plus rapide).

\end{description}
\index{processeur|)}

\subsection{Regitres 16 bits du 8086\index{registre|(}}

Le processeur 8086 original fournissait quatre registres généraux de 16 bits~:
AX, BX, CX et DX. Chacun de ces registres peut être décomposé en deux
registres de 8 bits. Par exemple, le registre AX pouvait être décomposé
en AH et AL comme le montre la  Figure~\ref{fig:AX_reg}. Le registre
AH contient les 8 bits de poids fort de AX et AL contient les 8 bits
de poids faivble. Souvent, AH et AL sont utilisés comme des registres
d'un octet indépendants~; cependant, il est important de réaliser qu'ils ne
sont pas indépendants de AX. Changer la valeur de AX changera les valeurs
de AH et AL et {\em vice versa}\/. Les registres généraux sont utilisés dans
beaucoup de déplacements de données et instructions arithmétiques.

\begin{figure}
\begin{center}
\begin{tabular}{cc}
\multicolumn{2}{c}{AX} \\
\hline
\multicolumn{1}{||c|}{AH} & \multicolumn{1}{c||}{AL} \\
\hline
\end{tabular}
\caption{Le registre AX\label{fig:AX_reg} }
\end{center}
\end{figure}

Il y a deux registres d'index \index{registre!index} de 16 bits~: SI et DI. Ils sont souvent utilisés
comme des pointeurs, mais peuvent être utilisés pour la plupart des mêmes choses
que les registres généraux. Cependant, ils ne peuvent pas être décomposés
en registres de 8 bits.

Les registres 16 bits BP et SP sont utilisés pour pointer sur des données dans
la pile du langage machine et sont appelés le pointeur de base \index{registre!pointeur de base}
et le ponteur de pile\index{registre!pointeur de pile}, respectivement. Nous en reparlerons plus tard.

Les registres 16 bits CS, DS, SS et ES sont des \emph{registres de segment}. \index{registre!segment} Ils
indiquent quelle zone de la mémoire est utilisée pour les différentes parties
d'un programme. CS signifie Code Segment, DS Data Segment, SS Stack Segment (segment
de pile) et ES Extra Segment. ES est utilisé en tant que registre de segment
temporaire. Des détails sur ces registres se trouvent dans les Sections~\ref{real_mode}
et \ref{16prot_mode}.

Le registre de pointeur d'instruction (IP) \index{registre!IP} est utilisé avec le registre CS pour
mémoriser l'adresse de la prochaine instruction à exécuter par le processeur.
Normalement, lorsqu'une instruction est exécutée, IP est incrémenté pour pointer vers
la prochaine instruction en mémoire.

Le registre FLAGS \index{registre!FLAGS} stocke des informations importantes sur les résultats
d'une instruction précédente. Ces résultats sont stockés comme des bits
individuels dans le registre. Par exemple, le bit Z est positionné à 1 si
le résultat de l'instruction précédente était 0 ou à 0 sinon. Toutes les
instructions ne modifient pas les bits dans FLAGS, consultez le tableau
dans l'appendice pour voir comment chaque instruction affecte le registre
FLAGS.

\subsection{Registres 32 bits du 80386\index{registre!32 bits}}

Les processeurs 80386 et plus récents ont des registres étendus. Par exemple
le registre AX 16 bits est étendu à 32 bits. Pour la compatibilité ascendante,
AX fait toujours référence au registre 16 bits et on utilise EAX pour faire référence
au registre 32 bits. AX représente les 16 bits de poids faible de EAX tout comme AL
représente les 8 bits de poids faible de AX (et de EAX). Il n'y a pas de moyen
d'accéder aux 16 bits de poids fort de EAX directement. Les autres registres étendus sont
EBX, ECX, EDX, ESI et EDI.

La plupart des autres registres sont également étendus. BP devient EBP\index{registre!pointeur de base}~;
SP devient ESP\index{registre!pointeur de pile}~; FLAGS devient EFLAGS\index{registre!EFLAGS} et IP devient
EIP\index{registre!EIP}. Cependant, contrairement aux registres généraux et d'index, en mode protégé
32 bits (dont nous parlons plus loin) seules les versions étendues de ces registres sont utilisées.

Les registres de segment sont toujours sur 16 bits dans le 80386. Il y a également deux
registres de segment supplémentaires~: FS et GS\index{registre!segment}. Leurs noms n'ont pas de signification
particulière. Ce sont des segments temporaires supplémentaires (comme ES).

Une des définitions du terme \emph{mot} \index{mot}se réfère à la taille des registres de données
du processeur. Pour la famille du 80x86, le terme est désormais un peu confus.
Dans le Tableau ~\ref{tab:mem_units}, on voit que le terme \emph{mot} est défini
comme faisant 2 octets (ou 16 bits). Cette signification lui a été attribuée
lorsque le premier 8086 est apparu. Lorsque le 80386 a été développé, il a été
décidé de laisser la définition de \emph{mot} inchangée, même si la taille des
registres avait changé.
\index{registre|)}

\newpage
\subsection{Mode Réel\label{real_mode} \index{mode réel|(}}

En \MarginNote{Alors d'où vient l'infâme limite des 640Ko du DOS~? Le BIOS
requiert une partie des 1Mo pour son propre code et pour les périphériques
matériels comme l'écran.} mode réel, la mémoire est limitée à seulement un
méga octet ($2^{20}$ octet). 
Les adresses valides vont de 00000 à FFFFF (en hexa).\@ % \@ signifie 'fin de phrase'
Ces adresses nécessitent un nombre sur 20 bits. Cependant, un nombre de 20 bits ne tiendrait
dans aucun des registres 16 bits du 8086. Intel a résolu le problème, en utilisant
deux valeurs de 16 bits pour déterminer une adresse. La première valeur de 16 bits
est appelée le \emph{sélecteur}. Les valeurs du sélecteur doivent être stockées dans des
registres de segment. La seconde valeur de 16 bits est appelée le \emph{déplacement}
(offset). L'adresse physique identifiée par un couple \emph{sélecteur}:\emph{déplacement}
32 bits est calculée par la formule \[ 16 * {\rm s\acute{e}lecteur} + {\rm d\acute{e}placement} \]
Multiplier par 16 en hexa est facile, il suffit d'ajouter un 0 à la droite du nombre.
Par exemple, l'adresse physique référencée par 047C:0048 est obtenue de la façon suivante~:
\begin{center}
\begin{tabular}{r}
047C0 \\
+0048 \\
\hline
04808 \\
\end{tabular}
\end{center}
De fait, la valeur du sélecter est un numéro de paragraphe
(voir Tableau~\ref{tab:mem_units}).

Les adresses réelles segmentées ont des inconvénients~:
\begin{itemize}
\item Une seule valeur de sélecteur peut seulement référencer 64Ko de mémoire
(la limite supérieure d'un déplacement de 16 bits). Que se passe-t-il si
un programme a plus de 64Ko de code~? Une seule valeur de CS ne peut pas être utilisée 
pour toute l'exécution du programme. Le programme doit être divisé en
sections (appelées \emph{segments}\index{mémoire!segments}) de moins de 64Ko. Lorsque l'exécution passe d'un
segment à l'autre, la valeur de CS doit être changée. Des problèmes similaires
surviennent avec de grandes quantités de données et le registre DS. Cela
peut être très gênant~!

\item Chaque octet en mémoire n'a pas une adresse segmentée unique. L'adresse
physique 04808 peut être référencée par  047C:0048, 047D:0038, 047E:0028
ou 047B:0058.\@ Cela complique la comparaison d'adresses segmentées.

\end{itemize}
\index{mode réel|)}

\subsection{Mode Protégé 16 bits\label{16prot_mode}\index{mode protégé!16 bits|(}}

Dans le mode protégé 16 bits du 80286, les valeurs du sélecteur sont
interprétées de façon totalement différente par rapport au mode réel. En mode
réel, la valeur d'un sélecteur est un numéro de paragraphe en mémoire. En mode
protégé, un sélecteur est un \emph{indice} dans un \emph{tableau de descripteurs}.
Dans les deux modes, les programmes sont divisés en segments\index{mémoire!segments}. En mode réel, ces
segments sont à des positions fixes en mémoire et le sélecteur indique le numéro
de paragraphe auquel commence le segment. En mode protégé, les segments ne sont
pas à des positions fixes en mémoire physique. De fait, il n'ont même
pas besoin d'être en mémoire du tout~!

Le mode protégé utilise une technique appelée \emph{mémoire virtuelle}\index{mémoire!virtuelle}.
L'idée de base d'un système de mémoire virtuelle est de ne garder en mémoire
que les programmes et les données actuellement utilisés. Le reste des données
et du code sont stockés temporairement sur le disque jusqu'à ce qu'on aie à
nouveau besoin d'eux. Dans le mode protégé 16 bits, les segments sont déplacés
entre la mémoire et le disque selon les besoins. Lorsqu'un segment est rechargé
en mémoire depuis le disque, il est très probable qu'il sera à un endroit en
mémoire différent de celui où il était avant d'être placé sur le disque. Tout
ceci est effectué de façon transparente par le système d'exploitation. Le programme
n'a pas a être écrit différemment pour que la mémoire virtuelle fonctionne.

En mode protégé, chaque segment est assigné à une entrée dans un tableau
de descripteurs. Cette entrée contient toutes les informations dont le système
a besoin à propos du segment. Ces informations indiquent~: s'il est actuellement
en mémoire~; s'il est en mémoire, où il se trouve~; les droits d'accès
({\em p.e.\/}, lecture seule). L'indice de l'entrée du segment est la valeur
du sélecteur stockée dans les registres de segment.

Un \MarginNote{Un journaliste PC bien connu a baptisé le processeur 286
``cerveau mort'' (brain dead)} gros inconvénient du mode protégé 16 bits
est que les déplacements sont toujours des quantités sur 16 bits. En conséquence,
Les tailles de segment sont toujours limitées au plus à 64Ko. Cela rend
l'utilisation de grands tableaux problématique.
\index{mode protégé!16 bits|)}

\subsection{Mode Protégé 32 bits\index{mode protégé!32 bits|(}}

Le 80386 a introduit le mode protégé 32 bits. Il y a deux différences majeures
entre les modes protégés 32 bits du 386 et 16 bits du 286~:
\begin{enumerate}
\item

Les déplacements sont étendus à 32 bits. Cela permet à un déplacement
d'aller jusqu'à 4 milliards. Ainsi, les segments peuvent avoir des
tailles jusqu'à 4Go.

\item

Les segments\index{mémoire!segments} peuvent être divisés en unités plus petites de 4Ko appelées
\emph{pages}\index{mémoire!pages}. Le système de mémoire virtuelle\index{mémoire!virtuelle} fonctionne maintenant avec
des pages plutôt qu'avec des segments. Cela implique que seules certaines
parties d'un segment peuvent être présentes en mémoire à un instant donné.
En mode 16 bits du 286, soit le segment en entier est en mémoire, soit rien
n'y est. Ce qui n'aurait pas été pratique avec les segments plus grands que permet
le mode 32 bits.

\end{enumerate}

\index{mode protégé!32 bits|)}

Dans Windows 3.x, le \emph{mode standard} fait référence au mode protégé
16 bits du 286 et le \emph{mode amélioré} (enhanced) fait référence au mode
32 bits. Windows 9X, Windows NT/2000/XP, OS/2 et Linux fonctionnent tous en
mode protégé 32 bits paginé.

\subsection{Interruptions\index{interruptions}}

Quelques fois, le flot ordinaire d'un programme doit être interrompu pour traiter
des évènements qui requièrent une réponse rapide. Le matériel d'un ordinateur
offre un mécanisme appelé \emph{interruptions} pour gérer ces évènements. Par
exemple, lorsqu'une souris est déplacée, la souris interrompt le programme en
cours pour gérer le déplacement de la souris (pour déplacer le curseur {\em etc.\/}).
Les interruptions provoquent le passage du contrôle à un \emph{gestionnaire d'interruptions}.
Les gestionnaires d'interruptions sont des routines qui traitent une interruption.
Chaque type d'interruption est assignée à un nombre entier. Au début de la mémoire
physique, réside un tableau de \emph{vecteurs d'interuptions} qui contient
les adresses segmentées des gestionnaires d'interruption. Le numéro d'une interruption
est essentiellement un indice dans ce tableau.

Les interruptions externes proviennent de l'extérieur du processeur (la souris
est un exemple de ce type). Beaucoup de périphériques d'E/S soulèvent des
interruptions ({\em p.e.\/}, le clavier, le timer, les lecteurs de disque, le CD-ROM
et les cartes son). Les interruptions internes sont soulevées depuis le processeur,
à cause d'une erreur ou d'une instruction d'interruption. Les interruptions
erreur sont également appelées \emph{traps}. Les interruptions générées par l'instruction
d'interruption sont également appelées \emph{interruptions logicielles}. Le DOS
utilise ce type d'interruption pour implémenter son API(Application Programming Interface).
Les systèmes d'exploitation plus récents (comme Windows et Unix) utilisent une interface
basée sur C\footnote{Cependant, ils peuvent également utiliser une interface
de plus bas niveau au niveau du noyau}.

Beaucoup de gestionnaires d'interruptions redonnent le contrôle au programme interrompu
lorsqu'ils se terminent. Ils restaurent tous les registres aux valeurs qu'ils avaient
avant l'interruption. Ainsi, le programme interrompu s'exécute comme si rien
n'était arrivé (excepté qu'il perd quelques cycles processeur). Les traps
ne reviennent généralement jamais. Souvent, elles arrêtent le programme.

\section{Langage Assembleur}

\subsection{Langage Machine\index{langage machine}}

Chaque type de processeur comprend son propre langage machine. Les instructions
dans le langage machine sont des nombres stockés sous forme d'octets en mémoire.
Chaque instruction a son propre code numérique unique appelé \emph{code d'opération}\index{opcode}
ou \emph{opcode} (operation code) en raccourci. Les instructions des processeurs
80x86 varient en taille. L'opcode est toujours au début de l'instruction. Beaucoup
d'instructions comprennent également les données ({\em p.e.\/} des constantes
ou des adresses) utilisées par l'instruction.

Le langage machine est très difficile à programmer directement. Déchiffrer
la signification d'instructions codées numériquement est fatigant pour des
humains. Par exemple, l'instruction qui dit d'ajouter les registres EAX et EBX
et de stocker le résultat dans EAX est encodée par les codes hexadécimaux suivants~:
\begin{quote}
   03 C3
\end{quote}
C'est très peu clair. Heureusement, un programme appelé un \emph{assembleur} \index{assembleur}
peut faire ce travail laborieux à la place du programmeur.

\subsection{Langage d'Assembleur\index{langage assembleur|(}}

Un programme en langage d'assembleur est stocké sous forme de texte (comme un programme
dans un langage de plus haut niveau). Chaque instruction assembleur représente
exactement une instruction machine. Par exemple, l'instruction d'addition décrite
ci-dessus serait représentée en langage assembleur comme suit~:
\begin{CodeQuote}
   add eax, ebx
\end{CodeQuote}
Ici, la signification de l'instruction est \emph{beaucoup plus} claire qu'en code machine.
Le mot {\code add} est un \emph{mnémonique}\index{mnémonique} pour l'instruction d'addition.
La forme générale d'une instruction assembleur est~:
\begin{CodeQuote}
  {\em mnémonique opérande(s)}
\end{CodeQuote}

Un \emph{assembleur} \index{assembleur}est un programme qui lit un fichier texte avec des instructions
assembleur et convertit l'assembleur en code machine.
Les \emph{compilateurs} \index{compilateur}sont des programmes qui font des conversions similaires
pour les langages de programmation de haut niveau. Un assembleur est beaucoup plus simple
qu'un compilateur. \MarginNote{Cela a pris plusieurs années aux scientifiques de
l'informatique pour concevoir le simple fait d'écrire un compilateur~!} Chaque instruction
du langage d'assembleur représente directement une instruction machine. Les instructions
d'un langage de plus haut niveau sont \emph{beaucoup} plus complexes et peuvent
requérir beaucoup d'instructions machine.

Une autre différence importante entre l'assembleur et les langages de haut niveau est
que comme chaque type de processeur a son propre langage machine, il a également son
propre langage d'assemblage. Porter des programmes assembleur entre différentes
architectures d'ordinateur est \emph{beaucoup} plus difficile qu'avec un langage de
haut niveau.

Les exemples de ce livre utilisent le Netwide Assembler ou NASM \index{NASM}en raccourci.
Il est disponible gratuitement sur Internet (voyez la préface pour l'URL). Des
assembleurs plus courants sont l'Assembleur de Microsoft (MASM) ou l'Assembleur
de Borland (TASM). Il y a quelques différences de syntaxe entre MASM/TASM et NASM.

\subsection{Opérandes d'Instruction}

Les instructions en code machine ont un nombre et un type variables d'opérandes~; 
cependant, en général, chaque instruction a un nombre fixé d'opérandes (0 à 3).
Les opérandes peuvent avoir les types suivants~:
\begin{description}
\item[registre~:]
Ces opérandes font directement référence au contenu des registres du processeur.
\item[mémoire~:]
Ils font référence aux données en mémoire. L'adresse de la donnée peut être une constante
codée en dur dans l'instruction ou calculée en utilisant les valeurs des registres.
Les adresses sont toujours des déplacements relatifs au début d'un segment.
\item[immédiat~:]
\index{immédiat}
Ce sont des valeurs fixes qui sont listées dans l'instruction elle-même. Elles sont
stockées dans l'instruction (dans le segment de code), pas dans le segment de données.
\item[implicite~:]
Ces opérandes ne sont pas entrés explicitement. Par exemple, l'instruction
d'incrémentation ajoute un à un registre ou à la mémoire. Le un est implicite.
\end{description}
\index{langage assembleur|)}

\subsection{Instructions de base}

L'instruction la plus basique est l'instruction {\code MOV}\index{MOV}. Elle déplace les données
d'un endroit à un autre (comme l'opérateur d'assignement dans un langage de
haut niveau). Elle prend deux opérandes~:
\begin{CodeQuote}
  mov {\em dest, src}
\end{CodeQuote}
La donnée spécifiée par {\em src} est copiée vers {\em dest\/}. Une restriction est
que les deux opérandes ne peuvent pas être tous deux des opérandes mémoire. Cela nous
montre un autre caprice de l'assembleur.  Il y a souvent des règles quelque peu
arbitraires sur la façon dont les différentes instructions sont utilisées. Les opérandes
doivent également avoir la même taille. La valeur de AX ne peut pas être stockée dans BL.

Voici un exemple (les points-virgules marquent un commentaire\index{commentaire})~:
\begin{AsmCodeListing}[frame=none, numbers=none]
      mov    eax, 3  ; stocke 3 dans le registre EAX (3 est un operande immediat)
      mov    bx, ax  ; stocke la valeur de AX dans le registre BX
\end{AsmCodeListing}

L'instruction {\code ADD} \index{ADD}est utilisée pour additionner des entiers.
\begin{AsmCodeListing}[frame=none, numbers=none]
      add    eax, 4   ; eax = eax + 4
      add    al, ah   ; al = al + ah 
\end{AsmCodeListing}

L'instruction {\code SUB} \index{SUB}soustrait des entiers.
\begin{AsmCodeListing}[frame=none, numbers=none]
      sub    bx, 10   ; bx = bx - 10
      sub    ebx, edi ; ebx = ebx - edi
\end{AsmCodeListing}

Les instructions {\code INC} \index{INC} et {\code DEC} \index{DEC} incrémentent ou décrémentent les valeurs de 1.
Comme le un est un opérande implicite, le code machine pour {\code INC} et {\code DEC} est
plus petit que celui des instructions {\code ADD} et {\code SUB} équivalentes.
\begin{AsmCodeListing}[frame=none, numbers=none]
      inc    ecx      ; ecx++
      dec    dl       ; dl--
\end{AsmCodeListing}

\subsection{Directives\index{directive|(}}

Une \emph{directive} est destinée à l'assembleur, pas au processeur. Elles sont généralement
utilisées pour indiquer à l'assembleur de faire quelque chose ou pour l'informer de quelque chose.
Elles ne sont pas traduites en code machine. Les utilisations courantes des directives sont~:
\begin{list}{$\bullet$}{\setlength{\itemsep}{0pt}}
\item la définition de constantes
\item la définition de mémoire pour stocker des données
\item grouper la mémoire en segment
\item inclure des codes sources de façon conditionnelle
\item inclure d'autres fichiers
\end{list}

Le code NASM est analysé par un préprocesseur, exactement comme en C. Il y a beaucoup
de commandes identiques à celles du préprocesseur C. Cependant, les directives du préprocesseur
NASM commencent par un \% au lieu d'un \# comme en C.

\subsubsection{La directive equ\index{directive!equ}}

La directive {\code equ} peut être utilisée pour définir un \emph{symbole}. Les symboles
sont des constantes nommées qui peuvent être utilisées dans le programme assembleur.
Le format est le suivant~:
\begin{quote}
  \code {\em symbole} equ {\em valeur}
\end{quote}
Les valeurs des symboles ne peuvent \emph{pas} être redéfinies plus tard.

\subsubsection{La directive \%define\index{directive!\%define}}

Cette directive est semblable à la directive {\code \#define} du C. Elle est
le plus souvent utilisée pour définir des macros, exactement comme en C.
\begin{AsmCodeListing}[frame=none, numbers=none]
%define SIZE 100
      mov    eax, SIZE
\end{AsmCodeListing}
Le code ci-dessus définit une macro appelée {\code SIZE} et montre son utilisation
dans une instruction {\code MOV}. Les macros sont plus flexibles que les symboles
de deux façons. Elles peuvent être redéfinies et peuvent être plus que de simples
nombres constants.

\subsubsection{Directives de données\index{directive!données|(}}

\begin{table}[t]
\centering
\begin{tabular}{||c|c||} \hline
{\bf Unité} & {\bf Lettre} \\
\hline
octet & B \\
mot & W \\
double mot & D \\
quadruple mot & Q \\
dix octets & T \\
\hline
\end{tabular}
\caption{Lettres des Directives {\code RESX} et {\code DX}
         \label{tab:size-letters} }
\end{table}

Les directives de données sont utilisées dans les segments de données
pour réserver de la place en mémoire. Il y a deux façons de réserver de la
mémoire. La première ne fait qu'allouer la place pour les données~; la seconde
alloue la place et donne une valeur initiale. La première méthode utilise une
des directives {\code RES{\em X}}\index{directive!RES\emph{X}}. Le {\em X} est
remplacé par une lettre qui détermine la taille de l'objet (ou des objets) qui
sera stocké. Le Tableau~\ref{tab:size-letters} montre les valeurs possibles.

La seconde méthode (qui définit une valeur initiale) utilise une des directives
{\code D{\em X}}\index{directive!D\emph{X}}. Les lettres {\em X} sont les mêmes
que celles de la directive {\code RES{\em X}}.

Il est très courant de marquer les emplacements mémoire avec des \emph{labels}\index{label}
(étiquettes). Les labels permettent de faire référence facilement aux emplacements
mémoire dans le code. Voici quelques exemples~:
\begin{AsmCodeListing}[frame=none, numbers=none]
L1    db     0        ; octet libelle L1 avec une valeur initiale de 0
L2    dw     1000     ; mot labelle L2 avec une valeur initiale de 1000
L3    db     110101b  ; octet initialise a la valeur binaire 110101 (53 en decimal)
L4    db     12h      ; octet initialise a la valeur hexa 12 (18 en decimal)
L5    db     17o      ; octet initialise a la valeur octale 17 (15 en decimal)
L6    dd     1A92h    ; double mot initialise a la valeur hexa 1A92
L7    resb   1        ; 1 octet non initialise
L8    db     "A"      ; octet initialise avec le code ASCII du A (65)
\end{AsmCodeListing}

Les doubles et simples quotes sont traitées de la même façon. Les définitions
de données consécutives sont stockées séquentiellement en mémoire. C'est à dire
que le mot L2 est stocké immédiatement après L1 en mémoire. Des séquences de
mémoire peuvent également être définies.
\begin{AsmCodeListing}[frame=none, numbers=none]
L9    db     0, 1, 2, 3              ; definit 4 octets
L10   db     "w", "o", "r", 'd', 0   ; definit une chaine C = "word"
L11   db     'word', 0               ; idem L10
\end{AsmCodeListing}

La directive {\code DD}\index{directive!DD} peut être utilisée pour définir
à la fois des entiers et des constantes à virgule flottante en simple
précision\footnote{Les nombres à virgule flottante en simple précision sont
équivalent aux variables {\code float} en C.}.
Cependant, la directive {\code DQ}\index{directive!DQ} ne peut être
utilisée que pour définir des constantes à virgule flottante en double
précision.

Pour les grandes séquences, la directive de NASM {\code TIMES} \index{directive!TIMES}
est souvent utile. Cette directive répète son opérande un certain nombre
de fois. Par exemple~:
\begin{AsmCodeListing}[frame=none, numbers=none]
L12   times 100 db 0                 ; equivalent a 100 (db 0)
L13   resw   100                     ; reserve de la place pour 100 mots
\end{AsmCodeListing}
\index{directive!données|)}
\index{directive|)}

\index{label|(}
Souvenez vous que les labels peuvent être utilisés pour faire référence à des données
dans le code. Il y a deux façons d'utiliser les labels. Si un label simple
est utilisé, il fait référence à l'adresse (ou offset) de la donnée. Si le
label est placé entre crochets ({\code []}), il est interprété comme la donnée
à cette adresse. En d'autres termes, il faut considérer un label comme un
\emph{pointeur} vers la donnée et les crochets déréférencent le pointeur, exactement
comme l'astérisque en C (MASM/TASM utilisent une convention différente).
En mode 32 bits, les adresses sont sur 32 bits. Voici quelques exemples~:
\begin{AsmCodeListing}[frame=none]
      mov    al, [L1]      ; Copie l'octet situe en L1 dans AL
      mov    eax, L1       ; EAX = addresse de l'octet en L1
      mov    [L1], ah      ; copie AH dans l'octet en L1
      mov    eax, [L6]     ; copie le double mot en L6 dans EAX
      add    eax, [L6]     ; EAX = EAX + double mot en L6
      add    [L6], eax     ; double mot en L6 += EAX
      mov    al, [L6]      ; copie le premier octet du double mot en L6 dans AL
\end{AsmCodeListing}
La ligne 7 de cet exemple montre une propriété importante de NASM. L'assembleur
ne garde \emph{pas} de trace du type de données auquel se réfère le label.
C'est au programmeur de s'assurer qu'il (ou elle) utilise un label correctement.
Plus tard, il sera courant de stocker des adresses dans les registres et
utiliser les registres comme un pointeur en C. Là encore, aucune vérification n'est
faite pour savoir si le pointeur est utilisé correctement. A ce niveau,
l'assembleur est beaucoup plus sujet à erreur que le C.

Considérons l'instruction suivante~:
\begin{AsmCodeListing}[frame=none, numbers=none]
      mov    [L6], 1             ; stocke 1 en L6
\end{AsmCodeListing}
Cette instruction produit une erreur {\code operation size not specified}. Pourquoi~?
Parce que l'assembleur ne sait pas s'il doit considérer 1 comme un octet, un mot ou un
double mot. Pour réparer cela, il faut ajouter un spécificateur de taille~:
\begin{AsmCodeListing}[frame=none, numbers=none]
      mov    dword [L6], 1       ; stocke 1 en L6
\end{AsmCodeListing}
\index{DWORD}Cela indique à l'assembleur de stocker un 1 dans le double mot qui commence
en {\code L6}. Les autres spécificateurs de taille sont~: {\code BYTE}\index{BYTE},
{\code WORD}\index{WORD}, {\code QWORD}\index{QWORD} et {\code TWORD}\footnote{{\code TWORD} 
définit une zone mémoire de 10 octets. Le coprocesseur virgule flottante utilise ce type de 
données.}\index{TWORD}.
\index{label|)}

\subsection{Entrées et Sorties\index{E/S|(}}

Les entrées et sorties sont des activités très dépendantes du système.
Cela implique un interfaçage avec le matériel. Les langages de plus haut niveau,
comme C, fournissent des bibliothèques standards de routines pour une interface de programmation
simple, uniforme pour les E/S. Les langages d'assembleur n'ont pas de bibliothèque standard.
Ils doivent soit accéder directement au matériel (avec une opération privilégiée
en mode protégé) ou utiliser les routines de bas niveau éventuellement fournies par
le système d'exploitation.

\index{E/S!bibliothèque asm\_io|(} 
Il est très courant pour les routines assembleur d'être interfacées avec du C. Un
des avantages de cela est que le code assembleur peut utiliser les routines
d'E/S de la bibliothèque standard du C. Cependant, il faut connaître les règles
de passage des informations aux routines que le C utilise. Ces règles sont trop
compliquées pour en parler ici (nous en parlerons plus tard~!). Pour simplifier
les E/S, l'auteur a développé ses propres routines qui masquent les règles complexes
du C et fournissent une interface beaucoup plus simple. Le Tableau~\ref{tab:asmio}
décrit les routines fournies. Toutes les routines préservent les valeurs de 
tous les registres, excepté les routines read. Ces routines modifient la
valeur du registre EAX. Pour utiliser ces routines, il faut inclure un fichier
contenant les informations dont l'assembleur a besoin pour les utiliser. Pour inclure
un fichier dans NASM, utilisez la directive du préprocesseur {\code \%include}.
La ligne suivante inclut le fichier requis par les routines d'E/S de l'auteur
\footnote{Le fichier {\code asm\_io.inc} (et le fichier objet {\code asm\_io}
que recquiert {\code asm\_io.inc}) sont dans les exemples de code à télécharger sur
la page web de ce tutorial, {\code http://www.drpaulcarter.com/pcasm}}~:
\begin{AsmCodeListing}[frame=none, numbers=none]
%include "asm_io.inc"
\end{AsmCodeListing}

\begin{table}[t]
\centering
\begin{tabular}{lp{3.5in}}
{\bf print\_int} & affiche à l'écran la valeur d'un entier stocké dans EAX\\
{\bf print\_char} & affiche à l'écran le caractère dont le code ASCII est stocké dans AL\\
{\bf print\_string} & affiche à l'écran le contenu de la chaîne à l'{\em adresse} stockée
                      dans EAX. La chaîne doit être une chaîne de type C ({\em i.e.} terminée par 0).\\
{\bf print\_nl} & affiche à l'écran un caractère de nouvelle ligne. \\
{\bf read\_int} & lit un entier au clavier et le stocke dans le registre EAX. \\
{\bf read\_char} & lit un caractère au clavier et stocke son code ASCII dans le registre EAX.\\
\end{tabular}
\caption{Routine d'E/S de l'assembleur\label{tab:asmio} \index{E/S!bibliothèque asm\_io!print\_int}
\index{E/S!bibliothèque asm\_io!print\_char} \index{E/S!bibliothèque asm\_io!print\_string} 
\index{E/S!bibliothèque asm\_io!print\_nl} \index{E/S!bibliothèque asm\_io!read\_int}
\index{E/S!bibliothèque asm\_io!read\_char}}
\end{table}

Pour utiliser une de ces routines d'affichage, il faut charger EAX avec la valeur correcte
et utiliser une instruction {\code CALL} pour l'invoquer. L'instruction {\code CALL} est
équivalente à un appel de fonction dans un langage de haut niveau. Elle saute à une
autre portion de code mais revient à son origine une fois la routine terminée. Le programme
d'exemple ci-dessous montre plusieurs exemples d'appel à ces routines d'E/S.

\subsection{Débogage\index{débogage|(}}

La bibliothèque de l'auteur contient également quelques routines utiles pour
déboguer les programmes. Ces routines de débogage affichent des informations sur
l'état de l'ordinateur sans le modifier. Ces routines sont en fait des
\emph{macros} qui sauvegardent l'état courant du processeur puis font appel à une
sous-routine. Les macros sont définies dans le fichier {\code asm\_io.inc} dont
nous avons parlé plus haut. Les macros sont utilisées comme des instructions ordinaires.
Les opérandes des macros sont séparés par des virgules.

Il y a quatre routines de débogage nommées {\code dump\_regs}, {\code
dump\_mem}, {\code dump\_stack} et {\code dump\_math}~; elles affichent
respectivement les valeurs des registres, de la mémoire, de la pile et
du coprocesseur arithmétique.
\begin{description}

\item[dump\_regs]
\index{E/S!bibliothèque asm\_io!dump\_regs} 
Cette macro affiche les valeurs des registres (en hexadécimal) de l'ordinateur
sur {\code stdout} (\emph{i.e.} l'écran). Elle affiche également les bits
positionnés dans le registre FLAGS\footnote{Le Chapitre~2 parle de ce registre}.
Par exemple, si le drapeau zéro (zero flag) est à 1, \emph{ZF} est affiché.
S'il est à 0, il n'est pas affiché. Elle prend en argument un entier qui est
affiché également. Cela peut aider à distinguer la sortie de différentes
commandes {\code dump\_regs}.

\item[dump\_mem]
\index{E/S!bibliothèque asm\_io!dump\_mem} 
Cette macro affiche les valeurs d'une région de la mémoire (en hexadécimal et
également en caractères ASCII). Elle prend trois arguments séparés par des
virgules. Le premier est un entier utilisé pour étiqueter la sortie
(comme l'argument de {\code dump\_regs}). Le second argument est
l'adresse à afficher (cela peut être un label). Le dernier argument est le
nombre de paragraphes de 16 octets à afficher après l'adresse. La mémoire
affichée commencera au premier multiple de paragraphe avant l'adresse
demandée.

\item[dump\_stack]
\index{E/S!bibliothèque asm\_io!dump\_stack} 
Cette macro affiche les valeurs de la pile du processeur (la pile sera
présentée dans le Chapitre~4). La pile est organisée en double mots
et cette routine les affiche de cette façon. Elle prend trois arguments
délimités par des virgules. Le premier est un entier (comme pour {\code
dump\_regs}). Le second est le nombre de double mots à afficher
\emph{après} l'adresse que le registre {\code EBP} contient et le troisième
argument est le nombre de double mots à afficher \emph{avant} l'adresse
contenue dans {\code EBP}.

\item[dump\_math]
\index{E/S!bibliothèque asm\_io!dump\_math} 
Cette macro affiche les valeurs des registres du coprocesseur arithmétique.
Elle ne prend qu'un entier en argument qui est utilisé pour étiqueter la
sortie comme le fait l'argument de {\code dump\_regs}.
\end{description}
\index{débogage|)}
\index{E/S!bibliothèque asm\_io|)} 
\index{E/S|)}

\section{Créer un Programme}

Aujourd'hui, il est très peu courant de créer un programme autonome écrit complètement
en langage assembleur. L'assembleur est habituellement utilisé pour optimiser
certaines routines critiques. Pourquoi~? Il est \emph{beaucoup} plus simple de
programmer dans un langage de plus haut niveau qu'en assembleur. De plus, utiliser
l'assembleur rend le programme très dur à porter sur d'autres plateformes.
En fait, il est rare d'utiliser l'assembleur tout court.

Alors, pourquoi apprendre l'assembleur~?
\begin{enumerate}
\item Quelques fois, le code écrit en assembleur peut être plus rapide et plus compact
	que le code généré par un compilateur.
\item L'assembleur permet l'accès à des fonctionnalités matérielles du système directement
	qu'il pourrait être difficile ou impossible à utiliser depuis un langage de plus haut
	niveau.
\item Apprendre à programmer en assembleur aide à acquérir une compréhension plus profonde
	de la façon dont fonctionne un ordinateur.
\item Apprendre à programmer en assembleur aide à mieux comprendre comment les compilateurs et
	les langage de haut niveau comme C fonctionnent.
\end{enumerate}
Ces deux dernier points démontrent qu'apprendre l'assembleur peut être utile même si on ne
programme jamais dans ce langage plus tard. En fait, l'auteur programme rarement en assembleur,
mais il utilise les leçons qu'il en a tiré tous les jours.

\subsection{Premier programme}

\begin{figure}[t]
\begin{lstlisting}[frame=tlrb]{}
int main()
{
  int ret_status;
  ret_status = asm_main();
  return ret_status;
}
\end{lstlisting}
\caption{Code de {\code driver.c} \label{fig:driverProg} \index{lanceur C}}
\end{figure}

Les programmes qui suivront dans ce texte partiront tous du programme
de lancement en C de la Figure~\ref{fig:driverProg}. Il appelle simplement
une autre fonction nommée {\code asm\_main}. C'est la routine qui sera
écrite en assemble proprement dit. Il y a plusieurs avantages à utiliser
un programme de lancement en C. Tout d'abord, cela laisse le système du C
initialiser le programme de façon à fonctionner correctement en mode protégé.
Tous les segments et les registres correspondants seront initialisés par le C.
L'assembleur n'aura pas à se préoccuper de cela. Ensuite, la bibliothèque
du C pourra être utilisée par le code assembleur. Les routines d'E/S de l'auteur
en tirent partie. Elles utilisent les fonctions d'E/S du C ({\code printf}, {\em etc.}).
L'exemple suivant montre un programme assembleur simple.

\begin{AsmCodeListing}[label=first.asm]
; fichier: first.asm
; Premier programme assembleur. Ce programme attend la saisie de deux
; entiers et affiche leur somme.
;
; Pour creer l'executable en utilisant djgpp :
; nasm -f coff first.asm
; gcc -o first first.o driver.c asm_io.o

%include "asm_io.inc"
;
; Les donnees initialisees sont placees dans le segment .data
;
segment .data
;
; Ces labels referencent les chaines utilisees pour l'affichage
;
prompt1 db    "Entrez un nombre : ", 0      ; N'oubliez pas le 0 final
prompt2 db    "Entrez un autre nombre : ", 0
outmsg1 db    "Vous avez entre ", 0
outmsg2 db    " et ", 0
outmsg3 db    ", leur somme vaut ", 0

;
; Les donnees non initialisees sont placees dans le segment .bss
;
segment .bss
;
; Ces labels referencent les doubles mots utilises pour stocker les entrees
;
input1  resd 1
input2  resd 1

;
; Le code est place dans le segment .text
;
segment .text
        global  _asm_main
_asm_main:
        enter   0,0              ; initialisation
        pusha

        mov     eax, prompt1     ; affiche un message
        call    print_string

        call    read_int         ; lit un entier
        mov     [input1], eax    ; le stocke dans input1

        mov     eax, prompt2     ; affiche un message
        call    print_string

        call    read_int         ; lit un entier
        mov     [input2], eax    ; le stocke dans input2

        mov     eax, [input1]    ; eax = dword en input1
        add     eax, [input2]    ; eax += dword en input2
        mov     ebx, eax         ; ebx = eax

        dump_regs 1              ; affiche les valeurs des registres
        dump_mem  2, outmsg1, 1  ; affiche le contenu de la memoire
;
; Ce qui suit affiche le message resultat en plusieurs etapes
;
        mov     eax, outmsg1
        call    print_string     ; affiche le premier message
        mov     eax, [input1]     
        call    print_int        ; affiche input1
        mov     eax, outmsg2
        call    print_string     ; affiche le second message
        mov     eax, [input2]
        call    print_int        ; affiche input2
        mov     eax, outmsg3
        call    print_string     ; affiche le troisieme message
        mov     eax, ebx
        call    print_int        ; affiche la somme (ebx)
        call    print_nl         ; affiche une nouvelle ligne

        popa
        mov     eax, 0           ; retourne dans le programme C
        leave                     
        ret
\end{AsmCodeListing}

La ligne~13 du programme définit une section qui spécifie
la mémoire à stocker dans le segment de données (dont le nom est
{\code .data})\index{segment de données}. Seules les données initialisées doivent
être définies dans ce segment. Dans les lignes~17 à 21, plusieurs chaînes
sont déclarées. Elle seront affichées avec la bibliothèque C et doivent donc
se terminer par un caractère \emph{null} (code ASCII 0). Souvenez vous
qu'il y a une grande différence entre {\code 0} et {\code '0'}.

Les données non initialisées doivent être déclarées dans le segment bss 
(appelé {\code .bss} à la ligne 26)\index{segment bss}. Ce segment
tient son nom d'un vieil opérateur assembleur UNIX qui signifiait
``block started by symbol''. Il y a également un segment de pile.
Nous en parlerons plus tard.

Le segment de code \index{segment de code} est appelé {\code .text}
historiquement. C'est là que les instructions sont placées. Notez que
le label de code pour la routine main (ligne~38) a un préfixe de
soulignement. Cela fait partie de la \emph{convention d'appel C}\index{convention d'appel!C}.
Cette convention spécifie les règles que le C utilise lorsqu'il
compile le code. Il est très important de connaître cette convention
lorsque l'on interface du C et de l'assembleur. Plus loin, la convention
sera présentée dans son intégralité~; cependant, pour l'instant, il suffit
de savoir que tous les symboles C ({\em i.e.}, les fonctions et les variables
globales) on un préfixe de soulignement qui leur est ajouté par le compilateur
C (cette règle s'applique spécifiquement pour DOS/Windows, le compilateur
C Linux n'ajoute rien du tout aux noms des symboles).

La directive {\code global} {\index{directive!global} ligne 37
indique à l'assembleur de rendre le label {\code \_asm\_main} global.
Contrairement au C, les labels ont une \emph{portée interne} par
défaut. Cela signifie que seul le code du même module peut utiliser
le label. La directive {\code global} donne au(x) label(s) spécifié(s)
une \emph{portée externe}. Ce type de label peut être accédé par
n'importe quel module du programme. Le module {\code asm\_io}
déclare les labels {\code print\_int}, {\em et.al.\/} comme étant globaux.
C'est pourquoi l'on peut les utiliser dans le module {\code first.asm}.

\subsection{Dépendance vis à vis du compilareur}

Le code assembleur ci-dessus est spécifique au compilateur C/C++
GNU\footnote{GNU est un projet de Free Software Foundation  ({\code
http://www.fsf.org})} gratuit
DJGPP\index{compilateur!DJGPP}\footnote{\code http://www.delorie.com/djgpp}. Ce compilateur peut
être téléchargé gratuitement depuis Internet. Il nécessite un PC
à base de 386 ou plus et tourne sous DOS, Windows 95/98 ou NT. Ce compilateur
utilise des fichiers objets au format COFF (Common Object File Format).
Pour assembler le fichier au format COFF, utilisez l'option  {\code -f~coff}
avec {\code nasm} (comme l'indiquent les commentaires du code).
L'extension du fichier objet sera {\code o}.

Le compilateur C Linux est également un compilateur GNU. \index{compilateur!gcc} 
Pour convertir le code ci-dessus afin qu'il tourne sous Linux, retirez simplement
les préfixes de soulignement aux lignes ~37 et 38. Linux utilise le format
ELF (Executable and Linkable Format) pour les fichiers objet. Utilisez
l'option {\code -f~elf} pour Linux. Il produit également un fichier objet
avec l'extension {\code o}.\MarginNote{Les fichiers spécifiques au compilateur,
disponibles sur le site de l'auteur, ont déjà été modifiés pour fonctionner
avec le compilateur approprié.}

Borland C/C++ \index{compilateur!Borland} est un autre compilateur populaire. Il
utilise le format Microsoft OMF pour les fichier objets. Utilisez l'option
{\code -f~obj} pour les compilateurs Borland. L'extension du fichier objet
sera {\code obj}. Le format OMF utilise des directives {\code segment}
différentes de celles des autres formats objet. Le segment data
(ligne~13) doit être changé en~:
\begin{CodeQuote}
segment \_DATA public align=4 class=DATA use32
\end{CodeQuote}
Le segment bss (ligne 26) doit être changé en~:
\begin{CodeQuote}
segment \_BSS public align=4 class=BSS use32
\end{CodeQuote}
Le segment text (ligne 36) doit être changé en~:
\begin{CodeQuote}
segment \_TEXT public align=1 class=CODE use32
\end{CodeQuote}
De plus, une nouvelle ligne doit être ajoutée avant la ligne 36~:
\begin{CodeQuote}
group DGROUP \_BSS \_DATA
\end{CodeQuote}

Le compilateur Microsoft C/C++ \index{compilateur!Microsoft} peut utiliser
soit le format OMF, soit le format Win32 pour les fichiers objet
(Si on lui passe un format OMF, il convertit les informations au format
Win32 en interne). Le format Win32 permet de définir les segment comme
pour DJGPP et Linux. Utilisez l'option {\code -f~win32} pour produire
un fichier objet dans ce format. L'extension du fichier objet sera {\code obj}.

\subsection{Assembler le code}

La première étape consiste à assembler le code. Depuis la ligne de commande, tappez~:
\begin{CodeQuote}
nasm -f {\em format-objet} first.asm
\end{CodeQuote}
où {\em format-objet} est soit {\em coff\/}, soit {\em elf\/}, soit {\em obj} soit
{\em win32} selon le compilateur C utilisé (Souvenez vous que le fichier source
doit être modifié pour Linux et pour Borland).


\subsection{Compiler le code C}

Compilez le fichier {\code driver.c} en utilisant un compilateur C. Pour DJGPP, utilisez~:
\begin{CodeQuote}
gcc -c driver.c
\end{CodeQuote}
L'option {\code -c} signifie de compiler uniquement, sans essayer de lier.
Cette option fonctionne à la fois sur les compilateurs Linux, Borland et Microsoft.

\subsection{Lier les fichiers objets\label{seq:linking} \index{édition de liens|(}}

L'édition de liens est le procédé qui consiste à combiner le code machine
et les données des fichiers objet et des bibliothèques afin de créer
un fichier exécutable. Comme nous allons le voir, ce processus est
compliqué.

Le code C nécessite la bibliothèque standard du C et un \emph{code de démarrage}
spécial afin de s'exécuter. Il est \emph{beaucoup} plus simple de laisser le
compilateur C appeler l'éditeur de liens avec les paramètres corrects que
d'essayer d'appeler l'éditeur de liens directement. Par exemple, pour lier
le code du premier programme en utilisant DJGPP, \index{compilateur!DJGPP} utilisez~:
\begin{CodeQuote}
gcc -o first driver.o first.o asm\_io.o
\end{CodeQuote}
Cela crée un exécutable appelé {\code first.exe} (ou juste {\code first}
sous Linux). 

Avec Borland, \index{compilateur!Borland} on utiliserait~:
\begin{CodeQuote}
bcc32 first.obj driver.obj asm\_io.obj
\end{CodeQuote}
Borland utilise le nom du premier fichier afin de déterminer le nom de l'exécutable.
Donc, dans le cas ci-dessus, le programme s'appelera {\code first.exe}.

Il est possible de combiner les étapes de compilation et d'édition de liens.
Par exemple~:
\begin{CodeQuote}
gcc -o first {\em driver.c} first.o asm\_io.o
\end{CodeQuote}
Maintenant {\code gcc} compilera {\code driver.c} puis liera.
\index{édition de liens|)}

\subsection{Comprendre un listing assembleur\index{fichier listing|(}}

L'option {\code -l {\em fichier-listing}} peut être utilisée pour indiquer
à {\code nasm} de créer un fichier listing avec le nom donné. Ce fichier
montre comment le code a été assemblé. Voici comment les lignes~17 et 18
(dans le segment data) apparaîssent dans le fichier listing (les numéros
de ligne sont dans le fichier listing~; cependant, notez que les
numéro de ligne dans le fichier source peuvent ne pas être les mêmes).
\MarginNote{Les nombres
diffèrent sur la version française car ce ne sont pas les mêmes caractères.}
\begin{Verbatim}[xleftmargin=\AsmMargin]
48 00000000 456E7465722061206E-     prompt1 db    "Entrez un nombre : ", 0
49 00000009 756D6265723A2000
50 00000011 456E74657220616E6F-     prompt2 db    "Entrez un autre nombre : ", 0
51 0000001A 74686572206E756D62-
52 00000023 65723A2000
 \end{Verbatim}
La première colonne de chaque ligne est le numéro de la ligne et la seconde
est le déplacement (en hexa) de la donnée dans le segment. La troisième colonne
montre les données hexa brutes qui seront stockées. Dans ce cas, les données
hexa correspondent aux codes ASCII. Enfin, le texte du fichier source est
affiché sur la droite. Les déplacements listés dans la seconde colonne ne sont
très probalement \emph{pas} les déplacements réels auxquels les données seront
placées dans le programme terminé. Chaque module peut définir ses propres labels
dans le segment de données (et les autres segments également). Lors de l'étape
d'édition de liens (voir Section~\ref{seq:linking}), toutes ces définitions
de labels de segment de données sont combinées pour ne former qu'un segment de
données. Les déplacements finaux sont alors calculés par l'éditeur
de liens.

Voici une petite portion (lignes~54 à 56 du fichier source) du segment
text dans le fichier listing~:
\begin{Verbatim}[xleftmargin=\AsmMargin]
94 0000002C A1[00000000]          mov     eax, [input1]
95 00000031 0305[04000000]        add     eax, [input2]
96 00000037 89C3                  mov     ebx, eax
\end{Verbatim}
La troisième colonne montre le code machine généré par l'assembleur.
Souvent le code complet pour une instruction ne peut pas encore
être calculé. Par exemple, ligne~94, le déplacement (ou l'adresse)
de {\code input1} n'est pas connu avant que le code ne soit lié.
L'assembleur peut calculer l'opcode pour l'instruction {\code mov}
(qui, d'après le listing, est A1), mais il écrit le déplacement entre
crochets car la valeur exacte ne peut pas encore être calculée. Dans
ce cas, un déplacement temporaire de 0 est utilisé car {\code input1}
est au début du segment bss déclaré dans ce fichier. Souvenez vous que
cela ne signifie \emph{pas} qu'il sera au début du segment bss final du
programme. Lorsque le code est lié, l'éditeur de liens insèrera le
déplacement correct à la place. D'autres instructions, comme ligne~96,
ne font référence à aucun label. Dans ce cas, l'assembleur peut calculer
le code machine complet.
\index{fichier listing|)}

\subsubsection{Réprésentations Big et Little Endian \index{endian|(}}

Si l'on regarde attentivement la ligne~95, quelque chose semble très bizarre
à propos du déplacement enrte crochets dans le code machine. Le label
{\code input2} est au déplacement 4 (comme défini dans ce fichier)~; cepedant,
le déplacement qui apparaît en mémoire n'est pas 00000004 mais 04000000. 
Pourquoi~? Des processeurs différents stockent les entiers multioctets dans
des ordres différents en mémoire. Il y a deux méthodes populaires pour
stocker les entiers~: \emph{big endian} et \emph{little endian}.
\MarginNote{Endian est prononcé comme \emph{indian}.} Le big endian est la méthode
qui semble la plus naturelle. L'octet le plus fort (\emph{i.e.} le plus significatif)
est stocké en premier, puis le second plus fort, \emph{etc.} Par exemple,
le dword 00000004 serait stocké sous la forme des quatre octets suivants~: 00~00~00~04. 
Les mainframes IBM, la plupart des processeurs RISC et les processeur Motorola utilisent
tous cette méthode big endian. Cependant, les processeurs de type Intel utilisent
la méthode little endian~! Ici, l'octet le moins significatif est stocké en premier.
Donc, 00000004 est stocké en mémoire sous la forme 04~00~00~00. Ce format est
codé en dur dans le processeur et ne peut pas être changé. Normalement, le programmeur n'a
pas besoin de s'inquiéter du format utilisé. Cependant, il y a des circonstances
où c'est important.
\begin{enumerate}
\item Lorsque des données binaires sont transférées entre différents ordinateurs (soit
	via des fichiers, soit via un réseau).
\item Lorsque des données binaires sont écrites en mémoire comme un entier multioctet puis
	relues comme des octets individuels ou \emph{vice versa}.
\end{enumerate}

Le caractère big ou little endian ne s'applique pas à l'ordre des éléments d'un tableau.
Le premier élément d'un tableau est toujours à l'adresse la plus petite. C'est également
valable pour les chaînes (qui sont juste des tableaux de caractères). Cependant, le caractère
big ou little endian s'applique toujours aux éléments individuels des tableaux.
\index{endian|)}

\begin{figure}[t]
\begin{AsmCodeListing}[label=skel.asm]
%include "asm_io.inc"
segment .data
;
; Les donnees initialisees sont placees dans ce segment de donnees
;

segment .bss
;
; Les donnees non initialisees sont placees dans le segment bss
;

segment .text
        global  _asm_main
_asm_main:
        enter   0,0               ; initialisation
        pusha

;
; Le code est place dans le segment text. Ne modifiez pas le code
; place avant ou apres ce commentaire.
;

        popa
        mov     eax, 0           ; retour au C
        leave                     
        ret
\end{AsmCodeListing}
\caption{Squelette de Programme\label{fig:skel}}
\end{figure}

\section{Fichier Squelette\index{fichier squelette}}

La Figure~\ref{fig:skel} montre un fichier squelette qui peut être
utilisé comme point de départ pour l'écriture de programmes assembleur.








