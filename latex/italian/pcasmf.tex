% -*-LaTex-*-
%front matter of pcasm book

\chapter{Prefazione}

\section*{Scopo}

Lo scopo di questo libro e' di dare al lettore una migliore comprensione
di come i computer in realta' lavorano ad un piu' basso livello rispetto
ai linguaggi di programmazione come il Pascal. Con l'approfondimento
di come funzionano i computer, il lettore puo' essere molto piu'
produttivo nello sviluppare software nei linguaggi di alto livello
come il C e C++. Imparare a programmare in linguaggio assembly e' un
eccellente modo per raggiungere questo scopo. Gli altri libri sul
linguaggio assembly per PC ancora insegnano come programmare il processore
8086 che i primi PC usavano nel 1981! Il processore 8086 supporta
solo la modalita' \emph{remota}. In questa modalita', ogni programma
puo' indirizzare ogni memoria o periferica di un computer. Questa
modalita' non e' adatta per un sistema operativo multitasking sicuro.
Questo libro invece discute su come programmare il processore 80386
e i successivi in modalita' \emph{protetta} (la modalita' che usano 
Windows e Linux). Questa modalita' supporta le caratteristiche che 
i moderni sistemi operativi si aspettano come la memoria virtuale e
la protezione della memoria.
Ci sono diverse ragione per utilizzare la modalita' protetta: 
\begin{enumerate}
\item E' piu' facile programmare in modalita' protetta piuttosto che 
			in modalita' reale dell'8086, che gli altri libri usano.
\item Tutti i moderni PC lavorano in modalita' protetta.
\item Esiste software libero disponibile che gira in questa modalita'.
\end{enumerate}
La mancanza di libri di testo per la programmazioni in linguaggio 
assembly in modalita' protetta e' la ragione principale che ha spinto
l'autore a scrivere questo libro.

Come accennato prima, questo ttesto fa uso di software Free/Open Source:
nella fattispecie, l'assembler NASM e il compilatore DJGPP per C/C++.
Entrambi sono disponibili per il download su Internet. Il testo inoltre
discute su come usare il codice assembly NASM nei sistemi operativi Linux
e, attraverso i compilatori per C/C++ di Borland e di Microsoft, sotto 
Windows. Gli esempi per tutte queste piattaforme possono essere trovati
sul mio sito:
{\code http://www.drpaulcarter.com/pcasm}.
Tu \emph{devi} scaricare il codice di sempio se vuoi assemblare ed
eseguire i molti esempi di questo libro.

E' importante sottolineare che questo testo non tenta di coprire ogni 
aspetto della programmazione in assembly. L'autore ha provato a coprire
gli aspetti piu' importante sui cui \emph{tutti} i programmatori dovrebbero
avere  familiarita'.

\section*{Ringraziamenti}

L'autore vuole ringraziare tutti i programmatori in giro per il mondo
che hanno contribuito al movimento Free/Open Source. Tutti i programmi
ed anche questo stesso libro sono stati prodotti utlizzando software
libero. L'autore vuole ringraziare in modo speciale John~S.~Fine,
Simon~Tatham, Julian~Hall e gli altri per lo sviluppo dell'assmbler
NASM su cui sono basati tutti gli esempi di questo libro; DJ Delorie
per lo sviluppo del compiliatore usato DJGPP per C/C++; le numerore
persone che hanno contribuito al compilatore GNU gcc su cui e' 
basato DJGPP.Donald Knuth e gli altri per lo sviluppo dei linguaggi
di composizione \TeX\ e \LaTeXe\ che sono stati utilizzati per 
produrre questo libro; Richard Stallman (fondatore della Free Software
Foundation), Linus Torvalds (creatore del Kernel Linux) e gli altri 
che hanno prodotto il sofware di base che l'autore ha usato per 
produrre questo lavoro.

Grazie alle seguenti persone per le correzioni:
\begin{itemize}
\item John S. Fine
\item Marcelo Henrique Pinto de Almeida
\item Sam Hopkins
\item Nick D'Imperio
\item Jeremiah Lawrence
\item Ed Beroset
\item Jerry Gembarowski
\item Ziqiang Peng
\item Eno Compton
\item Josh I Cates
\item Mik Mifflin
\item Luke Wallis
\item Gaku Ueda
\item Brian Heward
\item Chad Gorshing
\item F. Gotti
\item Bob Wilkinson
\item Markus Koegel
\item Louis Taber
\item Dave Kiddell
\item Eduardo Horowitz
\item S\'{e}bastien Le Ray
\item Nehal Mistry
\item Jianyue Wang
\item Jeremias Kleer
\item Marc Janicki
\end{itemize}


\section*{Risorse su Internet}
\begin{center}
\begin{tabular}{|ll|}
\hline
Pagina dell'Autore & {\code http://www.drpaulcarter.com/} \\
%NASM   & {\code http://nasm.2y.net/} \\
Pagine NASM SourceForge & {\code http://sourceforge.net/projects/nasm/} \\
DJGPP  & {\code http://www.delorie.com/djgpp} \\
Linux Assembly & {\code http://www.linuxassembly.org/} \\
The Art of Assembly & {\code http://webster.cs.ucr.edu/} \\
USENET & {\code comp.lang.asm.x86 } \\
Documentazione Intel & {\code http://developer.intel.com/design/Pentium4/documentation.htm} \\
\hline
\end{tabular}
\end{center}


\section*{Feedback}

L'autore accetta volentieri ogni ritorno su questo lavoro.
\begin{center}
\begin{tabular}{ll}
\textbf{E-mail:} & {\code pacman128@gmail.com} \\
\textbf{WWW:}    & {\code http://www.drpaulcarter.com/pcasm} \\
\end{tabular}
\end{center}



