% To create PDF version, type
%   pdflatex pcasm.tex
% This will produce errors the first time, type R at the error prompt
% Then rerun again (twice to get all the references.

\documentclass[11pt]{book}
\typeout{-----------------------------------------}
\typeout{Enter files to be included. (*=all)}
\typeout{(pcasm1,pcasm2,...)}
\typeout{-----------------------------------------}
%\typein[\infiles]{ }
%\if*\infiles\else\includeonly{\infiles}\fi


\newif\ifmypdf
\ifx\pdfoutput\undefined
    \pdffalse          % we are not running PDFLaTeX
\else
%    \pdfoutput=1       % we are running PDFLaTeX
%    \pdftrue
\fi
%----------------------------------------------------------------------
% The following is the construct that interests us in the end:
%\ifpdf
%   % Put PDF-specific stuff here
%\else
%   % Put LaTeX-specific stuff here
%\fi

\usepackage{indentfirst}  % indent first paragraph of sections
%\usepackage{graphicx}
\usepackage{listings}
\usepackage{epsfig}
\usepackage{longtable}
\usepackage{color}
\usepackage[italian]{babel}
\usepackage{makeidx}
\ifmypdf
\usepackage[pdftex,
            bookmarks=true,
            bookmarksnumbered=true,
            pdftitle={PC Assembly Language},
            pdfauthor={Paul A. Carter},
            pdfsubject={80x86 Assembly Language Programming},
	          pdfkeywords={80x86 assembly programming}]{hyperref}
\fi
\author{Paul~A.~Carter}
\title{Il linguaggio PC Assembly}
\date{20 Marzo 2005}
\usepackage{lecnote}
\makeindex


\hyphenation{num-bers SF OF CF DF fact dif-fer-ence inter-rupts op-er-ands}
\begin{document}
\maketitle
\newlength{\AsmMargin}
\setlength{\AsmMargin}{-1cm}
\DefineVerbatimEnvironment{AsmCodeListing}{Verbatim}
{numbers=left, frame=lines,xleftmargin=\AsmMargin, labelposition=all, commentchar=^ }

\newcommand{\MarginNote}[1]{\marginpar{\sloppy \em \small #1}}
\thispagestyle{empty}
\vspace*{\fill}
\noindent Copyright \copyright\  2001, 2002, 2003, 2004 by Paul Carter\\

\noindent \emph{Traduzione italiana a cura di Giacomo Bruschi}\\

\noindent Questo documento puo' essere riprodotto e distribuito interamente (incluse le note sull'autore, i copyrights e i permessi), ammesso che non siano apportate modifiche al documento stesso, senza il consenso dell'autore. Cio' include un'uso 'corretto' del documento nelle rassegne e nella pubblicita', oppure in lavori di traduzione.\\

\noindent Si noti che questa limitazione non e' intesa a proibire modifiche per il servizio di stampa o di copiatura del documento. \\

\noindent Gli insegnanti sono incoraggiati ad usare questo documento come risorsa per l'insegnamento; comunque l'autore apprezzerebbe essere informato in questi casi.\\

\noindent \emph{Nota alla traduzione italiana: Nel testo i termini tecnici sono
stati tradotti nel corrispondene italiano, ad eccezione di quei termini la cui
traduzione avrebbe determinato una perdita di chiarezza, o di quei termini che sono 
comunemente usati e conosciuti in lingua inglese.}

\index{sottoroutine|vedi{sottoprogramma}}
\index{REPNZ|vedi{REPNE}}
\index{REPZ|vedi{REPE}}
\index{C++!funzioni membro|vedi{metodi}}
\index{segmento testo|vedi{segmento codice}}

\vfill
\frontmatter
\include{toc}
% -*-LaTex-*-
%front matter of pcasm book

\chapter{Prefacio}

\section*{Prop�sito}

El prop�sito de este libro es dar al lector un mejor entendimiento de
c�mo trabajan realmente los computadores a un nivel m�s bajo que los
lenguajes de alto nivel como Pascal. Teniendo un conocimiento profundo
de c�mo trabajan los computadores, el lector puede ser m�s productivo
desarrollando software en lenguajes de alto nivel tales como C y C++.
Aprender a programar en lenguaje ensamblador es una manera excelente
de lograr este objetivo. Otros libros de lenguaje ensamblador a�n
ense�an a programar el procesador 8086 que us� el PC original en 1981.
El procesador 8086 s�lo soporta el modo \emph{real}. En este modo, 
cualquier programa puede acceder a cualquier direcci�n de memoria
o dispositivo en el computador. Este modo no es apropiado para un sistema
operativo multitarea seguro. Este libro, en su lugar discute c�mo
programar los procesadores 80386 y posteriores en modo \emph{protegido}
(el modo en que corren Windows y Linux). Este modo soporta las
caracter�sticas que los sistemas operativos modernos esperan, como
memoria virtual y protecci�n de memoria.
Hay varias razones para usar el modo protegido
\begin{enumerate}
\item Es m�s f�cil de programar en modo protegido que en el modo real del
8086 que usan los otros libros.
\item Todos los sistemas operativos de PC se ejecutan en modo protegido.
\item Hay disponible software libre que se ejecuta en este modos.
\end{enumerate}
La carencia de libros de texto para la programaci�n en ensamblador de PC
para modo protegido es la principal raz�n por la cual el autor escribi�
este libro.

C�mo lo dicho antes, este libro hace uso de Software Libre: es decir el
ensamblador NASM y el compilador de C/C++ DJGPP. Ambos se pueden
descargar de Internet. El texto tambi�n discute c�mo usar el c�digo del 
ensamblador NASM bajo el sistema operativo Linux y con los compiladores
de C/C++ de Borland y Microsoft bajo Windows. Todos los ejemplos de estas
plataformas se pueden encontrar en mi sitio web:
{\url{http://www.drpaulcarter.com/pcasm}}.
Debe descargar el c�digo de los ejemplos, si desea ensamblar y correr los
muchos ejemplos de este tutorial.

Tenga en cuenta que este libro no intenta cubrir cada aspecto de la
programaci�n en ensamblador. El autor ha intentado cubrir los t�picos m�s
importantes que \emph{todos} los programadores deber�an tener

\section*{Reconocimientos}

El autor quiere agradecer a los muchos programadores alrededor del mundo
que han contribuido al movimiento de Software Libre. Todos los programe y
a�n este libro en s� mismo fueron producidos usando software libre. 
El autor desear�a agradecerle especialmente a John~S.~Fine,
Simon~Tatham, Julian~Hall 
y otros por desarrollar el ensamblador NASM ya que todos los ejemplos de
este libro est�n basados en �l; a DJ Delorie por desarrollar el
compilador usado de C/C++ DJGPP; la numerosa gente que ha contribuido al
compilador GNU gcc en el cual est� basado DJGPP; a Donald Knuth y otros
por desarrollar los lenguajes de composici�n de textos \TeX\ y \LaTeXe\
que fueron usados para producir este libro; a Richar Stallman (fundador
de la Free Software Fundation), Linus Torvalds 
(creador del n�cleo de Linux) y a otros que han desarrollado el software
que el autor ha usado para producir este trabajo.

Gracias a las siguientes personas por correcciones:
\begin{itemize}
\item John S. Fine
\item Marcelo Henrique Pinto de Almeida
\item Sam Hopkins
\item Nick D'Imperio
\item Jeremiah Lawrence
\item Ed Beroset
\item Jerry Gembarowski
\item Ziqiang Peng
\item Eno Compton
\item Josh I Cates
\item Mik Mifflin
\item Luke Wallis
\item Gaku Ueda
\item Brian Heward
\item Chad Gorshing
\item F. Gotti
\item Bob Wilkinson
\item Markus Koegel
\item Louis Taber
\item Dave Kiddell
\item Eduardo Horowitz
\item S\'{e}bastien Le Ray
\item Nehal Mistry
\item Jianyue Wang
\item Jeremias Kleer
\item Marc Janicki
\end{itemize}


\section*{Recursos en Internet}
\begin{center}
\begin{tabular}{|ll|}
\hline
P�gina del autor & \url{http://www.drpaulcarter.com/} \\
%NASM   & {\code http://nasm.2y.net/} \\
P�gina de NASM en SourceForge & \url{http://nasm.sourceforge.net/} \\
DJGPP  & \url{http://www.delorie.com/djgpp} \\
Ensamblador con Linux & \url{http://www.linuxassembly.org/} \\
The Art of Assembly & \url{http://webster.cs.ucr.edu/} \\
USENET & {\code comp.lang.asm.x86} \\
Documentaci�n de Intel & \url{http://www.intel.com/design/Pentium4/documentation.htm} \\
\hline
\end{tabular}
\end{center}


\section*{Comentarios}

El autor agradece cualquier comentario sobre este trabajo.
\begin{center}
\begin{tabular}{ll}
\textbf{E-mail:} & {\code pacman128@gmail.com} \\
\textbf{WWW:}    & \url{http://www.drpaulcarter.com/pcasm} \\
\end{tabular}
\end{center}




\mainmatter
% -*-latex-*-
\chapter{���}
\section{����}

���������ڴ���������ɡ�������ڴ沢û����ʮ����(����Ϊ10)��������Щ���֡���Ϊ������Զ�����(����Ϊ2)��ʽ������������Ϣ�ܼ���ؼ�Ӳ�����������������ع�һ��ʮ�������ơ�

\subsection{ʮ����\index{ʮ����}}

����Ϊ10��������10������(0-9)��ɡ�һ������ÿһλ�л����������е�λ���������10�ij˷�ֵ�����磺
\begin{displaymath}
234 = 2 \times 10^2 + 3 \times 10^1 + 4 \times 10^0
\end{displaymath}

\subsection{������\index{������|(}}

����Ϊ2��������2������(0��1)��ɡ�һ������ÿһλ�л����������е�λ���������2�ij˷�ֵ��(һ����������λ����Ϊһ������λ��)���磺
\begin{eqnarray*}
11001_2 & = & 1 \times 2^4 + 1 \times 2^3 + 0 \times 2^2 + 0 \times 2^1
              + 1 \times 2^0 \\
 & = & 16 + 8 + 1 \\
 & = & 25
\end{eqnarray*}

��Щ��ʾ�˶��������ת����ʮ���ơ���~\ref{tab:dec-bin}չʾ�˿�ʼ�ļ���ʮ����������Զ����������
\begin{table}[t]
\begin{center}
\begin{tabular}{||c|c||cc||c|c||}
\hline
ʮ���� & ������ & & & ʮ���� & ������ \\
\hline
0       & 0000   & & & 8       & 1000 \\
\hline
1       & 0001   & & & 9       & 1001 \\
\hline
2       & 0010   & & & 10      & 1010 \\
\hline
3       & 0011   & & & 11      & 1011 \\
\hline
4       & 0100   & & & 12      & 1100 \\
\hline
5       & 0101   & & & 13      & 1101 \\
\hline
6       & 0110   & & & 14      & 1110 \\
\hline
7       & 0111   & & & 15      & 1111 \\
\hline
\end{tabular}
\caption{�ڶ�������ʮ����0��15�ı�ʾ\label{tab:dec-bin}}
\end{center}
\end{table}


\begin{figure}[h]
\begin{center}
\begin{tabular}{|rrrrrrrrp{.1cm}|p{.1cm}rrrrrrrr|}
\hline
& \multicolumn{7}{c}{��ǰ�޽�λ} & & & \multicolumn{7}{c}{��ǰ�н�λ} & \\
\hline
&  0 & &  0 & &  1 & &  1 & & &  0 & &  0 & &  1 & & 1  & \\
& +0 & & +1 & & +0 & & +1 & & & +0 & & +1 & & +0 & & +1 &  \\
\cline{2-2} \cline{4-4} \cline{6-6} \cline{8-8} \cline{11-11} \cline{13-13} \cline{15-15} \cline{17-17}
& 0  & & 1  & & 1  & & 0  & & & 1  & & 0  & & 0  & & 1 & \\
&    & &    & &    & & c  & & &    & & c  & & c  & & c & \\
\hline
\end{tabular}

\caption{�����Ƽӷ�(c����\emph{��λ})\label{fig:bin-add}}
%TODO: Change this so that it is clear that single bits are being added,
%not 4-bit numbers. Ideas: Table or do sums horizontally.
\index{������!�ӷ�}
\end{center}
\end{figure}

ͼ~\ref{fig:bin-add} ��ʾ�����Ķ���������({\em Ҳ����}:
λ)��ӡ�������һ�����ӣ�

\begin{tabular}{r}
 $11011_2$ \\
+$10001_2$ \\
\hline
$101100_2$ \\
\end{tabular}

������˿����������ʮ���Ƶij�����
\[ 1234 \div 10 = 123\; r\; 4 \]
�����Կ������������ȥ������������ұߵ�ʮ���������ҽ�������ʮ�����������ƶ���һλ������2Ҳ��ִ��ͬ���IJ�����������Ϊ�˵õ�һ�����Ķ�����λ�⡣����������������ij���\footnote{2����±���������������������Զ����Ʊ�ʾ��������ʮ����
}��
\[ 1101_2 \div 10_2 = 110_2\; r\; 1 \]
����������������һ��ʮ����ת�������ĵȼ۵Ķ����Ʊ�ʾ��ʽ����ͼ~\ref{fig:dec-convert}չʾ��һ�������ַ��������ҵ����ұߵ���λ�������λ����Ϊ\emph{��͵���Чλ}
(lsb)������ߵ���λ��Ϊ\emph{��ߵ���Чλ}
(msb)���ڴ�Ļ�����Ԫ��8λ��ɣ�����Ϊһ��\emph{�ֽ�}��\index{������|)}

\begin{figure}[t]
\centering
\fbox{\parbox{\textwidth}{
\begin{eqnarray*}
%\mathrm{ʮ����} & \mathrm{������} \\
\mathrm{Decimal} & \mathrm{Binary} \\
25 \div 2 = 12\;r\;1 & 11001 \div 10 = 1100\;r\;1 \\
12 \div 2 = 6\;r\;0  & 1100 \div 10 = 110\;r\;0 \\
6 \div 2 = 3\;r\;0   & 110 \div 10 = 11\;r\;0 \\
3 \div 2 = 1\;r\;1   & 11 \div 10 = 1\;r\;1 \\
1 \div 2 = 0\;r\;1   & 1 \div 10 = 0\;r\;1 \\
\end{eqnarray*}

\centering ��� $25_{10} = 11001_{2}$ }}
\caption{ʮ����ת��\label{fig:dec-convert}}
\end{figure}

\subsection{ʮ������\index{ʮ������|(}}

ʮ��������ʹ�õĻ���Ϊ16��ʮ������(���߼�̳�Ϊ\emph{hex})�������������������ټ���ʽ��ʮ��������Ҫ16�����롣��Ͳ�����һ�����⣬��Ϊû�з��ſ���������ʾ��9֮��Ķ�������֡�ͨ��Э������ĸ��������ʾ��Щ��������֡���16��ʮ������������0-9��Ȼ��A��
B�� C�� D�� E��
F����A�ȼ���ʮ���Ƶ�10��B��11���ȵȡ�һ��ʮ�����Ƶ�ÿһλ�л����������е�λ���������16�ij˷�ֵ�����磺
\begin{eqnarray*}
\rm
2BD_{16} & = & 2 \times 16^2 + 11 \times 16^1 + 13 \times 16^0 \\
         & = & 512 + 176 + 13 \\
         & = & 701 \\
\end{eqnarray*}
��ʮ����ת����ʮ�����ƣ�����ʹ�úͶ�����ת��ͬ���ķ��������˳���16�⡣������ͼ~\ref{fig:hex-conv}��

\begin{figure}[t]
\centering
\fbox{\parbox{\textwidth}{
\begin{eqnarray*}
589 \div 16 & = & 36\;r\;13 \\
36 \div 16 & = & 2\;r\;4 \\
2 \div 16 & = & 0\;r\;2 \\
\end{eqnarray*}

\centering
��� $589 = 24\mathrm{D}_{16}$
}}
\caption{\label{fig:hex-conv}}
\end{figure}

ʮ�����Ʒdz����õ�ԭ������Ϊʮ�����ƺͶ�����֮��ת����һ���dz��򵥵ķ��������������dz�����ҷdz�������ʮ�������ṩһ���Ƚ�����ķ�������ʾ����������

��һ��ʮ��������ת���ɶ���������ֻ��Ҫ�򵥵ؽ�ÿһλʮ��������ת����4λ�������������磺$\mathrm{24D}_{16}$ת����\mbox{$0010\;0100\;
1101_2$}��ע������Щ4λ������������ͷ��0�dz���Ҫ�����$\mathrm{24D}_{16}$�м����λ��4λ������������ͷ��0û��ʹ�õĻ�����ô������Ǵ��ġ��Ӷ�����ת����ʮ������ͬ���򵥡�ֻ�跴�������ղ��Ǹ���������������ÿ4λһ��ת����ʮ�����ơ��Ӷ������������Ҷ˿�ʼ������������ˡ��������ܱ�֤��������ʹ������ȷ��4λ��
\footnote{������������Ϊʲô����������ô�����������Ž�������Ӵ���߿�ʼת����}.
����:\newline

\begin{tabular}{cccccc}
$110$ & $0000$ & $0101$ & $1010$ & $0111$ & $1110_2$ \\
  $6$ & $0$    &   $5$  &   A  &  $7$   &    $\mathrm{E}_{16}$ \\
\end{tabular}\newline

һ����λ��������Ϊ\emph{���ֽ�}
\index{���ֽ�}�����ÿһλʮ�������൱��һ�����ֽڡ��������ֽ�Ϊһ���ֽڣ�����һ���ֽڿ�������λʮ������������ʾ��һ���ֽ�ֵ�ķ�Χ�Զ����Ʊ�ʾΪ0��11111111����ʮ�����Ʊ�ʾΪ0��FF����ʮ���Ʊ�ʾΪ0��255��
\index{ʮ������|)}

\section{������ṹ}

\subsection{�ڴ�\index{�ڴ�|(}}

�ڴ�Ļ�����Ԫ��һ���ֽڡ�\index{�ֽ�}
\MarginNote{�ڴ���ǧ�ֽ�(~$2^{10} = 1��024$�ֽ�)�����ֽ�(~$2^{20} =
1��048��576$ �ֽ�)��ʮ��λԪ��(~$2^{30} = 1��073��741��824$
�ֽ�)��������}һ̨��32���ڴ�ĵ��Դ��������3200���ֽڵ���Ϣ�����ڴ����ÿһ���ֽ�ͨ��һ��Ψһ����������ʶ��Ϊ���ĵ�ַ��ͼ~\ref{fig:memory}չʾ��һ����

\begin{figure}[ht]
\begin{center}
\begin{tabular}{rcccccccc}
Address & 0 & 1 & 2 & 3 & 4 & 5 & 6 & 7 \\
\cline{2-9}
Memory & \multicolumn{1}{|c}{2A}  & \multicolumn{1}{|c}{45}
       & \multicolumn{1}{|c}{B8} & \multicolumn{1}{|c}{20}
       & \multicolumn{1}{|c}{8F} & \multicolumn{1}{|c}{CD}
       & \multicolumn{1}{|c}{12} & \multicolumn{1}{|c|}{2E} \\
\cline{2-9}
\end{tabular}
\caption{ �ڴ��ַ \label{fig:memory} }
\end{center}
\end{figure}

\begin{table}
\begin{center}
\begin{tabular}{|l|l|}\hline
word(��) & 2���ֽ� \\ \hline
double word(˫��) & 4���ֽ� \\ \hline
quad word(����) & 8���ֽ� \\ \hline
paragraph(һ��) & 16���ֽ� \\ \hline
\end{tabular}
\caption{ �ڴ浥Ԫ \label{tab:mem_units} }
\end{center}
\end{table}

ͨ���ڴ涼�Ǵ�����ʹ�ö����ǵ����ֽڡ���PC���ṹ�У���������Щ�ڴ������~\ref{tab:mem_units}չʾ��һ����

���ڴ�������ݶ������ֵġ��ַ�ͨ������������ʾ�ַ���\emph{�ַ�����}�����档����һ�����ձ���ַ������Ϊ
\emph{ASCII}(������Ϣ������׼����)��һ���µģ�����ȫ�ģ��������ASCII�ı�����Unicode���������ֱ���������Ҫ��������ASCIIʹ��һ���ֽ�������һ���ַ�������Unicodeÿ���ַ�ʹ�������ֽ�(��һ��
\emph{��})�����磺ASCIIʹ��$41_{16}$
($65_{10}$)����ʾ�ַ���д\emph{A}��Unicodeʹ��$0041_{16}$����ʾ����ΪASCIIʹ��һ���ֽڣ����������ܱ�ʾ256�ֲ�ͬ���ַ�\footnote{��ʵ�ϣ�ASCII����ʹ�õ�7λ������ֻ��128�ֲ�ͬ��ֵ����ʹ�á�}��Unicode��ASCII��ֵ��չ��һ���֣�������ʾ������ַ�������ڱ�ʾȫ�������е����Էdz���Ҫ��\index{�ڴ�|)}

\subsection{CPU\index{CPU|(}}

���봦����(CPU)��ִ��ָ��������豸�� CPU
ִ�е�ָ��ͨ���dz��򵥡�ָ�����Ҫ������ʹ�õ����ݴ洢��һ��CPU��Ϊ\emph{�Ĵ���}������Ĵ���λ���С�\index{�Ĵ���}CPU���Աȷ����ڴ����ط��ʼĴ���������ݡ�Ȼ������CPU��ļĴ��������޵ģ����Գ���Ա����ע��ֻ��������ʹ�õ����ݵ��Ĵ����С�

����CPUִ�е�ָ������˸�CPU��\emph{��������}��\index{��������}��������ӵ�бȸ߼����Ը������Ľṹ����������ָ������δ�ӹ������֣��������Ѻõ��ı���ʽ��Ϊ�˸���Ч�����У�CPU�����ܺܿ�ؽ���һ��ָ���Ŀ�ġ��������Ծ���Ϊ�����Ŀ����Ƶģ������������Ǹ������������ơ�һ����������д�ij������ת����CPU�ı��ػ������ԣ������ڵ��������С�\emph{������}
\index{������}��һ�����ó�������д�ij����������ṹ�ĵ��ԵĻ������Եij���ͨ����ÿһ�����͵�CPU�������Լ�Ψһ�Ļ������ԡ�����ΪʲôΪ
Mac д�ij�������IBM����PC�����е�һ��ԭ��

����ͨ��ʹ��\emph{ʱ��}
\index{ʱ��}��ͬ��ָ���ִ��\MarginNote{\emph{GHz}����ʮ����ջ���ÿ��ʮ�ڴ�ѭ����1.5
GHz ��CPUÿ����15�ڵ�ʱ�����塣}��ʱ��������һ���̶���Ƶ��(��Ϊ
\emph{ʱ��Ƶ��})����������һ̨1.5 GHz �ĵ��ԣ�1.5 GHz
����ʱ��Ƶ��\footnote{ʵ���ϣ�ʱ������ʹ�������಻ͬ��CPU����С��������ͨ��ʹ����CPU��ͬ��ʱ��Ƶ�ʡ�}��ʱ�Ӳ�����¼�ֺ��롣���Բ�������ʼ����������Ӽ����ͨ��ʹ�������������ȷִ�����ǵIJ�����������������������������������ȷ�Ľ��ಥ�����֡�һ��ָ����Ҫ�����Ĵ���(��������Ǿ���˵��ִ��\emph{����})����CPU�IJ�����ģ�¡����ڵĴ���ȡ��������֮ǰ��ָ����������ء�


\subsection{CPU 80x86ϵ��\index{CPU!80x86}}

IBM�ͺŵ�PC��������һ������Intel
80x86����(�����Ŀ�¡)��CPU����������������CPU����һЩ�ձ��������������һ�ֻ����Ļ������ԡ�������Σ�����ij�Ա����ؼ�ǿ�����������
\begin{description}

\item[8088��8086:] ��ЩCPU�ӱ�̵Ĺ۵���������ȫ��ͬ�ġ���������������PC���ϵ�CPU�������ṩһЩ16λ�ļĴ�����AX��
BX��CX��DX��\\
SI��DI��BP��SP��CS��DS��SS�� ES��IP��
FLAGS�����ǽ���֧��1M�ֽڵ��ڴ棬����ֻ�ܹ�����\emph{ʵģʽ}�¡�������ģʽ�£�һ��������Է����κ��ڴ��ַ����������������ڴ棡���ʹ�ų����Ϻͱ�֤��ȫ��÷dz����ѣ����ң�������ڴ���Ҫ�ֳ�\emph{��}��ÿ�β��ܴ���64K��


\item[80286:] ����CPUʹ����ATϵ�е�PC���С�����8088/86�Ļ������������м�����һЩ�µ�ָ�Ȼ��������Ҫ���µ������� \emph{16λ����ģʽ}��������ģʽ�£������Է���16M�ֽڵ��ڴ��ͨ����ֹ��������������ڴ����������򡣿��ǣ�������Ȼ�Ƿֳɲ��ܴ���64K�ĶΡ�

\item[80386:]
����CPU�������ǿ��80286�����ܡ����ȣ�����չ������Ĵ���������32λ����(EAX��EBX��ECX��EDX��ESI��EDI��EBP��ESP��\\
EIP)���������������µ�16λ�Ĵ�����FS��GS������ͬ��������һ���µ�\emph{32λ����ģʽ}��������ģʽ�£������Է���4G�ֽڡ�����ͬ���ֳɶΣ���������ÿ�δ�Сͬ�����Ե�4G��

\item[80486/Pentium/Pentium Pro:] ��Щ80x86����ij�Ա�����˲�����µ�������������Ҫ�������ָ��ִ�е��ٶȡ�

\item[Pentium MMX:] ��Щ��������Pentium������������MMXָ�� (��ý����չ)����Щָ����������ͨ��ͼ����������ʡ�

\item[Pentium II:] ����ӵ�� MMX ָ���Pentium��������(Pentium III �����Ͼ���һ�������Pentium II��)
\end{description}
\index{CPU|)}

\subsection{8086 16λ �Ĵ���\index{�Ĵ���|(}}

�����8086CPU�ṩ4��16λͨ�üĴ�����AX��BX�� CX
��DX����Щ�Ĵ��������Էֽ������8λ�Ĵ��������磺AX�Ĵ������Էֽ��AH��AL�Ĵ�������ͼ~\ref{fig:AX_reg}չʾ��һ����AH�Ĵ�������AX����(���)8λ����AL����AX�ĵ�8λ��ͨ��AH��AL������������һ���ֽڵļĴ������ã����ǣ�������Dz��ܶ�����AX�Ƿdz���Ҫ�ġ��ı�AX��ֵ����ı�AH��AL��ֵ{\em
��֮��Ȼ}\/��ͨ�üĴ�������ʹ���������ƶ�������ָ���С�

\begin{figure}
\begin{center}
\begin{tabular}{cc}
\multicolumn{2}{c}{AX} \\
\hline
\multicolumn{1}{||c|}{AH} & \multicolumn{1}{c||}{AL} \\
\hline
\end{tabular}
\caption{AX�Ĵ��� \label{fig:AX_reg} }
\end{center}
\end{figure}

��������16λָ��Ĵ���\index{�Ĵ���!ָ��}�� SI �� DI
��ͨ�����Ƕ��ǵ���ָ����ʹ�ã����������������Ҳ������ͨ�üĴ���һ��ʹ�á����ǣ����Dz����Էֽ��8λ�Ĵ�����

16λBP�� SP
�Ĵ�������ָ��������Զ�ջ������ݣ������Գ�Ϊ��ַ�Ĵ���\index{�Ĵ���!��ַ}
�Ͷ�ջָ��Ĵ���\index{�Ĵ���!��ջָ��}����Щ�����Ժ����ۡ�

16λCS��DS��SS ��ES �Ĵ�����\emph{�μĴ���}��\index{�Ĵ���!��}
����ָ������ͬ������ʹ�õ��ڴ档CS��������Σ� DS �������ݶΣ� SS
������ջ�κ�ES�������ӶΡ�ES����һ����ʱ�μĴ�����ʹ�á���Щ�Ĵ�����ϸ��������С��~\ref{real_mode}
��\ref{16prot_mode}��

ָ��ָ��Ĵ���(IP) \index{�Ĵ���!ָ��ָ��}
��CS�Ĵ���һ��ʹ��������CPU��һ��ִ��ָ��ĵ�ַ��ͨ������һ��ָ��ִ��ʱ��IP��ǰָ���ڴ������һ��ָ�

FLAGS\index{�Ĵ���!FLAGS}�Ĵ���������ǰ��ָ��ִ�н������Ҫ��Ϣ����Щ����ڼĴ������Ե�����λ���档���磺���ǰ��ָ��ִ�н����0��ZλΪ1����֮Ϊ0������������ָ��޸�FLAGS���λ���鿴��¼��ı�������ָ�������Ӱ��FLAGS�Ĵ����ġ�

\subsection{80386 32λ �Ĵ���\index{�Ĵ���!32-bit}}

80386���Ժ�Ĵ�������չ�˼Ĵ��������磺16λAX�Ĵ�����չ����32λ��Ϊ�������ݣ�AX��Ȼ��ʾ16λ�Ĵ�����
EAX ������ʾ��չ��32λ�Ĵ�����AX�� EAX
�ĵ�16λ����AL��AX(EAX)�ĵ�8λһ��������û��ֱ�ӷ��� EAX
��16λ�ķ�������������չ�Ĵ����� EBX��ECX��EDX��\\ESI �� EDI ��

�����������͵ļĴ���ͬ��Ҳ��չ�ˡ�BP�����EBP\index{�Ĵ���!��ַ}��SP
�����ESP\index{�Ĵ���!��ջָ��}��FLAGS�����EFLAGSEFLAGS\index{�Ĵ���!EFLAGS}
��IP�����EIP\index{�Ĵ���!EIP}�����ǣ���ͬ��ָ��Ĵ�����ͨ�üĴ�������32λ����ģʽ��(���潫���۵�)ֻ����˼Ĵ�������չ��ʽ��ʹ�á�

��80386��μĴ�����Ȼ��16λ�ġ�����������µĶμĴ�����FS��GS\index{�Ĵ���!��}���������ֲ�������ʲô�������Ǹ��ӶμĴ���(��ESһ��)��

������\emph{��}\index{��}��һ������ΪCPU���ݼĴ����Ĵ�С������80x86���壬�������������һ������ˡ��ڱ�~\ref{tab:mem_units}����Կ���\emph{��}������������ֽڡ����ǵ�8086��һ�η���ʱ������������ġ���80386���������������������ɱ������\emph{��}���岻�ı䣬��ʹ�Ĵ����Ĵ�С�Ѿ��ı��ˡ�
\index{�Ĵ���|)}

\subsection{ʵģʽ \label{real_mode} \index{ʵģʽ|(}}

��ʵģʽ��\MarginNote{��ô���޳ܵ�DOS640K�����������BIOSΪ���Ĵ����Ӳ���豸����ʾ��Ҫ����1M�ڴ��е�һЩ�ڴ档}���ڴ汻����Ϊ����1M�ֽ�($2^{20}$
�ֽ�)����Ч�ĵ�ַ��00000 \\
�� FFFFF
(ʮ������)��\@  % \@ means end of sentence
��Щ��ַ��Ҫ��20λ��������ʾ����Ȼ��һ��20λ�������ʺ��κ�һ��8086��16λ�Ĵ�����Intelͨ����������16λ��ֵ������һ����ַ�ķ��������������⡣��ʼ��16λֵ��Ϊ\emph{�ε�ַ}(selector)���ε�ַ��ֵ����洢�ڶμĴ����С��ڶ���16λֵ��Ϊ\emph{ƫ�Ƶ�ַ}(offset)����32λ\emph{�ε�ַ��ƫ�Ƶ�ַ}��ʾ��������ַ����������Ĺ�ʽ���㣺
%\[ 16 * {\rm �ε�ַ} + {\rm ƫ�Ƶ�ַ} \]
\[ 16 * {\rm selector} + {\rm offset} \]
��ʮ�������г���16�Ƿdz����׵ģ�ֻ��Ҫ�������ұ߼�0�����磺��ʾΪ047C:0048��������ַͨ�������õ���
\begin{center}
\begin{tabular}{r}
047C0 \\
+0048 \\
\hline
04808 \\
\end{tabular}
\end{center}
ʵ���ϣ��ε�ַ��ֵ��һ�ڵ��׵�ַ(����~\ref{tab:mem_units})��

��ʵ�Ķε�ַ�����µ�ȱ�㣺
\begin{itemize}
\item һ���ε�ַֻ��ָ��64K�ڴ�(16λƫ�Ƶ�����)�����һ������ӵ�д���64K�Ĵ���������ô���أ���CS���һ����һ��ֵ����������������ִ�е���Ҫ���������ֳ�С��64K�Ķ�(\emph{segment}\index{�ڴ�!��})����ִ�д�һ���Ƶ���һ��ʱ��CS���ֵ����ı䡣ͬ�������ⷢ���ڴ��������ݺ�
DS �Ĵ���֮�䡣����ʹ���Ƿdz�������ģ�

\item ÿ���ֽ����ڴ��ﲢ��ֻ��Ψһ�Ķε�ַ��������ַ04808���Ա�ʾΪ��047C:0048�� 047D:0038�� 047E:0028
��047B:0058��\@ �⽫ʹ�ε�ַ�ıȽϱ�ø��ӡ�

\end{itemize}
\index{ʵģʽ|)}

\subsection{16λ����ģʽ \label{16prot_mode} \index{����ģʽ!16-bit|(}}

��80286��16λ����ģʽ�£��ε�ַ��ֵ��ʵģʽ��Ƚ��͵���ȫ��ͬ����ʵģʽ�£�һ���ε�ַ��ֵ�������ڴ����һ�ڵ��׵�ַ���ڱ���ģʽ�£�һ���ε�ַ��ֵ��һ��ָ��\emph{��������}��\emph{ָ��}������ģʽ�£������DZ��ֳɶ�\index{�ڴ�:��}����ʵģʽ�£���Щ���������ڴ�Ĺ̶�λ�ö��Ҷε�ַ��ֵ��ʾ�ο�ʼ�����ڽڵ��׵�ַ���ڱ���ģʽ�£���Щ�β����������ڴ�Ĺ̶��ĵ�ַ����ʵ�ϣ����Ǹ�����һ����Ҫ���ڴ��С�

����ģʽʹ����һ�ֽ���\emph{�����ڴ�}
\index{�ڴ�!����}�ļ����������ڴ�Ļ���˼���ǽ������������������ʹ�õĴ�������ݵ��ڴ��С��������ݺʹ�����ʱ������Ӳ����ֱ�������ٴ���Ҫʱ����һ�δ�Ӳ�����»ص��ڴ��У������п��ܷ��ڲ�ͬ�����ƶ���Ӳ��֮ǰʱ��λ�õ��ڴ��С�������Щ���ɲ���ϵͳ͸����ִ�С����򲢲���Ҫ��ΪҪ�������ڴ湤����ʹ�ò�ͬ����д������

�ڱ���ģʽ�£�ÿһ�ζ�������һ���������������Ŀ�������Ŀӵ��ϵͳ��֪���Ĺ�����ε�������Ϣ����Щ��Ϣ�����������Ƿ����ڴ��У�������ڴ��У����ģ�����Ȩ��({\em
���磺} ֻ��)���ε���Ŀ��ָ���Ǵ����ڶμĴ�����Ķε�ֵַ��

16λ����ģʽ��һ�����ȱ����ƫ�Ƶ�ַ��Ȼ��16λ��\MarginNote{һ���dz�������PCר�ҳ�286CPUΪ``���˵Ĵ���''}������ĺ�����ǶεĴ�С��Ȼ����Ϊ���64K����ᵼ��ʹ�ô������ʱ�������⡣
\index{����ģʽ!16-bit|)}

\subsection{32λ����ģʽ\index{����ģʽ!32-bit|(}}

80386������32λ����ģʽ��386 32λ����ģʽ��286
16λ����ģʽ֮������Ҫ�������ǣ�
\begin{enumerate}
\item

ƫ�Ƶ�ַ��չ����32λ���������ƫ�Ƶ�ַ��Χ����4G����ˣ��εĴ�СҲ����4G��

\item

��\index{�ڴ�!��}���Էֳɽ�С��4K��С�ĵ�Ԫ����Ϊ\emph{�ڴ�ҳ}\index{�ڴ�!ҳ}�������ڴ�\index{�ڴ�!����}ϵͳ������ҳ�ķ�ʽ�£������˶η�ʽ�������ζ��һ�����κ�һ��ʱ��ֻ�в��ֿ������ڴ��С���28616λ����ģʽ�£�Ҫô���������ڴ��У�Ҫô�������ڡ�������32λģʽ�������Ĵ�Ķε�����ºܲ�ʵ�á�

\end{enumerate}

\index{����ģʽ!32-bit|)}

��Windows 3.xϵͳ�У�\emph{��׼ģʽ}Ϊ286
16λ����ģʽ��\emph{��ǿģʽ}Ϊ32λ����ģʽ��Windows 9X��Windows
NT/2000/XP��OS/2��Linux�������ڷ�ҳ������32λ����ģʽ�¡�

\subsection{�ж�\index{�ж�}}

��ʱ����ͨ�ij�����������Ա�Ҫ����ٷ�Ӧ�Ĵ����¼��жϡ������ṩ��һ����Ϊ\emph{�ж�}�Ľṹ��������Щ�¼������磺��һ������ƶ��ˣ�Ӳ������ж����ڵij�������������ƶ�(�ƶ���꣬
{\em
�ȵ�\/})���жϵ��¿���Ȩת�Ƶ�һ��\emph{�жϴ�������}���жϴ��������Ǵ����жϵij���ÿ�����͵��ж϶�������һ���жϺš��������ڴ�Ŀ�ʼ��������һ�Ű����жϴ�������ε�ַ��\emph{�ж�����}
�����жϺ������ű����������ָ�롣

�ⲿ�ж���CPU���ⲿ����(��������һ���͵����ӡ�)����I/O�豸�����ж�({\em
���磺\/}���̣�ʱ�ӣ�Ӳ����������CD-ROM������)���ڲ��ж���CPU���ڲ�����Ҫô����һ����������Ҫô���ж�ָ�����𡣴����жϳ�Ϊ\emph{����}�����ж�ָ��������жϳ�Ϊ\emph{���ж�}��DOSʹ����Щ���͵��ж���ʵ������API(Ӧ�ó���ӿ�)�������ִ��IJ���ϵͳ(�磺Windows��UNIX)ʹ��һ������C�Ľӿڡ�
\footnote{Ȼ�����������ں˼����ܻ�ʹ��һ�����͵ȼ��Ľӿڡ�}

�����жϴ���������ִ�����ʱ��������Ȩ���ظ����жϵij������ǻָ��Ĵ����������ֵ���жϷ���֮ǰ��ֵ��ͬ����ˣ����жϵij������û���κ����鷢��һ������(������ʧȥ��һЩCPU����)������ͨ�������ء�ͨ��������ֹ����

\section{�������}

\subsection{��������\index{��������}}

ÿ�����͵�CPU�������������Լ��Ļ������ԡ������������ָ�������ֽ���ʽ���ڴ��д�������֡�ÿ��ָ������Ψһ���������Ϊ\emph{��������}������Ϊ\emph{������}
\index{������}��80x86��������ָ���С��ͬ��������ͨ������ָ��Ŀ�ʼ��������ָ�����ָ��ʹ�õ�����({\em
���磺\/}�������ַ)��

�������Ժ���ֱ�ӽ��б�̡�������Щ���ִ���ָ�����˼��������˵�dz��Ƶġ����磺ִ�н�
EAX �� EBX �Ĵ������Ȼ�󽫽���ͻص� EAX
��ָ����ʮ��������������£�
\begin{quote}
   03 C3
\end{quote}
����������⡣���˵��ǣ�һ������\emph{���ij���}
\index{�����}����Ϊ����Ա��������ƵĹ�����

\subsection{�������\index{�������|(}}

һ��������Գ������ı���ʽ����(����һ���߼����Գ���)��ÿ�����ָ�����ȷ�е�һ������ָ����磺���������ļӷ�ָ���ڻ�������н������ɣ�
\begin{CodeQuote}
   add eax, ebx
\end{CodeQuote}
����ָ�����˼���ڻ��������ʾ��\emph{��}���������{\code
add}�Ǽӷ�ָ���\emph{���Ƿ�}
\index{���Ƿ�}��һ�����ָ���ͨ����ʽΪ��
\begin{CodeQuote}
  {\em mnemonic(���Ƿ�) operand(s)(������)}
\end{CodeQuote}

\emph{������}
\index{�����}��һ�����������ָ����ı��ļ��ͽ��������ת���ɻ�������ij���\emph{������}
\index{������}��Ϊ�߼����������ͬ��ת���ij���һ���������һ��������Ҫ�򵥡�
\MarginNote{�������˵��Կ�ѧ�Ҽ����ʱ����������α�дһ����������}ÿ��������Ӵ���һ��Ψһ�Ļ���ָ��߼�����\emph{��}���Ӷ��ҿ���Ҫ�����Ļ���ָ�

���͸߼�����֮����һ������Ҫ����������Ϊÿ�ֲ�ͬ���͵�CPU�����Լ��Ļ������룬������ͬ�������Լ��Ļ�����ԡ��ڲ�ͬ�ĵ��Թ�������ֲ������Աȸ߼�����Ҫ����\emph{�ö�}��

�Ȿ��ʹ���� Netwide Assembler������Ϊ NASM
\index{NASM}������Internet��������ṩ��(Ҫ�õ�URL���뿴ǰ��)�����ձ�Ļ�������Microsoft
Assembler(MASM) \index{MASM}��Borland Assembler
(TASM)��\index{TASM}MASM/TASM�� NASM ֮����һЩ����﷨����

\subsection{ָ�������}

��������ָ��ӵ�и��������Ͳ�ͬ�IJ�������Ȼ����ͨ��ÿ��ָ���м����̶��IJ�����(0��3��)����������������������ͣ�
\begin{description}
\item[�Ĵ���:]
��Щ������ֱ��ָ��CPU�Ĵ���������ݡ�
\item[�ڴ�:]
��Щ������ָ���ڴ�������ݡ����ݵĵ�ַ������Ӳ���뵽ָ����ij��������ֱ��ʹ�üĴ�����ֵ����õ�������ε���ʼ��ַ��ƫ��ֵ��Ϊ�˵�ַ��
\item[������:]
\index{������}
��Щ��������ָ����г��Ĺ̶���ֵ�����Ǵ�����ָ���(�ڴ����)�����������ݶΡ�
\item[��ָ�IJ�����:]
��Щ������û����ȷ��ʾ�����磺���Ĵ������ڴ�����1�ļӷ�ָ�1�ǰ�ָ�ġ�
\end{description}
\index{�������|)}

\subsection{����ָ��}

�����ָ����{\code MOV}
\index{MOV}ָ��������ݴ�һ���ط��Ƶ���һ���ط�(��߼���������ĸ�ֵ����һ��)����Я��������������
\begin{CodeQuote}
  mov {\em dest (Ŀ�IJ�����), src(Դ������)}
\end{CodeQuote}
{\em src}ָ�������ݿ�������{\em
dest\/}��һ��ָ�����������������ͬʱ���ڴ�����������ָ����һ�����Źֵĵط���ͨ�������ڸ��ָ���ָ���ʹ�ö���ijЩǿ���ԵĹ涨��������������ͬ���Ĵ�С��AX���ֵ�Ͳ��ܴ��浽
BL ��ȥ��

�����һ������(�ֺű�ʾע��\index{ע��}�Ŀ�ʼ)��
\begin{AsmCodeListing}[frame=none, numbers=none]
      mov    eax, 3   ; ��3���� EAX �Ĵ���(3��һ��������)��
      mov    bx, ax   ; ��AX��ֵ���뵽BX�Ĵ�����
\end{AsmCodeListing}

{\code ADD} \index{ADD}ָ�����������������ݵ���ӡ�
\begin{AsmCodeListing}[frame=none, numbers=none]
      add    eax, 4   ; eax = eax + 4
      add    al, ah   ; al = al + ah
\end{AsmCodeListing}

{\code SUB} \index{SUB}ָ�����������������ݵ������
\begin{AsmCodeListing}[frame=none, numbers=none]
      sub    bx, 10   ; bx = bx - 10
      sub    ebx, edi ; ebx = ebx - edi
\end{AsmCodeListing}

{\code INC} \index{INC}��{\code DEC}
\index{DEC}ָ�ֵ��1���1����Ϊ1��һ����ָ�IJ�������{\code INC}
��{\code DEC}�Ļ�������ȵȼ۵�{\code ADD}��{\code SUB}ָ��Ҫ�١�
\begin{AsmCodeListing}[frame=none, numbers=none]
      inc    ecx      ; ecx++
      dec    dl       ; dl--
\end{AsmCodeListing}

\subsection{ָʾ��\index{ָʾ��|(}}


\emph{ָʾ��}���ɻ���������Ķ�������CPU����������ͨ������Ҫôָʾ��������ʲôҪô��ʾ������ʲô�����Dz�������ɻ������롣ָʾ���ձ��Ӧ���У�
\begin{list}{$\bullet$}{\setlength{\itemsep}{0pt}}
\item ���峣��
\item ���������������ݵ��ڴ�
\item ���ڴ���ϳɶ�
\item �������ذ���Դ����
\item ���������ļ�
\end{list}

NASM������Cһ��Ҫͨ��һ��Ԥ����������ӵ�������Cһ����Ԥ�������򡣵��ǣ�
NASM ��Ԥ������ָʾ����\%��ͷ��������Cһ����\#��ͷ��

\subsubsection{equ ָʾ��\index{ָʾ��!equ}}

{\code
equ}ָʾ��������������һ��\emph{����}�����ű�����Ϊ�����ڻ�������ʹ�õij�������ʽ�ǣ�
\begin{quote}
  \code {\em symbol} equ {\em value}
\end{quote}
���ŵ�ֵ�Ժ�\emph{��}�����ٶ��塣

\subsubsection{\%define ָʾ��\index{ָʾ��!\%define}}

���ָʾ����C�е�{\code
\#define}�dz����ơ���ͨ����������һ���곣��������C����һ����
\begin{AsmCodeListing}[frame=none, numbers=none]
%define SIZE 100
      mov    eax, SIZE
\end{AsmCodeListing}
����Ĵ��붨����һ����Ϊ {\code SIZE}�ĺ�ͨ��ʹ��һ��{\code
MOV}ָ�������������ȷ���Ҫ������Ա��ٴζ�����ҿ��Զ���ȼ򵥵ij�����ֵ�����ֵ��

\subsubsection{����ָʾ��\index{ָʾ��!data|(}}

\begin{table}[t]
\centering
\begin{tabular}{||c|c||} \hline
{\bf ��λ} & {\bf ��ĸ} \\
\hline
�ֽ� & B \\
�� & W \\
˫�� & D \\
���� & Q \\
ʮ���ֽ� & T \\
\hline
\end{tabular}
\caption{{\code RESX}��{\code DX}ָʾ������ĸ
         \label{tab:size-letters} }
\end{table}

����ָʾ��ʹ�������ݶ������������ڴ�ռ䡣�����ڴ������ַ�������һ�ַ�������Ϊ���ݶ���ռ䣻�ڶ��ַ����ڶ������ݿռ��ͬʱ����һ����ʼֵ����һ�ַ���ʹ��{\code
RES{\em X}}\index{ָʾ��!RES\emph{X}}ָʾ���е�һ����{\em
X}������ĸ�������ĸ����Ҫ����Ķ���Ĵ�С����������~\ref{tab:size-letters}�����˿��ܵ�ֵ��

�ڶ��ַ���(ͬʱ����һ����ʼֵ)ʹ��{\code D{\em
X}}ָʾ����\index{ָʾ��!D\emph{X}}��һ����{\em
X}��������ĸ�������ĸ��ֵ��{\code RES{\em X}}���ֵһ����

ʹ��\emph{����}\index{����}
������ڴ�λ���Ƿdz��ձ�ġ�����ʹ���ڴ�����ָ���ڴ�λ�ñ�����ס������Ǽ������ӣ�
\begin{AsmCodeListing}[frame=none, numbers=none]
L1    db     0        ;�ֽڱ���L1����ʼֵΪ0
L2    dw     1000     ;�ֱ���L2����ʼֵΪ1000
L3    db     110101b  ;�ֽڱ�����ʼ����110101(ʮ����Ϊ53)
L4    db     12h      ;�ֽڱ�����ʼ����ʮ������12(��ʮ������Ϊ18)
L5    db     17o      ;�ֽڱ�����ʼ���ɰ˽���17(��ʮ������Ϊ15)
L6    dd     1A92h    ;˫�ֱ�����ʼ����ʮ������1A92
L7    resb   1        ;1��δ��ʼ�����ֽ�
L8    db     "A"      ;�ֽڱ�����ʼ����ASCIIֵA(65)
\end{AsmCodeListing}

˫���ź͵����ű�ͬ�ȶԴ���������������ݴ������������ڴ��С�Ҳ����˵����L2�ʹ�����L1�ĺ��档�ڴ��˳�����ͬ�������塣
\begin{AsmCodeListing}[frame=none, numbers=none]
L9    db     0, 1, 2, 3              ; ����4���ֽ�
L10   db     "w", "o", "r", 'd', 0   ; ����һ������"word"��C�ַ���
L11   db     'word', 0               ; ��ͬ��L10
\end{AsmCodeListing}

ָʾ��{\code
DD}\index{ָʾ��!DD}���������������κ͵����ȵĸ���������\footnote{�����ȸ������ȼ���C��ĵ�{\code
float}����}�����ǣ�{\code
DQ}\index{ָʾ��!DQ}ָʾ������������������˫���ȵ���������

���ڴ�����У� NASM ��{\code TIMES}
\index{ָʾ��!TIMES}ָʾ�������dz����á����ָʾ��ÿ�ζ��ظ����IJ�������һ��ָ���Ĵ��������磺
\begin{AsmCodeListing}[frame=none, numbers=none]
L12   times 100 db 0                 ; �ȼ���100��ֵΪ0���ֽ�
L13   resw   100                     ; ����ռ�Ϊ100����
\end{AsmCodeListing}
\index{ָʾ��!data|)}
\index{ָʾ��|)}

\index{����|(}
��ס��������������ʾ�����е����ݡ�������ʹ�÷��������֡����һ��ƽ���ı�����ʹ���ˣ���������Ϊ���ݵĵ�ַ(��ƫ��)����������������ڷ�����({\code
[]})�У����ͱ�����Ϊ�������ַ�е����ݡ����仰˵�������ѱ�������һ��ָ�����ݵ�\emph{ָ��}���������������ָ�����*����C��һ����(MASM/TASMʹ�õ�������һ��������)��32λģʽ�£���ַ��32λ������м������ӣ�
\begin{AsmCodeListing}[frame=none]
      mov    al, [L1]      ; ����L1����ֽ����ݵ�AL
      mov    eax, L1       ; EAX = �ֽڱ���L1�����ĵ�ַ
      mov    [L1], ah      ; ��AH�������ֽڱ���L1
      mov    eax, [L6]     ; ����L6���˫�����ݵ� EAX
      add    eax, [L6]     ; EAX = EAX + L6���˫������
      add    [L6], eax     ; L6 = L6���˫������ + EAX
      mov    al, [L6]      ; ����L6������ݵĵ�һ���ֽڵ�AL
\end{AsmCodeListing}
���ӵĵ�7��չʾ�� NASM һ����Ҫ���ܡ�������\emph{��}
���ָ��ٱ������������͡����ɳ���Ա����������֤��(����)��ȷʹ����һ�������������һ�㽫���ݵĵ�ַ���浽�Ĵ����У�Ȼ������C��һ���ѼĴ�����һ��ָ�������ʹ�á�ͬ����û�м��ʹ��ָ������ȷʹ�á������ַ�ʽ���������C����и��׳���������

���������ָ�
\begin{AsmCodeListing}[frame=none, numbers=none]
      mov    [L6], 1             ; ����1��L6��
\end{AsmCodeListing}
����������һ��{\code operation size not
specified}(������Сû��ָ��)�Ĵ���Ϊʲô����Ϊ������֪���ǰ�1����һ���ֽڣ������֣�����˫�������档Ϊ�������������һ����Сָ����
\begin{AsmCodeListing}[frame=none, numbers=none]
      mov    dword [L6], 1       ; ����1��L6��
\end{AsmCodeListing}
\index{DWORD}������߻������1�����ڴ�{\code
L6}��ʼ��˫���С���һЩ��Сָ��Ϊ�� {\code BYTE}\index{�ֽ�}(�ֽ�)��
{\code WORD}\index{��}(��)�� {\code QWORD}\index{����}(����)��{\code
TWORD}(ʮ�ֽ�)��\footnote{{\code
TWORD}������ʮ���ֽڴ�С���ڴ档������Э������ʹ�����������͵�����}\index{ʮ�ֽ�}
\index{����|)}

\subsection{�������� \index{I/O|(}}

��������������ϵͳ�����Ļ�������ǣ�浽ϵͳӲ���Ľӿ����⡣�߼����ԣ���C���ṩ�˱�׼�ģ��򵥵ģ�ͳһ�ij���I/O�ӿڵij���⡣������Բ��ṩ��׼�⡣���DZ���Ҫôֱ�ӷ���Ӳ��(�ڱ���ģʽ��Ϊ��Ȩ������)��ʹ���κβ���ϵͳ�ṩ�ĵײ�ij���

\index{I/O!asm\_io library|(}
��������C����ʹ���Ƿdz��ձ�ġ���������һ���ŵ��ǻ��������ʹ�ñ�׼C
I/O����⡣���ǣ���������Cʹ�õij���֮�䴫����Ϣ�Ĺ�����Щ����������dz��鷳��(���ǽ����Ժ��ᵽ��)Ϊ�˼򵥻�I/O�������Ѿ��������������ڸ���C��������Լ��ij��򣬶����ṩ��һ�����򵥵Ľӿڡ���~\ref{tab:asmio}�������ṩ�ij���������Щ�����������мĴ�����ֵ�����˶��ij����⡣��Щ����ȷʵ�޸���
EAX
��ֵ��Ϊ��ʹ����Щ������������һ����������Ҫ�õ�����Ϣ���ļ���Ϊ����
NASM �а���һ���ļ��������ʹ��{\code
\%include}Ԥ����ָʾ�������漸�а����������ߵ�I/O������ļ�\footnote{
{\code asm\_io.inc} (��{\code asm\_io.inc}��Ҫ��{\code
asm\_io}Ŀ���ļ�)�����Ӵ����У����Դ����ָ�ϵ���ҳ�����ص���{\code
http://www.drpaulcarter.com/pcasm}}:
\begin{AsmCodeListing}[frame=none, numbers=none]
%include "asm_io.inc"
\end{AsmCodeListing}

\begin{table}[t]
\centering
\begin{tabular}{lp{3.5in}}
{\bf print\_int} & ����Ļ����ʾ�������� EAX �е�����ֵ\\
{\bf print\_char} & ����Ļ����ʾ����ASCII��ʽ������AL�е��ַ�\\
{\bf print\_string} & ����Ļ����ʾ������ EAX ���{\em
��ַ}ָ����ַ��������ݡ�����ַ���������C���͵��ַ�����({\em
Ҳ����:}��null�������ַ���)�� \\
{\bf print\_nl} & ����Ļ����ʾ���С� \\
{\bf read\_int} & �Ӽ����϶���һ��������Ȼ���������浽 EAX �Ĵ����� \\
{\bf read\_char} & �Ӽ��̶���һ�����ַ�Ȼ��������ASCII��ʽ���浽 EAX
�Ĵ����� \\
\end{tabular}
\caption{����I/O���� \label{tab:asmio} \index{I/O!asm\_io
library!print\_int} \index{I/O!asm\_io library!print\_char}
\index{I/O!asm\_io library!print\_string} \index{I/O!asm\_io
library!print\_nl} \index{I/O!asm\_io library!read\_int}
\index{I/O!asm\_io library!read\_char}}
\end{table}

Ϊ��ʹ��һ����ӡ��������������ȷ��ֵ�� EAX �У�Ȼ����{\code
CALL}ָ���������{\code
CALL}ָ��ȼ����ڸ߼�������ĺ���call������ת���������һ��ȥִ�У�Ȼ��ȳ���ִ����ɺ��ֻص�ԭʼ�ĵط�������ij�������չʾ�˵�����ЩI/O����ļ���������

\subsection{����\index{����|(}}

���ߵĿ�ͬ������һЩ���õĵ��Գ�����Щ���Գ�����ʾ����ϵͳ״̬����Ϣ�����ı����ǡ���Щ������һЩ����CPU�ĵ�ǰ״̬��ִ��һ���ӳ�����õ�\emph{��}����Щ�궨���������ᵽ��{\code
asm\_io.inc}�ļ��С����������ͨ��ָ��һ��ʹ�á���IJ������ɶ��Ÿ�����

������ĸ����Գ����Ϊ{\code dump\_regs}��{\code dump\_mem}��{\code
dump\_stack}��{\code
dump\_math}�����Ƿֱ���ʾ�Ĵ������ڴ棬��ջ������Э��������ֵ��
\begin{description}

\item[dump\_regs]
\index{I/O!asm\_io library!dump\_regs}
�������ʾϵͳ�ļĴ������ֵ(ʮ������)��{\code
stdout}(\emph{Ҳ���ǣ�} ��ʾ��)����ͬʱ��ʾ��FLAGS\footnote{��2��
����������Ĵ���}�Ĵ������λ�����磬������־λ��1��\emph{ZF}����ʾ�ġ������0�����Ͳ�����ʾ����Я��һ�����β������������ͬ������ʾ��������Ϳ�����������ͬ{\code
dump\_regs}����������

\item[dump\_mem]
\index{I/O!asm\_io library!dump\_mem}
�����ͬ����ASCII�ַ�����ʽ��ʾ�ڴ������ֵ(ʮ������)�������������ö��ŷֿ��IJ�������һ��������һ��������ʾ��������α���(����{\code
dump\_regs}����һ��)���ڶ���������Ҫ��ʾ���ڴ�ĵ�ַ��(�������Ǹ���š�)���һ���������ڴ˵�ַ����Ҫ��ʾ��16�ֽڵĽ������ڴ���ʾ����Ҫ��ĵ�ַ֮ǰ�ĵ�һ�ڵı߽翪ʼ��
\item[dump\_stack]
\index{I/O!asm\_io library!dump\_stack}
�������ʾCPU��ջ��ֵ��(�����ջ���ڵ�4�����ᵽ��)�����ջ��˫����ɣ������������Ҳ�����ָ�ʽ��ʾ���ǡ������������ö��Ÿ����IJ�������һ��������һ�����α�������{\code
dump\_regs}һ�����ڶ�����������{\code
EBP}�Ĵ�����ĵ�ַ\emph{����}��Ҫ��ʾ��˫�ֵ���Ŀ����������������{\code
EBP}�Ĵ�����ĵ�ַ\emph{����}��Ҫ��ʾ�ĵ���Ŀ��

\item[dump\_math]
\index{I/O!asm\_io library!dump\_math}
�������ʾ����Э�������Ĵ������ֵ����ֻ����һ�����β������������������ʾ����������{\code
dump\_regs}����һ����
\end{description}
\index{����|)}
\index{I/O!asm\_io library|)} \index{I/O|)}

\section{����һ������}

�ֽ���ȫ�û������д�Ķ����ij����Dz������ġ����һ������ijЩ������Ҫ�ij���Ϊʲô���ø߼���������̱��û��Ҫ��\emph{�ö�}��ͬ����ʹ�û�ཫʹ�ó�����ֲ����һ��ƽ̨�dz����ѡ���ʵ�ϣ���������ʹ�û�����

��ô��Ϊʲô�κ��˶���Ҫѧϰ�������أ�
\begin{enumerate}
\item ��ʱ���ñ��д�Ĵ����������������Ĵ���Ҫ�ٶ������еø��졣
\item �������ֱ�ӷ���ϵͳӲ����Ϣ��������ڸ߼������к��ѻ����������ʵ�֡�
\item ѧϰ����̽�����һ������̵�����ϵͳ������С�
\item ѧϰ����̽�����һ���˸��õ�����������͸߼�������C��ι�����
\end{enumerate}
���������ѧϰ����Ƿdz����õģ���ʹ���Ժ�������ﲻ�ڱ�����õ�������ʵ�ϣ����ߺ����û���̣�������ÿ�춼ʹ�����������뷨��

\subsection{��һ������}

\begin{figure}[t]
\begin{lstlisting}[frame=tlrb]{}
int main()
{
  int ret_status;
  ret_status = asm_main();
  return ret_status;
}
\end{lstlisting}
\caption{{\code driver.c}���� \label{fig:driverProg}
\index{C������}}
\end{figure}

����һ����ij��ڵij���ȫ����ͼ~\ref{fig:driverProg}��ļ�C��������ʼ�����򵥵ص�����һ����Ϊ{\code
asm\_main}�ĺ�����������������彫�û���д�ij���ʹ��C���������м����ŵ㡣���ȣ�����ʹCϵͳ��ȷ���ó����ڱ���ģʽ�����С����еĶκ�������صĶμĴ�������C��ʼ���������벻��ҪΪ������ġ���Σ�C��ͬ���ṩ��������ʹ�á����ߵ�I/O��������������ŵ㡣����ʹ����C��I/O����({\code
printf}�� {\em ��})��������ʾ��һ���򵥵Ļ�����

\begin{AsmCodeListing}[label=first.asm]
; �ļ�: first.asm
; ��һ����������������ܹ���Ҫ�������α�����Ϊ����Ȼ��������ǵĺ͡�
;
;
; ���� djgpp ����ִ���ļ���
; nasm -f coff first.asm
; gcc -o first first.o driver.c asm_io.o

%include "asm_io.inc"
;
; ��ʼ�����뵽���ݶ��������
;
segment .data
;
; ��Щ����ָ������������ַ���
;
prompt1 db    "Enter a number: ", 0       ; ��Ҫ���ǿս�����
prompt2 db    "Enter another number: ", 0
outmsg1 db    "You entered ", 0
outmsg2 db    " and ", 0
outmsg3 db    ", the sum of these is ", 0

;
; ��ʼ�����뵽.bss���������
;
segment .bss
;
; �������ָ���������������˫��
;
input1  resd 1
input2  resd 1

;
; ������뵽.text��
;
segment .text
        global  _asm_main
_asm_main:
        enter   0,0               ; ��ʼ����
        pusha

        mov     eax, prompt1      ; �����ʾ
        call    print_string

        call    read_int          ; �����α������浽input1
        mov     [input1], eax     ;

        mov     eax, prompt2      ; �����ʾ
        call    print_string

        call    read_int          ; �����α������浽input2
        mov     [input2], eax     ;

        mov     eax, [input1]     ; eax = ��input1���˫��
        add     eax, [input2]     ; eax = eax + ��input2���˫��
        mov     ebx, eax          ; ebx = eax

        dump_regs 1                ; ����Ĵ���ֵ
        dump_mem  2, outmsg1, 1    ; ����ڴ�
;
; ����ּ�����������Ϣ
;
        mov     eax, outmsg1
        call    print_string      ; �����һ����Ϣ
        mov     eax, [input1]
        call    print_int         ; ���input1
        mov     eax, outmsg2
        call    print_string      ; ����ڶ�����Ϣ
        mov     eax, [input2]
        call    print_int         ; ���input2
        mov     eax, outmsg3
        call    print_string      ; �����������Ϣ
        mov     eax, ebx
        call    print_int         ; �������(ebx)
        call    print_nl          ; ����

        popa
        mov     eax, 0            ; �ص�C��
        leave
        ret
\end{AsmCodeListing}

�������ĵ�13�ж�����ָ���������ݵ��ڴ�εIJ��ִ���(����Ϊ{\code
.data})\index{���ݶ�}��ֻ���dz�ʼ�������ݲ��趨����������С���17��20�������˼����ַ��������ǽ�ͨ��C����������Ա�����\emph{null}�ַ�(ASCIIֵΪ0)��������ס{\code
0}��{\code '0'}�кܴ������

����ʼ���������������� bss ��(��Ϊ{\code
.bss}����26��)\index{bss��}������ε��������������ڵĻ���UNIX������������˼��``�ɷ��ſ�ʼ�Ŀ顣''��ͬ������һ����ջ�Ρ��������Ժ����ۡ�

�����\index{�����}���ݹ���������Ϊ{\code
.text}�����Ƿ���ָ��ĵط���ע��������(38��)�Ĵ�������һ���»���ǰ׺�������\emph{��C�г�ΪԼ��}��һ���֡�\index{����Լ��!C}���Լ��ָ���˱������ʱCʹ�õĹ���C�ͻ�ཻ��ʹ��ʱ��֪�����Լ���Ƿdz���Ҫ�ġ��Ժ󽫻Ὣȫ��Լ�����֣����ǣ�������ֻ��Ҫ֪�����е�C���������C����({\em
Ҳ���ǣ�}
������ȫ�ֱ���)��һ�����ӵ��»���ǰ׺��(����涨��ΪDOS/Windowsָ���ģ���linux��C
����������ΪC�������ϼ��κζ�����)

��37�е�ȫ�ֱ���(global){\index{ָʾ��!ȫ��}}ָʾ�����߻�ඨ��{\code
\_asm\_main}Ϊȫ�ֱ�������C��ͬ���ǣ�������ȱʡ�����ֻ��ʹ����\emph{�ڲ���Χ}�С������ζ��ֻ����ͬһģ��Ĵ������ʹ�����������
{\code
global}ָʾ��ʹָ���ı�������ʹ����\emph{�ⲿ��Χ}�С��������͵ı������Ա������������ģ����ʡ�{\code
asm\_io}ģ��������ȫ�ֱ���{\code print\_int}��{\em
et.al.\/}�������Ϊʲô��{\code first.asm}ģ������ʹ�����ǵ�Ե�ʡ�

\subsection{����������}

����Ļ�����ָ��Ϊ����GNU\footnote{GNU��һ�����������Ϊ�����ļƻ�({\code
http://www.fsf.org})}�� DJGPP \index{������!DJGPP}
C/C++��������\footnote{\code http://www.delorie.com/djgpp}
������������Դ�Internet��������ء���Ҫ��һ������386����õ�PC��������DOS��
Windows 95/98 ��NT�����С����������ʹ�� COFF (Common Object File
Format����ͨĿ���ļ���ʽ)��ʽ��Ŀ���ļ���Ϊ�˷��������ʽ��{\code
nasm}����ʹ��{\code
-f~coff}ѡ��(�����������ע��չʾ��һ��)������Ŀ���ļ�����չ��Ϊ{\code
o}��

Linux C������ͬ����һ��GNU��������\index{������!gcc}
Ϊ��ת������Ĵ���ʹ������Linux�����У�ֻ��򵥽�37��38������»���ǰ׺�Ƴ���Linuxʹ��ELF(Executable
and Linkable
Format����ִ�кͿ����Ӹ�ʽ)��ʽ��Ŀ���ļ���Linux��ʹ��{\code
-f~elf}ѡ���ͬ������һ����չ��Ϊ{\code
o}��Ŀ���ļ���\MarginNote{ָ���������������ļ����Դ����ߵ���ַ�ϵõ������Ѿ��޸ij�����ǡ���ı������������ˡ�}

Borland C/C++ \index{������!Borland}����һ�����еı���������ʹ��΢��
OMF ��ʽ��Ŀ���ļ���Borland������ʹ��{\code
-f~obj}ѡ�Ŀ���ļ�����չ��������{\code obj}�� OMF
������Ŀ���ļ���ʽ���ʹ���˲�ͬ��{\code
��}ָʾ�������ݶΣ�13�У�����ijɣ�
\begin{CodeQuote}
segment \_DATA public align=4 class=DATA use32
\end{CodeQuote}
 bss �Σ�26������ijɣ�
\begin{CodeQuote}
segment \_BSS public align=4 class=BSS use32
\end{CodeQuote}
text �Σ�36������ijɣ�
\begin{CodeQuote}
segment \_TEXT public align=1 class=CODE use32
\end{CodeQuote}
������36��֮ǰ����һ���У�
\begin{CodeQuote}
group DGROUP \_BSS \_DATA
\end{CodeQuote}

΢��C/C++\index{������!Microsoft}����������ʹ�� OMF
��Win32��ʽ��Ŀ���ļ���(�����������OMF��ʽ���������ڲ�����Ϣת���Win32��ʽ��)Win32������DJGPP��Linuxһ��������Ρ������ģʽ��ʹ��{\code
-f~win32}ѡ���������Ŀ���ļ�����չ��������{\code obj}��

\subsection{������}

��һ���ǻ����롣�������У����룺
\begin{CodeQuote}
nasm -f {\em object-format} first.asm
\end{CodeQuote}
{\em object-format}Ҫô��{\em coff\/}�� {\em elf\/}�� {\em
obj}��Ҫô��{\em
win32}������ʹ�õ�C������������(��ס��Linux��Borland�£���Դ�ļ�ͬ������ı䡣)

\subsection{����C����}

ʹ��C����������{\code driver.c}�ļ������� DJGPP ��ʹ�ã�
\begin{CodeQuote}
gcc -c driver.c
\end{CodeQuote}
{\code
-c}ѡ����ζ�ű��룬��������ͼ�������ӡ�ͬ����ѡ����ʹ����Linux��
Borland��Microsoft�������ϡ�

\subsection{����Ŀ���ļ� \label{seq:linking} \index{����|(}}

������һ������Ŀ���ļ��Ϳ��ļ���Ļ�����������ݽ�ϵ�һ�����һ����ִ���ļ��Ĺ��̡��������潫չʾ�ģ���������Ƿdz����ӵġ�

C����Ҫ�����б�׼C��������\emph{��������}
\index{��������}����ֱ�ӵ������ӳ�����ȣ�C������\emph{��}���׵��ô�������ȷ�IJ��������ӳ������磺ʹ��
DJGPP �����ӵ�һ������Ĵ��룬\index{������!DJGPP}ʹ��:
\begin{CodeQuote}
gcc -o first driver.o first.o asm\_io.o
\end{CodeQuote}
��������һ��{\code first.exe}(����Linux��ֻ��{\code
first})��ִ���ļ���

����Borland��\index{������!Borland}����Ҫʹ�ã�
\begin{CodeQuote}
bcc32 first.obj driver.obj asm\_io.obj
\end{CodeQuote}
Borlandʹ���г��ĵ�һ���ļ�����ȷ����ִ���ļ������������������������򽫱�����Ϊ{\code
first.exe}��

��������������������������ǿ��ܵġ����磺
\begin{CodeQuote}
gcc -o first {\em driver.c} first.o asm\_io.o
\end{CodeQuote}
����{\code gcc}������{\code driver.c}Ȼ�����ӡ� \index{����|)}

\subsection{����һ������б��ļ� \index{�б��ļ�|(}}

{\code -l {\em listing-file}}ѡ�������������{\code
nasm}����һ��ָ�����ֵ��б��ļ�������ļ�����ʾ������α���ࡣ�����ʾ��17��18��(�����ݶ�)���б��ļ��������ʾ��(�к���ʾ���б��ļ��У�����ע����Դ�����ļ�����ʾ���кſ��ܲ�ͬ�����б��ļ�����ʾ���кš�)
\begin{Verbatim}[xleftmargin=\AsmMargin]
48 00000000 456E7465722061206E-     prompt1 db    "Enter a number: ", 0
49 00000009 756D6265723A2000
50 00000011 456E74657220616E6F-     prompt2 db    "Enter another number: ", 0
51 0000001A 74686572206E756D62-
52 00000023 65723A2000
 \end{Verbatim}
ÿһ�е�ͷһ�����кţ��ڶ����������ڶ����ƫ�Ƶ�ַ(ʮ��������ʾ)����������ʾ��Ҫ�����ʮ������ֵ����������£�ʮ���������ݷ���ASCII���롣���գ���ʾ������Դ�ļ������ġ����ڵڶ��е�ƫ�Ƶ�ַ�dz�����\emph{����}���ݴ������ɺ�ij����е���ʵƫ�Ƶ�ַ��ÿ��ģ����������ݶ�(��������)�������Լ��ı�������������һ��ʱ(С��~\ref{seq:linking})��������Щ���ݶεı����������γ�һ�����ݶΡ����յ�ƫ�������ӳ������õ���

�����һС����text�δ���(��Դ�ļ���54��56��)���б��ļ��������ʾ��
\begin{Verbatim}[xleftmargin=\AsmMargin]
94 0000002C A1[00000000]          mov     eax, [input1]
95 00000031 0305[04000000]        add     eax, [input2]
96 00000037 89C3                  mov     ebx, eax
\end{Verbatim}
��������ʾ���ɻ���������Ļ������롣ͨ��һ��ָ����������벻����ȫ������������磺��94�У�{\code
input1}��ƫ��(��ַ)Ҫֱ���������Ӻ����֪����������������{\code
mov}
ָ��(���б���ΪA1)�IJ����룬��������ƫ��д�ڷ������У���Ϊ׼ȷ��ֵ���������������������£�0��Ϊһ����ʱƫ�Ʊ�ʹ�ã���Ϊ{\code
input1}������ļ��У���������bss�εĿ�ʼ����ס��\emph{��}��ζ�������ڳ��������
bss
�εĿ�ʼ�����������Ӻ����ӳ�����λ���ϲ�����ȷ��ƫ�ơ�����ָ���96�У������漰�κα�������������������������Ļ������롣\index{�б��ļ�|)}

\subsubsection{Big��Little Endian ��ʾ�� \index{endianess|(}}
���������ϸ����95�У����ᷢ�ֻ��������еķ��������ƫ�Ƶ�ַ�dz���֡�{\code
input2}������ƫ�Ƶ�ַΪ4(���ļ������һ��)��������ʾ���ڴ���ƫ�Ʋ���00000004������04000000��Ϊʲô����ͬ�Ĵ��������ڴ����Բ�ͬ��˳�򴢴���ֽ����Σ�\emph{big
endian}��\emph{little endian}��\MarginNote{Endian�ķ�����
\emph{indian}һ����}Big
 endian��һ�ֿ���������Ȼ�ķ��������(\emph{Ҳ���ǣ�}
�����Чλ)���ֽ����ȱ����棬Ȼ����ǵڶ���ģ�\emph{��������}�����磺˫��00000004��������Ϊ�ĸ��ֽ�00~00~00~04��IBM����������
RISC ��������Motorola��������ʹ������big
 endian������Ȼ��������Intel�Ĵ�����ʹ��little
 endian���������ȱ���������С����Ч�ֽڡ�����00000004���ڴ��д���Ϊ
04~00~00~00�����ָ�ʽǿ������CPU���Ҳ����ܸ��ġ�ͨ������£�����Ա������Ҫ����ʹ�õ������ָ�ʽ�����ǣ������������£������Ƿdz���Ҫ�ġ�

\begin{enumerate}
\item �������������ڲ�ͬ�ĵ����ϴ���ʱ(���������ļ���������)��
\item ��������������Ϊһ�����ֽ�����д�뵽�ڴ���Ȼ�������������ֽڶ�����\emph{��֮��Ȼ}��
\end{enumerate}

Endian��ʽ����Ӧ�����������������ĵ�һ��Ԫ��ͨ������͵ĵ�ַ����Ӧ�����ַ�����(�ַ�����)��
Endian��ʽ��Ȼ��������ĵ���Ԫ���С� \index{endianess|)}

\begin{figure}[t]
\begin{AsmCodeListing}[label=skel.asm]
%include "asm_io.inc"
segment .data
;
; ��ʼ�����ݷ��뵽��������ݶ���
;

segment .bss
;
; δ��ʼ�������ݷ��뵽 bss ����
;

segment .text
        global  _asm_main
_asm_main:
        enter   0,0               ; ��ʼ����
        pusha

;
; �������text�Ρ���Ҫ�ı������ע��֮ǰ��֮��Ĵ��롣
;
;

        popa
        mov     eax, 0            ; ���ص�����
        leave
        ret
\end{AsmCodeListing}
\caption{�Ǽܳ���\label{fig:skel}}
\end{figure}

\section{�Ǽ��ļ� \index{�Ǽ��ļ�}}

ͼ~\ref{fig:skel}��ʾ��һ������������д������Ŀ�ʼ���ֵĹǼ��ļ���


\chapter{Grundlagen der Assemblersprache}

\section{Arbeiten mit Integern (Ganzzahlen)
  \index{Integer|(}}

\subsection{Die Darstellung von Integerwerten \index{Integer!Darstellung|(}}

\index{Integer!ohne Vorzeichen|(} Integer treten in zwei
Geschmacksrichtungen auf: mit und ohne Vorzeichen. Vorzeichenlose
Integer (die nicht-negativ sind) werden in einer nahe liegenden
bin\"{a}ren Weise repr\"{a}sentiert. Die Zahl 200 als eine ein-Byte
vorzeichenlose Ganzzahl w\"{u}rde als 11001000 (oder C8 in hex)
repr\"{a}sentiert werden. \index{Integer!ohne Vorzeichen|)}

\index{Integer!mit Vorzeichen|(} Vorzeichenbehaftete Integer (die
positiv oder negativ sein k\"{o}nnen) werden auf kompliziertere Weisen
dargestellt. Betrachten wir zum Beispiel $-56$. $+56$ w\"{u}rde als Byte
durch 00111000 dargestellt werden. Auf dem Papier k\"{o}nnte man $-56$
als $-111000$ repr\"{a}sentieren, aber wie w\"{u}rde das in einem Byte im
Computerspeicher repr\"{a}sentiert werden? Wie w\"{u}rde das Minuszeichen
gespeichert werden?

Es gibt drei allgemeine Techniken, die zur Darstellung von
vorzeichenbehafteten Integern im Computerspeicher benutzt wurden.
Alle diese Methoden benutzen das h\"{o}chstwertige Bit des Integers als
ein \emph{Vorzeichenbit}. \index{Integer!Vorzeichenbit}
\index{Vorzeichenbit} Dieses Bit ist 0, wenn die Zahl positiv ist
und 1, wenn negativ.

\subsubsection{Signed Magnitude
  \index{Integer!Darstellung!signed magnitude}}

Die erste Methode ist die einfachste und wird \emph{signed
magnitude} genannt. Sie stellt den Integer in zwei Teilen dar. Der
erste Teil ist das Vorzeichenbit und der zweite ist der Betrag des
Integers. So w\"{u}rde 56 als das Byte $\underline{0}0111000$ (das
Vorzeichenbit ist unterstrichen) dargestellt werden und $-56$ als
$\underline{1}0111000$. Der gr\"{o}{\ss}te Bytewert wird
$\underline{0}1111111$ oder $+127$ sein und der kleinste Bytewert
w\"{a}re $\underline{1}1111111$ oder $-127$. Um einen Wert zu negieren
wird das Vorzeichenbit umgekehrt. Diese Methode ist einfach, hat
aber ihre Nachteile. Zuerst gibt es zwei m\"{o}gliche Werte f\"{u}r Null,
$+0$ ($\underline{0}0000000$) und $-0$ ($\underline{1}0000000$). Da
Null weder positiv noch negativ ist, sollten sich beide dieser
Repr\"{a}sentationen gleich verhalten. Das kompliziert die Logik f\"{u}r die
Arithmetik der CPU\@. Zweitens ist die allgemeine Arithmetik
ebenfalls kompliziert. Wenn 10 zu $-56$ addiert wird, muss dies zu
10 subtrahiert von 56 umgedeutet werden. Wiederum kompliziert dies
die Logik der CPU\@.

\subsubsection{One's Complement (Einerkomplement)
  \index{Integer!Darstellung!Einerkomplement}
  \index{one's complement}
  \index{Einerkomplement}
  }

Die zweite Methode ist als Repr\"{a}sentation im \emph{Einerkomplement}
bekannt. Das Einerkomplement einer Zahl wird gefunden, indem jedes
Bit in der Zahl invertiert wird. (Eine andere Betrachtungsweise
besteht darin, den neuen Bitwert als $1 - \mathrm{alter Bitwert}$
anzusehen.) Das Einerkomplement von $\underline{0}0111000$ ($+56$)
zum Beispiel ist $\underline{1}1000111$. In Einerkomplement-Notation
ist das Berechnen des Einerkomplements gleichwertig zur Negation.
Deshalb ist $\underline{1}1000111$ die Repr\"{a}sentation von $-56$.
Beachte, dass das Vorzeichenbit automatisch durch die
Einerkomplementierung ge\"{a}ndert wurde und dass, wie man auch erwarten
w\"{u}rde, das Einerkomplement zwei Mal genommen, die urspr\"{u}ngliche Zahl
ergibt. Wie bei der ersten Methode gibt es zwei Repr\"{a}sentationen der
Null: $\underline{0}0000000$ ($+0$) und $\underline{1}1111111$
($-0$). Arithmetik mit Einerkomplement-Zahlen ist kompliziert.

Es gibt einen n\"{u}tzlichen Trick, um das Einerkomplement einer Zahl in
hexadezimal zu finden, ohne nach bin\"{a}r zu konvertieren. Der Trick
besteht darin, die Hexziffern von F (oder 15 in dezimal) abzuziehen.
Diese Methode nimmt an, dass die Anzahl Bits in der Zahl ein
Vielfaches von 4 ist. Hier ist ein Beispiel: $+56$ ist 38 in hex. Um
das Einerkomplement zu finden, zieht man jede Ziffer von F ab, um C7
in hex zu erhalten. Dies stimmt mit dem obigen Ergebnis \"{u}berein.

\subsubsection{Two's Complement (Zweierkomplement)
  \index{Integer!Darstellung!Zweierkomplement|(}
  \index{two's complement}
  \index{Zweierkomplement|(}
  }

Die ersten beiden beschriebenen Methoden wurden auf fr\"{u}hen Computern
benutzt. Moderne Computer benutzen eine dritte Methode, die
\emph{Zweierkomplement} genannt wird. Das Zweierkomplement einer
Zahl wird durch die folgenden zwei Schritte gefunden:
\begin{enumerate}
\parskip=-0.25em %reduce the spacing <<<<<<<<<<<<<<<<<<<<<<<<<<<<<<<<<<<<<<<<<<

\item Finde das Einerkomplement der Zahl

\item Addiere eins zum Ergebnis aus Schritt 1
\end{enumerate}
Hier ist ein Beispiel unter Verwendung von $\underline{0}0111000$
(56). Zuerst wird das Einerkomplement berechnet:
$\underline{1}1000111$. Dann wird eins addiert:
\[
\begin{array}{rr}
  & \underline{1}1000111 \\
 +&                    1 \\
 \hline
  & \underline{1}1001000
\end{array}
\]

In Zweierkomplement-Darstellung ist die Berechnung des
Zweierkomplements \"{a}quivalent zur Negation einer Zahl. So ist
$\underline{1}1001000$ die Repr\"{a}sentation von $-56$ im
Zweierkomplement. Zwei Negationen sollten wieder die urspr\"{u}ngliche
Zahl geben. \"{U}berraschenderweise erf\"{u}llt das Zweierkomplement diese
Forderung. Nimm das Zweierkomplement von $\underline{1}1001000$,
indem eins zum Einerkomplement addiert wird.
\[
\begin{array}{rr}
  & \underline{0}0110111 \\
 +&                    1 \\
 \hline
  & \underline{0}0111000
\end{array}
\]

Bei der Berechnung des Zweierkomplements kann die Addition der am
weitesten links stehenden Bits einen \"{U}bertrag produzieren. Dieser
\"{U}ber\-trag wird \emph{nicht} verwendet. Beachte, dass alle Daten im
Computer eine feste Gr\"{o}{\ss}e (in der Anzahl Bits) haben. Das Addieren
zweier Bytes liefert immer ein Byte als Ergebnis (genauso wie die
Addition zweier W\"{o}rter ein Wort liefert, usw.) Diese Eigenschaft ist
wichtig f\"{u}r die Zweierkomplement-Notation. Betrachte zum Beispiel
Null als eine ein-Byte Zweierkomplement-Zahl
($\underline{0}0000000$). Die Berechnung des Zweierkomplements
liefert die Summe:
\[
\begin{array}{rr}
  & \underline{1}1111111 \\
 +&                    1 \\
 \hline
 c& \underline{0}0000000
\end{array}
\]
wobei $c$ einen \"{U}bertrag repr\"{a}sentiert. (Sp\"{a}ter wird gezeigt werden,
wie dieser \"{U}bertrag entdeckt werden kann, er wird aber nicht im
Ergebnis gespeichert.) So gibt es in der Zweierkomplement-Notation
nur eine Null. Dies macht Arithmetik im Zweierkomplement einfacher
als die vorheriger Methoden.

Bei Benutzung der Notation im Zweierkomplement kann ein
vorzeichenbehaftetes Byte verwendet werden um die Zahlen $-128$ bis
$+127$ zu repr\"{a}sentieren. Tabelle~\ref{tab:twocomp} zeigt einige
ausgew\"{a}hlte Werte. Werden 16 Bits verwendet, k\"{o}nnen die
vorzeichenbehafteten Zahlen $-32\,768$ bis $+32\,767$ repr\"{a}sentiert
werden. $+32\,767$ wird dargestellt durch 7FFF, $-32\,768$ durch
8000, $-128$ als FF80 und $-1$ als FFFF\@. 32-bit
Zweierkomplement-Zahlen reichen von ungef\"{a}hr $-2$ Milliarden bis
$+2$ Milliarden.

\begin{table}[ht]
\centering
\begin{tabular}{||c|c||}
 \hline
 Zahl & Hex Repr\"{a}sentation \\
 \hline
    0 & 00 \\
    1 & 01 \\
  127 & 7F \\
 -128 & 80 \\
 -127 & 81 \\
   -2 & FE \\
   -1 & FF \\
 \hline
\end{tabular}
\caption{Darstellung im Zweierkomplement \label{tab:twocomp}}
\end{table}

Die CPU hat keine Vorstellung davon, was ein bestimmtes Byte (oder
Wort oder Doppelwort) repr\"{a}sentieren soll. Assembler hat nicht das
Konzept von Datentypen, die eine Hochsprache hat. Wie Daten
interpretiert werden, h\"{a}ngt davon ab, welche Befehle auf die Daten
angewendet werden. Ob der Hexwert FF dazu bestimmt ist, eine
vorzeichenbehaftete $-1$ oder eine vorzeichenlose $+255$ zu
repr\"{a}sentieren, h\"{a}ngt vom Programmierer ab. Die Sprache C definiert
vorzeichenbehaftete und vorzeichenlose Integertypen. Diese
erm\"{o}glicht dem C Compiler die richtigen Befehle zu bestimmen, um mit
den Daten umzugehen.

\index{Zweierkomplement|)}
\index{Integer!Darstellung!Zweierkomplement|)}
\index{Integer!mit Vorzeichen|)}

\subsection{Vorzeichenerweiterung \index{Integer!Vorzeichenerweiterung|(}}

In Assembler haben alle Daten eine festgelegte Gr\"{o}{\ss}e. Es ist nicht
un\-\"{u}b\-lich, die Gr\"{o}{\ss}e der Daten \"{a}ndern zu m\"{u}ssen, um sie mit
anderen Daten zu benutzen. Die Gr\"{o}{\ss}e zu verringern ist das
Einfachste.

\subsubsection{Einengung der Datengr\"{o}{\ss}e}

Um die Gr\"{o}{\ss}e der Daten zu verringern, entfernt man einfach die
h\"{o}herwertigen Bits der Daten. Hier ist ein triviales Beispiel:
\begin{AsmCodeListing}[numbers=left, frame=none]
      mov    ax, 0034h        ; ax = 52 (in 16 Bits gespeichert)
      mov    cl, al           ; cl = niedere 8 Bits von ax
\end{AsmCodeListing}

Wenn die Zahl nicht korrekt in der kleineren Gr\"{o}{\ss}e repr\"{a}sentiert
werden kann, schl\"{a}gt die Herabsetzung der Gr\"{o}{\ss}e nat\"{u}rlich fehl. Wenn
zum Beispiel {\code AX} 0134h (oder 308 in dezimal) w\"{a}re, w\"{u}rde der
obige Code {\code CL} immer noch auf 34h setzen. Diese Methode
funktioniert sowohl mit vorzeichenbehafteten als auch mit
vorzeichenlosen Zahlen. Betrachten wir vorzeichenbehaftete Zahlen.
Wenn {\code AX} FFFFh ($-1$ als Wort) w\"{a}re, dann w\"{u}rde {\code CL}
FFh ($-1$ als Byte) sein. Beachte jedoch, dass dies nicht korrekt
ist, wenn der Wert in {\code AX} vorzeichenlos w\"{a}re!

Die Regel f\"{u}r vorzeichenlose Zahlen ist, dass alle entfernten Bits 0
sein m\"{u}ssen, damit die Konversion korrekt ist. Die Regel f\"{u}r
vorzeichenbehaftete Zahlen ist, dass die entfernten Bits entweder
alle 1 oder alle 0 sein m\"{u}ssen. Zus\"{a}tzlich muss das erste nicht
entfernte Bit denselben Wert haben wie die entfernten Bits. Dieses
Bit wird zum neuen Vorzeichenbit des kleineren Wertes. Es ist
wichtig, dass es gleich dem originalen Vorzeichenbit ist!

\subsubsection{Ausweitung der Datengr\"{o}{\ss}e}

Heraufsetzen der Gr\"{o}{\ss}e der Daten ist komplizierter als herabsetzen.
Betrachten wir das Hexbyte FF\@. Wenn es zu einem Wort erweitert
wird, welchen Wert sollte dann das Wort haben? Es h\"{a}ngt davon ab,
wie FF interpretiert wird. Ist FF ein vorzeichenloses Byte (255 in
dezimal), dann sollte das Wort 00FF sein; wenn es jedoch ein
vorzeichenbehaftetes Byte ($-1$ in dezimal) ist, dann sollte das
Wort FFFF sein.

Um, ganz allgemein, eine vorzeichenlose Zahl zu erweitern, macht man
alle neuen Bits der erweiterten Zahl zu 0. So wird FF zu 00FF\@. Um
jedoch eine vorzeichenbehaftete Zahl zu erweitern, muss man das
Vorzeichenbit \emph{erweitern}. \index{Integer!Vorzeichenbit} Das
bedeutet, dass die neuen Bits Kopien des Vorzeichenbits werden. Da
das Vorzeichenbit von FF 1 ist, m\"{u}ssen die neuen Bits ebenso alle
Einsen sein, um dann FFFF zu liefern. Wenn die vorzeichenbehaftete
Zahl 5A (90 in dezimal) erweitert wird, w\"{u}rde das Ergebnis 005A
sein.

Es gibt mehrere Befehle, die die 80386 f\"{u}r die Zahlenerweiterung
bereitstellt. Erinnern wir uns, dass der Computer nicht wei{\ss}, ob
eine Zahl vorzeichenbehaftet oder vorzeichenlos ist. Es liegt am
Programmierer, den richtigen Befehl zu verwenden.

F\"{u}r vorzeichenlose Zahlen kann man mit einem {\code MOV} Befehl
einfach Nullen in die oberen Bits laden. Um zum Beispiel das Byte in
AL zu einem vorzeichenlosen Word in AX zu erweitern:
\begin{AsmCodeListing}[numbers=left, firstnumber=last, frame=none]
      mov    ah, 0            ; setze obere 8 Bits auf Null
\end{AsmCodeListing}
Jedoch ist es nicht m\"{o}glich, einen {\code MOV} Befehl zu verwenden,
um das vorzeichenlose Wort in AX zu einem vorzeichenlosen Doppelwort
in EAX zu konvertieren. Warum nicht? Es gibt keinen Weg, um mit
einem {\code MOV} die oberen 16 Bits von EAX zu spezifizieren. Die
80386 l\"{o}st dieses Problem, indem sie die neue Instruktion {\code
MOVZX} \index{Maschinenbefehl!MOVZX} bereitstellt. Dieser Befehl hat
zwei Operanden. Die Datensenke (erster Operand) muss ein 16 oder
32~bit Register sein. Die Quelle (zweiter Operand) kann ein 8 oder
16~bit Register oder ein Byte oder Wort im Speicher sein. Die andere
Einschr\"{a}nkung ist, dass die Senke gr\"{o}{\ss}er als die Quelle sein muss.
(Die meisten Befehle erfordern, dass Quelle und Ziel von der
gleichen Gr\"{o}{\ss}e sind.) Hier sind einige Beispiele:
\begin{AsmCodeListing}[frame=none, numbers=left, firstnumber=last]
      movzx  eax, ax          ; erweitert ax zu eax
      movzx  eax, al          ; erweitert al zu eax
      movzx  ax, al           ; erweitert al zu ax
      movzx  ebx, ax          ; erweitert ax zu ebx
\end{AsmCodeListing}

F\"{u}r vorzeichenbehaftete Zahlen gibt es keinen einfachen Weg, um den
{\code MOV} Befehl in jedem Fall zu benutzen. Die 8086 lieferte
mehrere Befehle, um vorzeichenbehaftete Zahlen zu erweitern. Der
{\code CBW} \index{Maschinenbefehl!CBW} (Convert Byte to Word)
Befehl f\"{u}hrt die Vorzeichenerweiterung des AL Registers nach AX
durch. Die Operanden sind implizit. Der {\code CWD}
\index{Maschinenbefehl!CWD} (Convert Word to Double word) Befehl
erweitert das Vorzeichen in AX nach DX:AX\@. Die Notation DX:AX
bedeutet, die DX und AX Register als ein 32~bit Register
aufzufassen, mit den oberen 16 Bits in DX und den unteren Bits in
AX\@. (Erinnern wir uns daran, dass die 8086 kein 32~bit Register
hat!) Die 80386 f\"{u}gte mehrere neue Befehle hinzu. Der {\code CWDE}
\index{Maschinenbefehl!CWDE} (Convert Word to Double word Extended)
Befehl erweitert das Vorzeichen von AX nach EAX\@. Der {\code CDQ}
\index{Maschinenbefehl!CDQ} (Convert Double word to Quad word)
Befehl erweitert das Vorzeichen von EAX nach
EDX:EAX\index{Register!EDX:EAX} (64~Bit!). Schlie{\ss}lich arbeitet der
{\code MOVSX} \index{Maschinenbefehl!MOVSX} Befehl wie {\code
MOVZX}, au{\ss}er dass er die Regeln f\"{u}r vorzeichenbehaftete Zahlen
benutzt.

\subsubsection{Anwendung in der C Programmierung}

Die Erweiterung \MarginNote{ANSI C definiert nicht, ob der Typ
{\code char} mit oder ohne Vorzeichen ist, es liegt an jedem
individuellen Compiler, das zu entscheiden. Deshalb wird der Typ in
Abbildung~\ref{fig:charExt} explizit definiert.} vorzeichenloser und
vorzeichenbehafteter Integer tritt auch in C auf. Variable in C
k\"{o}nnen entweder als vorzeichenbehaftet (signed) oder vorzeichenlos
(unsigned) deklariert werden ({\code int} ist mit Vorzeichen).
Betrachten wir den Code in Abbildung~\ref{fig:charExt}. In
Zeile~\ref{line:CharExt1} wird die Variable {\code a} unter
Verwendung der Regeln f\"{u}r vorzeichenlose Werte erweitert (unter
Benutzung von {\code MOVZX}), aber in Zeile~\ref{line:CharExt2}
werden die vorzeichenbehafteten Regeln f\"{u}r {\code b} benutzt (unter
Benutzung von {\code MOVSX}).

\begin{figure}[ht]
\begin{lstlisting}[frame=tlrb, numbers=left, escapeinside={@}{@}]{}
 unsigned char uchar = 0xFF;
 signed char   schar = 0xFF;
 int a = (int) uchar;     /* a = 255 (0x000000FF) */    @\label{line:CharExt1}@
 int b = (int) schar;     /* b = -1  (0xFFFFFFFF) */    @\label{line:CharExt2}@
\end{lstlisting}
\caption{Ausweitung von {\code char} Werten} \label{fig:charExt}
\end{figure}

Es gibt einen verbreiteten Programmierfehler in C, der direkt mit
diesem Thema in Verbindung steht. Betrachten wir den Code in
Abbildung~\ref{fig:IObug}. Der Prototyp von {\code fgetc()} ist:
\begin{CodeQuote}
 int fgetc( FILE * );
\end{CodeQuote}
Man k\"{o}nnte sich fragen, warum die Funktion einen {\code int}
zur\"{u}ckgibt, wenn sie doch Zeichen liest? Der Grund liegt darin, dass
sie normalerweise einen {\code char} (unter Verwendung der
Null-Erweiterung zu einem {\code int} erweitert) zur\"{u}ckgibt. Jedoch
gibt es einen Wert, den sie zur\"{u}ckgeben kann, der kein Zeichen ist,
n\"{a}mlich {\code EOF}\@. Das ist ein Makro, das gew\"{o}hnlich als $-1$
definiert ist. Folglich gibt {\code fgetc()} entweder ein zu einem
{\code int} erweiterten {\code char} Wert (das in hex {\code
000000{\em xx}} w\"{a}re) oder {\code EOF} (das in hex wie {\code
FFFFFFFF} aussieht) zur\"{u}ck.

\begin{figure}[ht]
\begin{lstlisting}[frame=tlrb, numbers=left, escapeinside={@}{@}]{}
 char ch;
 while( (ch = fgetc(fp)) != EOF ) {             @\label{line:IObug}@
   /* mache etwas mit ch */
 }
\end{lstlisting}
\caption{I/O Fehler}
\label{fig:IObug}
\end{figure}

Das grundlegende Problem mit dem Programm in
Abbildung~\ref{fig:IObug} ist, dass {\code fgetc()} einen {\code
int} zur\"{u}ckgibt, der Wert aber in einem {\code char} gespeichert
wird. C wird die h\"{o}herwertigen Bits abschneiden, um den {\code int}
Wert in einen {\code char} zu pressen. Das einzige Problem ist, dass
die Zahlen (in hex) {\code 000000FF} und {\code FFFFFFFF} beide zum
Byte {\code FF} verkleinert werden. Deshalb kann die while-Schleife
nicht zwischen dem von der Datei gelesenen Byte {\code FF} und dem
Dateiende unterscheiden.

Was der Code in diesem Fall genau tut, h\"{a}ngt davon ab, ob {\code
char} mit oder ohne Vorzeichen ist. Warum? Weil in
Zeile~\ref{line:IObug} {\code ch} mit {\code EOF} verglichen wird.
Da {\code EOF} ein {\code int} Wert ist,\footnote{Es ist ein
allgemeines Missverst\"{a}ndnis, dass Dateien ein EOF Zeichen an ihrem
Ende h\"{a}tten. Dies ist \emph{nicht} der Fall!} wird {\code ch} zu
einem {\code int} erweitert, sodass die beiden verglichenen Werte
von der gleichen Gr\"{o}{\ss}e sind.\footnote{Der Grund f\"{u}r diese Forderung
wird sp\"{a}ter gezeigt werden.} Wie Abbildung~\ref{fig:charExt} zeigte,
ist es sehr wichtig, ob die Variable mit oder ohne Vorzeichen ist.

Ist {\code char} ohne Vorzeichen, wird {\code FF} zu {\code
000000FF} erweitert. Dies wird mit {\code EOF} ({\code FFFFFFFF})
verglichen und als nicht gleich gefunden. Deshalb wird die Schleife
niemals beendet!

Ist {\code char} mit Vorzeichen, wird {\code FF} zu {\code FFFFFFFF}
erweitert. Der Vergleich wird wahr und die Schleife endet. Da das
Byte {\code FF} jedoch von der Datei gelesen werden kann, k\"{o}nnte die
Schleife vorzeitig beendet werden.

Die L\"{o}sung dieses Problems ist, die Variable {\code ch} als einen
{\code int}, nicht als {\code char} zu definieren. Wird dies getan,
wird in Zeile~\ref{line:IObug} weder abgeschnitten noch erweitert.
Innerhalb der Schleife ist es sicher, den Wert abzuschneiden, da
{\code ch} dort wirklich ein einfaches Byte sein \emph{muss}.

\index{Integer!Vorzeichenerweiterung|)}
\index{Integer!Darstellung|)}

\subsection{Arithmetik im Zweierkomplement \index{Zweierkomplement!Arithmetik|(}}

Wie fr\"{u}her gezeigt wurde, f\"{u}hrt der {\code add} Befehl Additionen
und der {\code sub} Befehl f\"{u}hrt Subtraktionen durch. Zwei der Bits
im FLAGS Register, die diese Befehle setzen, sind das
\emph{Overflow} und das \emph{Carry Flag}. Das Overflowflag wird
gesetzt, wenn das wahre Ergebnis der Operation zu gro{\ss} ist, um bei
vorzeichenbehafteter Arithmetik in das Ziel zu passen. Das Carryflag
wird gesetzt, wenn es einen \"{U}bertrag im MSB einer Addition oder
einer Subtraktion gibt. Deshalb kann es verwendet werden, um einen
\"{U}bertrag bei vorzeichenloser Arithmetik zu entdecken. Der Gebrauch
des Carryflags f\"{u}r vorzeichenbehaftete Arithmetik wird in K\"{u}rze
gezeigt werden. Einer der gro{\ss}en Vorteile des 2er~Komplements ist,
dass die Regeln f\"{u}r Addition und Subtraktion genau die gleichen sind
wie f\"{u}r vorzeichenlose Integer. Deshalb k\"{o}nnen {\code add} und
{\code sub} f\"{u}r Integer mit und ohne Vorzeichen verwendet werden.
\[
\begin{array}{rrcrr}
  & 002\mathrm{C} & & & 44~\\
 +& \mathrm{FFFF} & &+&(-1)\\
 \cline{1-2} \cline{4-5}
  & 002\mathrm{B} & & & 43~\\
\end{array}
\]
Dabei wird ein \"{U}bertrag gebildet, der aber nicht Bestandteil der
Antwort ist.

\index{Integer!Multiplikation|(} \index{Maschinenbefehl!MUL|(}
\index{Maschinenbefehl!IMUL|(} Es gibt zwei verschiedene
Multiplizier- und Divisionsbefehle. Um zu multiplizieren, verwendet
man entweder den {\code MUL} oder den {\code IMUL} Befehl. Der
{\code MUL} Befehl wird benutzt, um vorzeichenlose Integer zu
multiplizieren und {\code IMUL} wird benutzt, um vorzeichenbehaftete
Integer zu multiplizieren. Warum werden zwei verschiedene Befehle
ben\"{o}tigt? Die Regeln f\"{u}r die Multiplikation sind f\"{u}r vorzeichenlose
und vorzeichenbehaftete Zahlen im 2er~Komplement unterschiedlich.
Wie kommt das? Betrachten wir die Multiplikation des Bytes FF mit
sich selbst zu einem Ergebnis mit Wortgr\"{o}{\ss}e. Unter Benutzung von
vorzeichenloser Multiplikation ist dies 255 mal 255 oder 65\,025
(oder FE01 in hex). Mit vorzeichenbehafteter Multiplikation ist dies
$-1$ mal $-1$ oder 1 (0001 in hex).

Es gibt verschiedene Formen der Multiplikationsbefehle. Die \"{a}lteste
Form sieht so aus:
\begin{AsmCodeListing}[numbers=none, frame=none]
      mul   source
\end{AsmCodeListing}
\emph{source} ist entweder ein Register oder eine Speicherreferenz.
Es kann kein unmittelbarer Wert sein. Welche Multiplikation genau
ausgef\"{u}hrt wird, h\"{a}ngt von der Gr\"{o}{\ss}e des Quelloperanden ab. Ist der
Operand von Bytegr\"{o}{\ss}e, wird er mit dem Byte im AL Register
multipliziert und das Ergebnis wird in den 16 Bits von AX
gespeichert. Hat die Quelle 16 Bits, wird sie mit dem Wort in AX
multipliziert und das 32 bit Ergebnis wird in DX:AX gespeichert. Hat
die Quelle 32 Bits, wird sie mit EAX multipliziert und das 64 bit
Ergebnis wird nach EDX:EAX\index{Register!EDX:EAX} gespeichert.
\index{Maschinenbefehl!MUL|)}

\begin{table}[ht]
\centering
\begin{tabular}{|c|c|c|l|}
\hline
 { \bf dest} & { \bf source1 } & {\bf source2} & \multicolumn{1}{c|}{\bf Aktion} \\
\hline
             & reg/mem8        &               & AX = AL $\star\,$ source1 \\
             & reg/mem16       &               & DX:AX = AX $\star\,$ source1 \\
             & reg/mem32       &               & EDX:EAX = EAX $\star\,$ source1 \\
 reg16       & reg/mem16       &               & dest $\star$= source1 \\
 reg32       & reg/mem32       &               & dest $\star$= source1 \\
 reg16       & immed8          &               & dest $\star$= immed8 \\
 reg32       & immed8          &               & dest $\star$= immed8 \\
 reg16       & immed16         &               & dest $\star$= immed16 \\
 reg32       & immed32         &               & dest $\star$= immed32 \\
 reg16       & reg/mem16       & immed8        & dest = source1 $\star\,$ source2 \\
 reg32       & reg/mem32       & immed8        & dest = source1 $\star\,$ source2 \\
 reg16       & reg/mem16       & immed16       & dest = source1 $\star\,$ source2 \\
 reg32       & reg/mem32       & immed32       & dest = source1 $\star\,$ source2 \\
\hline
\end{tabular}
\caption{{\code imul} Befehle \label{tab:imul}}
\end{table}

Der {\code IMUL} Befehl hat die gleichen Formate wie {\code MUL},
f\"{u}gt aber einige weitere Befehlsformen hinzu. Es gibt Formate mit
zwei und drei Operanden:
\begin{AsmCodeListing}[numbers=none, frame=none]
      imul   dest, source1
      imul   dest, source1, source2
\end{AsmCodeListing}
\noindent Tabelle~\ref{tab:imul} zeigt die m\"{o}glichen Kombinationen.
\index{Maschinenbefehl!IMUL|)} \index{Integer!Multiplikation|)}

\index{Integer!Division|(} \index{Maschinenbefehl!DIV} Die zwei
Divisionsbefehle sind {\code DIV} und {\code IDIV}\@. Sie f\"{u}hren
Integerdivisionen ohne bzw.\ mit Vorzeichen aus. Das allgemeine
Format ist:  \enlargethispage*{3\baselineskip} % 2 is not enough; \samepage{} wont't work
\begin{AsmCodeListing}[numbers=none, frame=none]
      div   source
\end{AsmCodeListing}
Wenn die Quelle 8~bit gro{\ss} ist, dann wird AX durch den Operanden
geteilt. Der Quotient wird in AL gespeichert und der Rest in AH\@.
Hat die Quelle 16~Bits, dann wird DX:AX durch den Operanden
dividiert. Der Quotient wird in AX gespeichert, der Rest in DX\@.
Hat die Quelle 32~Bits, wird EDX:EAX \index{Register!EDX:EAX} durch
den Operanden geteilt, der Quotient in EAX gespeichert und der Rest
in EDX\@. Der {\code IDIV} \index{Maschinenbefehl!IDIV} Befehl
arbeitet auf die gleiche Weise. Es gibt keine speziellen {\code
IDIV} Befehlsformen wie bei {\code IMUL}\@. Wenn der Quotient zu
gro{\ss} ist um in sein Register zu passen oder der Teiler Null ist,
wird das Programm unterbrochen und beendet. Ein sehr verbreiteter
Fehler ist es, vor der Division zu vergessen DX oder EDX zu
initialisieren. \index{Integer!Division|)}

Der {\code NEG} \index{Maschinenbefehl!NEG} Befehl negiert seinen
einzigen Operanden, indem er dessen Zweierkomplement berechnet. Sein
Operand kann jedes 8-, 16- oder 32-bit Register oder Speicherstelle
sein.

\subsection{Beispielprogramm}
\index{math.asm|(}
\begin{AsmCodeListing}[label=math.asm, numbers=left, commandchars=\\\{\}]
 %include "asm_io.inc"
 segment .data                     ; Ausgabe-Strings
 prompt          db    "Enter a number: ", 0
 square_msg      db    "Square of input is ", 0
 cube_msg        db    "Cube of input is ", 0
 cube25_msg      db    "Cube of input times 25 is ", 0
 quot_msg        db    "Quotient of cube/100 is ", 0
 rem_msg         db    "Remainder of cube/100 is ", 0
 neg_msg         db    "The negation of the remainder is ", 0

 segment .bss
 input   resd    1

 segment .text
         global  _asm_main
 _asm_main:
         enter   0, 0              ; bereite Routine vor
         pusha

         mov     eax, prompt
         call    print_string

         call    read_int
         mov     [input], eax

         imul    eax               ; edx:eax = eax * eax
         mov     ebx, eax          ; sichere Antwort in ebx
         mov     eax, square_msg
         call    print_string
         mov     eax, ebx
         call    print_int
         call    print_nl

         mov     ebx, eax
         imul    ebx, [input]      ; ebx *= [input]
         mov     eax, cube_msg
         call    print_string
         mov     eax, ebx
         call    print_int
         call    print_nl

         imul    ecx, ebx, 25      ; ecx = ebx*25
         mov     eax, cube25_msg
         call    print_string
         mov     eax, ecx
         call    print_int
         call    print_nl

         mov     eax, ebx
         cdq                       ; initialisiere edx durch Vorzeichenerweiterung
         mov     ecx, 100          ; kann nicht durch unmittelbaren Wert teilen
         idiv    ecx               ; edx:eax / ecx
         mov     ecx, eax          ; sichere Quotient in ecx
         mov     eax, quot_msg
         call    print_string
         mov     eax, ecx
         call    print_int
         call    print_nl
         mov     eax, rem_msg
         call    print_string
         mov     eax, edx
         call    print_int
         call    print_nl

         neg     edx               ; negiere den Teilerrest
         mov     eax, neg_msg
         call    print_string
         mov     eax, edx
         call    print_int
         call    print_nl

         popa
         mov     eax, 0            ; kehre zu C zur\"{u}ck
         leave
         ret
\end{AsmCodeListing}
\index{math.asm|)}

\subsection{Arithmetik mit erh\"{o}hter Genauigkeit \label{sec:ExtPrecArith}
\index{Integer!erh\"{o}hte Genauigkeit|(}}

Die Assemblersprache besitzt ebenso Befehle, die einem erlauben,
Addition und Subtraktion auch mit Zahlen durchzuf\"{u}hren, die gr\"{o}{\ss}er
als Doppelw\"{o}rter sind. Diese Befehle benutzen das Carryflag. Wie
oben erw\"{a}hnt, modifizieren {\code ADD} \index{Maschinenbefehl!ADD}
und {\code SUB} \index{Maschinenbefehl!SUB} Befehle das Carryflag,
wenn ein \"{U}bertrag generiert wird. Diese im Carryflag gespeicherte
Information kann benutzt werden, um gro{\ss}e Zahlen zu addieren oder
subtrahieren, indem die Operation in einzelne Doppelwort- (oder
kleinere) St\"{u}cke aufgeteilt wird.

Die {\code ADC} \index{Maschinenbefehl!ADC} und {\code SBB}
\index{Maschinenbefehl!SBB} Befehle benutzen diese Information im
Carryflag. Der {\code ADC} Befehl f\"{u}hrt die folgende Operation
durch:
\begin{center}
{\code \emph{operand1} = \emph{operand1} + carry flag + \emph{operand2} }
\end{center}
Der {\code SBB} Befehl f\"{u}hrt aus:
\begin{center}
{\code \emph{operand1} = \emph{operand1} - carry flag - \emph{operand2} }
\end{center}
Wie werden diese benutzt? Betrachten wir die Summe von 64~bit
Integern in EDX:EAX \index{Register!EDX:EAX} und EBC:ECX\@. Der
folgende Code w\"{u}rde die Summe in EDX:EAX speichern:
\begin{AsmCodeListing}[frame=none, numbers=left, commandchars=\\\{\}]
      add    eax, ecx         ; addiere untere 32 Bits
      adc    edx, ebx         ; addiere obere 32 Bits und \"{U}bertrag
\end{AsmCodeListing}
Die Subtraktion ist sehr \"{a}hnlich. Folgender Code zieht EBX:ECX von
EDX:EAX ab:
\begin{AsmCodeListing}[frame=none, numbers=left, firstnumber=last, commandchars=\\\{\}]
      sub    eax, ecx         ; subtrahiere untere 32 Bits
      sbb    edx, ebx         ; subtrahiere obere 32 Bits und \"{U}bertrag
\end{AsmCodeListing}

F\"{u}r \emph{wirklich} gro{\ss}e Zahlen k\"{o}nnte eine Schleife benutzt werden
(siehe Abschnitt~\ref{sec:control}). In einer Summationsschleife
w\"{u}rde es bequemer sein, den {\code ADC} Befehl bei jeder Iteration
zu verwenden (anstatt f\"{u}r alle au{\ss}er der ersten Iteration). Das kann
getan werden, wenn der {\code CLC} \index{Maschinenbefehl!CLC}
(CLear Carry) Befehl direkt vor der Schleife verwendet wird, um das
Carryflag mit 0 zu initialisieren. Wenn das Carryflag 0 ist, gibt es
keine Unterschiede zwischen den {\code ADD} und {\code ADC}
Befehlen. Die gleiche Idee kann auch f\"{u}r die Subtraktion verwendet
werden. \index{Integer!erh\"{o}hte Genauigkeit|)}
\index{Zweierkomplement!Arithmetik|)}

\section{Kontrollstrukturen}
\label{sec:control}

Hochsprachen verf\"{u}gen \"{u}ber Kontrollstrukturen auf einem hohen Niveau
(z.\,B.\ die \emph{if} und \emph{while} Statements), die den
Ausf\"{u}hrungsfluss kontrollieren. Assembler bietet keine solchen
komplexen Kontrollstrukturen. Er benutzt stattdessen das ber\"{u}chtigte
\emph{goto} und unangemessen benutzt, kann es zu Spaghetticode
f\"{u}hren! Es \emph{ist} jedoch m\"{o}glich, strukturierte
Assemblerprogramme zu schreiben. Die grunds\"{a}tzliche Vorgehensweise
ist, die Programme m\"{o}glichst unter Verwendung der vertrauten
Kontrollstrukturen der Hochsprachen zu entwerfen und den Entwurf in
die entsprechende Assemblersprache zu \"{u}bersetzen (etwa so, wie es
ein Compiler machen w\"{u}rde).

\subsection{Vergleiche \index{Integer!Vergleiche|(} \index{Maschinenbefehl!CMP|(}}
%TODO: Make a table of all the FLAG bits

\index{Register!FLAGS|(} Kontrollstrukturen entscheiden auf der
Grundlage des Vergleichs von Daten, was zu tun ist. In Assembler
wird das Ergebnis eines Vergleichs im FLAGS Register (Tabelle
\ref{tab:FLAGS}) gespeichert, um sp\"{a}ter benutzt zu werden. Die 80x86
stellt den {\code CMP} Befehl zur Verf\"{u}gung, um Vergleiche
durchzuf\"{u}hren. Das FLAGS Register wird auf der Grundlage der
Differenz der beiden Operanden des {\code CMP} Befehls gesetzt. Die
Operanden werden subtrahiert und die FLAGS werden auf Grund des
Ergebnisses gesetzt, allerdings wird das Ergebnis \emph{nirgends}
gespeichert. Wenn man das Ergebnis ben\"{o}tigt, benutzt man den {\code
SUB} anstatt des {\code CMP} Befehls.

% <<<<<<<<<<<<<<<<<<<<<<<<<<<<<<<<<<<<<<<<<<<<<<<<<<<<<<<<<<<<<<<<<<<<<<<<<<<<<
\begin{table}[ht]
\center
 \begin {tabular}{r|c|c|c|c|c|c|c|c|}
 \cline{2-9}
 Bit  &   7  &   6  &    5    &  4  &    3    &    2   &    1    &   0  \\
 \cline{2-9}
 Flag &  SF  &  ZF  & \,~0~\, &  AF & \,~0~\, &   PF   & \,~1~\, &  CF  \\
 \cline{2-9}
      & sign & zero &         & aux &         & parity &         & carry\\
 \cline{2-9}
 \end{tabular}
\caption{Die Flagbits im unteren Byte des (E)FLAGS Registers \label{tab:FLAGS}}       % ### german ###
%\caption{The flag bits in the lower byte of the (E)FLAGS register \label{tab:FLAGS}} % ### english ###
\end{table}
% <<<<<<<<<<<<<<<<<<<<<<<<<<<<<<<<<<<<<<<<<<<<<<<<<<<<<<<<<<<<<<<<<<<<<<<<<<<<<

\index{Integer!ohne Vorzeichen|(} F\"{u}r vorzeichenlose Integer sind
zwei Flags (Bits im FLAGS Register) wichtig: das Zero- (ZF)
\index{Register!FLAGS!ZF} und das Carry-Flag (CF).
\index{Register!FLAGS!CF} Das Zero\-flag wird gesetzt (1), wenn die
resultierende Differenz Null sein w\"{u}rde. Das Carryflag wird als
Borrowflag bei der Subtraktion benutzt. Betrachten wir einen
Vergleich wie:
\begin{AsmCodeListing}[frame=none, numbers=none]
      cmp    vleft, vright
\end{AsmCodeListing}
Die Differenz {\code vleft~-~vright} wird berechnet und die Flags
entsprechend gesetzt. Ist die Differenz von {\code CMP} Null, {\code
vleft~=~vright}, dann wird ZF gesetzt (d.\,h.~1) und CF gel\"{o}scht
(d.\,h.~0). Ist {\code vleft~>~vright}, dann wird ZF gel\"{o}scht und CF
wird gel\"{o}scht (kein Borrow). Ist {\code vleft~<~vright}, dann wird
ZF gel\"{o}scht und CF wird gesetzt (Borrow). \index{Integer!ohne
Vorzeichen|)}

\index{Integer!mit Vorzeichen|(} F\"{u}r Integer mit Vorzeichen
\MarginNote{Warum ist SF~=~OF, wenn {\code vleft~>~vright}? Wenn es
keinen \"{U}berlauf gibt, dann hat die Differenz den richtigen Wert und
muss nicht-negativ sein. Deshalb ist SF~=~OF~=~0. Jedoch, wenn es
einen \"{U}berlauf gibt, wird die Differenz nicht den richtigen Wert
haben (und wird tats\"{a}chlich negativ sein). Folglich ist
SF~=~OF~=~1.} gibt es drei Flags, die wichtig sind: das Zero\-flag
\index{Register!FLAGS!ZF} (ZF), das Overflowflag
\index{Register!FLAGS!OF} (OF) und das Signflag
\index{Register!FLAGS!SF} (SF). Das Over\-flow\-flag wird gesetzt,
wenn das Ergebnis einer Operation \"{u}berl\"{a}uft (oder unterl\"{a}uft). Das
Sign\-flag wird gesetzt, wenn das Ergebnis einer Operation negativ
ist. Ist {\code vleft~= vright}, wird das ZF gesetzt (genauso wie
f\"{u}r vorzeichenlose Integer). Ist {\code vleft~> vright}, wird ZF
gel\"{o}scht und SF~=~OF\@. Ist {\code vleft~< vright}, wird ZF gel\"{o}scht
und SF~$\neq$~OF\@. \index{Integer!mit Vorzeichen|)}

Vergessen Sie nicht, dass auch andere Befehle das FLAGS Register
\"{a}ndern k\"{o}nnen, nicht nur {\code CMP}\@.
\index{Maschinenbefehl!CMP|)} \index{Integer!Vergleiche|)}
\index{Register!FLAGS|)} \index{Integer|)}

\subsection{Sprungbefehle}

Sprungbefehle k\"{o}nnen die Ausf\"{u}hrung zu beliebigen Punkten eines
Programms f\"{u}hren. In anderen Worten, sie wirken wie ein \emph{Goto}.
Es gibt zwei Arten von Sprungbefehlen: unbedingte und bedingte. Ein
unbedingter Sprung ist genau wie ein Goto, die Verzweigung wird
immer durchgef\"{u}hrt. Ein bedingter Sprung kann die Verzweigung
durchf\"{u}hren oder nicht, abh\"{a}ngig von den Flags im FLAGS Register.
F\"{u}hrt ein bedingter Sprung die Verzweigung nicht durch, geht die
Kontrolle zum n\"{a}chsten Befehl \"{u}ber.

\index{Maschinenbefehl!JMP|(} Der {\code JMP} (kurz f\"{u}r \emph{jump})
Befehl f\"{u}hrt unbedingte Spr\"{u}nge \index{Sprung!unbedingter|(} aus.
Sein einziges Argument ist gew\"{o}hnlich ein \emph{Codelabel} des
Befehls, zu dem gesprungen werden soll. Der Assembler oder Linker
wird das Label durch die korrekte Adresse des Befehls ersetzen. Dies
ist eine weitere der m\"{u}hseligen Operationen, die der Assembler
ausf\"{u}hrt, um das Leben des Programmierers einfacher zu machen. Es
ist wichtig, sich zu vergegenw\"{a}rtigen, dass der Befehl unmittelbar
nach dem {\code JMP} Befehle niemals ausgef\"{u}hrt wird, es sei denn,
ein anderer Befehl verzweigt zu ihm!

Es gibt verschiedene Varianten des Sprungbefehls:
\begin{description}
\parskip=-0.25em %reduce the spacing <<<<<<<<<<<<<<<<<<<<<<<<<<<<<<<<<<<<<<<<<<

\item[SHORT]
Dieser Sprung ist in der Reichweite sehr begrenzt. Er kann nur um
128 Bytes im Speicher vor oder zur\"{u}ck springen. Der Vorteil dieses
Typs ist, dass er weniger Speicher als die anderen ben\"{o}tigt. Er
verwendet ein einzelnes vorzeichenbehaftetes Byte um das
\emph{Displacement} des Sprungs zu speichern. Der Wert des
Displacements entscheidet, um wie viele Bytes vor oder zur\"{u}ck
gesprungen werden soll. (Das Displacement wird zu EIP addiert.) Um
einen kurzen Sprung zu spezifizieren, benutzt man das Schl\"{u}sselwort
{\code SHORT} unmittelbar vor dem Label im {\code JMP} Befehl.

\item[NEAR]
Dieser Sprung ist der vorgegebene Typ sowohl f\"{u}r unbedingte als auch
f\"{u}r bedingte Spr\"{u}nge; er kann verwendet werden, um zu jeder Stelle
in einem Segment zu springen. Tats\"{a}chlich unterst\"{u}tzt die 80386 zwei
Typen von nahen Spr\"{u}ngen. Einer verwendet zwei Bytes f\"{u}r das
Displacement. Dies erlaubt einem, sich ungef\"{a}hr 32\,000 Bytes vor
oder zur\"{u}ck zu bewegen. Der andere Typ benutzt vier Bytes f\"{u}r das
Displacement, das einem nat\"{u}rlich erm\"{o}glicht, sich zu jeder Stelle
im Codesegment zu bewegen. Der Typ mit vier Bytes ist der
vorgegebene im protected Mode der 386. Der Typ mit zwei Bytes kann
spezifiziert werden, indem das Schl\"{u}sselwort {\code WORD} vor das
Label im {\code JMP} Befehl gestellt wird.

\item[FAR]
Dieser Sprung erlaubt der Kontrolle, sich in ein anderes Codesegment
zu bewegen. Dies zu tun ist im protected Mode der 386 eine sehr
seltene Sache. \index{Sprung!unbedingter|)}
\end{description}

G\"{u}ltige Codelabels folgen denselben Regeln wie Datenlabels.
Codelabels werden definiert, indem sie im Codesegment vor die
Anweisung, die sie markieren, gesetzt werden. An das Label wird am
Ort seiner Definition ein Doppelpunkt angeh\"{a}ngt. Der Doppelpunkt ist
\emph{nicht} Bestandteil des Namens. \index{Maschinenbefehl!JMP|)}
\index{Sprung!bedingter|(}
\begin{table}[ht]
\center
\begin{tabular}{|ll|}
\hline
 JZ  & verzweigt nur, wenn ZF gesetzt ist \\
 JNZ & verzweigt nur, wenn ZF nicht gesetzt ist \\
 JO  & verzweigt nur, wenn OF gesetzt ist \\
 JNO & verzweigt nur, wenn OF nicht gesetzt ist \\
 JS  & verzweigt nur, wenn SF gesetzt ist \\
 JNS & verzweigt nur, wenn SF nicht gesetzt ist \\
 JC  & verzweigt nur, wenn CF gesetzt ist \\
 JNC & verzweigt nur, wenn CF nicht gesetzt ist \\
 JP  & verzweigt nur, wenn PF gesetzt ist \\
 JNP & verzweigt nur, wenn PF nicht gesetzt ist \\
\hline
\end{tabular}
\caption{Einfache bedingte Verzweigungen \label{tab:SimpBran}
%\index{Maschinenbefehl!JZ} \index{Maschinenbefehl!JNZ}
%\index{Maschinenbefehl!JO} \index{Maschinenbefehl!JNO}
%\index{Maschinenbefehl!JS} \index{Maschinenbefehl!JNS}
%\index{Maschinenbefehl!JC} \index{Maschinenbefehl!JNC}
%\index{Maschinenbefehl!JP} \index{Maschinenbefehl!JNP}
}
\end{table}
\index{Maschinenbefehl!J\emph{cc}|(}

Es gibt viele verschiedene bedingte Sprunganweisungen. Auch sie
ben\"{o}tigen ein Codelabel als ihren einzigen Operanden. Die
einfachsten betrachten nur ein einziges Flag im FLAGS Register, um
zu entscheiden, ob sie verzweigen oder nicht. Siehe
Tabelle~\ref{tab:SimpBran} f\"{u}r eine Liste dieser Instruktionen. (PF
ist das \emph{ Parityflag}, \index{Register!FLAGS!PF} das anzeigt,
ob die Anzahl der gesetzten Bits in den niederwertigen 8 bit des
Ergebnisses gerade oder ungerade ist.)

{\samepage Der folgende Pseudocode:
\begin{Verbatim}[numbers=none]
 if ( EAX == 0 )
   EBX = 1;
 else
   EBX = 2;
\end{Verbatim}
k\"{o}nnte in Assembler geschrieben werden als:}
\begin{AsmCodeListing}[frame=none, numbers=left, commandchars=\\\{\}]
      cmp    eax, 0           ; setze Flags (ZF gesetzt, wenn eax - 0 = 0)
      jz     thenblock        ; wenn ZF gesetzt ist verzweige zu thenblock
      mov    ebx, 2           ; ELSE Teil von IF
      jmp    next             ; \"{u}berspringe THEN Teil von IF
 thenblock:
      mov    ebx, 1           ; THEN Teil von IF
 next:
\end{AsmCodeListing}

Andere Vergleiche sind unter Verwendung der bedingten Verzweigungen
in Tabelle~\ref{tab:SimpBran} nicht so einfach. Um das zu zeigen,
betrachten wie den folgenden Pseudocode:
\begin{Verbatim}[numbers=none]
 if ( EAX >= 5 )
   EBX = 1;
 else
   EBX = 2;
\end{Verbatim}
Wenn EAX gr\"{o}{\ss}er als oder gleich f\"{u}nf ist, dann kann das ZF gesetzt
sein oder nicht und SF ist gleich OF\@. Hier ist Assemblercode, der
auf dieses Bedingungen testet (unter der Annahme, dass EAX
vorzeichenbehaftet ist):
\begin{AsmCodeListing}[frame=none, numbers=left]
      cmp    eax, 5
      js     signon           ; goto signon wenn SF = 1
      jo     elseblock        ; goto elseblock wenn SF = 0 und OF = 1
      jmp    thenblock        ; goto thenblock wenn SF = 0 und OF = 0
 signon:
      jo     thenblock        ; goto thenblock wenn SF = 1 und OF = 1
 elseblock:
      mov    ebx, 2
      jmp    next
 thenblock:
      mov    ebx, 1
 next:
\end{AsmCodeListing}

\begin{table}
\center
\begin{tabular}{|ll|ll|}
\hline
\multicolumn{2}{|c|}{\textbf{mit Vorzeichen}} & \multicolumn{2}{c|}{\textbf{ohne Vorzeichen}} \\
\hline
 JE       & Sprung bei {\code vleft $=$ vright}    & JE       & Sprung bei {\code vleft $=$ vright} \\
 JNE      & Sprung bei {\code vleft $\neq$ vright} & JNE      & Sprung bei {\code vleft $\neq$ vright} \\
 JL, JNGE & Sprung bei {\code vleft $<$ vright}    & JB, JNAE & Sprung bei {\code vleft $<$ vright} \\
 JLE, JNG & Sprung bei {\code vleft $\leq$ vright} & JBE, JNA & Sprung bei {\code vleft $\leq$ vright} \\
 JG, JNLE & Sprung bei {\code vleft $>$ vright}    & JA, JNBE & Sprung bei {\code vleft $>$ vright} \\
 JGE, JNL & Sprung bei {\code vleft $\geq$ vright} & JAE, JNB & Sprung bei {\code vleft $\geq$ vright} \\
\hline
\end{tabular}
\caption{Befehle f\"{u}r Vergleiche mit und ohne Vorzeichen
\label{tab:CompBran}
%\index{Maschinenbefehl!JE} \index{Maschinenbefehl!JNE}  % <<< we're inside "Jcc"
%\index{Maschinenbefehl!JL} \index{Maschinenbefehl!JNGE}
%\index{Maschinenbefehl!JLE} \index{Maschinenbefehl!JNG}
%\index{Maschinenbefehl!JG} \index{Maschinenbefehl!JNLE}
%\index{Maschinenbefehl!JGE} \index{Maschinenbefehl!JNL}
%\index{Maschinenbefehl!JB} \index{Maschinenbefehl!JNAE}
%\index{Maschinenbefehl!JBE} \index{Maschinenbefehl!JNA}
%\index{Maschinenbefehl!JA} \index{Maschinenbefehl!JNBE}
%\index{Maschinenbefehl!JAE} \index{Maschinenbefehl!JNB}
}
\end{table}

Der obige Code ist sehr unhandlich. Gl\"{u}cklicherweise besitzt die
80x86 zus\"{a}tzliche Sprunganweisungen, die diese Art von Tests
\emph{viel} einfacher machen. Von jedem gibt es vorzeichenbehaftete
und vorzeichenlose Versionen. Tabelle~\ref{tab:CompBran} zeigt diese
Befehle. Die gleich und ungleich Verzweigungen (JE und JNE) sind die
selben sowohl f\"{u}r Integer mit Vorzeichen als auch ohne Vorzeichen.
(In Wirklichkeit sind JE und JNE wirklich identisch mit jeweils JZ
und JNZ.) Jeder der anderen Sprunganweisungen hat zwei Synonyme. Zum
Beispiel betrachten wir JL (Jump Less than) und JNGE (Jump Not
Greater than or Equal to). Dies sind die gleichen Instruktionen, da:
\[ x < y \Longleftrightarrow \mathbf{not}( x \geq y ) \]
Die vorzeichenlosen Vergleiche verwenden A f\"{u}r \emph{above} und B
f\"{u}r \emph{below} anstatt L und G\@.

Unter Verwendung dieser neuen Sprunganweisungen kann der obige
Pseudocode viel leichter in Assembler \"{u}bersetzt werden.
% <<< here you have to code the ELSE before the THEN part <<<<<<<<<<<<<<<<<<<<<
\begin{AsmCodeListing}[frame=none, numbers=left, firstnumber=last]
      cmp    eax, 5
      jge    thenblock
      mov    ebx, 2
      jmp    next
 thenblock:
      mov    ebx, 1
 next:
\end{AsmCodeListing}
% <<< do as compilers do <<<<<<<<<<<<<<<<<<<<<<<<<<<<<<<<<<<<<<<<<<<<<<<<<<<<<<
%\begin{AsmCodeListing}[frame=none, numbers=left, firstnumber=last]
%      cmp    eax, 5
%      jnge   elseblock
%      mov    ebx, 1
%      jmp    next
% elseblock:
%      mov    ebx, 2
% next:
%\end{AsmCodeListing}
% <<<<<<<<<<<<<<<<<<<<<<<<<<<<<<<<<<<<<<<<<<<<<<<<<<<<<<<<<<<<<<<<<<<<<<<<<<<<<
\index{Sprung!bedingter|)} \index{Maschinenbefehl!J\emph{cc}|)}

\subsection{Der {\code LOOP} Befehl}

Die 80x86 stellt mehrere Befehle zur Verf\"{u}gung, die zur
Implementierung von \emph{for}-\"{a}hnlichen Schleifen entwickelt
wurden. Jeder dieser Befehle benutzt ein Code\-la\-bel als seinen
einzigen Operanden.
\begin{description}
\parskip=-0.10em %reduce the spacing <<<<<<<<<<<<<<<<<<<<<<<<<<<<<<<<<<<<<<<<<<

\item[LOOP]
\index{Maschinenbefehl!LOOP} Dekrementiert ECX und verzweigt zum
Label, wenn ECX $\neq$ 0

\item[LOOPE, LOOPZ]
\index{Maschinenbefehl!LOOPE/LOOPZ} Dekrementiert ECX (das FLAGS
Register wird nicht ver\-\"{a}n\-dert) und verzweigt, wenn ECX $\neq$ 0
und ZF = 1

\item[LOOPNE, LOOPNZ]
\index{Maschinenbefehl!LOOPNE/LOOPNZ} Dekrementiert ECX (FLAGS
unver\"{a}ndert), verzweigt, wenn ECX $\neq$ 0 und ZF = 0
\end{description}

Die letzten beiden Befehle sind f\"{u}r sequenzielle Suchschleifen
n\"{u}tzlich. Der folgende Pseudocode:
\begin{lstlisting}[numbers=none]{}
 sum = 0;
 for( i = 10; i > 0; i-- )
   sum += i;
\end{lstlisting}
\noindent k\"{o}nnte so in Assemblersprache \"{u}bersetzt werden:
\begin{AsmCodeListing}[frame=none, numbers=left]
      mov    eax, 0           ; eax ist sum
      mov    ecx, 10          ; ecx ist i
 loop_start:
      add    eax, ecx
      loop   loop_start
\end{AsmCodeListing}

\section[\"{U}bersetzung von Standard-Kontrollstrukturen]{Die \"{U}bersetzung von
Standard-Kontrollstrukturen}

Dieser Abschnitt betrachtet, wie die Standard-Kontrollstrukturen der
Hochsprachen in Assembler implementiert werden k\"{o}nnen.

\subsection{If Anweisungen \index{if Anweisung|(}}
Der folgende Pseudocode:
\begin{lstlisting}[numbers=none]{}
 if ( Bedingung )
   then_block;
 else
   else_block;
\end{lstlisting}
\noindent k\"{o}nnte implementiert werden als:
\begin{AsmCodeListing}[frame=none, numbers=left, commandchars=\\\{\}]
      ; Code um FLAGS zu setzen
      jxx    else_block       ; w\"{a}hle xx f\"{u}r Sprung wenn Bedingung falsch
      ; Code f\"{u}r then Block
      jmp    endif
 else_block:
      ; Code f\"{u}r else Block
 endif:
\end{AsmCodeListing}

Wenn es kein else gibt, dann kann der Sprung zu {\code else\_block}
durch einen Sprung zu {\code endif} ersetzt werden.
\begin{AsmCodeListing}[frame=none, numbers=left, firstnumber=last, commandchars=\\\{\}]
      ; Code um FLAGS zu setzen
      jxx    endif            ; w\"{a}hle xx f\"{u}r Sprung wenn Bedingung falsch
      ; Code f\"{u}r then Block
 endif:
\end{AsmCodeListing}
\index{if Anweisung|)}

\subsection{While Schleifen \index{while Schleife|(}}
Die \emph{while} Schleife ist eine kopfgesteuerte Schleife:
\begin{lstlisting}[numbers=none]{}
 while ( Bedingung ) {
   Rumpf der Schleife;
 }
\end{lstlisting}
\noindent Das k\"{o}nnte \"{u}bersetzt werden zu:
\begin{AsmCodeListing}[frame=none, numbers=left, commandchars=\\\{\}]
 while:
      ; Code um FLAGS auf Grundlage der Bedingung zu setzen
      jxx    endwhile         ; w\"{a}hle xx f\"{u}r Sprung wenn falsch
      ; Schleifen-Rumpf
      jmp    while
 endwhile:
\end{AsmCodeListing}
\index{while Schleife|)}

\subsection{Do while Schleifen \index{do while Schleife|(}}
Die \emph{do while} Schleife ist eine fu{\ss}gesteuerte Schleife:
\begin{lstlisting}[numbers=none]{}
 do {
   Rumpf der Schleife;
 } while ( Bedingung );
\end{lstlisting}
\noindent Das k\"{o}nnte \"{u}bersetzt werden zu:
\begin{AsmCodeListing}[frame=none, numbers=left, commandchars=\\\{\}]
 do:
      ; Schleifen-Rumpf
      ; Code um FLAGS auf Grundlage der Bedingung zu setzen
      jxx    do               ; w\"{a}hle xx f\"{u}r Sprung wenn wahr
\end{AsmCodeListing}
\index{do while Schleife|)}

\section{Beispiel: Primzahlsuche}
Dieser Abschnitt betrachtet ein Programm, das Primzahlen findet. Um
das zu tun, gibt es keine Formel. Erinnern wir uns, dass Primzahlen
nur durch 1 und sich selbst ohne Rest teilbar sind. Die grundlegende
Methode, die dieses Programm benutzt, ist, die Faktoren aller
ungeraden Zahlen\footnote{2 ist die einzige gerade Primzahl.} unter
einer gegebenen Grenze zu finden. Wenn f\"{u}r eine ungerade Zahl kein
Faktor gefunden werden kann, dann ist sie prim.
Abbildung~\ref{fig:primec} zeigt den grundlegenden Algorithmus,
geschrieben in C\@.

\begin{figure}[ht]
\begin{lstlisting}[frame=tlrb, numbers=left, escapeinside={@}{@}]{}
 unsigned guess;   /* laufende Testzahl @\itshape{f\"{u}r}@ Primtest */
 unsigned factor;  /* @\itshape{m\"{o}glicher}@ Faktor von guess */
 unsigned limit;   /* Finde PZ bis zu diesem Wert */

 printf("Find primes up to: ");
 scanf("%u", &limit);
 printf("2\n");    /* behandle die ersten beiden */
 printf("3\n");    /* Primzahlen als Spezialfall */
 guess = 5;        /* @\itshape{anf\"{a}ngliche}@ Testzahl */
 while ( guess <= limit ) {
   /* suche einen Faktor von guess */
   factor = 3;
   while ( factor*factor < guess &&
           guess % factor != 0 )
     factor += 2;
   if ( guess % factor != 0 )
     printf("%d\n", guess);
   guess += 2;     /* beachte nur ungerade Zahlen  */
 }
\end{lstlisting}
\caption{Primzahlsuche in C}\label{fig:primec}
\end{figure}

\pagebreak[2] Hier ist die Assemblerversion: \index{prime.asm|(}
\begin{AsmCodeListing}[label=prime.asm, commandchars=\\\{\}]
 %include "asm_io.inc"
 segment .data
 Message db      "Find primes up to: ", 0

 segment .bss
 Limit   resd    1                    ; finde PZ bis zu dieser Grenze
 Guess   resd    1                    ; laufende Testzahl f\"{u}r prime

 segment .text
         global  _asm_main
 _asm_main:
         enter   0, 0                 ; bereite Routine vor
         pusha

         mov     eax, Message
         call    print_string
         call    read_int             ; scanf("%u", &limit );
         mov     [Limit], eax

         mov     eax, 2               ; printf("2\symbol{"5C}n");
         call    print_int
         call    print_nl
         mov     eax, 3               ; printf("3\symbol{"5C}n");
         call    print_int
         call    print_nl

         mov     dword [Guess], 5     ; guess = 5;
 while_limit:                         ; while ( guess <= limit )
         mov     eax, [Guess]
         cmp     eax, [Limit]
         jnbe    end_while_limit      ; jnbe, da Zahlen ohne VZ sind

         mov     ebx, 3               ; ebx ist factor = 3;
 while_factor:
         mov     eax, ebx
         mul     eax                  ; edx:eax = eax*eax
         jo      end_while_factor     ; wenn Produkt nicht in eax allein passt
         cmp     eax, [Guess]
         jnb     end_while_factor     ; if !(factor*factor < guess)
         mov     eax, [Guess]
         mov     edx, 0
         div     ebx                  ; edx = edx:eax % ebx
         cmp     edx, 0
         je      end_while_factor     ; if !(guess % factor != 0)

         add     ebx, 2               ; factor += 2;
         jmp     while_factor
 end_while_factor:
         je      end_if               ; if !(guess % factor != 0)
         mov     eax, [Guess]         ; printf("%u\symbol{"5C}n")
         call    print_int
         call    print_nl
 end_if:
         add     dword [Guess], 2     ; guess += 2
         jmp     while_limit
 end_while_limit:

         popa
         mov     eax, 0               ; kehre zu C zur\"{u}ck
         leave
         ret
\end{AsmCodeListing}
\index{prime.asm|)}

% -*-latex-*-
\chapter{Operaciones con bits}
\section{Operaciones de desplazamientos\index{operaciones con bits!desplazamientos|(}}

El lenguaje ensamblador le permite al programador manipular bits
individuales de los datos. Una operaci�n com�n es llamada un
\emph{desplazamiento}. Una operaci�n de desplazamiento mueve la posici�n
de los bits de alg�n dato. Los desplazamientos pueden  ser hacia la
izquierda (hacia el bit m�s significativo) o hacia la derecha (el bit
menos significativo).

\subsection{Desplazamientos l�gicos\index{operaciones con
bits!desplazamientos!desplazamientos l�gicos|(}}

Un desplazamiento l�gico es el tipo m�s simple de desplazamiento.
Desplaza de una manera muy directa. La Figura~\ref{fig:logshifts}
muestra un ejemplo del desplazamiento de un byte.

\begin{figure}[h]
\centering
\begin{tabular}{l|c|c|c|c|c|c|c|c|}
\cline{2-9}
Original      & 1 & 1 & 1 & 0 & 1 & 0 & 1 & 0 \\
\cline{2-9}
Desplazado a la izquierda & 1 & 1 & 0 & 1 & 0 & 1 & 0 & 0 \\
\cline{2-9}
Desplazado a la derecha & 0 & 1 & 1 & 1 & 0 & 1 & 0 & 1 \\
\cline{2-9}
\end{tabular}
\caption{Desplazamientos l�gicos\label{fig:logshifts}}
\end{figure}

Observe que los nuevos bits que entran son siempre cero. Se usan las
instrucciones {\code SHL} \index{SHL} y {\code SHR}\index{SHR} para
realizar los desplazamientos a la izquierda y derecha respectivamente.
Estas instrucciones le permiten a uno desplazar cualquier n�mero de
posiciones. El n�mero de posiciones puede ser o una constante o puede
estar almacenado en el registro {\code CL}. El �ltimo bit desplazado se
almacena en la bandera de carry. A continuaci�n, algunos ejemplos:
\begin{AsmCodeListing}[frame=none]
      mov    ax, 0C123H
      shl    ax, 1           ; desplaza un bit a la izquierda,   
                             ; ax = 8246H, CF = 1
      shr    ax, 1           ; desplaza un bit a la derecha,  
                             ; ax = 4123H, CF = 0
      shr    ax, 1           ; desplaza un bit a la derecha,  
                             ; ax = 2091H, CF = 1
      mov    ax, 0C123H
      shl    ax, 2           ; desplaza dos bit a la izquierda,  
                             ;ax = 048CH, CF = 1
      mov    cl, 3
      shr    ax, cl          ; desplaza tres bit a la derecha,
                             ; ax = 0091H, CF = 1
\end{AsmCodeListing}

\subsection{Uso de los desplazamientos}

Los usos m�s comunes de las operaciones de desplazamientos son las
multiplicaciones y divisiones r�pidas. Recuerde que en el sistema decimal
la multiplicaci�n y divisi�n por una potencia de 10 es s�lo un
desplazamiento de los d�gitos. Lo mismo se aplica para las potencias de
dos en binario. Por ejemplo para duplicar el n�mero binario $1011_2$ (o
11 en decimal), al desplazar una vez a la izquierda obtenemos $10110_2$
(o 22). El cociente de una divisi�n por una potencia de dos es el
resultado de un desplazamiento a la derecha. Para dividir por 2 solo haga
un desplazamiento a la derecha; para dividir por 4 ($2^2$) desplace los
bits 2 lugares; para dividir por 8 desplace 3 lugares a la derecha etc.
Las instrucciones de desplazamiento son muy elementales y son
\emph{mucho} m�s r�pidas que las instrucciones correspondientes {\code
MUL}\index{MUL} y {\code DIV} \index{DIV}.

Los desplazamientos l�gicos se pueden usar para multiplicar y dividir
valores sin signo. Ellos no funcionan, en general, para valores con
signo.  Considere el valor de dos bytes FFFF ($-1$). Si �ste se desplaza
l�gicamente a la derecha una vez �el resultado es 7FFF que es $+32,767$!
Se pueden usar otro tipo de desplazamientos para valores con signo.
\index{operaciones con bits!desplazamientos!desplazamiento l�gicos|)}

\subsection{Desplazamientos aritm�ticos\index{operaciones con
bits!desplazamientos!desplazamientos aritm�ticos|(}}

Estos desplazamientos est�n dise�ados para permitir que n�meros con signo
se puedan multiplicar y dividir r�pidamente por potencias de 2. Ellos
aseguran que el bit de signo se trate correctamente.
\begin{description}
\item[SAL] \index{SAL} 
(\emph{Shift aritmetic left}).  Esta instrucci�n es solo sin�nimo para
{\code SHL}. Se ensambla con el mismo c�digo de m�quina que SHL. Como el
bit de signo no se cambia por el desplazamiento, el resultado ser�
correcto.  SAR \item[SAR] \index{SAR} (\emph{Shift Arithmetic Right}).
Esta es una instrucci�n nueva que no desplaza el bit de signo (el bit m�s
significativo) de su operando. Los otros bits se desplazan como es normal
excepto que los bits nuevos que entran por la derecha son copias del bit
de signo (esto es, si el bit de signo es 1, los nuevos bits son tambi�n
1). As�, si un byte se desplaza con esta instrucci�n, s�lo los 7 bits
inferiores se desplazan. Como las otras instrucciones de desplazamiento,
el �ltimo bit que sale se almacena en la bandera de carry.
\end{description}

\begin{AsmCodeListing}[frame=none]
      mov    ax, 0C123H
      sal    ax, 1           ; ax = 8246H, CF = 1
      sal    ax, 1           ; ax = 048CH, CF = 1
      sar    ax, 2           ; ax = 0123H, CF = 0
\end{AsmCodeListing}
\index{operaciones con bits!desplazamientos!desplazamientos aritm�ticos|)}

\subsection{Desplazamientos de rotaci�n\index{operaciones con
bits!desplazamientos!rotaciones|(}}

Los desplazamientos de rotaci�n trabajan como los desplazamientos l�gicos
excepto que los bits perdidos en un extremo del dato se desplazan al otro
lado. As�, el dato es tratado como si fuera una estructura circular. Las
dos rotaciones m�s simples son  {\code ROL} \index{ROL} y {\code ROR}
\index{ROR}, que hacen rotaciones a la izquierda y a  la derecha
respectivamente. Tal como los otros desplazamientos, estos
desplazamientos dejan una copia del �ltimo bit rotado en la bandera de
carry.
\begin{AsmCodeListing}[frame=none]
      mov    ax, 0C123H
      rol    ax, 1           ; ax = 8247H, CF = 1
      rol    ax, 1           ; ax = 048FH, CF = 1
      rol    ax, 1           ; ax = 091EH, CF = 0
      ror    ax, 2           ; ax = 8247H, CF = 1
      ror    ax, 1           ; ax = C123H, CF = 1
\end{AsmCodeListing}

Hay dos instrucciones de rotaci�n adicionales que desplazan los bits en
el dato y la bandera de carry llamadas {\code RCL}\index{RCL} y {\code
RCR}.  \index{RCR}. Por ejemplo, si el registro {\code AX} rota con estas
instrucciones, los 17 bits se desplazan y la bandera de carry se rota.
\begin{AsmCodeListing}[frame=none]
      mov    ax, 0C123H
      clc                    ; borra la bandera de carry (CF = 0)
      rcl    ax, 1           ; ax = 8246H, CF = 1
      rcl    ax, 1           ; ax = 048DH, CF = 1
      rcl    ax, 1           ; ax = 091BH, CF = 0
      rcr    ax, 2           ; ax = 8246H, CF = 1
      rcr    ax, 1           ; ax = C123H, CF = 0
\end{AsmCodeListing}
\index{operaciones con bits!desplazamientos!rotaciones|)}

\subsection{Aplicaci�n simple\label{sect:AddBitsExample}}

A continuaci�n est� un fragmento de  c�digo que cuenta el n�mero de bits
que est�n ``encendidos'' (1) en el registro EAX. 
%TODO: show how the ADC instruction could be used to remove the jnc
\begin{AsmCodeListing}
      mov    bl, 0           ; bl contendr� el n�mero de bits prendidos
      mov    ecx, 32         ; ecx es el contador del bucle
count_loop:
      shl    eax, 1          ; desplaza los bits en la bandera de carry
      jnc    skip_inc        ; si CF == 0, va a skip_inc
      inc    bl
skip_inc:
      loop   count_loop
\end{AsmCodeListing}
El c�digo anterior destruye el valor original de {\code EAX} ({\code EAX}
es cero al final del bucle). Si uno desea conservar el valor de EAX, la
l�nea~4 deber�a ser reemplazada con {\code rol eax,1}.
\index{operaciones con bits!desplazamientos|)}

\section{Operaciones booleanas entre bits}

Hay cuatro operadores booleanos b�sicos \emph{AND}, \emph{OR}, \emph{XOR}
y \emph{NOT}. Una \emph{tabla de verdad} muestra el resultado de cada
operaci�n por cada posible valor de los operandos.

\subsection{La operaci�n \emph{AND} \index{operaciones con bits!AND}}

\begin{table}[t]
\centering
\begin{tabular}{|c|c|c|}
\hline
\emph{X} & \emph{Y} & \emph{X} AND \emph{Y} \\
\hline \hline
0 & 0 & 0 \\
0 & 1 & 0 \\
1 & 0 & 0 \\
1 & 1 & 1 \\
\hline
\end{tabular}
\caption{La operaci�n AND \label{tab:and} \index{AND}}
\end{table}

El resultado del \emph{AND} de dos bits es 1 solo si ambos bits son 1, si
no el resultado es cero como muestra el Cuadro~\ref{tab:and}

\begin{figure}[t]
\centering
\begin{tabular}{rcccccccc}
    & 1 & 0 & 1 & 0 & 1 & 0 & 1 & 0 \\
AND & 1 & 1 & 0 & 0 & 1 & 0 & 0 & 1 \\
\hline
    & 1 & 0 & 0 & 0 & 1 & 0 & 0 & 0
\end{tabular}
\caption{ANDdo un byte \label{fig:and}}
\end{figure}

Los procesadores tienen estas operaciones como instrucciones que act�an
independientemente en todos los bits del dato en paralelo. Por ejemplo,
si el contenido de {\code AL} y {\code BL} se les opera con \emph{AND},
la operaci�n se aplica a cada uno de los 8 pares de bits correspondientes
en los dos registros como muestra la Figura~\ref{fig:and}. A continuaci�n
un c�digo de ejemplo:
\begin{AsmCodeListing}[frame=none]
      mov    ax, 0C123H
      and    ax, 82F6H          ; ax = 8022H
\end{AsmCodeListing}

\subsection{La operaci�n \emph{OR} \index{operaciones con bits!OR}}

\begin{table}[t]
\centering
\begin{tabular}{|c|c|c|}
\hline
\emph{X} & \emph{Y} & \emph{X} OR \emph{Y} \\
\hline \hline
0 & 0 & 0 \\
0 & 1 & 1 \\
1 & 0 & 1 \\
1 & 1 & 1 \\
\hline
\end{tabular}
\caption{La operaci�n OR\label{tab:or} \index{OR}}
\end{table}

El \emph{O} inclusivo entre dos bits es 0 solo si ambos bits son 0, si no
el resultado es 1 como se muestra  en el Cuadro~\ref{tab:or} . A
continuaci�n un c�digo de ejemplo:

\begin{AsmCodeListing}[frame=none]
      mov    ax, 0C123H
      or     ax, 0E831H          ; ax = E933H
\end{AsmCodeListing}

\subsection{La operaci�n \emph{XOR} \index{operaciones con bits!XOR}}

\begin{table}
\centering
\begin{tabular}{|c|c|c|}
\hline
\emph{X} & \emph{Y} & \emph{X} XOR \emph{Y} \\
\hline \hline
0 & 0 & 0 \\
0 & 1 & 1 \\
1 & 0 & 1 \\
1 & 1 & 0 \\
\hline
\end{tabular}
\caption{La operaci�n XOR \label{tab:xor}\index{XOR}}
\end{table}

El \emph{O} exclusivo entre 2 bits es 0 si y solo si ambos bits son
iguales, sino el resultado es 1 como muestra el Cuadro~\ref{tab:xor}.
Sigue un c�digo de ejemplo:

\begin{AsmCodeListing}[frame=none]
      mov    ax, 0C123H
      xor    ax, 0E831H          ; ax = 2912H
\end{AsmCodeListing}

\subsection{La operaci�n \emph{NOT} \index{operciones con bits!NOT}}

\begin{table}[t]
\centering
\begin{tabular}{|c|c|}
\hline
\emph{X} & NOT \emph{X} \\
\hline \hline
0 & 1 \\
1 & 0 \\
\hline
\end{tabular}
\caption{La operaci�n NOT \label{tab:not}\index{NOT}}
\end{table}

La operaci�n \emph{NOT} es \emph{unaria} (act�a sobre un solo operando,
no como las operaciones \emph{binarias} como \emph{AND}).  El \emph{NOT}
de un bit es el valor opuesto del bit como se muestra en el
Cuadro~\ref{tab:not}. Sigue un c�digo de ejemplo:

\begin{AsmCodeListing}[frame=none]
      mov    ax, 0C123H
      not    ax                 ; ax = 3EDCH
\end{AsmCodeListing}

Observe que \emph{NOT} halla el complemento a 1. A diferencia de las
otras operaciones entre bits, la instrucci�n {\code NOT} no cambian
ninguno de los bits en el registro {\code FLAGS}.

\subsection{La instrucci�n {\code TEST} \index{TEST}}

La instrucci�n {\code TEST} realiza una operaci�n \emph{AND}, pero no
almacena el resultado. Solo fija las banderas del registro {\code FLAGS}
dependiendo del resultado obtenido (muy parecido a lo que hace la
instrucci�n {\code CMP} con la resta que solo fija las banderas). Por
ejemplo, si el resultado fuera cero, {\code ZF} se fijar�a.

\begin{table}
\begin{tabular}{lp{3in}}
Prende el bit \emph{i} & \emph{OR} el n�mero con $2^i$ (es el n�mero con
�nicamente el bit \emph{i}-�simo prendido) \\
Apaga el bit \emph{i} & \emph{AND} el n�mero binario con s�lo el bit
\emph{i} apagado . Este operando es a menudo llamado \emph{m�scara} \\
Complementa el bit \emph{i} & \emph{XOR} el n�mero con $2^i$
\end{tabular}
\caption{Usos de las operaciones booleanas \label{tab:bool}}
\end{table}

\subsection{Usos de las operaciones con bits\index{operaciones con bits!ensamblador|(}}

Las operaciones con bits son muy �tiles para manipular bits individuales
sin modificar los otros bits. El Cuadro~\ref{tab:bool}  muestra los 3
usos m�s comunes de estas operaciones. Sigue un ejemplo de c�mo
implementar estas ideas.
\begin{AsmCodeListing}[frame=none]
      mov    ax, 0C123H
      or     ax, 8           ; prende el bit 3,   ax = C12BH
      and    ax, 0FFDFH      ; apaga el bit 5,  ax = C10BH
      xor    ax, 8000H       ; invierte el bit 31,   ax = 410BH
      or     ax, 0F00H       ; prende el nibble,  ax = 4F0BH
      and    ax, 0FFF0H      ; apaga nibble, ax = 4F00H
      xor    ax, 0F00FH      ; invierte nibbles,  ax = BF0FH
      xor    ax, 0FFFFH      ; complemento a uno ,  ax = 40F0H
\end{AsmCodeListing}

La operaci�n \emph{AND} se puede usar tambi�n para hallar el residuo de
una divisi�n por una potencia de dos. Para encontrar el residuo de una
divisi�n  por $2^i$, efect�a un AND entre el dividendo y una m�scara
igual a $2^i-1$,\emph{AND} el n�mero con una m�scara igual a $2^i - 1$.
Esta m�scara contendr� unos desde el bit 0 hasta el bit $i-1$. Son solo
estos bits los que contienen el residuo. El resultado de la \emph{AND}
conservar� estos bits y dejar� cero los otros. A continuaci�n un
fragmento de  c�digo que encuentra el cociente y el residuo de la
divisi�n de 100 por 16.
\begin{AsmCodeListing}[frame=none]
      mov    eax, 100        ; 100 = 64H
      mov    ebx, 0000000FH  ; m�cara = 16 - 1 = 15 or F
      and    ebx, eax        ; ebx = residuo = 4
      shr    eax, 4          ; eax = cociente de eax/2^4 = 6
\end{AsmCodeListing}
Usando el registro {\code CL} es posible modificar arbitrariamente bits.
El siguiente es un ejemplo que fija (prende) un bit arbitrario en {\code
EAX}. El n�mero del bit a prender se almacena en {\code BH}.
\begin{AsmCodeListing}[frame=none]
      mov    cl, bh          ; 
      mov    ebx, 1
      shl    ebx, cl         ; se desplaza a la derecha cl veces
      or     eax, ebx        ; prende el bit
\end{AsmCodeListing}
Apagar un bit es solo un poco m�s dif�cil.
\begin{AsmCodeListing}[frame=none]
      mov    cl, bh          ; 
      mov    ebx, 1
      shl    ebx, cl         ; se desplaza a la derecha cl veces
      not    ebx             ; invierte los bits
      and    eax, ebx        ; apaga el bit
\end{AsmCodeListing}
El c�digo para complementar un bit arbitrario es dejado como ejercicio al 
lector.

Es com�n ver esta instrucci�n en un programa 80x86.
\begin{AsmCodeListing}[frame=none,numbers=none]
      xor    eax, eax         ; eax = 0
\end{AsmCodeListing}
Un n�mero XOR con sigo mismo, el resultado es siempre cero. Esta
instrucci�n se usa porque su c�digo de m�quina es m�s peque�o que la
instrucci�n MOV equivalente. 

\index{operaciones con bits!ensamblador|)}

\begin{figure}[t]
\begin{AsmCodeListing}
      mov    bl, 0           ; bl contendr� el n�mero de bits prendidos
      mov    ecx, 32         ; ecx es el bucle contador
count_loop:
      shl    eax, 1          ; se desplaza el bit en la bandera de carry
      adc    bl, 0           ; a�ade solo la bandera de carry a bl
      loop   count_loop
\end{AsmCodeListing}
\caption{Contando bits con {\code ADC}\label{fig:countBitsAdc}}
\end{figure}

\section{Evitando saltos condicionales}
\index{predicci�n de ramificaciones|(} 

Los procesadores modernos usan t�cnicas muy sofisticadas para ejecutar el
c�digo  tan r�pido como sea posible. Una t�cnica com�n se conoce como
\emph{ejecuci�n especulativa}\index{ejecuci�n especulativa}.  Esta
t�cnica usa las capacidades de procesamiento paralelo  de la CPU para
ejecutar m�ltiples instrucciones a la vez. Las instrucciones
condicionales tienen un problema con esta idea. El procesador, en
general, no sabe si se realiza o no la ramificaci�n.Si se efect�a, se
ejecutar� un conjunto de instrucciones diferentes que si no se efect�a
(el salto). El procesador trata de predecir si ocurrir� la ramificaci�n o
no. Si la predicci�n fue err�nea el procesador ha perdido su tiempo
ejecutando un c�digo err�neo.

\index{predicci�n de ramificaciones|)}

Una manera de evitar este problema, es evitar usar ramificaciones
condicionales cuando es posible. El c�digo de ejemplo en
\ref{sect:AddBitsExample} muestra un programa muy simple donde uno podr�a
hacer esto. En el ejemplo anterior,  los bits ``encendidos'' del registro
EAX se cuentan. Usa una ramificaci�n para saltarse la instrucci�n {\code
INC}. La figura~\ref{fig:countBitsAdc} muestra c�mo se puede quitar la
ramificaci�n usando la instrucci�n {\code ADC}\index{ADC} para sumar el
bit de carry directamente.

Las instrucciones {\code SET\emph{xx}}\index{SET\emph{xx}} suministran
una manera de suprimir ramificaciones en ciertos casos. Esta instrucci�n
fija el valor de un registro o un lugar de memoria de 8 bits  a cero,
basado en el estudio del registro FLAGS. Los caracteres luego de {\code
SET} son los mismos caracteres usados en los saltos condicionales. Si la
condici�n correspondiente de {\code SET\emph{xx}} es verdadero, el
resultado almacenado es uno, si es falso se almacena cero. Por ejemplo,
\begin{AsmCodeListing}[frame=none,numbers=none]
      setz   al        ; AL = 1 if Z flag is set, else 0
\end{AsmCodeListing}
Usando estas instrucciones, uno puede desarrollar algunas t�cnicas
ingeniosas que calculan valores sin ramificaciones.

Por ejemplo, considere el problema de encontrar el mayor de dos valores.
La aproximaci�n normal para resolver este problema ser�a el uso de {\code
CMP} y usar un salto condicional y proceder con el valor que fue m�s
grande. El programa de ejemplo de abajo muestra c�mo se puede encontrar
el mayor sin ninguna ramificaci�n.


\begin{AsmCodeListing}
; Archivo: max.asm
%include "asm_io.inc"
segment .data

message1 db "Digite un n�mero: ",0
message2 db "Digite otro n�mero: ", 0
message3 db "El mayor n�mero es: ", 0

segment .bss

input1  resd    1        ; primer n�mero ingresado 

segment .text
        global  _asm_main
_asm_main:
        enter   0,0               ; 
        pusha

        mov     eax, message1     ; imprime el primer mensaje
        call    print_string
        call    read_int          ; ingresa el primer n�mero
        mov     [input1], eax

        mov     eax, message2     ; imprime el segundo mensaje
        call    print_string
        call    read_int          ; ingresa el segundo n�mero (en  eax)

        xor     ebx, ebx          ; ebx = 0
        cmp     eax, [input1]     ; compara el segundo y el primer n�mero
        setg    bl                ; ebx = (input2 > input1) ?          1 : 0
        neg     ebx               ; ebx = (input2 > input1) ? 0xFFFFFFFF : 0
        mov     ecx, ebx          ; ecx = (input2 > input1) ? 0xFFFFFFFF : 0
        and     ecx, eax          ; ecx = (input2 > input1) ?     input2 : 0
        not     ebx               ; ebx = (input2 > input1) ?          0 : 0xFFFFFFFF
        and     ebx, [input1]     ; ebx = (input2 > input1) ?          0 : input1
        or      ecx, ebx          ; ecx = (input2 > input1) ?     input2 : input1

        mov     eax, message3     ; imprime los resultado
        call    print_string
        mov     eax, ecx
        call    print_int
        call    print_nl

        popa
        mov     eax, 0            ; retorna a C
        leave                     
        ret
\end{AsmCodeListing}

El truco es crear una m�scara de bits que se pueda usar para seleccionar
el valor mayor.  La instrucci�n {\code SETG}\index{SETG} en la l�nea 30
fija BL a 1. Si la segunda entrada es mayor o 0 en otro caso. Esta no es
la m�scara deseada. Para crear la m�scara de bits requerida la l�nea 31
usa la instrucci�n {\code NEG}\index{NEG} en el registro EBX. (Observe
que se borr� EBX primero). Si EBX es 0 no hace nada; sin embargo si EBX
es 1, el resultado es la representaci�n en complemento a dos de -1 o
0xFFFFFFFF. Esta es la m�scara que se necesita.  El resto del c�digo usa
esta m�scara para seleccionar la entrada correcta como e la mayor.

Un truco alternativo es usar la instrucci�n {\code DEC}. En el c�digo de
arriba, si NEG se reemplaza con un {\code DEC}, de nuevo el resultado
ser� 0 o 0xFFFFFFFF.  Sin embargo, los valores son invertidos que cuando
se usa la instrucci�n {\code NEG}.

\section{Manipulando bits en C\index{operaciones con bits!C|(}}

\subsection{Las operacones entre bits de C}

A diferencia de algunos lenguajes de alto nivel C suministra operadores
para operaciones entre bits. La operaci�n {\code AND} se representa con
el operador {\code \&}\footnote{�Este operador es diferente del operador
binario \&\&  y del unario \&!}.  La operaci�n \emph{OR} se representa
por el operador binario {\code |}.  La operaci�n \emph{XOR} se representa
con el operador binario {\code \verb|^| }. Y la operaci�n \emph{NOT} se
representa con el operador unario {\code \verb|~| }.

Los desplazamientos son realizados por C con los operadores binarios
{\code \verb|<<| } y {\code \verb|>>| }. El operador {\code \verb|<<| }
realiza desplazamientos a la izquierda y el operador {\code \verb|>>| }
hace desplazamientos a la derecha. Estos operadores toman 2 operandos. El
de la derecha es el valor a desplazar y el de la izquierda es el n�mero
de bits a desplazar. Si el valor a desplazar es un tipo sin signo, se
realiza un desplazamiento l�gico. Si el valor es con signo (como {\code
int}), entonces se usa un desplazamiento aritm�tico a continuaci�n, un
ejemplo en C del uso de estos operadores:
\begin{lstlisting}{}
short int s;          /* se asume que short int es de 16 bits */
short unsigned u;
s = -1;               /* s = 0xFFFF (complemento a dos) */
u = 100;              /* u = 0x0064 */
u = u | 0x0100;       /* u = 0x0164 */
s = s & 0xFFF0;       /* s = 0xFFF0 */
s = s ^ u;            /* s = 0xFE94 */
u = u << 3;           /* u = 0x0B20 (desplazamiento l�gico) */
s = s >> 2;           /* s = 0xFFA5 (desplazamiento aritm�tico) */
\end{lstlisting}

\subsection{Usando las operaciones entre bits en C}

Los operadores entre bits se usan en C para los mismos prop�sitos que en
lenguaje ensamblador. Ellos le permiten a uno manipular bits individuales
y se pueden usar para multiplicaciones y divisiones r�pidas. De hecho, un
compilador de C inteligente usar� desplazamientos autom�ticamente para
multiplicaciones como {\code X*=2},  
\begin{table}
\centering
\begin{tabular}{|c|l|}
\hline
Macro & \multicolumn{1}{c|}{Meaning} \\
\hline \hline
{\code S\_IRUSR} & el propietario puede leer \\
{\code S\_IWUSR} & el propietario puede escribir \\
{\code S\_IXUSR} & el propietario puede ejecutar \\
\hline
{\code S\_IRGRP} & el grupo de propietario puede leer \\
{\code S\_IWGRP} & el grupo del propietario puede escribir \\
{\code S\_IXGRP} & el grupo del propietario puede ejecutar\\
\hline
{\code S\_IROTH} & los otros pueden leer \\
{\code S\_IWOTH} & los otros pueden escribir \\
{\code S\_IXOTH} & los otros pueden ejecutar \\
\hline
\end{tabular}
\caption{Macros POSIX para permisos de archivos \label{tab:posix}}
\end{table}
Muchos API\footnote{Aplication Programming Interface} (Como \emph{POSIX}
\footnote{Significa Portable Operatting System Interface for Computer
Enviroments. Una norma desarrollada por el IEEE basado en UNIX.} y Win
32).  tienen funciones que usan operandos que tienen datos codificados
como bits. Por ejemplo, los sistemas POSIX mantienen los permisos de los
archivos para 3 tipos diferentes de usuarios (un mejor nombre ser�a
\emph{propietario}), \emph{grupo} y \emph{otros}.  A cada tipo de usuario
se le puede  conceder permisos para leer, escribir o ejecutar un archivo.
Para cambiar los permisos de un archivo requiere que el programador de C
manipule bits individuales. POSIX define varios macros para ayudar (vea
el Cuadro~\ref{tab:posix}).  La funci�n {\code chmod} se puede usar para
establecer los permisos de un archivo.  Esta funci�n toma dos par�metros,
una cadena con el nombre del archivo sobre el que se va a actuar y un
entero\footnote{Actualmente un par�metro de tipo {\code mode\_t} que es
un typedef a un tipo integral.} Con los bits apropiados  para los
permisos deseados. Por ejemplo, el c�digo de abajo fija los permisos para
permitir que el propietario del archivo leerlo y escribirlo, a los
usuarios en e, grupo leer en archivo y que los otros no tengan acceso.
\begin{lstlisting}[stepnumber=0]{}
chmod("foo", S_IRUSR | S_IWUSR | S_IRGRP );
\end{lstlisting}

La funci�n POSIX {\code stat} se puede usar para encontrar los bits de
permisos actuales de un archivo. Usada con la funci�n {\code chmod}, es
posible modificar algunos de los permisos sin cambiar los otros. A
continuaci�n un ejemplo que quita el acceso de la escritura a los otros y
a�ade el accesode lectura. Los otros permisos no son alterados Los otros
permisos no se alteran.
\begin{lstlisting}{}
struct stat file_stats;    /* estructura usada por stat() */
stat("foo", &file_stats);  /* lee la informaci�n del archivo. 
                              file_stats.st_mode holds permission bits */
chmod("foo", (file_stats.st_mode & ~S_IWOTH) | S_IRUSR);
\end{lstlisting}
\index{operaciones con bits!C|)}

\section{Representaciones Littel Endian y Big Endian\index{endianess|(}}

El Cap�tulo~1 introdujo el concepto de las representaciones big y little
endian de datos multibyte. Sin embargo, el autor ha encontrad que este
tema confunde a muchas personas. Esta secci�n cubre el t�pico con m�s
detalle.

El lector recordar� que  lo endian se refiere al orden en que los bytes
individuales se almacenan en memoria (\emph{no} bits) de un elemento
multibyte se almacena en memoria. Big endian es el m�todo m�s directo.
Almacena el byte m�s significativo primero, luego el siguiente byte en
peso y as� sucesivamente. En otras palabras los bits de \emph{m�s peso}
se almacenan primero. Little endian almacena los bytes en el orden
contrario (primero el menos significativo). La familia de procesadores
X86 usa la representaci�n little endian.

Como un ejemplo, considere la palabra doble $12345678_{16}$. En la
representaci�n big endian, los bytes se almacenar�an como 12~34~56~78. En
la representaci�n little endian los bytes se almacenar�an como
78~56~34~12.

El lector probablemente se preguntar� as� mismo, �por qu� cualquier
dise�ador sensato de circuitos integrados usar�a la representaci�n little
endian? �Eran los ingenieros de Intel s�dicos para infligir esta confusa
representaci�n a multitud de programadores? Parecer�a que la CPU no tiene
que hacer trabajo extra para almacenar los bytes hacia atr�s en la
memoria (e invertirlos cuando los lee de la memoria). La respuesta es que
la CPU no hace ning�n trabajo extra para leer y escribir en la memoria
usando el formato little endian. Uno sabe que la CPU est� compuesta de
muchos circuitos electr�nicos que simplemente trabajan con bits. Los bits
(y bytes) no est�n en un orden en particular en la CPU.

Considere el registro de 2 bytes {\code AX}. Se puede descomponer en
registros de un byte ({\code AH} y {\code AL}). Hay circuitos en la CPU
que mantienen los valores de {\code AH} y {\code AL}. Los circuitos no
est�n en un orden particular en la CPU. Esto significa que los circuitos
para {\code AH} no est�n antes o despu�s que los circuitos para {\code
AL}.Una instrucci�n mov que y el valor de AX en memoria  el valor de Al y
luego AH Quiere decir que, no es dif�cil para la CPU hacer que almacene
{\code AH} primero.

\begin{figure}[t]
\begin{lstlisting}[stepnumber=0,frame=tblr]{}
  unsigned short word = 0x1234;   /* se asume sizeof(short) == 2 */
  unsigned char * p = (unsigned char *) &word;

  if ( p[0] == 0x12 )
    printf("M�quina Big Endian\n");
  else
    printf("M�quina Little Endian\n");
\end{lstlisting}
\caption{C�mo determinar lo Endianness \label{fig:determineEndian}}
\end{figure}

El mismo argumento se aplica a los bits individuales dentro de un byte,
no hay realmente ning�n orden en los circuitos de la CPU (o la memoria).
Sin embargo, ya que los bits individuales no se  pueden direccionar
directamente en la CPU o en la memoria, no hay manera de conocer que
orden  parece que conservaran internamente en la CPU.


El c�digo en C  en la Figura~\ref{fig:determineEndian} muestra c�mo se
puede determinar lo endian de una CPU. El apuntador  \lstinline|p| trata
la variable  \lstinline|word| como dos elementos de un arreglo de
caracteres. As�,  \lstinline|p[0]| eval�a el primer byte de
\lstinline|word| en la memoria que depende de lo endian en la CPU.

\subsection{Cuando tener cuidado con Little and Big Endian}

Para la programaci�n t�pica, lo endian de la CPU no es importante. La
mayor�a de las veces esto es importante cuando se transfiere datos
binarios entre sistemas de c�mputo diferente. Esto es ocurre normalmente
usando un medio de datos f�sico (como un disco) o una red.
\MarginNote{Ahora con los conjuntos de caracteres multibyte como UNICODE
\index{UNICODE}, lo endian es importante a�n para texto. UNICODE soporta
ambos tipos de representaci�n y tiene un mecanismo para especificar cu�l
se est� usando para representar los datos.} Ya que el c�digo ASCII es de
1 byte la caracter�stica endian no le es importante.

Todos los encabezados internos de TCP/IP almacena enteros en big Endian
(llamado \emph{orden de byte de la red}). Y las bibliotecas de TCP/IP
suministran funciones de C para tratar la cuesti�n endian de una manera
port�til. Por ejemplo la funci�n \lstinline|htonl()| convierte una
palabra doble ( o long int) del formato del \emph{host} al de \emph{red}.
La funci�n \lstinline|ntohl()| hace la transformaci�n
inversa.\footnote{Ahora, invertir lo endian de un entero simplemente
coloca al rev�s los bytes; as� convertir de big a little o de little a
big es la misma operaci�n. As�, ambas funciones hacen la misma cosa.}
Para un sistema big endian, las dos funciones s�lo retornan su entrada
sin cambio alguno. Esto le permite a uno escribir programas  de red que
compilar�n y se ejecutar�n correctamente en cualquier sistema sin
importar lo endian. Para m�s informaci�n sobre lo endian  programaci�n de
redes vea el excelente libro de W. Richard Steven \emph{UNIX Network
Programming}.

\begin{figure}[t]
\begin{lstlisting}[frame=tlrb]{}
unsigned invert_endian( unsigned x )
{
  unsigned invert;
  const unsigned char * xp = (const unsigned char *) &x;
  unsigned char * ip = (unsigned char *) & invert;

  ip[0] = xp[3];   /* invierte los bytes individuales */
  ip[1] = xp[2];
  ip[2] = xp[1];
  ip[3] = xp[0];

  return invert;   /* retorna los bytes invertidos */
}
\end{lstlisting}
\caption{Funci�n invert\_endian \label{fig:invertEndian}\index{endianess!invert\_endian}}
\end{figure}

La Figura~\ref{fig:invertEndian} muestra una funci�n de C que invierte lo
endian de una palabra doble. El 486 ha introducido una nueva instrucci�n
de m�quina llamada {\code BSWAP}\index{BSWAP} que invierte los bytes de
cualquier registro de 32 bits. Por ejemplo:
\begin{AsmCodeListing}[frame=none,numbers=none]
      bswap   edx          ; intercambia los bytes de edx
\end{AsmCodeListing}
Esta instrucci�n no se puede usar en los registros de 16 bits, sin
embargo la instrucci�n {\code XCHG}\index{XCHG} se puede usar para
intercambiar los bytes en los registros de 16 bits que se pueden
descomponer en registros de 8 bits. Por ejemplo:
\begin{AsmCodeListing}[frame=none,numbers=none]
      xchg    ah,al        ; intercambia los bytes de ax
\end{AsmCodeListing}
\index{endianess|)}

\section{Contando bits\index{contando bits|(}}

Al principio se dio una t�cnica directa para contar el n�mero de bits que
est�n ``encendidos'' en una palabra doble. Esta secci�n mira otros
m�todos menos directos de hacer esto, como un ejercicio que usa las
operaciones de bits discutidas en este cap�tulo.

\begin{figure}[t]
\begin{lstlisting}[frame=tblr]{}
int count_bits( unsigned int data )
{
  int cnt = 0;

  while( data != 0 ) {
    data = data & (data - 1);
    cnt++;
  }
  return cnt;
}
\end{lstlisting}
\caption{Contando bits: m�todo uno \label{fig:meth1}}
\end{figure}

\subsection{M�todo uno\index{contando bits!m�todo uno|(}}

El primer m�todo es muy simple, pero no obvio. La figura~\ref{fig:meth1}
muestra el c�digo.

�C�mo trabaja este m�todo? En cada iteraci�n el bucle, se apaga un bit de
data {\code dato}. Cuando todos los bits se han apagado (cuando el {\code
dato} es cero), el bucle finaliza. El n�mero de iteraciones requerido
para hacer el {\code dato} cero es igual al n�mero de bits en el valor
original de data {\code data}.

La l�nea~6 es donde se apaga un bit del {\code dato}. �C�mo se hace esto?
Considere la forma general de la representaci�n en binario del {\code
dato} y el 1 del extremo derecho de esta representaci�n. Por definici�n
cada bit despu�s de este 1 debe ser cero. Ahora, �Cu�l ser�  la
representaci�n de {\code data -1}? Los bits a la izquierda del 1 del
extremo derecho ser�n los mismos que para {\code data}, pero en el punto
del 1 del extremo derecho ellos ser�n el  complemento de los bits
originales de {\code data}. Por ejemplo:\\
\begin{tabular}{lcl}
{\code data}     & = & xxxxx10000 \\
{\code data - 1} & = & xxxxx01111
\end{tabular}\\
donde X es igual para ambos n�meros. Cuando se hace {\code data}
\emph{AND} {\code data -1}, el resultado ser� cero el 1 del extremo
derecho en {\code data} y deja todos los otros bits sin cambio.

\begin{figure}[t]
\begin{lstlisting}[frame=tlrb]{}
static unsigned char byte_bit_count[256];  /* lookup table */

void initialize_count_bits()
{
  int cnt, i, data;

  for( i = 0; i < 256; i++ ) {
    cnt = 0;
    data = i;
    while( data != 0 ) {	/* m�todo uno */
      data = data & (data - 1);
      cnt++;
    }
    byte_bit_count[i] = cnt;
  }
}

int count_bits( unsigned int data )
{
  const unsigned char * byte = ( unsigned char *) & data;

  return byte_bit_count[byte[0]] + byte_bit_count[byte[1]] +
         byte_bit_count[byte[2]] + byte_bit_count[byte[3]];
}
\end{lstlisting}
\caption{M�todo dos \label{fig:meth2}}
\end{figure}
\index{contando bits!m�todo uno|)}

\subsection{M�todo dos\index{contando bits!m�todo dos|(}}

Una b�squeda en una tabla se puede usar para contar bits de una palabra
doble arbitraria. La aproximaci�n directa ser�a precalcular el n�mero de
bits para cada palabra doble y almacenar esto en un arreglo. Sin embargo,
hay dos problemas relacionados con esta aproximaci�n. �Hay alrededor de
\emph{4 mil millones} de palabras dobles! Esto significa que el arreglo
ser� muy grande e iniciarlo consumir�a mucho tiempo. (de hecho, a menos
que uno vaya a utilizar realmente el arreglo de m�s que 4 mil millones de
veces, se tomar� m�s tiempo iniciando el arreglo que el que se requerir�a
para calcular la cantidad de bits usando el m�todo uno).

Un m�todo m�s realista  ser�a precalcular la cantidad de bits para todos
los valores posibles de un byte y almacenar esto en un arreglo. Entonces
la palabra doble se puede dividir en 4 bytes. Se hallan los b y se suman
para encontrar la cantidad de bits de la palabra doble original. La
figura~\ref{fig:meth2} muestra la implementaci�n de esta aproximaci�n.

La funci�n {\code initialize\_count\_bits} debe ser llamada antes, del
primer llamado a la funci�n {\code count\_bits} . Esta funci�n inicia el
arreglo global {\code byte\_bit\_count}. La funci�n {\code count\_bits}
mira la variable {\code data} no como una palabra doble, sino como un
arreglo de 4 bytes. El apuntador {\code dword} act�a como un apuntador
a este arreglo de 4 bytes. As� {\code dword [0]} es uno de los bytes en
{\code data} ( el byte menos significativo o el m�s significativo
dependiendo si es little o big endian respectivamente). Claro est� uno
podr�a usar una instrucci�n como:
\begin{lstlisting}[stepnumber=0]{}
(data >> 24) & 0x000000FF
\end{lstlisting}
\noindent Para encontrar el byte m�s significativo y hacer algo parecido
con los otros bytes; sin embargo, estas construcciones ser�n m�s lentas
que una referencia al arreglo.

Un �ltimo punto, se podr�a usar f�cilmente un bucle {\code for} para
calcular la suma en las l�neas~22 y 23. Pero el bucle {\code for}
incluir�a el trabajo extra de iniciar el �ndice del bucle, comparar el
�ndice luego de cada iteraci�n e incrementar el �ndice. Calcular la suma
como la suma expl�cita de 4 valores ser� m�s r�pido. De hecho un
compilador inteligente podr�a convertir la versi�n del bucle {\code for}
a la suma expl�cita. Este proceso de reducir o eliminar iteraciones de
bucles es una t�cnica de optimizaci�n conocida como \emph{loop
unrolling}.
\index{contando bits!m�todo dos|)}

\subsection{M�todo tres\index{contando bits!m�todo tres|(}}

\begin{figure}[t]
\begin{lstlisting}[frame=tlrb]{}
int count_bits(unsigned int x )
{
  static unsigned int mask[] = { 0x55555555,
                                 0x33333333,
                                 0x0F0F0F0F,
                                 0x00FF00FF,
                                 0x0000FFFF };
  int i;
  int shift;   /* n�mero de posiciones a desplazarse a la derecha */

  for( i=0, shift=1; i < 5; i++, shift *= 2 )
    x = (x & mask[i]) + ( (x >> shift) & mask[i] );
  return x;
}
\end{lstlisting}
\caption{M�todo tres \label{fig:method3}}
\end{figure}

Hay otro m�todo ingenioso de contar bits que est�n en un dato. Este
m�todo literalmente a�ade los unos y ceros del dato unido. Esta suma debe
ser igual al n�mero de unos en el dato. Por ejemplo considere calcular
los unos en un byte almacenado en una variable llamada {\code data}. El
primer paso es hacer la siguiente operaci�n:
\begin{lstlisting}[stepnumber=0]{}
data = (data & 0x55) + ((data >> 1) & 0x55);
\end{lstlisting}
�Qu� hace esto? La constante hexadecimal {\code 0X55} es $01010101$ en
binario. En el primer operando de la suma {\code data} es \emph{AND} con
�l, los bits en las posiciones pares  se sacan. El Segundo operando
({\code data \verb|>>| 1 \& 0x55}), primero mueve todos los bits de
posiciones pares a impares y usa la misma m�scara para sacar estos mismos
bits. Ahora, el primer operando contiene los bits pares y el segundo los
bits impares de {\code data}. Cuando estos dos operandos se suman, se
suman los bits pares e impares de {\code data}. Por ejemplo si data es
$10110011_2$, entonces:\\
\begin{tabular}{rcr|l|l|l|l|}
\cline{4-7}
{\code data \&} $01010101_2$          &    &   & 00 & 01 & 00 & 01 \\
+ {\code (data \verb|>>| 1) \&} $01010101_2$ & or & + & 01 & 01 & 00 & 01 \\
\cline{1-1} \cline{3-7}
                                      &    &   & 01 & 10 & 00 & 10 \\
\cline{4-7}
\end{tabular}

La suma de la derecha muestra los bits sumados. Los bits del byte se
dividen en 4 campos de 2 bits para mostrar que se realizan 4 sumas
independientes. Ya que la mayor�a de estas sumas pueden ser dos, no hay
posibilidad de que la suma desborde este campo y da�e otro de los campos
de la suma.

Claro est�, el n�mero total de bits no se ha calculado a�n. Sin embargo
la misma t�cnica que se us� arriba se puede usar para calcular el total
en una serie de pasos similares. El siguiente paso podr�a ser:
\begin{lstlisting}[stepnumber=0]{}
data = (data & 0x33) + ((data >> 2) & 0x33);
\end{lstlisting}
Continuando con el ejemplo de arriba (recuerde que {\code data} es ahora
$01100010_2$):\\
\begin{tabular}{rcr|l|l|}
\cline{4-5}
{\code data \&} $00110011_2$          &    &   & 0010 & 0010 \\
+ {\code (data \verb|>>| 2) \&} $00110011_2$ & or & + & 0001 & 0000 \\
\cline{1-1} \cline{3-5}
                                      &    &   & 0011 & 0010 \\
\cline{4-5}
\end{tabular}\\
Ahora hay 2 campos de 4 bits que se suman independientemente.

El pr�ximo paso es sumar estas dos sumas unidas para conformar el
resultado final:
\begin{lstlisting}[stepnumber=0]{}
data = (data & 0x0F) + ((data >> 4) & 0x0F);
\end{lstlisting} 

Usando el ejemplo de arriba (con {\code data} igual a $00110010_2$):\\
\begin{tabular}{rcrl}
{\code data \&} $00001111_2$          &    &   & 00000010 \\
+ {\code (data \verb|>>| 4) \&} $00001111_2$ & or & + & 00000011 \\
\cline{1-1} \cline{3-4}
                                      &    &   & 00000101 \\
\end{tabular}\\
Ahora {\code data} es 5 que es el resultado correcto. La
Figura~\ref{fig:method3} muestra una implementaci�n de este m�todo que
cuenta los bits en una palabra doble.  Usa un bucle {\code for} para
calcular la suma. Podr�a ser m�s r�pido deshacer el bucle; sin embargo,
el bucle clarifica c�mo el m�todo generaliza a diferentes tama�os de
datos.
\index{contando bits!m�todo tres|)}
\index{contando bits|)}

%-*- latex -*-
\chapter{副程式}

本章主要著眼於使用副程式來構成模組化程式和得到與高階語言(比如說C)的介面。函式和進程是高階語言中副程式的例子。

調用了一個子程式的代碼和這個子程式必須協商它們之間的資料如何傳輸。資料如何傳輸的這些規則稱為\emph{調用約定}。\index{調用約定}這一章的很大一部分都是在討論使用在彙編副程式和C程式介面上的標準C調用約定。這個約定(和其他約定)通常都是通過傳遞資料的位址(\emph{也就是}指標)來允許副程式訪問記憶體中的資料。

\section{間接定址\index{間接定址|(}}

間接定址允許寄存器像指標變數一樣運作。要指出寄存器像一個指標一樣被間接使用,需要用方括號({\code []})將它括起來。例如:
\begin{AsmCodeListing}[frame=none]
      mov    ax, [Data]     ; 一個字的標準的直接記憶體位址
      mov    ebx, Data      ; ebx = & Data
      mov    ax, [ebx]      ; ax = *ebx
\end{AsmCodeListing}
因為AX可以容納一個字,所以第三行代碼從EBX儲存的位址開始讀取一個字。如果用AL替換AX,那麼只有一個位元組會被讀取。認識到寄存器不像在C中的變數一樣有類型是非常重要的。到底EBX具體指向什麼完全取決於使用了什麼指令。而且,甚至EBX是一個指標這個事實都完全取決於使用的指令。如果EBX錯誤地使用了,通常不會有編譯錯誤;但是,程式將不會正確運行。這就是為什麼相比於高階語言組合語言程式較容易犯錯的原因之一。

所有的32位通用寄存器(EAX, EBX, ECX, EDX)和指標寄存器(ESI, EDI)都可以用來間接定址。一般來說,16位或8位的寄存器是不可以的。
\index{間接定址|)}

\section{副程式的簡單例子\index{副程式|(}}

副程式是代碼中的一個的獨立的單元,它可以使用在程式的不同的地方。換句話說,一個子程式就像一個C中的函式。可以使用跳轉來調用副程式,但是返回會是一個問題。如果子程式要求能使用在程式中的任何地方,它必須要返回到調用它的代碼段處。因此,副程式的跳轉返回最好不要硬編碼為標號。下面的代碼展示了如何使用{\code
JMP}指令的間接方式來做這件事。此指令方式使用一個寄存器的值來決定跳轉到哪(因此,這個寄存器與C中的\emph{函式指標}非常相似。)
下面使用副程式的方法來重寫第一章中的第一個程式。
\begin{AsmCodeListing}[label=sub1.asm]
; file: sub1.asm
; 副程式的簡單例子
%include "asm_io.inc"

segment .data
prompt1 db    "Enter a number: ", 0       ; 不要忘記空結束符
prompt2 db    "Enter another number: ", 0
outmsg1 db    "You entered ", 0
outmsg2 db    " and ", 0
outmsg3 db    ", the sum of these is ", 0

segment .bss
input1  resd 1
input2  resd 1

segment .text
        global  _asm_main
_asm_main:
        enter   0,0               ; 程式開始運行
        pusha

        mov     eax, prompt1      ; 顯示提示資訊
        call    print_string

        mov     ebx, input1       ; 儲存input1的位址到ebx中
        mov     ecx, ret1         ; 儲存返回位址到ecx中
        jmp     short get_int     ; 讀整形
ret1:
        mov     eax, prompt2      ; 輸出提示資訊
        call    print_string

        mov     ebx, input2
        mov     ecx, $ + 7        ; ecx = 當前地址 + 7
        jmp     short get_int

        mov     eax, [input1]     ; eax = 在input1中的雙字
        add     eax, [input2]     ; eax += 在input2中的雙字
        mov     ebx, eax          ; ebx = eax

        mov     eax, outmsg1
        call    print_string      ; 輸出第一條資訊
        mov     eax, [input1]
        call    print_int         ; 輸出input1
        mov     eax, outmsg2
        call    print_string      ; 輸出第二條資訊
        mov     eax, [input2]
        call    print_int         ; 輸出input2
        mov     eax, outmsg3
        call    print_string      ; 輸出第三條資訊
        mov     eax, ebx
        call    print_int         ; 輸出總數(ebx)
        call    print_nl          ; 換行

        popa
        mov     eax, 0            ; 返回到C中
        leave
        ret
; 副程式 get_int
; 參數:
;   ebx - 儲存整形雙字的位址
;   ecx - 返回指令的位址
; 注意:
;   eax的值被已經被破壞掉了
get_int:
        call    read_int
        mov     [ebx], eax         ; 儲存輸入到記憶體中
        jmp     ecx                ; 返回到調用處
\end{AsmCodeListing}

副程式{\code get\_int}使用了一個簡單,基於寄存器的調用約定。它認為EBX寄存器中存的是輸入雙字的儲存位址而ECX寄存器中存的是跳轉返回指令的位址。25行到28行,使用了{\code ret1}標號來計算返回位址。在32行到34行,使用了{\code \$}運算子來計算返回的地址。{\code \$}運算子返回出現\$這一行的當前地址。{\code \$ + 7}運算式計算在36行的{\code MOV}指令的位址。

這兩種計算返回位址的方法都是不方便的。第一種方法要求為每一次子程式調用定義一個標號。第二種方法不需要標號,但是需要仔細的思量。如果使用了近跳轉來替代短跳轉,那麼與{\code \$}相加的數就不會是7!幸運的是,有一個更簡單的方法來調用副程式。這種方法使用\emph{堆疊}。

\section{堆疊\index{堆疊|(}}

許多CPU都支援內置堆疊。堆疊是一個先進後出(\emph{LIFO})的佇列。它是以這種方式組織的一塊記憶體區域。{\code
PUSH}指令添加一個資料到堆疊中而 {\code
POP}指令從堆疊中移除數據。移除的資料就是最後入堆疊的資料(這就是稱為先進後出佇列的緣故)。

SS段寄存器指定包含堆疊的段(通常它與儲存資料的段是一樣)。ESP寄存器包含將要移除出堆疊資料的位址。這個資料也被稱為堆疊頂。資料只能以雙字的形式入堆疊。也就是說,你不可以將一個位元組推入堆疊中。

{\code
PUSH}指令通過把ESP減4來向堆疊中插入一個雙字\footnote{實際上,多字入堆疊也是可以的,但是在32位元保護模式下,在堆疊上最好只操作單個雙字。},然後把雙字儲存到{\code
[ESP]}中。 {\code POP}指令從{\code
[ESP]}中讀取雙字,然後再把ESP加4.下面的代碼演示了這些指令如何工作,假定在ESP初始值為{\code
1000H}。
\begin{AsmCodeListing}[frame=none]
      push   dword 1    ; 1儲存到0FFCh中,ESP = 0FFCh
      push   dword 2    ; 2儲存到0FF8h中, ESP = 0FF8h
      push   dword 3    ; 3儲存到0FF4h中, ESP = 0FF4h
      pop    eax        ; EAX = 3, ESP = 0FF8h
      pop    ebx        ; EBX = 2, ESP = 0FFCh
      pop    ecx        ; ECX = 1, ESP = 1000h
\end{AsmCodeListing}

堆疊可以方便地用來臨時儲存資料。它同樣可以用來形成副程式調用和傳遞參數和局部變數。

80x86同樣提供一條{\code PUSHA}指令來把EAX, EBX, ECX, EDX, ESI,
EDI\\和EBP寄存器的值推入堆疊中(不是以這個順序)。 {\code
POPA}指令可以用來將它們移除出堆疊。 \index{堆疊|)}

\section{CALL和RET指令\index{副程式!調用|(}}
\index{CALL|(}
\index{RET|(}
80x86提供了兩條使用堆疊的指令來使副程式調用變得快速而簡單。CALL指令執行一個跳到副程式的無條件跳轉,同時將下一條指令的位址\emph{推入}堆疊中。RET指令\emph{彈出}一個位址並跳轉到這個位址去執行。使用這些指令的時候,正確處理堆疊以便RET指令能彈出正確的數值是非常重要的!

前面的例子可以使用這些新的指令來重寫。把25行到34行改成:
\begin{AsmCodeListing}[numbers=none]
      mov    ebx, input1
      call   get_int

      mov    ebx, input2
      call   get_int
\end{AsmCodeListing}
同時把副程式{\code get\_int}改成:
\begin{AsmCodeListing}[numbers=none]
get_int:
      call   read_int
      mov    [ebx], eax
      ret
\end{AsmCodeListing}

CALL和RET指令有幾個優點:
\begin{itemize}
\item 它們很簡單!
\item 它們使副程式嵌套變得簡單。注意:副程式
{\code get\_int}調用了{\code read\_int}。這個調用將另一個位址壓入到堆疊中了。在{\code read\_int}代碼的末尾是一條彈出返回位址的RET指令,通過執行指令重新回到{\code get\_int}代碼中去執行。然後,當
{\code get\_int}的RET指令被執行時,它彈出跳回到{\code asm\_main}的返回地址。這個之所以能正確運行,是因此堆疊的LIFO特性。
\end{itemize}

記住彈出壓入到堆疊的所有資料是非常重要的。例如,考慮下面的代碼:
\begin{AsmCodeListing}[frame=none]
get_int:
      call   read_int
      mov    [ebx], eax
      push   eax
      ret                  ; 彈出EAX的值,沒有返回地址!!
\end{AsmCodeListing}
這個代碼將不會正確返回!
\index{RET|)}
\index{CALL|)}

\section{調用約定\index{調用約定|(}}

當調用了一個子程式,調用的代碼和副程式(\emph{被調用的代碼})必須協商好在它們之間如何傳遞資料。高階語言有標準傳遞資料的方法稱為\emph{調用約定}。要讓高階語言介面於組合語言,組合語言代碼就一定要使用與高階語言一樣的約定。不同的編譯器有不同的調用約定或者說不同的約定可能取決於代碼如何被編譯。(\emph{例如:}是否進行了優化)。一個廣泛的約定是:使用一條{\code CALL}指令來調用代碼再通過{\code RET}指令返回。

所有PC的C編譯器支持的調用約定將在本章的後續部分階段進行描述。這些約定允許你創建\emph{可重入的}副程式。一個可重入的副程式可以在程式中的任意一點被安全調用(甚至在副程式內部)。

\subsection{在堆疊上傳遞參數\index{堆疊|(}\index{堆疊!參數|(}}

給副程式的參數需要在堆疊中傳遞。它們在{\code CALL}指令之前就已經被壓入堆疊中了。和在C中是一樣的,如果參數被副程式改變了,那麼需要傳遞的是資料的\emph{位址},而不是\emph{值}。如果參數的大小小於雙字,它就需要在壓入堆疊之前轉換成雙字。

在堆疊裏的參數並沒有由副程式彈出,取而代之的是:它們自己從堆疊中訪問本身。為什麼?考慮\MarginNote{當使用了間接定址,\\80x86通過看間接定\\址運算式裏使用了\\什麼寄存器來決定\\訪問哪不同的段。ESP\\(和EBP)使用堆疊段,\\而EAX,EBX,ECX和\\EDX使用資料段。但\\是,這個對於保護模式\\通常是不重要的,因為\\對於保護模式資料段和\\堆疊段是同一段。}
\begin{itemize}
\item 因為它們在{\code CALL}指令之前被壓入堆疊中,所以返回時首先彈出的是返回位址(然後修改堆疊指標使其指向參數入堆疊以前的值)。
\item 參數往往將會使用在副程式中幾個的地方。通常在整個程式中,它們不可以保存在一個寄存器中,而應該儲存在記憶體中。把它們留在堆疊裏就相當於把資料複製到了記憶體中,這樣就可以在副程式的任意一點訪問資料。
\end{itemize}

\begin{figure}
\centering
\begin{tabular}{l|c|}
\cline{2-2}
&  \\ \cline{2-2}
ESP + 4 & 參數 \\ \cline{2-2}
ESP &返回地址\\ \cline{2-2}
& \\ \cline{2-2}
\end{tabular}
\caption{}
\label{fig:stack1}
\end{figure}
通過堆疊傳遞了一個參數的副程式。當副程式被調用了,堆疊狀態如圖~\ref{fig:stack1}。這個參數可以通過間接定址訪問到。({\code
[ESP+4]}
\footnote{使用間接定址時,寄存器加上一個常量是合法的。許多更複雜的運算式也是合法的。這個話題將在下一章中介紹。})。
\begin{figure}
\centering
\begin{tabular}{l|c|}
\cline{2-2}
&  \\ \cline{2-2}
ESP + 8 & 參數 \\ \cline{2-2}
ESP + 4 & 返回地址 \\ \cline{2-2}
ESP     & 副程式資料 \\ \cline{2-2}
\end{tabular}
\caption{}
\label{fig:stack2}
\end{figure}

\begin{figure}[t]
\begin{AsmCodeListing}[frame=single]
subprogram_label:
      push   ebp           ; 把EBP的原始值保存在堆疊中
      mov    ebp, esp      ; 新EBP的值 = ESP
; subprogram code
      pop    ebp           ; 恢復EBP的原始值
      ret
\end{AsmCodeListing}
\caption{副程式的一般格式 \label{fig:subskel1}}
\end{figure}

如果在副程式內部使用了堆疊儲存資料,那麼與ESP相加的數將要改變。例如:
圖~\ref{fig:stack2}展示了如果一個雙字壓入堆疊中後堆疊的狀態。現在參數在{\code ESP + 8}中,而不在{\code
ESP + 4}中。因此,引用參數時若使用ESP就很容易犯錯了。為了解決這個問題,80386提供使用另外一個寄存器:EBP。這個寄存器的唯一目的就 是引用堆疊中的資料。C調用約定要求副程式首先把EBP的值保存到堆疊中,然後再使EBP的值等於ESP。當資料壓入或彈出堆疊時,這就允許ESP值被改變的同時EBP不會被改變。在副程式的結束處,EBP的原始值必須恢復出來
(這就是為什麼它在副程式的開始處被保存的緣故。)圖~\ref{fig:subskel1}展示了遵循這些約定的副程式的一般格式。

\begin{figure}[t]
\centering
\begin{tabular}{ll|c|}
\cline{3-3} &  & \\ \cline{3-3}
ESP + 8 & EBP + 8 & 參數 \\ \cline{3-3}
ESP + 4 & EBP + 4 & 返回地址 \\ \cline{3-3}
ESP     & EBP     & 保存的EBP值 \\ \cline{3-3}
\end{tabular}
\caption{}
\label{fig:stack3}
\end{figure}


圖~\ref{fig:subskel1}中的第2行和第3行組成了一個子程式的大體上的\emph{開始部分}。第5行和第6行組成了\emph{結束部分}。圖~\ref{fig:stack3}展示了剛執行完開始部分之後堆疊的狀態。現在參數可以在副程式中的任何地方通過{\code [EBP + 8]}來訪問,而不用擔心在副程式中有什麼資料壓入到堆疊中了。

執行完副程式之後,壓入堆疊中的參數必須移除掉。C調用約定\index{調用約定!C}規定調用的代碼必須做這件事。其他約定可能不同。例如:Pascal 調用約定
\index{調用約定!Pascal}規定副程式必須移除參數。(RET\index{RET}指令的另一種格式可以很容易做這件事。)一些C編譯器同樣支持這種約定。關鍵字{\code pascal}用在函式的原型和定義中來告訴編譯器使用這種約定。事實上,MS Windows API的C函式使用的{\code stdcall}調用約定\index{調用約定!stdcall}同樣以這種方式運作。這種方式有什麼優點?它比C調用約定更有效一點。那為什麼所有的C函式都使用C調用約定呢?一般說來,C允許一個函式的參數為變化的個數(\emph{例如}:{\code printf}和{\code scanf}函式)。對於這種類型的函式,將參數移除出堆疊的操作在這次函式調用中和下次函式調用中是不同的。C調用約定能使指令簡單地執行這種不同的操作。Pascal和stdcall調用約定執行這種操作是非常困難的。因此,
Pascal調用約定(和Pascal語言一樣)不允許有這種類型的函式。MS Windows只有當它的API函式不可以攜帶變化個數的參數時才可以使用這種約定。

\begin{figure}[t]
\begin{AsmCodeListing}[frame=single]
      push   dword 1        ; 傳遞參數1
      call   fun
      add    esp, 4         ; 將參數移除出堆疊
\end{AsmCodeListing}
\caption{副程式調用示例 \label{fig:subcall}}
\end{figure}

圖~\ref{fig:subcall}展示了一個將被調用的副程式如何使用C調用約定。第3行通過直接操作堆疊指標將參數移除出堆疊。同樣可以使用{\code POP}指令來做這件事,但是常常使用在要求將無用的結果儲存到一個寄存器的情況下。實際上,對於這種情況,許多編譯器常常使用一條{\code POP ECX}來移除參數。編譯器會使用{\code
POP}指令來代替{\code ADD}指令,因為{\code ADD}指令需要更多的位元組。但是,{\code POP}會改變ECX的值。下面是一個有兩個副程式的例子,它們使用了上面討論的C調用約定。54行(和其他行)展示了多個資料和文本段可以在同一個原始檔案中聲明。進行連接處理時,它們將會組合成單一的資料段和文本段。把資料和文本段分成單獨的幾段就允許資料定義在副程式代碼附近,這也是副程式經常做的。
\index{堆疊!參數|)}

\begin{AsmCodeListing}[label=sub3.asm]
%include "asm_io.inc"

segment .data
sum     dd   0

segment .bss
input   resd 1

;
; 虛擬碼演算法
; i = 1;
; sum = 0;
; while( get_int(i, &input), input != 0 ) {
;   sum += input;
;   i++;
; }
; print_sum(num);
segment .text
        global  _asm_main
_asm_main:
        enter   0,0               ; 程式開始運行
        pusha

        mov     edx, 1            ; edx就是虛擬碼裏的i
while_loop:
        push    edx               ; 保存i到堆疊中
        push    dword input       ; 把input的位址壓入堆疊
        call    get_int
        add     esp, 8            ; 將i和&input移除出堆疊

        mov     eax, [input]
        cmp     eax, 0
        je      end_while

        add     [sum], eax        ; sum += input

        inc     edx
        jmp     short while_loop

end_while:
        push    dword [sum]       ; 將總數壓入堆疊
        call    print_sum
        pop     ecx               ; 將[sum]移除出堆疊

        popa
        leave
        ret

; 副程式get_int
; 參數(順序壓入堆疊中)
;   輸入的個數(儲存在[ebp + 12]中)
;   儲存輸入字的位址(儲存在[ebp + 8]中)
; 注意:
;   eax和ebx的值被毀掉了
segment .data
prompt  db      ") Enter an integer number (0 to quit): ", 0

segment .text
get_int:
        push    ebp
        mov     ebp, esp

        mov     eax, [ebp + 12]
        call    print_int

        mov     eax, prompt
        call    print_string

        call    read_int
        mov     ebx, [ebp + 8]
        mov     [ebx], eax         ; 將輸入儲存到記憶體中

        pop     ebp
        ret                        ; 返回到調用代碼

; 副程式print_sum
; 輸出總數
; 參數:
;   需要輸出的總數(儲存在[ebp+8]中)
; 注意: eax的值被毀掉了
;
segment .data
result  db      "The sum is ", 0

segment .text
print_sum:
        push    ebp
        mov     ebp, esp

        mov     eax, result
        call    print_string

        mov     eax, [ebp+8]
        call    print_int
        call    print_nl

        pop     ebp
        ret
\end{AsmCodeListing}


\subsection{堆疊上的局部變數\index{堆疊!局部變數|(}}

堆疊可以方便地用來儲存局部變數。這實際上也是C儲存普通變數(或C
lingo中的\emph{自動變數})的地方。如果你希望副程式是可重入的,那麼使用堆疊存儲變數是非常重要的。一個可重入的副程式不管在任何地方被調用都能正常運行,包括副程式本身。換句話說,可重入副程式可以\emph{嵌套}調用。儲存變數的堆疊同樣在記憶體中。不儲存在堆疊裏的資料從程式開始到程式結束都使用記憶體(C稱這種類型的變數為
\emph{總體變數}或\emph{靜態變數})。儲存在堆疊裏的資料只有當定義它們的副程式是活動的時候才使用記憶體。

\begin{figure}[t]
\begin{AsmCodeListing}[frame=single]
subprogram_label:
      push   ebp                ; 保存原始EBP值到堆疊中
      mov    ebp, esp           ; 新EBP的值 = ESP
      sub    esp, LOCAL_BYTES   ; = #局部變數需要的位元組數
; subprogram code
      mov    esp, ebp           ; 釋放局部變數
      pop    ebp                ; 恢復原始EBP值
      ret
\end{AsmCodeListing}
\caption{帶有局部變數的副程式的一般格式\label{fig:subskel2}}
\end{figure}

\begin{figure}[t]
\begin{lstlisting}[frame=tlrb]{}
void calc_sum( int n, int * sump )
{
  int i, sum = 0;

  for( i=1; i <= n; i++ )
    sum += i;
  *sump = sum;
}
\end{lstlisting}
\caption{求總數的C語言版 \label{fig:Csum}}
\end{figure}

\begin{figure}[t]
\begin{AsmCodeListing}[frame=single]
cal_sum:
      push   ebp
      mov    ebp, esp
      sub    esp, 4               ; 為局部變數sum分配空間

      mov    dword [ebp - 4], 0   ; sum = 0
      mov    ebx, 1               ; ebx (i) = 1
for_loop:
      cmp    ebx, [ebp+8]         ; is i <= n?
      jnle   end_for

      add    [ebp-4], ebx         ; sum += i
      inc    ebx
      jmp    short for_loop

end_for:
      mov    ebx, [ebp+12]        ; ebx = sump
      mov    eax, [ebp-4]         ; eax = sum
      mov    [ebx], eax           ; *sump = sum;

      mov    esp, ebp
      pop    ebp
      ret
\end{AsmCodeListing}
\caption{求總數的組合語言版 \label{fig:Asmsum}}
\end{figure}

\MarginNote{儘管{\code ENTER}和{\code LEAVE}指令\\事實上簡化了開始部分\\和結束部分,但是它們\\並沒有經常被使用。這\\是為什麼呢?因為與等\\價的簡單的指令相比,\\它們執行較慢!當你並\\不認為執行一隊指令序\\列比執行一條複合的指\\令要快的時候,這就是\\一個榜樣。}在堆疊中,局部變數恰好儲存在保存的EBP值之後。它們通過在副程式的開始部分用ESP減去一定的位元組數來分配存儲空間。
圖~\ref{fig:subskel2}展示了副程式新的骨架。EBP用來訪問局部變數。考慮圖~\ref{fig:Csum}中的C函式。 圖~\ref{fig:Asmsum}
展示如何用組合語言編寫等價的副程式。

\begin{figure}[t]
\centering
\begin{tabular}{ll|c|}
\cline{3-3}
ESP + 16 & EBP + 12 & {\code sump} \\ \cline{3-3}
ESP + 12 & EBP + 8  & {\code n} \\ \cline{3-3}
ESP + 8  & EBP + 4  & Return address \\ \cline{3-3}
ESP + 4  & EBP      & saved EBP \\ \cline{3-3}
ESP      & EBP - 4  & {\code sum} \\ \cline{3-3}
\end{tabular}
\caption{}
\label{fig:SumStack}
\end{figure}

圖~\ref{fig:SumStack}展示了執行完圖~\ref{fig:Asmsum}中程式的開始部分後的堆疊狀態。這一節的堆疊包含了參數,返回資訊和局部變數,這樣堆疊稱為一個\emph{堆疊幀}。C函式的每一次調用都會在堆疊上創建一個新的堆疊幀。

\begin{figure}[t]
\begin{AsmCodeListing}[frame=single]
subprogram_label:
      enter  LOCAL_BYTES, 0     ; = #局部變數需要的位元組數
; subprogram code
      leave
      ret
\end{AsmCodeListing}
\caption{通過使用{\code ENTER}和{\code LEAVE}指令,帶有局部變數的副程式的一般格式
\label{fig:subskel3}}
\end{figure}

可以使用兩條專門的指令來簡化一個子程式的開始部分和結束部分,它們是為這個目的而專門設計的。{\code
ENTER}指令執行開始部分的代碼,而 {\code
LEAVE}指令執行結束部分。{\code
ENTER}指令攜帶兩個立即數。對於C調用約定,第二個運算元總是為0。第一個運算元是局部變數所需要的位元組數。{\code
LEAVE}指令沒有運算元。
圖~\ref{fig:subskel3}展示了如何使用這些指令。注意程式skeleton(圖~\ref{fig:skel})同樣使用了{\code
ENTER}和{\code LEAVE}指令。 \index{堆疊!局部變數|)} \index{堆疊|)}
\index{調用約定|)} \index{副程式!調用|)}

\section{多模組程式\index{多模組程式|(}}

\emph{多模組程式}是由不止一個目標檔組成的程式。這裏出現的所有程式都是多模組程式。它們由C驅動目標檔和彙編目標檔(加上C庫目標檔)組成。回憶一下連接程式將目標檔組合成一個可執行程式。連接程式必須把在一個模組(\emph{也就是}
目標檔)中引用的每個變數匹配到定義該變數的模組。
為了讓模組A能使用定義在模組B裏的變數,就必須使用{\code
extern(外部)}指示符。在{\code extern}
\index{directive!extern}指示符後面是用逗號隔開的變數列表。這個指示符告訴編譯器把這些變數視為是模組\emph{外部的}。也就是說,這些變數可以在這個模組中使用,但是卻定義在另一模組中。{\code
asm\_io.inc}檔中就將{\code read\_int}\emph{等}程式定義為外部的。

在編譯語言中,缺省情況下變數不可以由外部程式訪問。如果一個變數可以被一個模組訪問,而這個模組又不是定義它的,那麼在定義它的模組中,它一定被聲明為\emph{global(全局的)}。{\code
global} \index{指示符!全局}指示符就可以用來做這件事情。圖~\ref{fig:skel}的程式skeleton中的第13行定義了一個總體變數 {\code
\_asm\_main}。若沒有這個聲明,就可能會出錯。為什麼?因為C代碼將會找不到\emph{內部的} {\code \_asm\_main}變數。

下面是用兩個模組重寫的以前例子的代碼。副程式({\code get\_int}和{\code print\_sum})在不同的原始檔案中,而不是在{\code \_asm\_main}程式中。

\begin{AsmCodeListing}[label=main4.asm,commandchars=\\\{\}]
%include "asm_io.inc"

segment .data
sum     dd   0

segment .bss
input   resd 1

segment .text
        global  _asm_main
\textit{        extern  get_int, print_sum}
_asm_main:
        enter   0,0               ; 程式開始運行
        pusha

        mov     edx, 1            ; edx就是虛擬碼中的i
while_loop:
        push    edx               ; 保存i到堆疊中
        push    dword input       ; 把input的位址壓入堆疊
        call    get_int
        add     esp, 8            ; 將i和&input移除出堆疊

        mov     eax, [input]
        cmp     eax, 0
        je      end_while

        add     [sum], eax        ; sum += input

        inc     edx
        jmp     short while_loop

end_while:
        push    dword [sum]       ; 將總數壓入堆疊
        call    print_sum
        pop     ecx               ; 將總數[sum]移除出堆疊

        popa
        leave
        ret
\end{AsmCodeListing}

\begin{AsmCodeListing}[label=sub4.asm,commandchars=\\\{\}]
%include "asm_io.inc"

segment .data
prompt  db      ") Enter an integer number (0 to quit): ", 0

segment .text
\textit{        global  get_int, print_sum}
get_int:
        enter   0,0

        mov     eax, [ebp + 12]
        call    print_int

        mov     eax, prompt
        call    print_string

        call    read_int
        mov     ebx, [ebp + 8]
        mov     [ebx], eax         ; 將輸入儲存到記憶體中

        leave
        ret                        ; 返回到調用代碼

segment .data
result  db      "The sum is ", 0

segment .text
print_sum:
        enter   0,0

        mov     eax, result
        call    print_string

        mov     eax, [ebp+8]
        call    print_int
        call    print_nl

        leave
        ret
\end{AsmCodeListing}

上面的例子只有全局的\index{指示符!全局}代碼變數;同樣,全局資料變數也可以使用一模一樣的方法。
\index{多模組程式|)}

\section{C與彙編的介面技術\index{與C介面|(}\index{調用約定!C|(}}

現今,完全用彙編書寫的程式是非常少的。編譯器能很好地將高階語言轉換成有效的機器代碼。因為用高階語言書寫代碼非常容易,所以高階語言變得很流行。此外,高階語言比組合語言\emph{更}容易移植!

當使用組合語言時,我們經常將它使用在代碼中的一小部分上。有兩種使用組合語言的方法:在C中調用彙編副程式或內嵌彙編。內嵌彙編允許程式師把彙編語句直接放入到C代碼中。這樣是非常方便的;但是,內嵌彙編同樣存在缺點。組合語言的書寫格式必須是編譯器使用的格式。目前沒有一個編譯器支持NASM格式。不同的編譯器要求使用不同的格式。Borland和Microsoft要求使用
MASM格式。DJGPP和Linux中gcc要求使用GAS\footnote{GAS是所有的GNU編譯器使用的組合語言程式。它使用AT\&T語法,這是一種完全不同於MASM,TASM 和NASM的語法。}
格式。在PC機上,調用彙編副程式是更標準的技術。

在C中使用組合語言程式通常是因為以下幾個原因:
\begin{itemize}
\item 需要直接訪問電腦的硬體特性,而用C語言很難或不可能做到。
\item 程式執行必須盡可能地快,而且相比于編譯器,程式師手動優化的代碼更好。
\end{itemize}

最後一個原因不像它以前一樣有根據。因為這些年編譯器技術提高了,而且編譯器通常可以產生非常有效的代碼
(特別是當開啟編譯器優化的時候)。調用組合語言程式的缺點:可攜性和可讀性減弱了。

絕大部分的C調用約定已經確定了。但是,還需要描述一些額外的特徵。

\subsection{保存寄存器\index{調用約定!C!寄存器|(}}
首先, \MarginNote{關鍵字{\code register}可以使\\用在一個C變數聲明中\\來暗示編譯器:這個變\\數使用一個寄存器而不\\是記憶體空間。這些變\\數稱為寄存器變數。\\現在的編譯器會自動做\\這件事而不需要給它暗示。}
C假定副程式保存了下面這幾個寄存器的值:EBX,ESI,EDI,\\EBP,CS,DS,SS,ES。這並不意味著不能在副程式內部修改它們。相反,它表示如果子程式改變了它們的值,那麼在副程式返回之前必須恢復它們的原始值。EBX,ESI和EDI的值不能被改變,因為C將這些寄存器用於\emph{寄存器變數}。通常都是使用堆疊來保存這些寄存器的原始值。

\begin{figure}[t]
\begin{AsmCodeListing}[frame=single]
segment .data
x            dd     0
format       db     "x = %d\n", 0

segment .text
...
      push   dword [x]     ; 將x的值壓入堆疊中
      push   dword format  ; 將format字串的位址壓入堆疊中
      call   _printf       ; 注意下劃線!
      add    esp, 8        ; 從堆疊中移除參數
\end{AsmCodeListing}
\caption{調用{\code printf} \label{fig:Cprintf}}
\end{figure}

\begin{figure}[t]
\centering
\begin{tabular}{l|c|}
\cline{2-2}
EBP + 12 & {\code x}的值 \\ \cline{2-2}
EBP + 8  & format字串的位址 \\ \cline{2-2}
EBP + 4  & 返回地址 \\ \cline{2-2}
EBP      & 保存的EBP值 \\ \cline{2-2}
\end{tabular}
\caption{{\code printf}的堆疊結構\label{fig:CprintfStack}}
\end{figure}
\index{調用約定!C!寄存器|)}

\subsection{函式名\index{調用約定!C!函式名|(}}
大多數C編譯器都在函式名和全局或靜態變數前附加一個下劃線字元。例如,函式名{\code f}將指定為{\code \_f}。因此,如果這是一個組合語言程式,那麼它必須標記為{\code \_f},而不是{\code f}。Linux gcc編譯器並\emph{不}附加任何字元。
在可執行的Linux ELF下,對於C函式{\code f},你只需要簡單使用函式名{\code f}即可。但是,
 DJGPP的gcc卻附加了一個下劃線。注意,在組合語言程式skeleton中
(圖~\ref{fig:skel}),主程序函式名是{\code
\_asm\_main}。
\index{調用約定!C!函式名|)}

\subsection{傳遞參數\index{調用約定!C!參數|(}}
按照C調用約定,一個函式的參數將以一定順序壓入堆疊中,這個順序與它們出現在函式調用裏的順序\emph{相反}。

考慮這條C語句:\verb|printf("x = %d\n",x);|
圖~\ref{fig:Cprintf}展示了如何編譯這條語句(用等價的NASM格式)。圖~\ref{fig:CprintfStack}展示了執行完{\code printf}函式的開始部分後,堆疊的狀態。{\code printf}函式一個可以攜帶任意個參數的C語言庫函式。C調用約定的規則就是專門為允許這些類型的函式而規定的。\MarginNote{沒有必要使用組合語言\\來處理C中的可變參數\\函式。可以通過移植\\{\code stdarg.h}頭檔中定義的\\宏來處理它們。看一本\\好的C語言的書來得到更\\詳細的資訊。} 因為format字串的位址最後壓入堆疊,所以不管有多少參數傳遞到函式,它在堆疊裏的位置將總是
{\code EBP + 8}。然後{\code printf}代碼就可以通過看format字串的位置來決定需要傳遞多少參數和在堆疊上如何找到它們。

當然,如果有錯誤發生, \verb|printf("x = %d\n")|,
{\code printf}代碼仍然會將{\code [EBP + 12]}中的雙字值輸出,而這並不是{\code x}的值!
\index{調用約定!C!參數|)}

\subsection{計算局部變數的位址\index{堆疊!局部變數|(}}

找到定義在{\code data}或{\code
bss}段的變數的位址是非常容易的。基本上,連接程式做的就是這件事情。但是,要計算出在堆疊上的一個局部變數(或參數)的位址就不簡單了。可是,當調用副程式的時候,這種需求是非常普通的。考慮傳遞一個變數(讓我們稱它為{\code x})的位址到一個函式(讓我們稱它為
{\code foo})的情況。如果{\code x}處在堆疊的EBP $-$ 8的位置,你不可以這樣使用:
\begin{AsmCodeListing}[numbers=none,frame=none]
      mov    eax, ebp - 8
\end{AsmCodeListing}
為什麼?因為指令{\code
MOV}儲存到EAX裏的值必須能由彙編器計算出來(也就是說,它最後必須是一個常量)。但是,有一條指令能做這種需求的計算。它就是\index{LEA|(}
{\code LEA}  (即\emph{Load Effective
Address,載入有效位址})。下面的代碼就能計算出{\code
x}的位址並將它儲存到EAX中:
\begin{AsmCodeListing}[numbers=none,frame=none]
      lea    eax, [ebp - 8]
\end{AsmCodeListing}
現在EAX中存有了{\code x}的位址,而且當調用函式{\code foo}的時候,就可以將其壓入到堆疊中。不要搞混了,這條指令看起來是從[EBP\nolinebreak$-$\nolinebreak8]中讀數據;然而,這並\emph{不}正確。{\code LEA}指令\emph{永遠不會}從記憶體中讀數據。它僅僅計算出一個將會被其他指令使用到的位址,然後將這個位址儲存到它的第一個運算元裏。因為它並沒有實際讀記憶體,所以不指定記憶體大小(\emph{例如:}
{\code dword})是必須的或說是允許的。

\index{LEA|)}
\index{堆疊!局部變數|)}

\subsection{返回值\index{調用約定!C!返回值|(}}

返回值不為空的C函式執行完後會返回一個值。C調用約定規定了這個要如何去做。返回值需通過寄存器傳遞。所有的整形類型({\code
char}, {\code int}, {\code enum},
\emph{等})通過EAX寄存器返回。如果它們小於32位元,那麼儲存到EAX的時候,它們將被擴展成32位。
(它們如何擴展取決於是有符號類型還是無符號類型。)
64位的值通過EDX:EAX\index{寄存器!EDX:EAX}寄存器對返回。浮點數儲存在數學輔助運算器中的ST0寄存器中。
(這個寄存器將在浮點數這一章來討論。) \index{調用約定!C!返回值|)}
\index{調用約定!C|)}

\subsection{其他調用約定\index{調用約定|(}}

所有的80x86
C編譯器中都支援上面描述的標準C調用約定的規則。通常編譯器也支援其他調用約定。當與組合語言進行介面時,知道編譯器調用你的函式時使用的是什麼調用約定是\emph{非常}重要的。通常,缺省時,使用的是標準的調用約定;但是,並不總是這一種情況\footnote{Watcom
C\index{編譯器!Watcom}編譯器就是一個在缺省情況下\emph{不}使用標準調用的例子。看Watcom的樣例原始檔案代碼來得到更詳細的資訊}。
使用多種約定的編譯器通常都擁有可以用來改變缺省約定的命令行開關。它們同樣提供擴展的C語法來為單個函式指定調用約定。但是,各個編譯器的這些擴展標準可以是不一樣的。

GCC編譯器允許不同的調用約定。一個函式的調用約定可以通過擴展語法{\code
\_\_attribute\_\_}
\index{編譯器!gcc!\_\_attribute\_\_}明確指定。例如,要聲明一個返回值為空的函式{\code
f},它帶有一個{\code
int}參數,使用標準調用約定\index{調用約定!C},需使用下面的語法來聲明它的原型:
\begin{lstlisting}[stepnumber=0]{}
void f( int ) __attribute__((cdecl));
\end{lstlisting}
GCC同樣支援\emph{標準call} \index{調用約定!stdcall}調用約定。通過把{\code cdecl}替換成{\code stdcall},上面的函式可以指定為使用這種約定。{\code stdcall}約定和{\code cdecl}約定的不同點是
{\code stdcall}要求副程式將參數移除出堆疊(和Pascal調用約定一樣)。因此,{\code stdcall}調用約定只能使用在帶有固定參數的函式上 (\emph{也就是說},不可以是函式{\code printf}和{\code scanf})。

GCC同樣支持稱為{\code regparm}
\index{調用約定!寄存器}的約定,這種約定告訴編譯器前3個整形參數通過寄存器傳遞給函式,而不是通過堆疊。這是許多編譯器支援的一個共同的優化模式。

Borland和Microsoft使用一樣語法來聲明調用約定。它們在C代碼中加上關鍵字{\code \_\_cdecl}\index{調用約定!\_\_cdecl}和 {\code \_\_stdcall}\index{調用約定!\_\_stdcall}。這些關鍵字用來修飾函式。在原型聲明中,它們出現在函式名的前面例如,上面的函式{\code f}用Borland和Microsoft定義如下:
\begin{lstlisting}[stepnumber=0]{}
void __cdecl f( int );
\end{lstlisting}

每種調用約定都有各自的優缺點。{\code cdecl}\index{調用約定!C}調用約定的主要優點是它非常簡單而且非常靈活。它可以用於任何類型的C函式和C編譯器。使用其他約定會限制副程式的可攜性。它的主要缺點是與其他約定相比它執行較慢而且使用更多的記憶體(因為函式的每次調用都需要用代碼將參數移除出堆疊。)。

{\code stdcall}\index{調用約定!標準call}調用約定的主要優點是相比於{\code cdecl}它使用較少的記憶體。在{\code CALL}指令之後 ,不需要清理堆疊。它的主要缺點是它不能使用於可變參數的函式。

使用寄存器傳遞參數的調用約定的優點是速度非常快。主要缺點是這種約定太複雜。有些參數可能在寄存器中,而另一些可能在堆疊中。

\index{調用約定|)}

\subsection{樣例}

下面是一個展示組合語言程式如何與C程式介面的例子。(注意:這個程式並沒有使用skeleton組合語言程式(圖~\ref{fig:skel})或driver.c模組。)

\LabelLine{main5.c}
\lstset{escapeinside=`',language=Pascal,%
}
\begin{lstlisting}{}
#include <stdio.h>
/* `組合語言程式的原型聲明' */
void calc_sum( int, int * ) __attribute__((cdecl));

int main( void )
{
  int n, sum;

  printf("Sum integers up to: ");
  scanf("%d", &n);
  calc_sum(n, &sum);
  printf("Sum is %d\n", sum);
  return 0;
}
\end{lstlisting}
\LabelLine{main5.c}

\begin{AsmCodeListing}[label=sub5.asm, commandchars=\\\%|]
; 副程式 _calc_sum
; 求整形1到n的和
; 參數:
;   n - 從1加到多少(儲存在[ebp + 8])
;   sump - 指向總數儲存位址的整形指標(儲存在[ebp + 12])
; C虛擬碼:
; void calc_sum( int n, int * sump )
; {
;   int i, sum = 0;
;   for( i=1;i <= n; i++ )
;     sum += i;
;   *sump = sum;
; }

segment .text
        global  _calc_sum
;
; 局部變數:
;   儲存在[ebp-4]裏的sum值
_calc_sum:
        enter   4,0               ; 在堆疊上為sum分配空間
        push    ebx               ; 重要! \label%line:pushebx|

        mov     dword [ebp-4],0   ; sum = 0
        dump_stack 1, 2, 4        ; 輸出堆疊中從ebp-8到ebp+16的值 \label%line:dumpstack|
        mov     ecx, 1            ; ecx是虛擬碼中的i
for_loop:
        cmp     ecx, [ebp+8]      ; 比較i和n
        jnle    end_for           ; 如果i > n,則退出迴圈

        add     [ebp-4], ecx      ; sum += i
        inc     ecx
        jmp     short for_loop

end_for:
        mov     ebx, [ebp+12]     ; ebx = sump
        mov     eax, [ebp-4]      ; eax = sum
        mov     [ebx], eax

        pop     ebx               ; 恢復ebx的值
        leave
        ret
\end{AsmCodeListing}

\begin{figure}[t]
\begin{Verbatim}[frame=single]
Sum integers up to: 10
Stack Dump # 1
EBP = BFFFFB70 ESP = BFFFFB68
 +16  BFFFFB80  080499EC
 +12  BFFFFB7C  BFFFFB80
  +8  BFFFFB78  0000000A
  +4  BFFFFB74  08048501
  +0  BFFFFB70  BFFFFB88
  -4  BFFFFB6C  00000000
  -8  BFFFFB68  4010648C
Sum is 55
\end{Verbatim}
\caption{sub5程式的運行示例 \label{fig:dumpstack}}
\end{figure}

為什麼程式{\code sub5.asm}中的第~\ref{line:pushebx}行非常重要?因為C調用約定要求EBX的值不能被調用的函式更改。如果沒有做這個,程式很可能不會正確運行。

第~\ref{line:dumpstack}行演示了巨集{\code dump\_stack}如何運作。它的第一個參數只是一個數字標號,第二個參數決定需顯示EBP以下多少個雙字而第三個參數決定需顯示EBP以上多少個雙字。圖~\ref{fig:dumpstack}展示了這個程式的運行示例。對於這次轉儲,你可以看到儲存總數的雙字位址是FBFFFFB80 (儲存在EBP~+~12);n值為0000000A
(儲存在EBP~+~8);程式的返回位址為08048501 (儲存在EBP~+~4);保存的EBP的值為BFFFFB88 (儲存在EBP);局部變數的值為
0(儲存在EBP~-~4);最後保存的EBX的值為4010648C (儲存在EBP~-~8)。

{\code calc\_sum}函式可以這樣重寫:把sum當作返回值返回,替代使用的指針參數。因為sum是一個整形值,所以應保存到EAX寄存器中。{\code main5.c}檔中的第11行應該改成:
\begin{lstlisting}[stepnumber=0]{}
  sum = calc_sum(n);
\end{lstlisting}
同樣,{\code calc\_sum}的原型也需要改變。下面是修改後的彙編代碼:
\begin{AsmCodeListing}[label=sub6.asm]
; 副程式 _calc_sum
; 求整形1到n的和 ; 參數:
;   n - 從1加到多少(儲存在[ebp + 8]
; 返回值:
;   sum的值
; C虛擬碼:
; int calc_sum( int n )
; {
;   int i, sum = 0;
;   for( i=1; i <= n; i++)
;     sum += i;
;   return sum;
; }
segment .text
        global  _calc_sum
;
; 局部變數:
;   儲存在[ebp-4]裏的sum值
_calc_sum:
        enter   4,0               ; 在堆疊上為sum分配空間

        mov     dword [ebp-4],0   ; sum = 0
        mov     ecx, 1            ; ecx是虛擬碼中的i
for_loop:
        cmp     ecx, [ebp+8]      ; 比較i和n
        jnle    end_for           ; 如果i > n,則退出迴圈

        add     [ebp-4], ecx      ; sum += i
        inc     ecx
        jmp     short for_loop

end_for:
        mov     eax, [ebp-4]      ; eax = sum

        leave
        ret
\end{AsmCodeListing}

\subsection{在組合語言程式中調用C函式}

\begin{figure}[t]
\begin{AsmCodeListing}[frame=single]
segment .data
format       db "%d", 0

segment .text
...
      lea    eax, [ebp-16]
      push   eax
      push   dword format
      call   _scanf
      add    esp, 8
...
\end{AsmCodeListing}
\caption{在組合語言程式中調用{\code scanf}函式\label{fig:scanf}}
\end{figure}

C與彙編介面的一個主要優點是允許彙編代碼訪問大型C庫和用戶寫的函式。例如,如果你想調用一下{\code
scanf}函式來從鍵盤讀一個整形,該怎麼辦?圖~\ref{fig:scanf}展示了完成這件事的代碼。需要記住的非常重要的一點就是{\code
scanf}函式遵循字面意義的C調用標準。這就意味著它保存了EBX,ESI和\\EDI寄存器的值;但是,
EAX,ECX和EDX寄存器的值可能會被修改。事實上,EAX肯定會被修改,因為它將保存{\code
scanf}調用的返回值。至於與C介面的其他例子,可以看用來產生{\code
asm\_io.obj}的{\code asm\_io.asm}檔中的代碼。 \index{與C介面|)}

\section{可重入和遞迴副程式\index{遞迴|(}}

\index{副程式!可重入|(}
一個可重入副程式必須滿足下面幾個性質:
\begin{itemize}
\item 它不能修改代碼指令。在高階語言中,修改代碼指令是非常難的;但是在組合語言中,一個程式要修改自己的代碼並不是一件很難的事。例如:
\begin{AsmCodeListing}[frame=none, numbers=none]
      mov    word [cs:$+7], 5      ; 將5複製到前面七個位元組的字中
      add    ax, 2                 ; 前面的語句將2改成了5!
\end{AsmCodeListing}
這些代碼在實模式下可以運行,但是在保護模式下的作業系統上不行,因為代碼段被標識為唯讀。在這些作業系統上,當執行了上面的第一行代碼,程式將被終止。這種類型的程式從各個方面來看都非常差。它很混亂,很難維護而且不允許代碼共用(看下面)。

\item 它不能修改總體變數(比如在{\code data}和
{\code bss}段裏的數據)。所有的變數應儲存在堆疊裏。

\end{itemize}

書寫可重入性代碼有幾個好處。
\begin{itemize}
\item 一個可重入副程式可以遞迴調用。
\item 一個可重入程式可以被多個進程共用。在許多多工作業系統上,如果一個程式有許多實例正在運行,那麼只有\emph{一份}代碼的拷貝在記憶體中。共用庫和DLL(\emph{Dynamic Link Libraries,動態連結程式庫})同樣使用了這種技術。
\item 可重入副程式可以運行在\emph{多線程}
\footnote{一個多線程程式同時有多條線程在執行。也就是說,程式本身是多工的。} 程式中。 Windows 9x/NT和大多數類
UNIX作業系統(Solaris, Linux,\emph{等})都支援多線程程式。
\end{itemize}
\index{副程式!可重入|)}

\subsection{遞迴副程式}

這種類型的副程式調用它們自己。遞迴可以是
\emph{直接的}或是\emph{間接的}。當一個名為{\code foo}的副程式在{\code foo}內部調用自己就產生直接遞迴。當一個子程式雖然自己沒有直接調用自己,但是其他副程式調用了它,就產生間接遞迴。例如:副程式{\code foo}可以調用
{\code bar}且{\code bar}也可以調用{\code foo}。

遞迴副程式必須有一個\emph{終止條件}。當這個條件為真時,就不再進行遞迴調用了。如果一個子程式沒有終止條件或條件永不為真,那麼遞迴將不會結束(非常像一個無窮迴圈)。

\begin{figure}
\begin{AsmCodeListing}[frame=single]
; 求n!
segment .text
      global _fact
_fact:
      enter  0,0

      mov    eax, [ebp+8]    ; eax = n
      cmp    eax, 1
      jbe    term_cond       ; 如果n <= 1,則終止
      dec    eax
      push   eax
      call   _fact           ; eax = fact(n-1)
      pop    ecx             ; 結果在eax中
      mul    dword [ebp+8]   ; edx:eax = eax * [ebp+8]
      jmp    short end_fact
term_cond:
      mov    eax, 1
end_fact:
      leave
      ret
\end{AsmCodeListing}
\caption{求n!的遞迴函式\label{fig:factorial}}
\end{figure}

\begin{figure}
\centering
%\includegraphics{factStack.eps}
\input{factStack.latex}
\caption{n!函式的堆疊幀\label{fig:factStack}}
\end{figure}

圖~\ref{fig:factorial}展示了一個遞迴求n!的函式。在C中它可以這樣被調用:
\begin{lstlisting}[stepnumber=0]{}
x = fact(3);         /* find 3! */
\end{lstlisting}
圖~\ref{fig:factStack}展示了上面的函式調用的最深點的堆疊狀態。

\begin{figure}[t]
\begin{lstlisting}[frame=tlrb]{}
void f( int x )
{
  int i;
  for( i=0; i < x; i++ ) {
    printf("%d\n", i);
    f(i);
  }
}
\end{lstlisting}
\caption{另一個例子(C語言版)\label{fig:rec2C}}
\end{figure}

\begin{figure}
\begin{AsmCodeListing}[frame=single]
%define i ebp-4
%define x ebp+8          ; useful macros
segment .data
format       db "%d", 10, 0     ; 10 = '\n'
segment .text
      global _f
      extern _printf
_f:
      enter  4,0           ; 在堆疊上為i分配空間

      mov    dword [i], 0  ; i = 0
lp:
      mov    eax, [i]      ; is i < x?
      cmp    eax, [x]
      jnl    quit

      push   eax           ; 調用printf
      push   format
      call   _printf
      add    esp, 8

      push   dword [i]     ; 調用f
      call   _f
      pop    eax

      inc    dword [i]     ; i++
      jmp    short lp
quit:
      leave
      ret
\end{AsmCodeListing}
\caption{另一個例子(組合語言版)\label{fig:rec2Asm}}
\end{figure}

圖~\ref{fig:rec2C}展示了另一個更複雜的遞迴樣例的C語言版而\ref{fig:rec2Asm}展示了它的組合語言版。對於{\code f(3)},輸出是什麼?注意:每一次遞迴調用,{\code ENTER}指令都會在堆疊上給新的{\code i}值分配空間。因此,{\code f}的每一次遞迴調用都有它自己獨立的變數{\code i}。若是在{\code data}段定義{\code i}為一雙字,結果就不一樣了。
\index{遞迴|)}

\subsection{回顧一下C變數的儲存類型}

C提供了幾種變數儲存類型。
\begin{description}
\item[global,全局]
\index{儲存類型!全局}
這些變數定義在任何函式的外面,且儲存在固定的記憶體空間中(在{\code data}或{\code
bss}段),而且從程式的開始一直到程式的結束都存在。缺省情況下,它們能被程式中的任何一個函式訪問;但是,如果它們被聲明為{\code static},那麼只有在同一模組中的函式才能訪問它們(\emph{也就是說,} 依照彙編的術語,這個變數是內部的,不是外部的)。

\item[static,靜態]
\index{儲存類型!靜態}
在一個函式中,它們是被聲明為{\code靜態}的\emph{局部}變數。(不幸的是,C使用關鍵字{\code
static}有兩種目的!)這些變數同樣儲存在固定的記憶體空間中(在{\code data}或{\code bss}段),但是只能被定義它的函式直接訪問。

\item[automatic,自動]
\index{儲存類型!自動}
它是定義在一個函式內的C變數的缺省類型。當定義它們的函式被調用了,這些變數就被分配在堆疊上,而當函式返回了又從堆疊中移除。因此,它們沒有固定的記憶體空間。

\item[register,寄存器]
\index{儲存類型!寄存器}
這個關鍵字要求編譯器使用寄存器來儲存這個變數的資料。這僅僅是一個\emph{要求}。編譯器並\emph{不}一定要遵循。如果變數的位址使用在程式的任意的地方,那麼就不會遵循(因為寄存器沒有位址)。同樣,只有簡單的整形資料可以是寄存器變數。結構類型不可以;因為它們的大小不匹配寄存器!C編譯器通常會自動將普通的自動變數轉換成寄存器變數,而不需要程式師給予暗示。

\item[volatile,不穩定]
\index{儲存類型!不穩定}
這個關鍵字告訴編譯器這個變數值隨時都會改變。這就意味著當變數被更改了,編譯器不能做出任何推斷。通常編譯器會將一個變數的值暫時存在寄存器中,而且在出現這個變數的代碼部分使用這個寄存器。但是,編譯器不能對{\code 不穩定}類型的變數做這種類型的優化。一個不穩定變數的最普遍的例子就是:它可以被多線程程式的兩個線程修改。考慮下面的代碼:
\begin{lstlisting}{}
x = 10;
y = 20;
z = x;
\end{lstlisting}
如果{\code x}可以被另一個線程修改。那麼其他線程可以會在第1行和第3行之間修改{\code x}的值,以致於{\code z}將不會等於10.但是, 如果{\code x}沒有被聲明為不穩定類型,編譯器就會推斷{\code x}沒有改變,然後再將{\code z}置為10。

{\code 不穩定類型}的另一個使用就是避免編譯器為一個變數使用一個寄存器。

\end{description}
\index{副程式|)}


% -*-latex-*-
\chapter{Arrays}
\index{Arrays|(}
\section{Einf\"{u}hrung}

Ein \emph{Array} ist ein zusammenh\"{a}ngender Block einer Liste von
Daten im Speicher. Jedes Element der Liste muss den gleichen Typ
haben und genau die gleiche Anzahl Bytes f\"{u}r die Speicherung
benutzen. Wegen diesen Eigenschaften erlauben Arrays effizienten
Zugriff auf die Daten \"{u}ber ihre Position (oder Index) im Array. Die
Adresse jeden Elements kann berechnet werden, wenn drei Angaben
bekannt sind:
\begin{itemize}
\parskip=-0.25em %reduce the spacing <<<<<<<<<<<<<<<<<<<<<<<<<<<<<<<<<<<<<<<<<<

\item
Die Adresse des ersten Elements des Arrays

\item
Die Anzahl Bytes in jedem Element

\item
Der Index des Elements

\end{itemize}

Es ist bequem, den Index des ersten Arrayelements als Null zu
betrachten (genau wie in C). Es ist m\"{o}glich, andere Werte f\"{u}r den
ersten Index zu verwenden, aber es kompliziert die Berechnungen.

\subsection{Arrays definieren\index{Arrays!Definition|(}}

\begin{figure}[t]
\begin{AsmCodeListing}[frame=single, numbers=left, commandchars=\\\{\}]
 segment .data
 ; definiere Array aus 10 Doppelw\"{o}rtern initialisiert mit 1,2,..,10
 a1           dd    1, 2, 3, 4, 5, 6, 7, 8, 9, 10
 ; definiere Array aus 10 W\"{o}rtern initialisiert mit 0
 a2           dw    0, 0, 0, 0, 0, 0, 0, 0, 0, 0
 ; das Gleiche wie zuvor unter Benutzung von TIMES
 a3           times 10 dw 0
 ; definiere Array aus Bytes mit 200 0en und dann 100 1en
 a4           times 200 db 0
              times 100 db 1

 segment .bss
 ; definiere einen Array aus 10 uninitialisierten Doppelw\"{o}rtern
 a5           resd  10
 ; definiere einen Array aus 100 uninitialisierten W\"{o}rtern
 a6           resw  100
\end{AsmCodeListing}
\caption{Arrays definieren \label{fig:DataArrays}}
\end{figure}

\subsubsection{Arrays im {\code data} und {\code bss} Segment definieren
\index{Arrays!Definition!statisch}}

Um einen initialisierten Array im {\code data} Segment zu
definieren, benutzt man die normalen {\code db}, {\code dw}, usw.\/
\index{Direktive!D\emph{x}}Direktiven. NASM stellt auch eine
n\"{u}tzliche Direktive namens {\code TIMES} \index{Direktive!TIMES} zur
Verf\"{u}gung, die verwendet werden kann, um eine Anweisung viele Male
zu wiederholen, ohne die Anweisung von Hand duplizieren zu m\"{u}ssen.
Abbildung~\ref{fig:DataArrays} zeigt verschiedene Bespiele dazu.

Um einen uninitialisierten Array im {\code bss} Segment zu
definieren, benutzt man die {\code resb}, {\code resw},
\index{Direktive!RES\emph{x}} usw.\ Direktiven. Erinnern wir uns,
dass diese Direktiven einen Operanden haben, der angibt, wie viele
Speichereinheiten zu reservieren sind.
Abbildung~\ref{fig:DataArrays} zeigt ebenso Beispiele dieses Typs
von Definitionen.

\subsubsection{Arrays als lokale Variable auf dem Stack definieren
\index{Arrays!Definition!lokale Variable}}

Es gibt keinen direkten Weg, eine lokale Arrayvariable auf dem Stack
zu definieren. Wie zuvor berechnet man die gesamte Bytezahl, die f\"{u}r
\emph{alle} lokalen Variable ben\"{o}tigt werden, einschlie{\ss}lich Arrays
und zieht dies von ESP (entweder direkt oder unter Verwendung des
{\code ENTER} Befehls) ab. Wenn eine Funktion zum Beispiel eine
Charaktervariable br\"{a}uchte, zwei Doppelwortinteger und einen
50-elementigen Wortarray, w\"{u}rde man  $1 + 2 \times 4 + 50 \times 2 =
109$ Byte ben\"{o}tigen. Jedoch sollte die von ESP subtrahierte Zahl ein
Vielfaches von vier sein (112 in diesem Fall), um ESP auf einer
Doppelwortgrenze zu halten. Man k\"{o}nnte die Variablen innerhalb
dieser 109 Byte auf verschiedene Weisen anordnen.
Abbildung~\ref{fig:StackLayouts} zeigt zwei m\"{o}gliche Arten. Der
unbenutzte Teil der ersten Anordnung ist dazu da, die Doppelw\"{o}rter
auf Doppelwortgrenzen zu halten, um die Speicherzugriffe zu
beschleunigen. \index{Arrays!Definition|)}

\begin{figure}[ht]
\centering
\begin{tabular}{l|c|ll|c|}
 \cline{2-2} \cline{5-5}
 EBP - 1   & char      & \hspace{2em} &           & \\
 \cline{2-2}
           & unbenutzt &              &           & \\
 \cline{2-2}
 EBP - 8   & dword 1   &              &           & \\
 \cline{2-2}
 EBP - 12  & dword 2   &              &           & word \\
 \cline{2-2}
           &           &              &           & Array \\
           &           &              &           & \\
           & word      &              &           & \\
           & Array     &              & EBP - 100 & \\
 \cline{5-5}
           &           &              & EBP - 104 & dword 1 \\
 \cline{5-5}
           &           &              & EBP - 108 & dword 2 \\
 \cline{5-5}
           &           &              & EBP - 109 & char \\
 \cline{5-5}
 EBP - 112 &           &              &           & unbenutzt \\
 \cline{2-2} \cline{5-5}
\end{tabular}
\caption{Anordnungen des Stacks \label{fig:StackLayouts}}
\end{figure}

\subsection{Auf Elemente des Arrays zugreifen
\index{Arrays!Zugriff|(}}

Es gibt in Assembler keinen {\code [\,]} Operator wie in C\@. Um auf
ein Element eines Arrays zuzugreifen, muss seine Adresse berechnet
werden. Betrachten wir die folgenden zwei Arraydefinitionen:
\begin{AsmCodeListing}[frame=none, numbers=left, commandchars=\\\{\}]
 array1       db     5, 4, 3, 2, 1     ; Array von Bytes
 array2       dw     5, 4, 3, 2, 1     ; Array von W\"{o}rtern
\end{AsmCodeListing}
Hier sind einige Beispiele, die diese Arrays benutzen:
\begin{AsmCodeListing}[frame=none, numbers=left, firstnumber=last, commandchars=\\\{\}]
      mov    al, [array1]     ; al = array1[0]
      mov    al, [array1 + 1] ; al = array1[1]
      mov    [array1 + 3], al ; array1[3] = al
      mov    ax, [array2]     ; ax = array2[0]
      mov    ax, [array2 + 2] ; ax = array2[1] (NICHT array2[2]!)   \label{line:ArrayRef1}
      mov    [array2 + 6], ax ; array2[3] = ax
      mov    ax, [array2 + 1] ; ax = ??                             \label{line:ArrayRef2}
\end{AsmCodeListing}
In Zeile~\ref{line:ArrayRef1} wird Element 1 des Wortarrays
referenziert, nicht Element 2. Warum? W\"{o}rter sind zwei-Byte
Einheiten, so muss man zwei Bytes weitergehen, um sich zum n\"{a}chsten
Element in einem Wortarray zu bewegen, nicht eins.
Zeile~\ref{line:ArrayRef2} liest ein Byte vom ersten Element und
eins vom zweiten. In C schaut der Compiler auf den Typ eines
Zeigers, um zu bestimmen, wie viele Bytes er in einem Ausdruck, der
Zeiger Arithmetik verwendet, voranschreiten muss, sodass es der
Programmierer nicht tun muss. Jedoch liegt es in Assembler beim
Programmierer, die Gr\"{o}{\ss}e der Arrayelemente zu ber\"{u}cksichtigen, wenn
er sich von Element zu Element bewegt.

\begin{figure}[ht]
\begin{AsmCodeListing}[frame=single, numbers=left, commandchars=\\\{\}]
      mov    ebx, array1      ; ebx = Adresse von array1
      mov    dx, 0            ; dx enth\"{a}lt die Summe
      mov    ah, 0            ; ?                       \label{line:SumArray10}
      mov    ecx, 5
 lp:
      mov    al, [ebx]        ; al = *ebx               \label{line:SumArray11}
      add    dx, ax           ; dx += ax (nicht al!)    \label{line:SumArray12}
      inc    ebx              ; ebx++
      loop   lp
\end{AsmCodeListing}
\caption{Die Elemente eines Arrays zusammenz\"{a}hlen (Version 1)
\label{fig:SumArray1}}
\end{figure}

Abbildung~\ref{fig:SumArray1} zeigt ein Codefragment, das alle
Elemente von {\code array1} aus dem vorigen Beispielcode
aufsummiert. In Zeile~\ref{line:SumArray12} wird AX zu DX summiert.
Warum nicht AL? Erstens m\"{u}ssen die beiden Operanden des {\code ADD}
Befehls von der gleichen Gr\"{o}{\ss}e sein. Zweitens k\"{o}nnte es leicht
passieren, Bytes aufzusummieren und eine Summe zu erhalten, die zu
gro{\ss} war, um in ein Byte zu passen. Indem DX benutzt wird, sind
Summen bis hinauf zu 65\,535 erlaubt. Es ist jedoch wichtig, sich
klar zu machen, dass AH ebenfalls addiert wird. Das ist der Grund,
warum AH in Zeile~\ref{line:SumArray10} auf Null\footnote{Indem AH
auf Null gesetzt wird, wird implizit angenommen, dass AL eine
vorzeichenlose Zahl ist. Wenn sie vorzeichenbehaftet w\"{a}re, w\"{u}rde die
passende Aktion sein, stattdessen einen {\code CBW} Befehl zwischen
Zeilen~\ref{line:SumArray11} und \ref{line:SumArray12} einzuf\"{u}gen.}
gesetzt wurde.

\begin{figure}[t]
\begin{AsmCodeListing}[frame=single, numbers=left, commandchars=\\\{\}]
      mov    ebx, array1      ; ebx = Adresse von array1
      mov    dx, 0            ; dx enth\"{a}lt die Summe
      mov    ecx, 5
 lp:
 \textit{     add    dl, [ebx]        ; dl += *ebx}
 \textit{     jnc    next             ; if no carry goto next}
 \textit{     inc    dh               ; inc dh}
 \textit{next:}
      inc    ebx              ; ebx++
      loop   lp
\end{AsmCodeListing}
\caption{Die Elemente eines Arrays zusammenz\"{a}hlen (Version 2)
\label{fig:SumArray2}}
\end{figure}

\begin{figure}[t]
\begin{AsmCodeListing}[frame=single, numbers=left, commandchars=\\\{\}]
      mov    ebx, array1      ; ebx = Adresse von array1
      mov    dx, 0            ; dx enth\"{a}lt die Summe
      mov    ecx, 5
 lp:
 \textit{     add    dl, [ebx]        ; dl += *ebx}
 \textit{     adc    dh, 0            ; dh += carry flag + 0}
      inc    ebx              ; ebx++
      loop   lp
\end{AsmCodeListing}
\caption{Die Elemente eines Arrays zusammenz\"{a}hlen (Version 3)
\label{fig:SumArray3}}
\end{figure}

Abbildungen~\ref{fig:SumArray2} und \ref{fig:SumArray3} zeigen zwei
alternative Wege, um die Summe zu berechnen. Die Zeilen in
Schr\"{a}gschrift ersetzen Zeilen~\ref{line:SumArray11} und
\ref{line:SumArray12} von Abbildung~\ref{fig:SumArray1}.

\subsection{Fortgeschrittenere indirekte Adressierung
                 \index{indirekte Adressierung!Arrays|(}}

Es d\"{u}rfte nicht \"{u}berraschen, dass indirekte Adressierung oft mit
Arrays verwendet wird. Die allgemeinste Form einer indirekten
Speicherreferenz ist:
\begin{center}
{\code [ \emph{base reg} + \emph{factor}\,*\,\emph{index reg} +
      \emph{constant}]}
\end{center}
wobei:
\begin{description}
\parskip=-0.25em %reduce the spacing <<<<<<<<<<<<<<<<<<<<<<<<<<<<<<<<<<<<<<<<<<

\item[base reg]
eines der Register EAX, EBX, ECX, EDX, EBP, ESP, ESI oder EDI ist.

\item[factor]
ist entweder 1, 2, 4 oder 8. (Wenn 1, wird {\code factor}
weggelassen.)

\item[index reg]
ist eines der Register  EAX, EBX, ECX, EDX, EBP, ESI, EDI. (Beachte,
dass ESP nicht in der Liste ist.)

\item[constant]
ist eine 8- oder 32-bit Konstante. Die Konstante kann ein Label
(oder ein Labelausdruck) sein.
\end{description}

\subsection{Beispiel}
Hier ist ein Beispiel, das einen Array benutzt und ihn an eine
Funktion \"{u}bergibt. Es benutzt als Treiber das {\code array1c.c}
Programm (unten aufgef\"{u}hrt), nicht das {\code driver.c} Programm.
\index{array1.asm|(}
\begin{AsmCodeListing}[label=array1.asm, numbers=left, commandchars=\\\{\}]
 %define ARRAY_SIZE 100
 %define NEW_LINE 10

 segment .data
 FirstMsg        db   "First 10 elements of array", 0
 Prompt          db   "Enter index of element to display: ", 0
 SecondMsg       db   "Element %d is %d", NEW_LINE, 0
 ThirdMsg        db   "Elements 20 through 29 of array", 0
 InputFormat     db   "%d", 0

 segment .bss
 array   resd    ARRAY_SIZE

 segment .text
         extern  _puts, _printf, _scanf, _dump_line
         global  _asm_main
 _asm_main:
         enter   4, 0                   ; lokale Dword Variable bei EBP - 4
         push    ebx
         push    esi

 ; initialisiere Array mit 100, 99, 98, 97, ...

         mov     ecx, ARRAY_SIZE
         mov     ebx, array
 init_loop:
         mov     [ebx], ecx
         add     ebx, 4
         loop    init_loop

         push    dword FirstMsg         ; gebe FirstMsg aus
         call    _puts
         pop     ecx

         push    dword 10
         push    dword array
         call    _print_array           ; gebe erste 10 Elemente von array aus
         add     esp, 8

 ; frage Benutzer nach Index des Elements
 Prompt_loop:
         push    dword Prompt
         call    _printf
         pop     ecx

         lea     eax, [ebp-4]           ; eax = Adresse des lokalen Dwords
         push    eax
         push    dword InputFormat
         call    _scanf
         add     esp, 8
         cmp     eax, 1                 ; eax = R\"{u}ckgabewert von scanf
         je      InputOK

         call    _dump_line             ; bei ung\"{u}ltiger Eingabe verwerfe Rest
         jmp     Prompt_loop            ; der Zeile und beginne nochmals

 InputOK:
         mov     esi, [ebp-4]
         push    dword [array + 4*esi]
         push    esi
         push    dword SecondMsg        ; gebe Wert des Elements aus
         call    _printf
         add     esp, 12

         push    dword ThirdMsg         ; gebe Elemente 20-29 aus
         call    _puts
         pop     ecx

         push    dword 10
         push    dword array + 20*4     ; Adresse von array[20]
         call    _print_array
         add     esp, 8

         pop     esi
         pop     ebx
         mov     eax, 0                 ; kehre zu C zur\"{u}ck
         leave
         ret

 ;
 ; Routine _print_array
 ; Von C aufrufbare Routine die die Elemente eines Doppelwort-Arrays
 ; als Integer mit Vorzeichen ausgibt.
 ; C Prototyp:
 ; void print_array( const int *a, int n );
 ; Parameter:
 ;   a - Zeiger zum auszugebenden Array (bei ebp + 8 auf Stack)
 ;   n - Anzahl auszugebender Integer (bei ebp + 12 auf Stack)

 segment .data
 OutputFormat    db   "%-5d %5d", NEW_LINE, 0

 segment .text
         global  _print_array
 _print_array:
         enter   0, 0
         push    esi
         push    ebx

         xor     esi, esi               ; esi = 0
         mov     ecx, [ebp + 12]        ; ecx = n
         mov     ebx, [ebp + 8]         ; ebx = Adresse des Arrays
 print_loop:
         push    ecx                    ; printf k\"{o}nnte ecx \"{a}ndern!

         push    dword [ebx + 4*esi]    ; push array[esi]
         push    esi
         push    dword OutputFormat
         call    _printf
         add     esp, 12                ; entferne Parameter (lasse ecx!)

         inc     esi
         pop     ecx
         loop    print_loop

         pop     ebx
         pop     esi
         leave
         ret
\end{AsmCodeListing}

\LabelLine{array1c.c}
\begin{lstlisting}[numbers=left, escapeinside={@}{@}]{}
 #include <stdio.h>

 int asm_main( void );
 void dump_line( void );

 int main()
 {
   int ret_status;
   ret_status = asm_main();
   return ret_status;
 }

 /*
  * Funktion dump_line
  * verwirft alle im Eingabepuffer @\itshape{\"{u}brig}@ gebliebenen Zeichen
  */
 void dump_line()
 {
   int ch;

   while( (ch = getchar()) != EOF && ch != '\n')
     /* leerer Rumpf */ ;
 }@\\[-20pt]@
\end{lstlisting}
\LabelLine{array1c.c}
 \index{array1.asm|)}
 \index{indirekte Adressierung!Arrays|)}
 \index{Arrays!Zugriff|)}

\subsubsection{Nochmals der {\code LEA} Befehl\index{Maschinenbefehl!LEA|(}}

Der {\code LEA} Befehl kann noch f\"{u}r weitere Aufgaben verwendet
werden, als nur Adressen zu berechnen. Eine ziemlich einfache ist
f\"{u}r schnelle Berechnungen. Betrachten wir das Folgende:
\begin{AsmCodeListing}[numbers=none, frame=none]
      lea    ebx, [4*eax + eax]
\end{AsmCodeListing}
Dies speichert effektiv den Wert von $5 \times \mathtt{EAX}$ in
{\code EBX}\@. Die Verwendung von {\code LEA} f\"{u}r diesen Zweck ist
sowohl einfacher als auch schneller als die Verwendung von {\code
MUL}\@. \index{Maschinenbefehl!MUL} Jedoch muss man sich klarmachen,
dass der Ausdruck innerhalb der eckigen Klammern eine g\"{u}ltige
indirekte Adresse sein \emph{muss}. Deshalb kann dieser Befehl zum
Beispiel nicht verwendet werden, um schnell mit 6 zu multiplizieren.
\index{Maschinenbefehl!LEA|)}


\subsection{Mehrdimensionale Arrays\index{Arrays!mehrdimensionale|(}}

Mehrdimensionale Arrays unterscheiden sich nicht wirklich sehr stark
von den bereits betrachteten einfachen eindimensionalen Arrays.
Tats\"{a}chlich werden sie im Speicher als genau das repr\"{a}sentiert, als
ein einfacher eindimensionaler Array.

\subsubsection{Zweidimensionale Arrays\index{Arrays!mehrdimensionale!zwei-dimensionale|(}}
Nicht \"{u}berraschend ist der einfachste mehrdimensionale Array ein
zweidimensionaler. Ein zweidimensionaler Array wird oft als Gitter
von Elementen dargestellt. Jedes Element wird durch ein Paar von
Indizes identifiziert. Per \"{U}ber"-einkunft wird der erste Index mit
der Reihe des Elements identifiziert und der zweite Index mit der
Spalte.

Betrachten wir einen Array mit drei Reihen und zwei Spalten, der
definiert ist als:
\begin{lstlisting}[stepnumber=0]{}
  int a[3][2];
\end{lstlisting}
Der C Compiler w\"{u}rde Platz f\"{u}r einen 6 ($= 2 \times 3$) elementigen
Integerarray reservieren und die Elemente wie folgt anlegen:

\parbox{\textwidth}{
\vspace{0.5em}
\centering
\begin{tabular}{||l|c|c|c|c|c|c||}
 \hline
 Index   &    0    &    1    &    2    &    3    &    4    &    5    \\
 \hline
 Element & a[0][0] & a[0][1] & a[1][0] & a[1][1] & a[2][0] & a[2][1] \\
 \hline
\end{tabular}
\vspace{0.5em} } \noindent Was die Tabelle zu zeigen versucht, ist,
dass das Element, auf das mit {\code a[0][0]} zugegriffen wird, am
Anfang des 6-elementigen eindimensionalen Arrays gespeichert wird.
Element {\code a[0][1]} wird an der n\"{a}chsten Position (Index~1)
gespeichert und so weiter. Jede Reihe des zweidimensionalen Arrays
wird fortlaufend im Speicher abgelegt. Das letzte Element einer
Reihe wird vom ersten Element der n\"{a}chsten Reihe gefolgt. Das ist
als eine \emph{reihenweise} Repr\"{a}sentation des Arrays bekannt und
ist, wie ein C/C++ Compiler einen Array repr\"{a}sentieren w\"{u}rde.

Wie bestimmt der Compiler, wo {\code a[i][j]} in einer reihenweisen
Repr\"{a}sentation erscheint? Eine einfache Formel berechnet den Index
aus {\code i} und {\code j}. Die Formel ist in diesem Fall $2i + j$.
Es ist nicht zu schwer zu sehen, wovon sich diese Formel ableitet.
Jede Zeile ist zwei Elemente lang; so liegt das erste Element von
Reihe $i$ an der Stelle $2i$. Dann wird die Position von Spalte $j$
gefunden, indem $j$ zu $2i$ addiert wird. Diese Analyse zeigt auch,
wie die Formel f\"{u}r einen Array mit {\code N} Spalten verallgemeinert
wird: $N \times i + j$. Beachte, dass die Formel \emph{nicht} von
der Anzahl der Reihen abh\"{a}ngt.

\begin{figure}[ht]
\begin{AsmCodeListing}[numbers=left]
     mov    eax, [ebp-44]       ; ebp - 44 ist i's Platz
     sal    eax, 1              ; multipliziere i mit 2
     add    eax, [ebp-48]       ; addiere j
     mov    eax, [ebp+4*eax-40] ; ebp - 40 ist die Adresse von a[0][0]
     mov    [ebp-52], eax       ; speichere Ergebnis in x (bei ebp - 52)
\end{AsmCodeListing}
\caption{ Assemblercode f\"{u}r \lstinline|x = a[i][j]| \label{fig:aij}}
\end{figure}

Als ein Beispiel werden wir uns ansehen, wie \emph{gcc}
\index{Compiler!gcc} den folgenden Code kompiliert (unter Verwendung
des oben definierten Arrays {\code a}):
\begin{lstlisting}[stepnumber=0]{}
  x = a[i][j];
\end{lstlisting}
Abbildung~\ref{fig:aij} zeigt den Assemblercode, in den dies
\"{u}bersetzt wurde. Somit konvertiert der Compiler den Code im
Wesentlichen zu:
\begin{lstlisting}[stepnumber=0]{}
  x = *(&a[0][0] + 2*i + j);
\end{lstlisting}
und in der Tat k\"{o}nnte der Programmierer ihn in dieser Weise mit
demselben Ergebnis schreiben.

Es ist nichts Magisches an der Wahl der reihenweisen Repr\"{a}sentation
des Arrays. Eine spaltenweise Repr\"{a}sentation w\"{u}rde genauso gut
arbeiten:

\parbox{\textwidth}{
\vspace{0.5em}
\centering
\begin{tabular}{||l|c|c|c|c|c|c||}
 \hline
 Index   &    0    &    1    &    2    &    3    &    4    &    5    \\
 \hline
 Element & a[0][0] & a[1][0] & a[2][0] & a[0][1] & a[1][1] & a[2][1]  \\
 \hline
\end{tabular}
\vspace{0.5em} }

\noindent In der \emph{spaltenweisen} Repr\"{a}sentation wird jede
Spalte fortlaufend gespeichert. Element {\code [i][j]} wird an
Position $i + 3j$ gespeichert. Andere Sprachen (FORTRAN zum
Beispiel\footnote{mit 1- statt 0-basierten Indizes [Anm.\ d.\
\"{U}\@.]}) benutzen die spaltenweise Repr\"{a}sentation. Das ist wichtig,
wenn man Code mit mehreren Sprachen verbindet.
\index{Arrays!mehrdimensionale!zwei-dimensionale|)}

\subsubsection{Dimensionen \"{u}ber zwei}
Bei Dimensionen \"{u}ber zwei wird die gleiche grundlegende Idee
angewandt. Betrachten wir einen dreidimensionalen Array:
\begin{lstlisting}[stepnumber=0]{}
  int b[4][3][2];
\end{lstlisting}
Dieser Array w\"{u}rde gespeichert, wie wenn er vier zweidimensionale
Arrays, jeder mit Gr\"{o}{\ss}e {\code [3][2]} fortlaufend im Speicher w\"{a}re.
Die unten stehende Tabelle zeigt, wie er beginnt:

\begin{center} % <<< ok, here <<<<<<<<<<<<<<<<<<<<<<<<<<<<<<<<<<<<<<<<<<<<<<<<<
\parbox{\textwidth}{
\vspace{0.5em} % brings table onto next page <<<<<<<<<<<<<<<<<<<<<<<<<<<<<<<<<<
%\centering % <<< will not work here <<<<<<<<<<<<<<<<<<<<<<<<<<<<<<<<<<<<<<<<<<
\begin{tabular}{||l|c|c|c|c|c|c||}
 \hline
 Index   &      0     &      1     &      2     &      3     & 4 & 5  \\
 \hline
 Element & b[0][0][0] & b[0][0][1] & b[0][1][0] & b[0][1][1]
         & b[0][2][0]  &  b[0][2][1]  \\
 \hline
 \hline
 Index   &      6     &      7     &      8     &      9     & 10 & 11 \\
 \hline
 Element & b[1][0][0] & b[1][0][1] & b[1][1][0] & b[1][1][1]
         & b[1][2][0] & b[1][2][1] \\
 \hline
\end{tabular}
\vspace{0.5em} %helps later on filling page on 'Bespiel' <<<<<<<<<<<<<<<<<<<<<<
}
\end{center} % <<<<<<<<<<<<<<<<<<<<<<<<<<<<<<<<<<<<<<<<<<<<<<<<<<<<<<<<<<<<<<<<

\noindent Die Formel, um die Position von {\code b[i][j][k]} zu
berechnen, ist $6i + 2j + k$. Die 6 ist gegeben durch die Gr\"{o}{\ss}e des
{\code [3][2]} Arrays. Im Allgemeinen wird die Position des Elements
{\code a[i][j][k]} in einem als {\code a[L][M][N]} dimensionierten
Array $M\times N\times i + N \times j + k$ sein. Beachte wieder,
dass die erste Dimension ({\code L}) nicht in der Formel erscheint.

F\"{u}r h\"{o}here Dimensionen wird derselbe Prozess generalisiert. F\"{u}r
einen $n$ dimensionalen Array mit Dimensionen $D_1$ bis $D_n$ ist
die Position des durch die Indizes $i_1$ bis $i_n$ bezeichneten
Elements durch die Formel:
\begin{displaymath}
D_2 \times D_3 \cdots \times D_n \times i_1 + D_3 \times D_4 \cdots \times D_n
\times i_2 + \cdots + D_n \times i_{n-1} + i_n
\end{displaymath}
gegeben oder f\"{u}r die Mathefreaks kann es pr\"{a}gnanter geschrieben
werden als:
\begin{displaymath}
\sum_{j=1}^{n} \: \left( \prod_{k=j+1}^{n} D_k \right) \: i_j
\end{displaymath}
\MarginNote{Hier ist die Stelle, an der man erkennen kann, dass der
Autor Physik als Hauptfach hatte. (Oder hat ihn die Erw\"{a}hnung von
FORTRAN verraten?)} Die erste Dimension, $D_1$, tritt in der Formel
nicht auf.

F\"{u}r die spaltenweise Repr\"{a}sentation, w\"{a}re die allgemeine Formel:
\begin{displaymath}
i_1 + D_1 \times i_2 + \cdots + D_1 \times D_2 \times \cdots \times D_{n-2}
\times i_{n-1} + D_1 \times D_2 \times \cdots \times D_{n-1} \times i_n
\end{displaymath}
oder in der Notation f\"{u}r Mathefreaks:
\begin{displaymath}
\sum_{j=1}^{n} \: \left( \prod_{k=1}^{j-1} D_k \right) \: i_j
\end{displaymath}
In diesem Fall ist es die letzte Dimension, $D_n$, die in der Formel
nicht auftritt.

\subsubsection{Die \"{U}bergabe mehrdimensionaler Arrays als Parameter in C
\index{Arrays!mehrdimensionale!Parameter|(}}

Die reihenweise Repr\"{a}sentation mehrdimensionaler Arrays hat einen
direkten Einfluss auf die C Programmierung. F\"{u}r eindimensionale
Arrays wird die Gr\"{o}{\ss}e des Arrays nicht ben\"{o}tigt, um zu berechnen, wo
irgendein spezifisches Element im Speicher liegt. Das trifft auf
mehrdimensionale Arrays nicht zu. Um auf die Elemente dieser Arrays
zuzugreifen, muss der Compiler alle au{\ss}er der ersten Dimension
kennen. Dies wird offenbar, wenn man den Prototypen einer Funktion
betrachtet, die einen mehrdimensionalen Array als Parameter hat. Das
Folgende wird nicht kompiliert:
%\begin{small}
\begin{lstlisting}[stepnumber=0, escapeinside={@}{@}]{}
  void f( int a[@\(\!\,\,\)@][@\(\,\,\!\)@] );  /* keine Dimensionsinformation */
\end{lstlisting}
%\end{small}
Jedoch wird das Folgende kompiliert:
%\begin{small}
\begin{lstlisting}[stepnumber=0, escapeinside={@}{@}]{}
  void f( int a[@\(\!\,\,\)@][2] );
\end{lstlisting}
%\end{small}

Jeder zweidimensionale Array mit zwei Spalten kann an diese Funktion
\"{u}bergeben werden. Die erste Dimension wird nicht
ben\"{o}tigt.\footnote{Eine Gr\"{o}{\ss}e kann hier angegeben werden, wird aber
vom Compiler ignoriert.}

Nicht verwirren lassen durch eine Funktion mit diesem Prototypen:
\begin{lstlisting}[stepnumber=0, escapeinside={@}{@}]{}
  void f( int *a[@\(\,\,\)@] );
\end{lstlisting}
Dies definiert einen eindimensionalen Array von Integerzeigern (der
nebenbei dazu verwendet werden kann, um einen Array von Arrays zu
schaffen, der sich ganz so wie ein zweidimensionaler Array verh\"{a}lt).

F\"{u}r h\"{o}herdimensionale Arrays m\"{u}ssen bei Parametern alle au{\ss}er der
ersten Dimension angegeben werden. Zum Beispiel k\"{o}nnte ein
vierdimensionaler Array so \"{u}bergeben werden:
\begin{lstlisting}[stepnumber=0, escapeinside={@}{@}]{}
  void f( int a[@\(\!\!\,\)@][4][3][2] );
\end{lstlisting}
\index{Arrays!mehrdimensionale!Parameter|)}
\index{Arrays!mehrdimensionale|)}

\section{Array/String Befehle}
\index{String Befehle|(}

Die 80x86 Familie von Prozessoren stellt verschiedene Befehle, die
f\"{u}r die Arbeit mit Arrays geschaffen wurden, zur Verf\"{u}gung. Diese
Befehle werden \emph{Stringbefehle} genannt. Sie benutzen die
Indexregister (ESI und EDI) \index{Register!EDI, ESI} um eine
Operation durchzuf\"{u}hren und erh\"{o}hen oder vermindern dann automatisch
eines oder beide der Indexregister. Das \emph{Richtungsflag}
(\emph{direction flag}, DF) \index{Register!FLAGS!DF} im FLAGS
Register bestimmt, ob die Indexregister erh\"{o}ht oder vermindert
werden. Es gibt zwei Befehle, die das Richtungsflag \"{a}ndern:
\begin{description}
\parskip=-0.25em %reduce the spacing <<<<<<<<<<<<<<<<<<<<<<<<<<<<<<<<<<<<<<<<<<

\item[CLD]
\index{Maschinenbefehl!CLD} l\"{o}scht das Richtungsflag. In diesem
Zustand werden die Indexregister erh\"{o}ht.

\item[STD]
\index{Maschinenbefehl!STD} setzt das Richtungsflag. In diesem
Zustand werden die Indexregister vermindert.
\end{description}
Ein \emph{sehr} verbreitertes Versehen in der 80x86 Programmierung
ist, zu vergessen, das Richtungsflag explizit in den richtigen
Zustand zu setzen. Das f\"{u}hrt oft zu Code, der die meiste Zeit
funktioniert (wenn sich das Richtungsflag zuf\"{a}llig im gew\"{u}nschten
Zustand befindet), aber er funktioniert nicht \emph{immer}.

\begin{figure}[t]
\centering
{\code
\begin{tabular}{|lp{1.5in}|lp{1.5in}|}
 \hline
 LODSB & AL = [DS:ESI]\newline ESI = ESI $\pm$ 1 &
 STOSB & [ES:EDI] = AL\newline EDI = EDI $\pm$ 1 \\
 \hline
 LODSW & AX = [DS:ESI]\newline ESI = ESI $\pm$ 2 &
 STOSW & [ES:EDI] = AX\newline EDI = EDI $\pm$ 2 \\
 \hline
 LODSD & EAX = [DS:ESI]\newline ESI = ESI $\pm$ 4 &
 STOSD & [ES:EDI] = EAX\newline EDI = EDI $\pm$ 4 \\
 \hline
\end{tabular}
} \caption{Lesende und schreibende Stringbefehle
\label{fig:rwString} \index{Maschinenbefehl!LODS\emph{x}}
\index{Maschinenbefehl!STOS\emph{x}}}
\end{figure}

\begin{figure}[t]
\begin{AsmCodeListing}[frame=single, numbers=left, commandchars=\\\{\}]
 segment .data
 array1  dd     1, 2, 3, 4, 5, 6, 7, 8, 9, 10

 segment .bss
 array2  resd   10

 segment .text
         cld                     ; dies nicht vergessen!
         mov    esi, array1
         mov    edi, array2
         mov    ecx, 10
 lp:                   \label{line:lodEx0}
         lodsd         \label{line:lodEx1}
         stosd         \label{line:lodEx2}
         loop   lp     \label{line:lodEx3}
\end{AsmCodeListing}
\caption{Load und store Beispiel\label{fig:lodEx}}
\end{figure}

\subsection{ Speicherbereiche lesen und schreiben}

Die einfachsten Stringbefehle lesen entweder aus oder schreiben in
den Speicher oder beides. Sie k\"{o}nnen auf einmal ein Byte, Wort oder
Doppelwort lesen oder schreiben. Abbildung~\ref{fig:rwString} zeigt
diese Befehle mit einer kurzen Beschreibung in Pseudocode dessen was
sie tun. Es gibt hier verschiedene Punkte zu beachten. Zuerst wird
ESI zum Lesen verwendet und EDI zum Schreiben. Es ist einfach, sich
an das zu erinnern, wenn man bedenkt, dass SI f\"{u}r \emph{Source
Index} und DI f\"{u}r \emph{Destination Index} \index{Register!EDI, ESI}
steht. Als n\"{a}chstes muss man beachten, dass das Register, das die
Daten h\"{a}lt, festgelegt ist (entweder AL, AX oder EAX). Schlie{\ss}lich
muss man beachten, dass die speichernden Befehle ES benutzen, um das
Segment zu bestimmen in das sie schreiben, nicht DS\@. In der
protected Mode Programmierung ist dies gew\"{o}hnlich kein Problem, da
es nur ein einziges Datensegment gibt und ES automatisch
initialisiert sein sollte um sich auf dieses zu beziehen (genauso
wie DS es ist). In der real Mode Programmierung jedoch, ist es f\"{u}r
den Programmierer \emph{sehr} wichtig, ES mit dem korrekten
\index{Register!Segment} Segmentwert\footnote{Eine weitere
Komplikation ist, dass man, unter Benutzung eines einzelnen {\code
MOV} Befehls, den Wert des DS Registers nicht direkt in das ES
Register kopieren kann. Stattdessen muss der Wert von DS in ein
Allzweckregister (wie AX) kopiert werden, um dann aus diesem
Register nach ES kopiert zu werden, unter Verwendung von zwei {\code
MOV} Befehlen.} zu initialisieren. Abbildung~\ref{fig:lodEx} zeigt
ein Beispiel der Benutzung dieser Befehle, das einen Array in einen
anderen kopiert.

\begin{figure}[t]
\centering
{\code
\begin{tabular}{|lp{2.5in}|}
 \hline
 MOVSB & byte [ES:EDI] = byte [DS:ESI] \newline
        ESI = ESI $\pm$ 1 \newline
        EDI = EDI $\pm$ 1 \\
 \hline
 MOVSW & word [ES:EDI] = word [DS:ESI] \newline
        ESI = ESI $\pm$ 2 \newline
        EDI = EDI $\pm$ 2 \\
 \hline
 MOVSD & dword [ES:EDI] = dword [DS:ESI] \newline
        ESI = ESI $\pm$ 4 \newline
        EDI = EDI $\pm$ 4 \\
 \hline
\end{tabular}
} \caption{Die Memory move String Befehle \label{fig:movString}
\index{Maschinenbefehl!MOVS\emph{x}}}
\end{figure}

Die Kombination eines {\code LODS\emph{x}} mit einem {\code
STOS\emph{x}} Befehl (wie in Zeilen~\ref{line:lodEx1} und
\ref{line:lodEx2}  von Abbildung~\ref{fig:lodEx}) ist sehr
verbreitet. Tats\"{a}chlich kann diese Kombination mit einem einzelnen
{\code MOVS\emph{x}} Stringbefehl durchgef\"{u}hrt werden.
Abbildung~\ref{fig:movString} beschreibt die Operationen, die diese
Befehle ausf\"{u}hren. Zeilen~\ref{line:lodEx1} und \ref{line:lodEx2}
von Abbildung~\ref{fig:lodEx} k\"{o}nnten mit dem gleichen Effekt durch
einen einzelnen {\code MOVSD} Befehl ersetzt werden. Der einzige
Unterschied w\"{a}re, dass das EAX Register in der Schleife \"{u}berhaupt
nicht verwendet werden w\"{u}rde.

\subsection{Das {\code REP} Befehlspr\"{a}fix
\index{Maschinenbefehl!REP|(}}

Die 80x86 Familie stellt ein spezielles Befehlspr\"{a}fix\footnote{Ein
Befehlspr\"{a}fix ist kein Befehl, es ist ein spezielles Byte, das vor
einen Stringbefehl gesetzt wird, um das Verhalten des Befehls zu
modifizieren. Andere Pr\"{a}fixe werden auch benutzt, um die
Segmentvoreinstellungen f\"{u}r die Speicherzugriffe zu \"{u}berschreiben.},
{\code REP} genannt, zur Verf\"{u}gung, das mit den obigen
Stringbefehlen verwendet werden kann. Dieses Pr\"{a}fix sagt der CPU,
den n\"{a}chsten Stringbefehl eine gegebene Anzahl mal zu wiederholen.
Das ECX Register wird benutzt, um die Iterationen zu z\"{a}hlen (genauso
wie bei einem {\code LOOP} Befehl). Unter Benutzung des {\code REP}
Pr\"{a}fixes k\"{o}nnte die Schleife in Abbildung~\ref{fig:lodEx}
(Zeilen~\ref{line:lodEx0} bis \ref{line:lodEx3}) durch eine einzelne
Zeile ersetzt werden:
\begin{AsmCodeListing}[frame=none, numbers=none]
      rep movsd
\end{AsmCodeListing}
Abbildung~\ref{fig:zeroArrayEx} zeigt ein weiteres Beispiel, das den
Inhalt eines Arrays l\"{o}scht. \index{Maschinenbefehl!REP|)}

\begin{figure}[ht]
\begin{AsmCodeListing}[frame=single, numbers=left]
 segment .bss
 array   resd   10

 segment .text
         cld                     ; dies nicht vergessen!
         mov    edi, array
         mov    ecx, 10
         xor    eax, eax
         rep stosd
\end{AsmCodeListing}
\caption{Beispiel einen Array zu l\"{o}schen\label{fig:zeroArrayEx}}
\end{figure}

\begin{figure}[ht]
\centering
{\code
\begin{tabular}{|lp{3.5in}|}
 \hline CMPSB & vergleicht Byte [DS:ESI] mit Byte [ES:EDI]
 \newline ESI = ESI $\pm$ 1
        \newline EDI = EDI $\pm$ 1 \\
 \hline
 CMPSW & vergleicht Word [DS:ESI] mit Word [ES:EDI] \newline ESI = ESI $\pm$ 2
        \newline EDI = EDI $\pm$ 2 \\
 \hline
 CMPSD & vergleicht Dword [DS:ESI] mit Dword [ES:EDI] \newline ESI = ESI $\pm$ 4
        \newline EDI = EDI $\pm$ 4 \\
 \hline
 SCASB & vergleicht AL mit [ES:EDI] \newline EDI $\pm$ 1 \\
 \hline
 SCASW & vergleicht AX mit [ES:EDI] \newline EDI $\pm$ 2 \\
 \hline
 SCASD & vergleicht EAX mit [ES:EDI] \newline EDI $\pm$ 4 \\
 \hline
\end{tabular}
} \caption{Vergleichende Stringbefehle \label{fig:cmpString}}
\index{Maschinenbefehl!CMPS\emph{x}}
\index{Maschinenbefehl!SCAS\emph{x}}
\end{figure}

\begin{figure}[ht]
\begin{AsmCodeListing}[frame=single, numbers=left, commandchars=\\\{\}]
 segment .bss
 array resd  100

 segment .text
      cld
      mov    edi, array       ; Zeiger zum Anfang des Arrays
      mov    ecx, 100         ; Anzahl Elemente
      mov    eax, 12          ; zu suchende Zahl
 lp:
      scasd                                                     \label{line:scasdSrchStrEx}
      je     found
      loop   lp
      ; Code auszuf\"{u}hren, wenn nicht gefunden
      jmp    onward
 found:
      sub    edi, 4           ; edi zeigt nun zur 12 im Array   \label{line:subSrchStrEx}
      ; Code auszuf\"{u}hren, wenn gefunden
 onward:
\end{AsmCodeListing}
\caption{Suchbeispiel \label{fig:srchStrEx}}
\end{figure}

\subsection{Vergleichende Stringbefehle}

Abbildung~\ref{fig:cmpString} zeigt verschiedene neue Stringbefehle,
die verwendet werden k\"{o}nnen, um Speicher mit anderem Speicher oder
einem Register zu vergleichen. Sie sind n\"{u}tzlich, um Arrays zu
vergleichen oder zu durchsuchen. Sie setzen das FLAGS Register
genauso wie der {\code CMP} Befehl. Die {\code CMPS\emph{x}} Befehle
vergleichen entsprechende Speicherstellen und die {\code
SCAS\emph{x}} \index{Maschinenbefehl!SCAS\emph{x}} suchen
Speicherstellen nach einem bestimmten Wert ab.

Abbildung~\ref{fig:srchStrEx} zeigt ein kurzes Codefragment, das die
Zahl 12 in einem Doppelwortarray sucht. Der {\code SCASD} Befehl in
Zeile~\ref{line:scasdSrchStrEx} addiert immer 4 zu EDI, sogar wenn
der gesuchte Wert gefunden wurde. Folglich, wenn man w\"{u}nscht, die
Adresse der im Array gefundenen 12 zu erhalten, ist es notwendig, 4
von EDI \index{Register!EDI, ESI} abzuziehen (wie in
Zeile~\ref{line:subSrchStrEx} getan).

\begin{figure}[ht]
\centering
\begin{tabular}{|l|p{4in}|}
 \hline
 {\code REPE}, {\code REPZ} & wiederhole Befehl solange ZF gesetzt ist,
        aber h\"{o}chstens ECX mal \\
 \hline
 {\code REPNE}, {\code REPNZ} & wiederhole Befehl solange ZF gel\"{o}scht ist,
        aber h\"{o}chstens ECX mal \\
 \hline
\end{tabular}
\caption{Die {\code REP\emph{x}} Befehls-Pr\"{a}fixe \label{fig:repx}
          \index{Maschinenbefehl!REPE/REPZ}
          \index{Maschinenbefehl!REPNE/REPNZ}}
\end{figure}

\begin{figure}[ht]
\begin{AsmCodeListing}[frame=single, numbers=left, commandchars=\\\{\}]
 segment .text
      cld
      mov    esi, block1      ; Adresse des ersten Blocks
      mov    edi, block2      ; Adresse des zweiten Blocks
      mov    ecx, size        ; Gr\"{o}{\ss}e der Bl\"{o}cke in Byte
      repe   cmpsb            ; wiederhole, solange ZF gesetzt
      je     equal            ; Bl\"{o}cke sind gleich, wenn ZF = 1 \label{line:cmpBlocksEx}
      ; Code ausf\"{u}hren, wenn Bl\"{o}cke nicht gleich sind
      jmp    onward
 equal:
      ; Code ausf\"{u}hren, wenn gleich
 onward:
\end{AsmCodeListing}
\caption{Speicherbl\"{o}cke vergleichen \label{fig:cmpBlocksEx}}
\end{figure}

\subsection{Die {\code REP\emph{x}} Befehlspr\"{a}fixe}

Es gibt verschiedene andere {\code REP}-\"{a}hnliche Befehlspr\"{a}fixe, die
mit den vergleichenden Stringbefehlen verwendet werden k\"{o}nnen.
Abbildung~\ref{fig:repx} zeigt die beiden neuen Pr\"{a}fixe und
beschreibt ihre Operationen. {\code REPE}
\index{Maschinenbefehl!REPE/REPZ} und {\code REPZ} sind nur
Sy"-nonyme f\"{u}r dasselbe Pr\"{a}fix (so wie {\code REPNE}
\index{Maschinenbefehl!REPNE/REPNZ} und {\code REPNZ}). Wenn der
wiederholte vergleichende Stringbefehl auf Grund des Vergleichs
stoppt, wird das oder die Indexregister noch erh\"{o}ht und ECX
vermindert; jedoch h\"{a}lt das FLAGS Register noch den Zustand, der die
Wiederholung beendete. \MarginNote{Warum kann man nicht einfach
nachsehen, ob ECX nach dem wiederholten Vergleich Null ist?} So ist
es m\"{o}glich, das Z Flag zu benutzen, um festzustellen, ob die
wiederholten Vergleiche auf Grund eines Vergleichs oder weil ECX
Null wurde, beendet wurden.

Abbildung~\ref{fig:cmpBlocksEx} zeigt als Beispiel ein Codefragment,
das bestimmt, ob zwei Speicherbl\"{o}cke gleich sind. Das {\code JE} in
Zeile~\ref{line:cmpBlocksEx} des Beispiels testet, um das Ergebnis
des vorangehenden Befehls zu sehen. Wenn der wiederholte Vergleich
anhielt, da er zwei ungleiche Bytes fand, wird das Z Flag immer noch
gel\"{o}scht sein und es wird kein Sprung durchgef\"{u}hrt; wenn die
Vergleiche jedoch anhielten, weil ECX Null wurde, wird das Z Flag
immer noch gesetzt sein und der Code verzweigt zum {\code equal}
Label.

\subsection{Beispiel}

Dieser Abschnitt enth\"{a}lt eine Assembler Quelldatei mit mehreren
Funktionen, die Arrayoperationen mit den Stringbefehlen ausf\"{u}hren.
Viele der Funktionen duplizieren bekannte C Bibliotheksfunktionen.
\pagebreak
\index{memory.asm|(}
\begin{AsmCodeListing}[label=memory.asm, numbers=left, commandchars=\\\{\}]
 global _asm_copy, _asm_find, _asm_strlen, _asm_strcpy

 segment .text
 ; Funktion _asm_copy
 ; kopiert einen Speicherblock
 ; C Prototyp:
 ; void asm_copy( void *dest, const void *src, unsigned sz );
 ; Parameter:
 ;   dest - Zeiger zum Ziel-Puffer
 ;   src  - Zeiger zum Quell-Puffer
 ;   sz   - Anzahl der zu kopierenden Bytes

 ; als n\"{a}chstes werden einige hilfreiche Symbole definiert

 %define dest [ebp+8]
 %define src  [ebp+12]
 %define sz   [ebp+16]
 _asm_copy:
         enter   0, 0
         push    esi
         push    edi

         mov     esi, src        ; esi = Adresse des Quell-Puffers
         mov     edi, dest       ; edi = Adresse des Ziel-Puffers
         mov     ecx, sz         ; ecx = Anzahl zu kopierender Bytes

         cld                     ; l\"{o}sche Richtungsflag
         rep     movsb           ; f\"{u}hre movsb ECX mal aus

         pop     edi
         pop     esi
         leave
         ret


 ; Funktion _asm_find
 ; durchsucht Speicher nach einem gegebenen Byte
 ; void *asm_find( const void *src, char target, unsigned sz );
 ; Parameter:
 ;   src    - Zeiger zum zu durchsuchenden Puffer
 ;   target - zu suchender Bytewert
 ;   sz     - Anzahl der Bytes im Puffer
 ; R\"{u}ckgabewert:
 ;   wenn target gefunden wird, wird der Zeiger zum ersten Auftreten
 ;     von target im Puffer zur\"{u}ckgegeben
 ;   sonst
 ;     wird NULL zur\"{u}ckgegeben
 ; Hinweis: target ist ein Bytewert, wird aber als Doppelwort auf den Stack geschoben.
 ;          Der Bytewert wird in den niederen 8 Bits gespeichert.
 ;
 ^\pagebreak[4]  ^% here are not two empty lines !!! \pagebreak needs one, the comment the other
 ^\pagebreak will not work here, an (additional!!!) empty line  forces the page to break


 %define src    [ebp+8]
 %define target [ebp+12]
 %define sz     [ebp+16]

 _asm_find:
         enter   0, 0
         push    edi

         mov     eax, target     ; zu suchender Wert in al
         mov     edi, src
         mov     ecx, sz
         cld

         repne   scasb           ; scan bis ECX == 0 oder [ES:EDI] == AL

         je      found_it        ; wenn ZF gesetzt, wurde Wert gefunden
         mov     eax, 0          ; wenn nicht gefunden, gebe NULL zur\"{u}ck
         jmp     short quit
 found_it:
         mov     eax, edi
         dec     eax             ; wenn gefunden, gebe (EDI - 1) zur\"{u}ck
 quit:
         pop     edi
         leave
         ret

 ; Funktion _asm_strlen
 ; liefert die Gr\"{o}{\ss}e eines Strings
 ; unsigned asm_strlen( const char * );
 ; Parameter:
 ;   src - Zeiger zum String
 ; R\"{u}ckgabewert:
 ;   Anzahl Zeichen im String (ohne 0 am Ende) (in EAX)

 %define src [ebp+8]
 _asm_strlen:
         enter   0, 0
         push    edi

         mov     edi, src        ; edi = Zeiger zum String
         mov     ecx, 0FFFFFFFFh ; benutze gr\"{o}{\ss}tm\"{o}gliches ECX
         xor     al, al          ; al = 0
         cld

         repnz   scasb           ; durchsuche nach 0 Terminator
\enlargethispage{\baselineskip} ^% will  n o t  work!!!!! <<<<<<<<<<<<<<<<<<<<<
^% here is not an empty line, this damned comment needs one to gobble <<<<<<<<<

 ;
 ; repnz geht einen Schritt zu weit, deshalb ist L\"{a}nge FFFFFFFE - ECX,
 ; nicht FFFFFFFF - ECX
 ;
         mov     eax, 0FFFFFFFEh
         sub     eax, ecx        ; length = 0FFFFFFFEh - ecx

         pop     edi
         leave
         ret

 ; Funktion _asm_strcpy
 ; kopiert einen String
 ; void asm_strcpy( char *dest, const char *src );
 ; Parameter:
 ;   dest - Zeiger zum Ziel-String
 ;   src  - Zeiger zum Quell-String
 ;
 %define dest [ebp+8]
 %define src  [ebp+12]
 _asm_strcpy:
         enter   0, 0
         push    esi
         push    edi

         mov     edi, dest
         mov     esi, src
         cld
 cpy_loop:
         lodsb                   ; lade AL & inc ESI
         stosb                   ; speichere AL & inc EDI
         or      al, al          ; setze Bedingungsflags
         jnz     cpy_loop        ; wenn nicht hinter 0 Terminator, weiter

         pop     edi
         pop     esi
         leave
         ret
\end{AsmCodeListing}

\LabelLine{memex.c}
\begin{lstlisting}[numbers=left, escapeinside={@}{@}]{}
 #include <stdio.h>

 #define STR_SIZE 30
 /* Prototypen */

 void asm_copy( void *, const void *, unsigned ) __attribute__((cdecl));
 void *asm_find( const void *,
                 char target, unsigned ) __attribute__((cdecl));
 unsigned asm_strlen( const char * ) __attribute__((cdecl));
 void asm_strcpy( char *, const char * ) __attribute__((cdecl));
@\pagebreak@
 int main()
 {
   char st1[STR_SIZE] = "test string";
   char st2[STR_SIZE];
   char *st;
   char  ch;

   asm_copy(st2, st1, STR_SIZE); /* kopiere alle 30 Zeichen des Strings */
   printf("%s\n", st2);

   printf("Enter a char: "); /* suche nach Byte im String */
   scanf("%c%*[^\n]", &ch);
   st = asm_find(st2, ch, STR_SIZE);
   if ( st )
     printf("Found it: %s\n", st);
   else
     printf("Not found\n");

   st1[0] = 0;
   printf("Enter string:");
   scanf("%s", st1);
   printf("len = %u\n", asm_strlen(st1));

   asm_strcpy(st2, st1); /* kopiere nur bedeutungsvolle Daten im String */
   printf("%s\n", st2);

   return 0;
 }
\end{lstlisting}
\LabelLine{memex.c}
\index{memory.asm|)}
\index{String Befehle|)}
\index{Arrays|)}

% -*-latex-*-
\chapter{浮點\index{浮點|(}}

\section{浮點表示法\index{浮點!表示法|(}}

\subsection{非整形的二進位數字}

在第一章討論數制的時候,我們只討論了整形。顯然,和十進位一樣,其他進制必須也能表示非整形數。在十進位中,在小數點右邊的數字關聯了10的負乘方值:
\[ 0.123 = 1 \times 10^{-1} + 2 \times 10^{-2} + 3 \times 10^{-3} \]

不必驚訝,二進位也是以同樣的方法表示:
\[ 0.101_2 = 1 \times 2^{-1} + 0 \times 2^{-2} + 1 \times 2^{-3} = 0.625 \]
這個辦法與第一章中的整形辦法相結合就可以用來轉換一個一般數值:
\[ 110.011_2 = 4 + 2 + 0.25 + 0.125 = 6.375 \]

將十進位轉換成二進位也不是很難。一般來說,需將十進位數字分成兩塊:整數部分和分數部分。使用第一章中的方法來將整數部分轉換成二進位。分數部分的轉換可以使用下面描述的方法。

\begin{figure}[t]
\centering
\fbox{
\begin{tabular}{p{2in}p{2in}}
\begin{eqnarray*}
0.5625 \times 2 & = & 1.125 \\
0.125 \times 2 & = & 0.25 \\
0.25 \times 2 & = & 0.5 \\
0.5 \times 2 & = & 1.0 \\
\end{eqnarray*}
&
\begin{eqnarray*}
\mbox{第一個比特位} & = & 1 \\
\mbox{第二個比特位} & = & 0 \\
\mbox{第三個比特位} & = & 0 \\
\mbox{第四個比特位} & = & 1 \\
\end{eqnarray*}
\end{tabular}
}
\caption{將0.5625轉換成二進位\label{fig:binConvert1}}
\end{figure}

考慮一個用$a, b, c, \ldots$標記比特位元的二進位分數。這個數用二進位表示為:
\[ 0.abcdef\ldots \]
將此數乘2.新得到的數的二進位表示將是:
\[ a.bcdef\ldots \]
注意,第一個比特位現在在權值為1的位置。用$0$替換$a$得到:
\[ 0.bcdef\ldots \]
再乘以2得到:
\[ b.cdef\ldots \]
現在第二個比特位($b$)在權值為1的位置。重複這個過程,直到得到了需要的盡可能多的比特位。圖~\ref{fig:binConvert1}展示了一個實例:將0.5625轉換成二進位。這種方法當分數部分為0了才停止。

\begin{figure}[t]
\centering
\fbox{\parbox{2in}{
\begin{eqnarray*}
0.85 \times 2 & = & 1.7 \\
0.7 \times 2 & = &  1.4 \\
0.4 \times 2 & = &  0.8 \\
0.8 \times 2 & = &  1.6 \\
0.6 \times 2 & = &  1.2 \\
0.2 \times 2 & = &  0.4 \\
0.4 \times 2 & = &  0.8 \\
0.8 \times 2 & = &  1.6 \\
\end{eqnarray*}
}}
\caption{將0.85轉換成二進位\label{fig:binConvert2}}
\end{figure}

另一個例子,將23.85轉換成二進位。將整數部分($23 = 10111_2$)轉換成二進位是容易的,但是分數部分呢?圖~\ref{fig:binConvert2}展示了這個計算的開始部分。如果你仔細看了這個數值,就會發現一個無限迴圈。這就意味著0.85是一個無限迴圈的二進位數字(與基數為10的無限迴圈十進位數字相對應)\footnote{不要大驚小怪,一個數值在一種數制下是一個無限迴圈,而在另一種數制下可能不是。考慮下
$\frac{1}{3}$,以十進位表示,它是一個無窮數,但是以三進制(基數為3)表示,它就為
$0.1_3$。}。這裏顯示了這個數的計算模式。在這個模式中,你可以看到$0.85 = 0.11\overline{0110}_2$。因此,
$23.85 = 10111.11\overline{0110}_2$。

上面計算的一個重要結論是23.85不可以用有限的比特位來\emph{精確}表示成二進位數字。
(就像$\frac{1}{3}$不能表示成有限的十進位數字。)正如這一章展示的,C語言中的{\code float}和{\code double}變數是以二進位儲存的。因此,類似23.85的數值不能精確地儲存到這些變數中。只能儲存23.85的近似值。

為了簡化硬體,採用固定的格式來儲存浮點數。這種格式採用科學計數法(但是是在二進位中,是2的乘方,不是10)。例如,23.85或 $10111.11011001100110\ldots_2$將儲存為:
\[ 1.011111011001100110\ldots \times 2^{100} \]
(其中指數(100)是二進位形式)。\emph{規範}的浮點數有下面的形式:
\[ 1.ssssssssssssssss \times 2^{eeeeeee} \]
其中$1.sssssssssssss$是\emph{有效數}而$eeeeeeee$是
\emph{指數}。

\subsection{IEEE浮點表示法\index{浮點!表示法!IEEE|(}}

IEEE(Institute of Electrical and Electronic Engineers,電氣與電子工程師學會)是一個國際組織,它已經設計了存儲浮點數的特殊的二進位格式。這種格式應用在大多數(但不是全部)現在的電腦上。通常電腦本身的硬體就支援它。例如,Intel的數學輔助運算器(從Pentium開始,就嵌入到所有它的CPU中了)就使用它。IEEE為不同的精度定義了不同的格式:單或雙精度。在C語言中,{\code float}變數使用單精確度,而{\code double}變數使用雙精度。

Intel數學輔助運算器使用第三種,更高的精度,稱為
\emph{擴展精度}。事實上,在數學輔助運算器自身裏的所有資料都是這種格式。當資料從輔助運算器儲存到記憶體中時,將自動轉換成單或雙精度。\footnote{
有些編譯器的(例如Borland) {\code long double}類型使用這種擴展精度。但是,其他的編譯器的{\code double}和{\code long double}都使用雙精度。(在ANSI C就允許這樣做。)}跟IEEE的浮點雙精度格式相比,擴展精度使用了一種有細微差別的格式,所以將不在這討論。

\subsubsection{IEEE單精確度\index{浮點!表示法!單精確度|(}}

\begin{figure}[t]
\fbox{
\centering
\parbox{5in}{
\begin{tabular}{|c|c|c|}
\multicolumn{1}{p{0.3cm}}{31} &
\multicolumn{1}{p{2.5cm}}{30 \hfill 23} &
\multicolumn{1}{p{6cm}}{22 \hfill 0} \\
\hline
s & e & f \\
\hline
\end{tabular}
\\[0.4cm]
\begin{tabular}{cp{4.5in}}
s & sign bit,符號位元 - 0 = 正數, 1 = 負數 \\
e & biased exponent,偏置指數 (8-bits) = 真實的指數 + 7F (十進位為127)。值00和FF有特殊的含義。(看正文) \\
f & fraction,分數 - 有效數中,1後面的前23個比特位。
\end{tabular}
}}
\caption{IEEE單精確度\label{fig:IEEEsingle}}
\end{figure}

單精確度浮點使用32個比特位元來編碼數位。通常它精確到小數點後七位。相比於整數,浮點數的儲存格式更複雜。圖~\ref{fig:IEEEsingle}展示了IEEE單精確度數的基本格式。這種格式有幾個古怪的地方。負的浮點數並不使用補數表示法。它們使用符號量值表示法。如圖顯示,第31位元決定數的符號。

二進位的指數並不會直接儲存。取而代之的是將指數和7F的和儲存到位23~30中。這個
\emph{偏置指數}總是非負的。

分數部分假定是一個規範的有效數(格式為
$1.sssssssss$)。因為第一個比特位總是1,所以領頭的1是
\emph{不儲存的!}這就允許在後面儲存一額外的比特位,稍微地擴展了精度。這個想法稱為
\emph{隱藏一的表示法}\index{浮點!表示法!隱藏一}.

怎樣儲存23.85呢?\MarginNote{你必須永遠記住:這些\\位元組41 BE CC CD可以用不同的方\\法解釋,使用什麼方法\\解釋取決於程式如何使\\用它們。因為,當作為\\一個單精確度浮點數時\\,它表示23.850000381\\,但是當它作為一個\\雙字整形時,它表示\\1,103,023,309!CPU並\\不知道哪種才是正確的\\解釋!} 首先,它是個正數,所以符號位元為0。其次,真實的指數為4,所以偏置指數為$7\mathrm{F}
+ 4 = 83_{16}$。最後,分數部分應表示為01111101100110011001100
(記住領頭的1是隱藏的)。把這些放到一起得到
(為了幫助澄清浮點格式的不同部分,符號位元和分數部分都加了下劃線,而且所有的比特位都分成了四個比特位一組。):
\[ \underline{0}\,100\;0001\;1
   \,\underline{011\;1110\;1100\;1100\;1100\;1100}_2 = 41 \mathrm{BE}
\mathrm{CC} \mathrm{CC}_{16} \]
這不是準確的23.85(因為它是一個無限迴圈的二進位數字)。如果你將上面的數值轉換回十進位形式,你會發現它大約等於
23.849998474。這個數與23.85非常接近,但是它並不準確。實際上,在C語言中,23.85的描述和上面的是一樣的。因為該數的精確描述被截去後的最左邊的位為1,所以最後一個比特位經四捨五入後為1。因此單精確度數23.85將表示成十六進位
41 BE CC CD。將這個轉換成十進位得23.850000381,這個數就更接近23.85。

怎麼描述-23.85呢?只需要改變符號位元得:C1 BE CC
CD。\emph{不要}使用補數!

\begin{table}[t]
\fbox{
\begin{tabular}{lp{3.1in}}
$e=0 \quad\mathrm{and}\quad f=0$ & 表示數0(它不可以被規範化)。注意這兒有+0和-0之分。\\
$e=0 \quad\mathrm{and}\quad f \neq 0$ & 表示一個\emph{非規範數}。它們將在下一節中討論。 \\
$e=\mathrm{FF} \quad\mathrm{and}\quad f=0$ &  表示無窮大($\infty$),包括正無窮大和負無窮大。\\
$e=\mathrm{FF} \quad\mathrm{and}\quad f\neq 0$ &  表示一個不可以定義的結果,稱為\emph{NaN} (Not a
Number,不是數)。
\end{tabular}
} \caption{\emph{f}和\emph{e}的特殊值\label{tab:floatSpecials}}
\end{table}

IEEE浮點格式中,\emph{e}和\emph{f}的某些組合有特殊的含義。表~\ref{tab:floatSpecials}描述了這些特殊的值。溢出或除以0將產生一個無窮數。一個無效的操作將產生一個不確定的結果,例如:試圖求一個負數的平方根,將兩個無窮數相加,\emph{等等。}

規範的單精確度數的數量級範圍為從
$1.0 \times 2^{-126}$ ($\approx 1.1755 \times 10^{-35}$) 到
$1.11111\ldots \times 2^{127}$ ($\approx 3.4028 \times 10^{35}$)。

\subsubsection{非規範化數\index{浮點!表示法!非規範|(}}

非規範化數可以用來表示那些值太小了以致於不能以規範格式描述的數(\emph{也就是}小於$1.0 \times 2^{-126}$)。例如:考慮下數$1.001_2 \times 2^{-129}$
($\approx 1.6530 \times 10^{-39}$)。在約定的規範格式中,這個指數太小了。但是,它可以用非規範的格式來描述:$0.01001_2 \times 2^{-127}$。為了儲存這個數,偏置指數被置為0(看表~\ref{tab:floatSpecials}),而且分數部分是以$2^{-127}$方式書寫得到的所有有效數
({\emph{也就是說}儲存了所有的比特位,包括小數點左邊的1).$1.001 \times 2^{-129}$將表示成:
\[ \underline{0}\,000\;0000\;0
   \,\underline{001\;0010\;0000\;0000\;0000\;0000} \]
\index{浮點!表示法!非規範|)}
\index{浮點!表示法!單精確度|)}


\subsubsection{IEEE雙精度\index{浮點!表示法!雙精度|(}}

\begin{figure}[t]
\centering
\begin{tabular}{|c|c|c|}
\multicolumn{1}{p{0.3cm}}{63} &
\multicolumn{1}{p{3cm}}{62 \hfill 52} &
\multicolumn{1}{p{7cm}}{51 \hfill 0} \\
\hline
s & e & f \\
\hline
\end{tabular}
\caption{IEEE雙精度\label{fig:IEEEdouble}}
\end{figure}

IEEE雙精度使用64位元來表示數字,而且通常精確到小數點後15位。如圖~\ref{fig:IEEEdouble}所示,基本的結構和單精確度是非常相似的。只是相比於單精確度,它使用了更多的位來描述偏置指數(11)和分數(52)。

更大範圍的偏置指數會導致兩個後果。一是計算的將是真實指數和3FF(1023)的和(而不是單精確度中的
7F)。二是,允許描述更大範圍的真實的指數(因此也可以描述更大範圍的數量級)。雙精度的數量級範圍大約為從$10^{-308}$到$10^{308}$。

雙精度值中增加的有效位元是增大分數欄位的原因。

作為一個例子,再次考慮下。偏置指令用十六進位表示為
$4 + \mathrm{3FF} = 403$。因此,該數用雙精度表示為:
\[ \underline{0}\,100\;0000\;0011\;\underline{0111\;1101\;1001\;1001\;1001\;
   1001\;1001\;1001\;1001\;1001\;1001\;1001\;1010} \]
或在十六進位中為40 37 D9 99 99 99 99 9A。如果你將它轉換回十進位,你將得到23.8500000000000014 (這有12個0!),這個數就更接近23.85。

雙精度和單精確度一樣有一些特殊的值\footnote{唯一的區別是:對於無窮數和不確定的值,偏置指數是7FF,而不是FF。}。非規範化數同樣也是一樣的。最主要的區別是雙精度的非規範數使用$2^{-1023}$替換$2^{-127}$。
\index{浮點!表示法!雙精度|)}
\index{浮點!表示法!IEEE|)}
\index{浮點!表示法|)}

\section{浮點運算\index{浮點!運算|(}}

電子電腦裏的浮點運算和持續精確的數學運算是不同的。數學中,所有的數都可以精確表示。但就如前面的章節所示,在電子電腦裏,許多數不能用有限個比特位來描述。所有的計算都在一定的精度下執行。在這節的例子中,為了簡單化,將使用8位的有效數。

\subsection{加法}
要將兩個浮點數相加,它們的指數必須是相等的。如果它們並不相等,那麼通過移動較小指數的數的有效數來使它們相等。例如:考慮$10.375 + 6.34375 = 16.71875$或在十進位中:
\[
\begin{array}{rr}
 & 1.0100110 \times 2^3 \\
+& 1.1001011 \times 2^2 \\ \hline
\end{array}
\]
這兩個數位的指數不一樣,所以通過移動有效數使指數相同,然後再相加:
\[
\begin{array}{rr@{.}l}
 &  1&0100110 \times 2^3 \\
+&  0&1100110 \times 2^3 \\ \hline
 & 10&0001100 \times 2^3
\end{array}
\]
注意,移位丟掉了$1.1001011 \times 2^2$中的末尾的1,經過四捨五入後得$0.1100110 \times 2^3$。加法的結果,$10.0001100 \times 2^3$ (or $1.00001100 \times 2^4$)等於
$10000.110_2$或16.75。這個數並\emph{不}等於準確的答案
(16.71875)!它只是一個近似值,是在進行加法操作時四捨五入後的應有誤差。

認識到在電子電腦(或計算器)裏的浮點運算得到的結果經常是近似值是非常重要的。對於電子電腦裏的浮點運算,算術法則不總是對的。算術中假定的無窮精度是任何電子電腦都無法做的。例如,算術法則告訴我們$(a + b) - b = a$;但是,在電子電腦裏,並不能完全保證它正確。

\subsection{減法}
減法和加法一樣運作,而且有和加法一樣的問題。作為一個例子,考慮$16.75 - 15.9375 = 0.8125$:
\[
\begin{array}{rr}
 & 1.0000110 \times 2^4 \\
-& 1.1111111 \times 2^3 \\ \hline
\end{array}
\]
移位$1.1111111 \times 2^3$後得到(四捨五入) $1.0000000 \times 2^4$
\[
\begin{array}{rr}
 & 1.0000110 \times 2^4 \\
-& 1.0000000 \times 2^4 \\ \hline
 & 0.0000110 \times 2^4
\end{array}
\]
$0.0000110 \times 2^4 = 0.11_2 = 0.75$ 它並不完全正確的。

\subsection{乘法和除法}

對於乘法,有效數執行乘法操作而指數執行相加操作。考慮$10.375 \times 2.5 = 25.9375$:
\[
\begin{array}{rr@{}l}
 &  1.0&100110 \times 2^3 \\
\times &  1.0&100000 \times 2^1 \\ \hline
 &     &10100110 \\
+&   10&100110   \\ \hline
 &   1.1&0011111000000 \times 2^4
\end{array}
\]
當然,真正的結果需四捨五入成8位,得:
\[1.1010000 \times 2^4 = 11010.000_2 = 26 \]

除法更複雜,但是也有同樣的四捨五入的誤差問題。

\subsection{分支程式設計}

這一節的重點是浮點運算的結果並不準確。程式師必須意識到這點。一個程式師經常犯的浮點運算錯誤就是在假定一個運算是精確的情況下,用它們去比較。例如,考慮一個執行複雜運算的\lstinline|f(x)|函式和一個求這個函式的根的程式\footnote{函式的根是滿足$f(x) = 0$條件的$x$值}。你可能會試圖用下面的語句來檢查\lstinline|x|是不是一個根:
\begin{lstlisting}[stepnumber=0]{}
  if ( f(x) == 0.0 )
\end{lstlisting}
但是,如果\lstinline|f(x)|返回$1 \times 10^{-30}$又該怎麼辦呢?這個數的最合適的含義是\lstinline|x|是一個實根的\emph{非常}好的近似值。可能沒有一個IEEE浮點值\lstinline|x|能恰好返回0,因為\lstinline|f(x)|的四捨五入誤差。

一個比較好的方法是使用:
\begin{lstlisting}[stepnumber=0]{}
  if ( fabs(f(x)) < EPS )
\end{lstlisting}
其中的\lstinline|EPS|是一個宏,定義為一個非常小的正數
(比如說$1 \times 10^{-10}$)。當\lstinline|f(x)|非常接近0時,它就為真。一般來說,一個浮點數(譬如
\lstinline|x|)和另一個浮點數(\lstinline|y|)的比較,需使用:
\begin{lstlisting}[stepnumber=0]{}
  if ( fabs(x - y)/fabs(y) < EPS )
\end{lstlisting}
\index{浮點!運算|)}

\section{數字輔助運算器}
\index{浮點輔助運算器|(}
\subsection{硬體}
\index{浮點輔助運算器!硬體|(}
早期的Intel處理器並沒有提供支援浮點操作的硬體。這並不意味著它們不可以執行浮點操作。它僅僅表示它們需要通過由許多非浮點指令組成的程式來執行這些操作。對於早期的系統,Intel提供了一片額外的稱為\emph{數學輔助運算器}的晶片。相比於使用軟體程式,數學輔助運算器擁有能快速執行許多浮點操作的機器指令(在早期的處理器上,至少快10倍!)。8086/8088的輔助運算器為8087。80286的輔助運算器為80287,80386的為80387。80486DX處理器將數學輔助運算器內置到80486中了。\footnote{但是,80486SX並\emph{沒有}內置數學輔助運算器。這些機器有分離的80487SX晶片。}從Pentium開始,所有生產的80x86\\處理器都內置數學輔助運算器;但是,它依然被規劃成好像它是一個分離的單元。即使是早期沒有輔助運算器的系統都可以安裝一個類比數學輔助運算器的軟體。當一個程式執行了一條輔助運算器指令時,這個類比套裝軟體將自動啟動並執行一個軟體程式來得到與真實輔助運算器一樣的結果(雖然,毫無疑問,會比較慢)。

數學輔助運算器有八個浮點數寄存器。每個寄存器儲存著80位元的資料。在這些寄存器中,浮點數總是儲存成80位元的擴展精度。這些寄存器稱為{\code ST0},{\code ST1},{\code
ST2},$\ldots$ {\code ST7}。浮點寄存器與主CPU中的整形寄存器的使用方法是不同的。浮點寄存器被當作一個堆疊來管理。回想一下堆疊是一個\emph{後進先出} (LIFO)佇列。{\code ST0}總是指向堆疊頂的值。所有新的數都被加入到堆疊頂中。已經存在的數被壓入到堆疊中,為了為新來的數提供空間。

在數學輔助運算器中同樣有一個狀態寄存器。它有幾個標誌位元。只有4個用來比較的標誌位元將會提到:C$_0$,
C$_1$, C$_2$ and C$_3$。這些位的使用將在以後討論。
\index{浮點輔助運算器!硬體|)}

\subsection{指令}

為了很容易地將普通的CPU指令和輔助運算器指令區分開來,所有的輔助運算器助詞符都是以{\code F}開頭。

\subsubsection{導入和儲存\index{浮點輔助運算器!資料的導入和儲存|(}}
用來將資料導入到輔助運算器寄存器堆疊頂的指令有幾條:\\
\begin{tabular}{lp{4in}}
{\code FLD \emph{source}} \index{FLD} &
從記憶體導入一個浮點數到堆疊頂。
\emph{source}可以是單,雙或擴展精度數或是一個輔助運算器寄存器。\\
{\code FILD \emph{source}} \index{FILD} &
從記憶體中讀出一個\emph{整形數},將它轉換成浮點數,再將結果儲存到堆疊頂。\emph{source}可以是字,雙字或四字。\\
{\code FLD1} \index{FLD1} &
將1儲存到堆疊頂。\\
{\code FLDZ} \index{FLDZ} &
將0儲存到堆疊頂。\\
\end{tabular}

將堆疊中的資料儲存到記憶體的指令同樣也有幾條。其中有幾條指令當它們儲存好一個數後,會將這個數從堆疊中\emph{彈出}(\emph{也就是}刪除)。
\\
\begin{tabular}{lp{4in}}
{\code FST \emph{dest}} \index{FST} &
將堆疊頂的值({\code ST0})儲存到記憶體中。
\emph{dest}可以是單,雙精度數或是一個輔助運算器寄存器。\\
{\code FSTP \emph{dest}} \index{FSTP} &
像{\code FST}一樣,將堆疊頂的值儲存到記憶體中;但是,當儲存完這個數後,它的值將被彈出堆疊。
\emph{dest}可以是單,雙或擴展精度數或是一個輔助運算器寄存器。\\
{\code FIST \emph{dest}} \index{FIST} &
將堆疊頂的值轉換成整形後再儲存到記憶體中。\emph{dest}可以是字或雙字。堆疊本身的值是不改變的。浮點數如何轉換成整形取決於輔助運算器的\emph{控制字}中的某些比特位。這是一個特殊的(非浮點)字寄存器,用來控制輔助運算器如何工作。缺省情況下,控制字會被初始化,以便於當需要轉換成整形時,它會四捨五入成最接近的整形數。但是,
{\code FSTCW} (Store Control Word,儲存控制字)和{\code FLDCW} (Load Control Word,導入控制字)指令可以用來改變這種行為。\index{FSTCW} \index{FLDCW} \\
{\code FISTP \emph{dest}} \index{FIST} & 它和{\code
FIST}是一樣,除了兩件事:堆疊頂的值會被彈出,\emph{dest}同樣可以是四字的。
\end{tabular}

同樣有兩條其他的指令用來從堆疊自身中移動或刪除資料。\\
\begin{tabular}{lp{4in}}
{\code FXCH ST\emph{n}} \index{FXCH}  &
將堆疊中的{\code ST0}的值和{\code ST\emph{n}}的值相互交換
(其中\emph{n}是一個從1到7的寄存器號)。 \\
{\code FFREE ST\emph{n}} \index{FFREE} &
通過標記寄存器為未被使用或為空來釋放堆疊中的一個寄存器。
\end{tabular}
\index{浮點輔助運算器!資料的導入和儲存|)}

\subsubsection{加法和減法\index{浮點輔助運算器!加法和減法|(}}

每一條加法指令都是計算{\code ST0}和另一個運算元的和。結果總是儲存到一個輔助運算器寄存器中。\\
\begin{tabular}{p{1.5in}p{3.5in}}
{\code FADD \emph{src}} \index{FADD} &
{\code ST0 += \emph{src}}。\emph{src}可以是任何輔助運算器寄存器或記憶體中的單或雙精度數。\\
{\code FADD \emph{dest}, ST0} &
{\code \emph{dest} += ST0}。 \emph{dest}可以是任何輔助運算器寄存器。\\
{\code FADDP \emph{dest}} or \newline {\code FADDP \emph{dest}, STO} \index{FADDP} &
{\code \emph{dest} += ST0}然後再被彈出堆疊。\emph{dest}可以是任何輔助運算器寄存器。\\
{\code FIADD \emph{src}} \index{FIADD} &
{\code ST0 += (float) \emph{src}}。{\code ST0}和一個整形相加。
\emph{src}必須是記憶體中的字或雙字。
\end{tabular}

\begin{figure}[t]
\begin{AsmCodeListing}[frame=single]
segment .bss
array        resq SIZE
sum          resq 1

segment .text
      mov    ecx, SIZE
      mov    esi, array
      fldz                  ; ST0 = 0
lp:
      fadd   qword [esi]    ; ST0 += *(esi)
      add    esi, 8         ; 移動到下個雙字
      loop   lp
      fstp   qword sum      ; 將結果儲存到sum中
\end{AsmCodeListing}
\caption{陣列求和的例子\label{fig:addEx}}
\end{figure}

減法指令是加法指令的兩倍,因為在減法中,運算元的次序是重要的。(\emph{也就是說,}
$a + b = b + a$,但是,$a - b \neq b - a$!)。對於每一條指令,都有一條跟它次序相反的反向指令。這些反向指令要都是以{\code R}或{\code RP}結尾。圖~\ref{fig:addEx}展示了一小段代碼:對一個雙字陣列的元素求和。在第10和第13行中,你必須指定記憶體運算元的大小。否則彙編器將不會知道記憶體運算元是一個單精確度浮點數(雙字)還是雙精度數(四字)。

\begin{tabular}{p{1.5in}p{3.5in}}
{\code FSUB \emph{src}} \index{FSUB} &
{\code ST0 -= \emph{src}}。\emph{src}可以是任何輔助運算器寄存器或記憶體中單,雙精度數。\\
{\code FSUBR \emph{src}} \index{FSUBR} &
{\code ST0 = \emph{src} - ST0}。\emph{src}可以是任何輔助運算器寄存器或記憶體中單,雙精度數。\\
{\code FSUB \emph{dest}, ST0} &
{\code \emph{dest} -= ST0}。\emph{dest}可以是任何輔助運算器寄存器。\\
{\code FSUBR \emph{dest}, ST0} &
{\code \emph{dest} = ST0 - \emph{dest}}。\emph{dest}可以是任何輔助運算器寄存器。\\
{\code FSUBP \emph{dest}} or \newline {\code FSUBP \emph{dest}, STO} \index{FSUBP} &
{\code \emph{dest} -= ST0}然後被彈出堆疊。\emph{dest}可以是任何輔助運算器寄存器。\\
{\code FSUBRP \emph{dest}} or \newline {\code FSUBRP \emph{dest}, STO} \index{FSUBRP} &
{\code \emph{dest} = ST0 - \emph{dest}}然後被彈出堆疊。\emph{dest}可以是任何輔助運算器寄存器。\\
{\code FISUB \emph{src}} \index{FISUB} &
{\code ST0 -= (float) \emph{src}}。用{\code ST0}減去一個整數。
\emph{src}必須是記憶體中的一個字或雙字。\\
{\code FISUBR \emph{src}} \index{FISUBR} &
{\code ST0 = (float) \emph{src} - ST0}。用一個整數減去{\code ST0}。\emph{src}必須是記憶體中的一個字或雙字。
\end{tabular}

\index{浮點輔助運算器!加法和減法|)}

\subsubsection{乘法和除法\index{浮點輔助運算器!乘法和除法|(}}

乘法指令和加法指令完全類似。\\
\begin{tabular}{p{1.5in}p{3.5in}}
{\code FMUL \emph{src}} \index{FMUL} &
{\code ST0 *= \emph{src}}。\emph{src}可以是任何輔助運算器寄存器或記憶體中的單或雙精度數。\\
{\code FMUL \emph{dest}, ST0} &
{\code \emph{dest} *= ST0}。\emph{dest}可以是任何輔助運算器寄存器。\\
{\code FMULP \emph{dest}} or \newline {\code FMULP \emph{dest}, STO} \index{FMULP} &
{\code \emph{dest} *= ST0}然後被彈出堆疊。\emph{dest}可以是任何的輔助運算器寄存器。\\
{\code FIMUL \emph{src}} \index{FMUL} &
{\code ST0 *= (float) \emph{src}}。{\code ST0}與一個整數相乘。
\emph{src}必須是記憶體中的一個字或雙字。
\end{tabular}

不要驚訝,除法指令和減法指令非常類似。除以0結果將是一個無窮數。\\
\begin{tabular}{p{1.5in}p{3.5in}}
{\code FDIV \emph{src}} \index{FDIV} &
{\code ST0 /= \emph{src}}。\emph{src}可以是任何輔助運算器寄存器或記憶體中的單或雙精度數。\\
{\code FDIVR \emph{src}} \index{FDIVR} &
{\code ST0 = \emph{src} / ST0}。\emph{src}可以是任何輔助運算器寄存器或記憶體中的單或雙精度數。\\
{\code FDIV \emph{dest}, ST0} &
{\code \emph{dest} /= ST0}。\emph{dest}可以是任何輔助運算器寄存器。\\
{\code FDIVR \emph{dest}, ST0} &
{\code \emph{dest} = ST0 / \emph{dest}}。\emph{dest}可以是任何輔助運算器寄存器。\\
{\code FDIVP \emph{dest}} or \newline {\code FDIVP \emph{dest}, STO} \index{FDIVP} &
{\code \emph{dest} /= ST0}然後被彈出堆疊。\emph{dest}可以是任何輔助運算器寄存器。\\
{\code FDIVRP \emph{dest}} or \newline {\code FDIVRP \emph{dest}, STO} \index{FDIVRP} &
{\code \emph{dest} = ST0 / \emph{dest}}然後被彈出堆疊。\emph{dest}可以是任何輔助運算器寄存器。\\
{\code FIDIV \emph{src}} \index{FIDIV} &
{\code ST0 /= (float) \emph{src}}。{\code ST0}除以一個整數。
\emph{src}必須是記憶體中的一個字或雙字。\\
{\code FIDIVR \emph{src}} \index{FIDIVR} &
{\code ST0 = (float) \emph{src} / ST0}。一個整數除以{\code ST0}。
 The \emph{src}必須是記憶體中的一個字或雙字。
\end{tabular}
\index{浮點輔助運算器!乘法和除法|)}

\subsubsection{比較 \index{浮點輔助運算器!比較|(}}

輔助運算器同樣能執行浮點數的比較操作。
{\code FCOM}家族的指令就是執行比較操作的。\\
\begin{tabular}{lp{4in}}
{\code FCOM \emph{src}} \index{FCOM} &
比較{\code ST0}和{\code \emph{src}}。\emph{src}可以是輔助運算器寄存器或記憶體中的單或雙精度數。\\
{\code FCOMP \emph{src}} \index{FCOMP} &
比較{\code ST0}和{\code \emph{src}},然後再彈出堆疊。\emph{src}可以是輔助運算器寄存器或記憶體中的單或雙精度數。\\
{\code FCOMPP} \index{FCOMPP} &
比較{\code ST0}和{\code ST1},然後執行兩次出堆疊操作。\\
{\code FICOM \emph{src}} \index{FICOM} &
比較{\code ST0}和{\code (float) \emph{src}}。\emph{src}可以是記憶體中的一個整形字或整形雙字。\\
{\code FICOMP \emph{src}} \index{FICOMP} &
比較{\code ST0}和{\code (float)\emph{src}},然後再彈出堆疊。\emph{src}可以是記憶體中的一個整形字或整形雙字。\\
{\code FTST } \index{FTST} &
比較{\code ST0}和0。
\end{tabular}

\begin{figure}[t]
\begin{AsmCodeListing}[frame=single]
;     if ( x > y )
;
      fld    qword [x]       ; ST0 = x
      fcomp  qword [y]       ; 比較STO和y
      fstsw  ax              ; 將C狀態標誌位元儲存到FLAGS中
      sahf
      jna    else_part       ; 如果x不大於y,則跳轉到else_part
then_part:
      ; then部分的代碼
      jmp    end_if
else_part:
      ; else部分的代碼
end_if:
\end{AsmCodeListing}
\caption{比較指令的例子\label{fig:compEx}}
\end{figure}

這些指令會改變輔助運算器狀態寄存器中的C$_0$,C$_1$,C$_2$和C$_3$比特位的值。不幸的是,CPU直接訪問這些位是不可能的。條件分支指令使用FLAGS寄存器,而不是輔助運算器中的狀態寄存器。但是,使用幾條新的指令可以相當容易地將狀態字的比特位元傳遞到FLAGS寄存器上相同的比特位中。\\
\begin{tabular}{lp{4in}}
{\code FSTSW \emph{dest}} \index{FSTSW} &
存儲輔助運算器狀態字到記憶體的一個字或AX寄存器中。\\
{\code SAHF} \index{SAHF} &
將AH寄存器中的值儲存到FLAGS寄存器中。\\
{\code LAHF} \index{LAHF} &
將FLAGS寄存器中的比特位導入到AH寄存器中。\\
\end{tabular}

\begin{figure}[t]
\begin{AsmCodeListing}[frame=single]
global _dmax

segment .text
; 函式 _dmax
; 返回兩個參數中的較大的一個
; C函式原型
; double dmax( double d1, double d2 )
; 參數:
;   d1   - 第一個雙精度數
;   d2   - 第二個雙精度數
; 返回值:
;   d1和d2中較大的一個 (儲存在ST0中)
%define d1   ebp+8
%define d2   ebp+16
_dmax:
        enter   0, 0

        fld     qword [d2]
        fld     qword [d1]          ; ST0 = d1, ST1 = d2
        fcomip  st1                 ; ST0 = d2
        jna     short d2_bigger
        fcomp   st0                 ; 從堆疊中彈出d2
        fld     qword [d1]          ; ST0 = d1
        jmp     short exit
d2_bigger:                     ; 如果d2不是較大的那個數,不做任何事
exit:
        leave
        ret
\end{AsmCodeListing}
\caption{{\code FCOMIP}指令的例子\label{fig:fcomipEx}}
\index{FCOMIP}
\end{figure}

圖~\ref{fig:compEx}展示了一小段樣例代碼。第5行和第6行將C$_0$,C$_1$,C$_2$和C$_3$比特位傳遞到FLAGS寄存器相同的比特位中了。傳遞了這些比特位,所以它們就類似於兩個\emph{無符號}整形的比較結果。這也是為什麼第7行使用{\code JNA}指令的緣故。

Pentium處理器(和它以後的處理器(Pentium II and III))支援兩條新比較指令,用來直接改變CPU中FLAGS寄存器的值。

\begin{tabular}{lp{4in}}
{\code FCOMI \emph{src}} \index{FCOMI} &
比較{\code ST0}和{\code \emph{src}}。\emph{src}必須是一個輔助運算器寄存器。\\
{\code FCOMIP \emph{src}} \index{FCOMIP} &
比較{\code ST0}和{\code \emph{src}},然後再彈出堆疊。\emph{src}必須是一個輔助運算器寄存器。\\
\end{tabular}
圖~\ref{fig:fcomipEx}展示了一個子程式例子:使用{\code FCOMIP}指令來找出兩個雙精度數的較大值。不要把這些指令和整形比較函式({\code FICOM}
和{\code FICOMP})混起來。
\index{浮點輔助運算器!比較|)}

\subsubsection{雜項指令}
%FINIT?

\begin{figure}[t]
\begin{AsmCodeListing}[frame=single]
segment .data
x            dq  2.75          ; 轉換成雙精度格式
five         dw  5

segment .text
      fild   dword [five]      ; ST0 = 5
      fld    qword [x]         ; ST0 = 2.75,ST1 = 5
      fscale                   ; ST0 = 2.75 * 32,ST1 = 5
\end{AsmCodeListing}
\caption{{\code FSCALE}指令的例子\label{fig:fscaleEx}}
\index{FSCALE}
\end{figure}

這一節包括了輔助運算器提供的其他雜項指令。

\begin{tabular}{lp{4in}}
{\code FCHS} \index{FCHS} &
{\code ST0 = - ST0}改變{\code ST0}的符號位元  \\
{\code FABS} \index{FABS} &
$\mathtt{ST0} = |\mathtt{ST0}|$ 求{\code ST0}的絕對值\\
{\code FSQRT} \index{FSQRT} &
$\mathtt{ST0} = \sqrt{\mathtt{STO}}$ 求{\code ST0}的平方根 \\
{\code FSCALE} \index{FSCALE} &
$\mathtt{ST0} = \mathtt{ST0} \times 2^{\lfloor \mathtt{ST1} \rfloor}$
快速執行{\code ST0}乘以2的幾次方的操作。{\code ST1}並不會從輔助運算器堆疊中移除。圖~\ref{fig:fscaleEx}展示了一個如何使用這些指令的例子。
\end{tabular}


\subsection{樣例}

\subsection{二次方程求根公式\index{quad.asm|(}}

第一個例子展示了如何用組合語言編寫二次方程求根公式。回憶一下如何用求根公式計算二次方程等式的根:
\[ a x^2 + b x + c = 0 \]
公式本身給出兩個根$x$:$x_1$和$x_2$。
\[ x_1, x_2 = \frac{-b \pm \sqrt{b^2 - 4 a c}}{2 a} \]
在平方根($b^2 - 4 a c$)裏的運算式稱為
\emph{判別式}。這個值在判定是下面三種可能的根的情況中的哪一種時非常有用。
\begin{enumerate}
\item 只有一個實根。當$b^2 - 4 a c = 0$時
\item 有兩個實根。當$b^2 - 4 a c > 0$時
\item 有兩個複根。當$b^2 - 4 a c < 0$時
\end{enumerate}

這是一個使用彙編副程式的小的C程式:
\LabelLine{quadt.c}
\begin{lstlisting}{}
#include <stdio.h>

int quadratic( double, double, double, double *, double *);

int main()
{
  double a,b,c, root1, root2;

  printf("Enter a, b, c: ");
  scanf("%lf %lf %lf", &a, &b, &c);
  if (quadratic( a, b, c, &root1, &root2) )
    printf("roots: %.10g %.10g\n", root1, root2);
  else
    printf("No real roots\n");
  return 0;
}
\end{lstlisting}
\LabelLine{quadt.c}

彙編副程式如下:
\begin{AsmCodeListing}[label=quad.asm,commentchar=$]
; 函式 quadratic
; 求二次等式的根:
;       a*x^2 + b*x + c = 0
; C函式原型:
;   int quadratic( double a, double b, double c,
;                  double * root1, double *root2 )
; 參數:
;   a, b, c - 二次等式中各次方的系統(看上面)
;   root1   - 指向存儲第一個根的雙精度變數的指標
;   root2   - 指向存儲第二個根的雙精度變數的指標
; 返回值:
;   如果存在實根,則返回1,否則返回0

%define a               qword [ebp+8]
%define b               qword [ebp+16]
%define c               qword [ebp+24]
%define root1           dword [ebp+32]
%define root2           dword [ebp+36]
%define disc            qword [ebp-8]
%define one_over_2a     qword [ebp-16]

segment .data
MinusFour       dw      -4

segment .text
        global  _quadratic
_quadratic:
        push    ebp
        mov     ebp, esp
        sub     esp, 16         ; 分配兩個雙精度數大小的空間(disc & one_over_2a)
        push    ebx             ; 必須保存原始的ebx值

        fild    word [MinusFour]; stack -4
        fld     a               ; stack: a, -4
        fld     c               ; stack: c, a, -4
        fmulp   st1             ; stack: a*c, -4
        fmulp   st1             ; stack: -4*a*c
        fld     b
        fld     b               ; stack: b, b, -4*a*c
        fmulp   st1             ; stack: b*b, -4*a*c
        faddp   st1             ; stack: b*b - 4*a*c
        ftst                    ; test with 0
        fstsw   ax
        sahf
        jb      no_real_solutions ; 如果disc < 0,則沒有實根
        fsqrt                   ; stack: sqrt(b*b - 4*a*c)
        fstp    disc            ; 儲存然後再彈出堆疊
        fld1                    ; stack: 1.0
        fld     a               ; stack: a, 1.0
        fscale                  ; stack: a * 2^(1.0) = 2*a, 1
        fdivp   st1             ; stack: 1/(2*a)
        fst     one_over_2a     ; stack: 1/(2*a)
        fld     b               ; stack: b, 1/(2*a)
        fld     disc            ; stack: disc, b, 1/(2*a)
        fsubrp  st1             ; stack: disc - b, 1/(2*a)
        fmulp   st1             ; stack: (-b + disc)/(2*a)
        mov     ebx, root1
        fstp    qword [ebx]     ; store in *root1
        fld     b               ; stack: b
        fld     disc            ; stack: disc, b
        fchs                    ; stack: -disc, b
        fsubrp  st1             ; stack: -disc - b
        fmul    one_over_2a     ; stack: (-b - disc)/(2*a)
        mov     ebx, root2
        fstp    qword [ebx]     ; 儲存到*root2中
        mov     eax, 1          ; 返回值為1
        jmp     short quit

no_real_solutions:
        mov     eax, 0          ; 返回值為0

quit:
        pop     ebx
        mov     esp, ebp
        pop     ebp
        ret
\end{AsmCodeListing}
\index{quad.asm|)}

\subsection{從檔中讀數組\index{read.asm|(}}

在這個例子中,有一個從檔中讀取雙精度數的組合語言程式。這是一個簡短的C測試程式:
\LabelLine{readt.c}
\lstset{escapeinside=`',language=Pascal,%
}
\begin{lstlisting}{}
/*`
   這個程式用來測試32位元的read\_doubles()組合語言程式。
   它從stdin中讀取雙精度數。(使用重定向從檔中讀取。)
 '*/
#include <stdio.h>
extern int read_doubles( FILE *, double *, int );
#define MAX 100

int main()
{
  int i,n;
  double a[MAX];

  n = read_doubles(stdin, a, MAX);

  for( i=0; i < n; i++ )
    printf("%3d %g\n", i, a[i]);
  return 0;
}
\end{lstlisting}
\LabelLine{readt.c}

這是組合語言程式:
\begin{AsmCodeListing}[label=read.asm]
segment .data
format  db      "%lf", 0        ; format for fscanf()

segment .text
        global  _read_doubles
        extern  _fscanf

%define SIZEOF_DOUBLE   8
%define FP              dword [ebp + 8]
%define ARRAYP          dword [ebp + 12]
%define ARRAY_SIZE      dword [ebp + 16]
%define TEMP_DOUBLE     [ebp - 8]

;
; 函式 _read_doubles
; C函式原型:
;   int read_doubles( FILE * fp, double * arrayp, int array_size );
; 這個函式從一個文字檔案中讀取雙精度數,並將它們儲存到一個陣列裏,直到遇到
; EOF或陣列滿了。
; 參數:
;   fp         - 指向需要讀取的檔的指標(必須允許輸入)
;   arrayp     - 指向寫入的雙精度陣列的指標
;   array_size - 陣列的元素個數
; 返回值:
;   儲存到陣列中的雙精度數的個數(保存在EAX中)

_read_doubles:
        push    ebp
        mov     ebp,esp
        sub     esp, SIZEOF_DOUBLE      ; 在堆疊中定義一個雙精度數

        push    esi                     ; 保存esi
        mov     esi, ARRAYP             ; esi = ARRAYP
        xor     edx, edx                ; edx = 陣列的下標(最開始為0)

while_loop:
        cmp     edx, ARRAY_SIZE         ; edx < ARRAY_SIZE?
        jnl     short quit              ; 如果不是,退出迴圈
;
; 調用fscanf()函式讀一個雙精度數到TEMP_DOUBLE中
; fscanf()會改變edx,所以需要保存它
;
        push    edx                     ; 保存edx
        lea     eax, TEMP_DOUBLE
        push    eax                     ; 將&TEMP_DOUBLE壓入堆疊中
        push    dword format            ; 將&format壓入堆疊中
        push    FP                      ; 將檔案指針壓入堆疊中
        call    _fscanf
        add     esp, 12
        pop     edx                     ; 恢復edx的值
        cmp     eax, 1                  ; fscanf函式是否返回1?
        jne     short quit              ; 如果不是,則退出迴圈

;
; 複製TEMP_DOUBLE到ARRAYP[edx]中
; (8個位元組的雙精度數是通過分成兩個4位元組的數來完成複製的)
;
        mov     eax, [ebp - 8]
        mov     [esi + 8*edx], eax      ; 首先複製低4位元組
        mov     eax, [ebp - 4]
        mov     [esi + 8*edx + 4], eax  ; 接著複製高4位元組

        inc     edx
        jmp     while_loop

quit:
        pop     esi                     ; 恢復esi

        mov     eax, edx                ; 將返回值儲存到eax中

        mov     esp, ebp
        pop     ebp
        ret
\end{AsmCodeListing}
\index{read.asm|)}

\subsection{查找素數\index{prime2.asm|(}}

最後一個例子又是查找素數的例子。這次的實現方法比以前的方法更有效。它將找到的素數儲存到一個陣列中,而且在查找新的素數時,只除以它已經找到的素數,而不是去除以每一個奇數。

另一個區別是它會計算出猜想的下一個素數的平方根來決定查找因數時應停在哪一個數。它修改了輔助運算器的控制字,所以當它把平方根當作一個整數來儲存時,是通過直接截去來得到整數,而不是四捨五入。這是由控制字中的第10位和第11位來控制的。這些位稱為RC(Rounding Control,四捨五入控制)位。如果這兩位都是0(缺省值),則輔助運算器轉換成整數時,採用四捨五入的方法。如果是1,則通過直接截去來得到整數。注意:程式必須小心保存好控制字的原始值,當它返回時須恢復它的值。

這是C驅動程式:
\LabelLine{fprime.c}
\lstset{escapeinside=`',language=Pascal,%
}
\begin{lstlisting}{}
#include <stdio.h>
#include <stdlib.h>
/*`
 * 函式 find\_primes
 * 查找給定範圍的素數
 * 參數:
 *   a - 保存素數的陣列
 *   n - 找到的素數的個數
 '*/
extern void find_primes( int * a, unsigned n );

int main()
{
  int status;
  unsigned i;
  unsigned max;
  int * a;

  printf("How many primes do you wish to find? ");
  scanf("%u", &max);

  a = calloc( sizeof(int), max);

  if ( a ) {

    find_primes(a,max);

    /* print out the last 20 primes found */
    for(i= ( max > 20 ) ? max - 20 : 0; i < max; i++ )
      printf("%3d %d\n", i+1, a[i]);

    free(a);
    status = 0;
  }
  else {
    fprintf(stderr, "Can not create array of %u ints\n", max);
    status = 1;
  }

  return status;
}
\end{lstlisting}
\LabelLine{fprime.c}

下麵是組合語言程式:


\begin{AsmCodeListing}[label=prime2.asm]
segment .text
        global  _find_primes
;
; 函式 find_primes
; 查找給定範圍的素數
; 參數:
;   array  - 保存素數的陣列
;   n_find - 找到的素數的個數
; C函式原型:
;extern void find_primes( int * array, unsigned n_find )
;
%define array         ebp + 8
%define n_find        ebp + 12
%define n             ebp - 4           ; 目前為止找到的素數的個數
%define isqrt         ebp - 8           ; 猜想的下一個素數開平方後得到的整數
%define orig_cntl_wd  ebp - 10          ; 原始控制字
%define new_cntl_wd   ebp - 12          ; 新的控制字

_find_primes:
        enter   12,0                    ; 為局部變數分配空間

        push    ebx                     ; 保存可能的寄存器變數
        push    esi

        fstcw   word [orig_cntl_wd]     ; 得到當前控制字
        mov     ax, [orig_cntl_wd]
        or      ax, 0C00h               ; 設置RC位為11(截去)
        mov     [new_cntl_wd], ax
        fldcw   word [new_cntl_wd]

        mov     esi, [array]            ; esi指向陣列
        mov     dword [esi], 2          ; array[0] = 2
        mov     dword [esi + 4], 3      ; array[1] = 3
        mov     ebx, 5                  ; ebx = guess = 5
        mov     dword [n], 2            ; n = 2
;
; 這個外部的迴圈用來查找一個新的素數,新的素數將被加到
; 陣列的末尾。跟以前的查找素數程式不同的是,這個函式
; 並不是通過除以所有的奇數來決定它是不是素數。它僅僅
; 除以已經找到的素數。(這也是為什麼它們
; 被儲存到陣列中的緣故。)
;
while_limit:
        mov     eax, [n]
        cmp     eax, [n_find]           ; while ( n < n_find )
        jnb     short quit_limit

        mov     ecx, 1                  ; ecx用來表示陣列的下標
        push    ebx                     ; 將猜想的素數儲存到堆疊中
        fild    dword [esp]             ; 將猜想的素數導入到輔助運算器堆疊中
        pop     ebx                     ; 將猜想的素數移除出堆疊
        fsqrt                           ; 求sqrt(guess)
        fistp   dword [isqrt]           ; isqrt = floor(sqrt(quess))
;
; 這個內部的迴圈用猜想的素數(ebx)除以已經找到的素數,
; 直到找到一個猜想的素數的因數(也就意味著這個猜想的素數不是素數),
; 或直到猜想的素數除以的找到的素數大於floor(sqrt(guess))
;
while_factor:
        mov     eax, dword [esi + 4*ecx]        ; eax = array[ecx]
        cmp     eax, [isqrt]                    ; while ( isqrt < array[ecx]
        jnbe    short quit_factor_prime
        mov     eax, ebx
        xor     edx, edx
        div     dword [esi + 4*ecx]
        or      edx, edx                        ; && guess % array[ecx] != 0 )
        jz      short quit_factor_not_prime
        inc     ecx                             ; 試下一個素數
        jmp     short while_factor

;
; found a new prime !
;
quit_factor_prime:
        mov     eax, [n]
        mov     dword [esi + 4*eax], ebx        ; 將猜想的素數加到陣列的末尾
        inc     eax
        mov     [n], eax                        ; inc n

quit_factor_not_prime:
        add     ebx, 2                          ; 試下一個奇數
        jmp     short while_limit

quit_limit:

        fldcw   word [orig_cntl_wd]             ; 恢復控制字
        pop     esi                             ; 恢復寄存器變數
        pop     ebx

        leave
        ret
\end{AsmCodeListing}
\index{prime2.asm|)}
\index{浮點輔助運算器|)}
\index{浮點|)}


% -*-latex-*-
\chapter{Structures et C++}

\section{Structures\index{structures|(}}

\subsection{Introduction}

Les structures sont utilis�es en C pour regrouper des donn�es ayant un rapport entre
elles dans une variable composite. Cette technique a plusieurs avantages :
\begin{enumerate}
\item Cela clarifie le code en montrant que les donn�es d�finies dans la structure sont
      intimement li�es.
\item Cela simplifie le passage des donne�s aux fonctions. Au lieu de passer plusieurs
      variables s�par�ment, elles peuvent �tre pass�es en une seule entit�.
\item Cela augmente la \emph{localit�}\index{localit�}\footnote{Votez le chapitre sur
la gestion de la m�moire virtuelle de n'importe quel livre sur les Syst�mes d'Exploitation
pour une explication de ce terme.} du code.
\end{enumerate}

Du point de vue de l'assembleur, une structure peut �tre consid�r�e comme
un tableau avec des �l�ments de taille \emph{variable}. Les �l�ments des
vrais tableaux sont toujours de la m�me taille et du m�me type. C'est cette
propri�t� qui permet de calculer l'adresse de n'importe quel �l�ment
en connaissant l'adresse de d�but du tableau, la taille des �l�ments et
l'indice de l'�l�ment voulu.

Les �l�ments d'une structure ne sont pas n�cessairement de la m�me taille
(et habituellement, ils ne le sont pas). A cause de cela, chaque �l�ment
d'une structure doit �tre explicitement sp�cifi� et doit recevoir un
\emph{tag} (ou nom) au lieu d'un indice num�rique.

En assembleur, on acc�de � un �l�ment d'une structure d'une fa�on similaire
� l'acc�s � un �l�ment de tableau. Pour acc�der � un �l�ment, il faut conna�tre
l'adresse de d�part de la structure et le \emph{d�placement relatif} de cet
�l�ment par rapport au d�but de la structure. Cependant, contrairement � un tableau
o� ce d�placement peut �tre calcul� gr�ce � l'indice de l'�l�ment, c'est le
compilateur qui affecte un d�placement aux �l�ments d'une structure.

Par exemple, consid�rons la structure suivante :
\begin{lstlisting}[stepnumber=0]{}
struct S {
  short int x;    /* entier sur 2 octets */
  int       y;    /* entier sur 4 octets */
  double    z;    /* flottant sur 8 octets */
};
\end{lstlisting}

\begin{figure}
\centering
\begin{tabular}{r|c|}
\multicolumn{1}{c}{D�placement} & \multicolumn{1}{c}{ El�ment } \\
\cline{2-2}
0 & {\code x} \\
\cline{2-2}
2 & \\
  & {\code y} \\
\cline{2-2}
6 & \\
  & \\
  & {\code z} \\
  & \\
\cline{2-2}
\end{tabular}
\caption{Structure S \label{fig:structPic1}}
\end{figure}

La Figure~\ref{fig:structPic1} montre � quoi pourrait ressembler une
variable de type {\code S} pourrait ressembler en m�moire. Le standard
ANSI C indique que les �l�ments d'une structure sont organis�s en
m�moire dans le m�me ordre que celui de leur d�finition dans le
{\code struct}. Il indique �galement que le premier �l�ment est au
tout d�but de la structure (\emph{i.e.} au d�placement z�ro). Il d�finit
�galement une macro utile dans le fichier d'en-t�te {\code stddef.h}
appel�e {\code offsetof()}. \index{structures!offsetof()} Cette macro
calcule et renvoie le d�placement de n'importe quel �l�ment d'une structure.
La macro prend deux param�tres, le premier est le nom du \emph{type} de la
structure, le second est le nom de l'�l�ment dont on veut le d�placement.
Donc le r�sultat de {\code offsetof(S, y)} serait 2 d'apr�s la
Figure~\ref{fig:structPic1}.

%TODO: talk about definition of offsetof() ??

\subsection{Alignement en m�moire}

\begin{figure}
\centering
\begin{tabular}{r|c|}
\multicolumn{1}{c}{Offset} & \multicolumn{1}{c}{ El�ment } \\
\cline{2-2}
0 & {\code x} \\
\cline{2-2}
2 & \emph{inutilis�} \\
\cline{2-2}
4 & \\
  & {\code y} \\
\cline{2-2}
8 & \\
  & \\
  & {\code z} \\
  & \\
\cline{2-2}
\end{tabular}
\caption{Structure S \label{fig:structPic2}}

\end{figure}
\index{structures!alignement|(}
Si l'on utilise la macro {\code offsetof} pour trouver le d�placement
de {\code y} en utilisant le compilateur gcc, on s'aper�oit qu'elle
renvoie 4, pas 2 ! Pourquoi ? \MarginNote{Souvenez vous qu'une adresse
est sur un multiple de double mot si elle est divisible par 4}
Parce que \emph{gcc} (et beaucoup d'autre compilateurs) aligne les
variables sur des multiples de doubles mots par d�faut. En mode prot�g�
32~bits, le processeur lit la m�moire plus vite si la donn�e commence
sur un multiple de double mot. La Figure~\ref{fig:structPic2} montre
� quoi ressemble la structure {\code S} en utilisant \emph{gcc}. Le compilateur
ins�re deux octets inutilis�s dans la structure pour aligner {\code y} (et
{\code z}) sur un multiple de double mot. Cela montre pourquoi c'est une bonne
id�e d'utiliser {\code offsetof} pour calculer les d�placements, au lieu de les
calculer soi-m�me lorsqu'on utilise des structures en C.

Bien s�r, si la structure est utilis�e uniquement en assembleur, le programmeur
peut d�terminer les d�placements lui-m�me. Cependant, si l'on interface du C et
de l'assembleur, il est tr�s important que l'assembleur et le C s'accordent sur
les d�placements des �l�ments de la structure ! Une des complications est que
des compilateurs C diff�rents peuvent donner des d�placements diff�rents aux
�l�ments. Par exemple, comme nous l'avons vu, le compilateur \emph{gcc} cr�e
une structure {\code S} qui ressemble � la Figure~\ref{fig:structPic2} ; 
cependant, le compilateur de Borland cr�erait une structure qui ressemble �
la Figure~\ref{fig:structPic1}. Les compilateurs C fournissent le moyen
de sp�cifier l'alignement utilis� pour les donn�es. Cependant, le standard
ANSI C ne sp�cifie pas comment cela doit �tre fait et donc, des compilateurs
diff�rents proc�dent diff�remment.

%Borland's compiler has a flag, {\code -a}, that can be
%used to define the alignment used for all data. Compiling with {\code -a 4}
%tells \emph{bcc} to use double word alignment. Microsoft's compiler 
%provides a {\code \#pragma pack} directive that can be used to set
%the alignment (consult Microsoft's documentation for details). Borland's
%compiler also supports Microsoft's pragma 

Le compilateur \emph{gcc}\index{compilateur!gcc!\_\_attribute\_\_} a une m�thode
flexible et compliqu�e de sp�cifier l'alignement. Le compilateur permet de sp�cifier
l'alignement de n'importe quel type en utilisant une syntaxe sp�ciale. Par exemple,
la ligne suivante : 
\begin{lstlisting}[stepnumber=0]{}
  typedef short int unaligned_int __attribute__((aligned(1)));
\end{lstlisting}
\noindent d�finit un nouveau type appel� {\code unaligned\_int} qui
est align� sur des multiples d'octet (Oui, toutes les parenth�ses suivant
{\code \_\_attribute\_\_} sont n�cessaires !)  Le param�tre 1 de {\code aligned}
peut �tre remplac� par d'autres puissances de deux pour sp�cifier d'autres alignements
(2 pour s'aligner sur les mots, 4 sur les doubles mots, \emph{etc.}). Si l'�l�ment
{\code y} de la structure �tait chang� en un type {\code unaligned\_int}, \emph{gcc}
placerait {\code y} au d�placement 2.
Cependant, {\code z} serait toujours au d�placement 8 puisque les doubles sont �galement
align�s sur des doubles mots par d�faut. La d�finition du type de {\code z} devrait
aussi �tre chang�e pour le placer au d�placement 6.

\begin{figure}[t]
\begin{lstlisting}[frame=tlrb,stepnumber=0]{}
struct S {
  short int x;    /* entier sur 2 octets */
  int       y;    /* entier sur 4 octets */
  double    z;    /* flottant sur 8 octets */
} __attribute__((packed));
\end{lstlisting}
\caption{Structure comprim�e sous \emph{gcc} \label{fig:packedStruct}\index{compilateur!gcc!\_\_attribute\_\_}}
\end{figure}

Le compilateur \emph{gcc} permet �galement de \emph{comprimer} (pack) une structure.
Cela indique au compilateur d'utiliser le minimum d'espace possible pour la structure.
La Figure~\ref{fig:packedStruct} montre comment {\code S} pourrait �tre r��crite de cette
fa�on. Cette forme de {\code S} utiliserait le moins d'octets possible, soit 14 octets.

Les compilateurs de Microsoft et Borland supportent tous les deux la m�me m�thode pour
indiquer l'alignement par le biais d'une directive {\code \#pragma}.\index{compilateur!Microsoft!pragma pack}
\begin{lstlisting}[stepnumber=0]{}
#pragma pack(1)
\end{lstlisting}
La directive ci-dessus indique au compilateur d'aligner les �l�ments
des structures sur des multiples d'un octet (\emph{i.e.}, sans d�calage
superflu). Le un peut �tre remplac� par deux, quatre, huit ou seize
pour sp�cifier un alignement sur des multiples de mots, doubles mots, quadruples mots
ou de paragraphe, respectivement. La directive reste active jusqu'� ce qu'elle
soit �cras�e par une autre. Cela peut poser des probl�mes puisque
ces directives sont souvent utilis�es dans des fichiers d'en-t�te.
Si le fichier d'en-t�te est inclus avant d'autres fichiers d'en-t�te d�finissant
des structures, ces structures peuvent �tre organis�es diff�remment de ce
qu'elles auraient �t� par d�faut. Cela peut conduire � des erreurs
tr�s difficiles � localiser. Les diff�rents modules d'un programmes
devraient organiser les �l�ments des structures � \emph{diff�rents}
endroits !

\begin{figure}[t]
\begin{lstlisting}[frame=tlrb,stepnumber=0]{}
#pragma pack(push)    /* sauve l'�tat de l'alignement */
#pragma pack(1)       /* d�finit un alignement sur octet*/

struct S {
  short int x;    /* entier sur 2 octets */
  int       y;    /* entier sur 4 octets */
  double    z;    /* flottant sur 8 octets */
};

#pragma pack(pop)     /* restaure l'alignement original */
\end{lstlisting}
\caption{Structure comprim�e sous les compilateurs Microsoft ou Borland \label{fig:msPacked}\index{compilateur!Microsoft!pragma pack}}
\end{figure}

Il y a une fa�on d'�viter ce probl�me. Microsoft et Borland permettent la
sauvegarde de l'�tat de l'alignement courant et sa restauration.
La Figure~\ref{fig:msPacked} montre comment on l'utilise.
\index{structures!alignement|)}

\subsection{Champs de Bits\index{structures!champs de bits|(}}

\begin{figure}[t]
\begin{lstlisting}[frame=tlrb,stepnumber=0]{}
struct S {
  unsigned f1 : 3;   /* champ de 3 bits */
  unsigned f2 : 10;  /* champ de 10 bits */
  unsigned f3 : 11;  /* champ de 11 bits */
  unsigned f4 : 8;   /* champ de 8 bits */
};
\end{lstlisting}
\caption{Exemple de Champs de Bits\label{fig:bitStruct}}
\end{figure}

Les champs de bits permettent de d�clarer des membres d'une structure qui n'utilisent
qu'un nombre de bits donn�. La taille en bits n'a pas besoin d'�tre un multiple de huit.
Un membre champ de bit est d�fini comme un \lstinline|unsigned int| ou un \lstinline|int|
suivi de deux-points et de sa taille en bits. La Figure~\ref{fig:bitStruct} en montre
un exemple. Elle d�finit une variable 32 bits d�compos�e comme suit :
\begin{center}
\begin{tabular}{|c|c|c|c|}
\multicolumn{1}{c}{8 bits} & \multicolumn{1}{c}{11 bits} 
& \multicolumn{1}{c}{10 bits} & \multicolumn{1}{c}{3 bits} \\ \hline
\hspace{2em} f4 \hspace{2em} & \hspace{3em} f3 \hspace{3em}
& \hspace{3em} f2 \hspace{3em} & f1 \\
\hline
\end{tabular}
\end{center}
Le premier champ de bits est assign� aux bits les moins significatifs du
double mot\footnote{En fait, le standard ANSI/ISO C laisse une certaine
libert� au compilateur sur la fa�on d'organiser les bits. Cependant, les
compilateurs C courants (\emph{gcc}, \emph{Microsoft} et
\emph{Borland}) organisent les champs comme cela.}.

N�anmoins, le format n'est pas si simple si l'on observe comment les bits
sont stock�s en m�moire. La difficult� appara�t lorsque les champs de bits
sont � cheval sur des multiples d'octets. Car les octets, sur un processeur
little endian seront invers�s en m�moire. Par exemple, les champs de bits
de la structure {\code S} ressembleront � cela en m�moire :
\begin{center}
\begin{tabular}{|c|c||c|c||c||c|}
\multicolumn{1}{c}{5 bits} & \multicolumn{1}{c}{3 bits} 
& \multicolumn{1}{c}{3 bits} & \multicolumn{1}{c}{5 bits} 
& \multicolumn{1}{c}{8 bits} & \multicolumn{1}{c}{8 bits} \\ \hline
f2l & f1 &  f3l  & f2m & \hspace{1em} f3m \hspace{1em} 
& \hspace{1.5em} f4 \hspace{1.5em} \\
\hline
\end{tabular}
\end{center}
L'�tiquette \emph{f2l} fait r�f�rence aux cinq derniers bits (\emph{i.e.}, les cinq
bits les moins significatifs) du champ de bits \emph{f2}. L'�tiquette \emph{f2m} fait
r�f�rence aux cinq bits les plus significatifs de \emph{f2}. Les lignes verticales
doubles montrent les limites d'octets. Si l'on inverse tous les octets, les morceaux
des champs \emph{f2} et \emph{f3} seront r�unis correctement.

\begin{figure}[t]
\centering
\begin{tabular}{|c*{8}{|p{1.3em}}|}
\hline
Byte $\backslash$ Bit & 7 & 6 & 5 & 4 & 3 & 2 & 1 & 0 \\ \hline
0 & \multicolumn{8}{c|}{Code Op�ration (08h) } \\ \hline
1 & \multicolumn{3}{c|}{N� d'Unit� Logique} & \multicolumn{5}{c|}{msb de l'ABL} \\ \hline
2 & \multicolumn{8}{c|}{milieu de l'Adresse de Bloc Logique} \\ \hline
3 & \multicolumn{8}{c|}{lsb de l'Adresse de Bloc Logique} \\ \hline
4 & \multicolumn{8}{c|}{Longueur du Transfert} \\ \hline
5 & \multicolumn{8}{c|}{Contr�le} \\ \hline
\end{tabular}
\caption{Format de la Commande de Lecture SCSI\label{fig:scsi-read}}
\end{figure}

\begin{figure}[t]
\begin{lstlisting}[frame=lrtb]{}
#define MS_OR_BORLAND (defined(__BORLANDC__) \
                        || defined(_MSC_VER))

#if MS_OR_BORLAND
#  pragma pack(push)
#  pragma pack(1)
#endif

struct SCSI_read_cmd {
  unsigned opcode : 8;
  unsigned lba_msb : 5;
  unsigned logical_unit : 3;
  unsigned lba_mid : 8;    /* bits du milieu */
  unsigned lba_lsb : 8;
  unsigned transfer_length : 8;
  unsigned control : 8;
}
#if defined(__GNUC__)
   __attribute__((packed))
#endif
;

#if MS_OR_BORLAND
#  pragma pack(pop)
#endif
\end{lstlisting}
\caption{Structure du Format de la Commande de Lecture SCSI\label{fig:scsi-read-struct}\index{compilateur!gcc!\_\_attribute\_\_}
         \index{compilateur!Microsoft!pragma pack}}
\end{figure}

L'organisation de la m�moire physique n'est habituellement pas importante � moins que
des donn�es de soient transf�r�es depuis ou vers le programme (ce qui est en fait
assez courant avec les champs de bits). Il est courant que les interfaces de
p�riph�riques mat�riels utilisent des nombres impairs de bits dont les champs de bits
facilitent la repr�sentation.

\begin{figure}[t]
\centering
\begin{tabular}{|c||c||c||c||c|c||c|}
\multicolumn{1}{c}{8 bits} & \multicolumn{1}{c}{8 bits} 
& \multicolumn{1}{c}{8 bits} & \multicolumn{1}{c}{8 bits} 
& \multicolumn{1}{c}{3 bits} & \multicolumn{1}{c}{5 bits} 
& \multicolumn{1}{c}{8 bits} \\ \hline
control & transfer\_length & lba\_lsb  & lba\_mid &  
logical\_unit  & lba\_msb & opcode \\
\hline
\end{tabular}
\caption{Organisation des champs de {\code SCSI\_read\_cmd} \label{fig:scsi-read-map}}
\end{figure}
\index{SCSI|(}
Un bon exemple est SCSI\footnote{Small Computer Systems Interface, un standard de
l'industrie pour les disques durs, \emph{etc.}}. Une commande de lecture directe
pour un p�riph�rique SCSI est sp�cifi�e en envoyant un message de six octets au
p�riph�rique selon le format indiqu� dans la Figure~\ref{fig:scsi-read}. 
La difficult� de repr�sentation en utilisant les champs de bits est l'\emph{adresse
de bloc logique} qui est � cheval sur trois octets diff�rents de la commande.
D'apr�s la Figure~\ref{fig:scsi-read}, on constate que les donn�es sont stock�es au
format big endian. La Figure~\ref{fig:scsi-read-struct} montre une d�finition qui
essaie de fonctionner avec tous les compilateurs. Les deux premi�res lignes
d�finissent une macro qui est vraie si le code est compil� avec un compilateur
Borland ou Microsoft. La partie qui peut porter � confusion va des lignes 11 � 14.
Tout d'abord, on peut se demander pourquoi les champs \lstinline|lba_mid| et
\lstinline|lba_lsb| sont d�finis s�par�ment et non pas comme un champ
unique de 16 bits. C'est parce que les donn�es sont stock�es au format
big endian. Un champ de 16 bits serait stock� au format little endian par
le compilateur. Ensuite, les champs \lstinline|lba_msb| et
\lstinline|logical_unit| semblent �tre invers�s ; cependant, ce n'est pas
le cas. Ils doivent �tre plac�s dans cet ordre. La Figure~\ref{fig:scsi-read-map}
montre comment les champs sont organis�s sous forme d'une entit� de 48 bits (les
limites d'octets sont l� encore repr�sent�es par des lignes doubles). Lorsqu'elle
est stock�e en m�moire au format little endian, les bits sont r�arrang�s au
format voulu (Figure~\ref{fig:scsi-read}).

\begin{figure}[t]
\begin{lstlisting}[frame=lrtb]{}
struct SCSI_read_cmd {
  unsigned char opcode;
  unsigned char lba_msb : 5;
  unsigned char logical_unit : 3;
  unsigned char lba_mid;    /* bits du milieu */
  unsigned char lba_lsb;
  unsigned char transfer_length;
  unsigned char control;
}
#if defined(__GNUC__)
   __attribute__((packed))
#endif
;
\end{lstlisting}
\caption{Structure du Format de la Commande de Lecture SCSI Alternative\label{fig:scsi-read-struct2}
         \index{compilateur!gcc!\_\_attribute\_\_}\index{compilateur!Microsoft!pragma pack}}
\end{figure}

Pour compliquer encore plus le probl�me, la d�finition de \lstinline|SCSI_read_cmd|
ne fonctionne pas correctement avec le C Microsoft. Si l'expression 
\lstinline|sizeof(SCSI_read_cmd)| est �valu�e, le C Microsoft renvoie 8 et non
pas~6 ! C'est parce que le compilateur Microsoft utilise le type du champ de bits
pour d�terminer comment organiser les bits. Comme tous les bits sont d�clar�s
comme \lstinline|unsigned|, le compilateur ajoute deux octets � la fin de la
structure pour qu'elle comporte un nombre entier de double mots. Il est possible
d'y rem�dier en d�clarant tous les champs \lstinline|unsigned short|. Maintenant,
le compilateur Microsoft n'a plus besoin d'ajouter d'octets d'alignement puisque
six octets forment un nombre entier de mots de deux octets\footnote{M�langer
diff�rents types de champs de bits conduit � un comportement tr�s �trange !
Le lecteur est invit� � tester.}. Les autres compilateurs fonctionnent �galement
correctement avec ce changement. La Figure~\ref{fig:scsi-read-struct2} montre
une autre d�finition qui fonctionne sur les trois compilateurs. Il ne d�clare
plus que deux champs de bits en utilisant le type \lstinline|unsigned char|.
\index{SCSI|)}

Le lecteur ne doit pas se d�courager s'il trouve la discussion ci-dessus
confuse. C'est confus ! L'auteur trouve souvent moins confus d'�viter
d'utiliser des champs de bits en utilisant des op�rations niveau bit
pour examiner et modifier les bits manuellement.

\index{structures!champs de bits|)}

%TODO:discuss alignment issues and struct size issues

\subsection{Utiliser des structures en assembleur}

Comme nous l'avons dit plus haut, acc�der � une structure en assembleur
ressemble beaucoup � acc�der � un tableau. Prenons un exemple simple,
regardons comment l'on pourrait �crire une routine assembleur qui mettrait
� z�ro l'�l�ment {\code y} d'une structure {\code S}. Supposons que le
prototype de la routine soit :
\begin{lstlisting}[stepnumber=0]{}
void zero_y( S * s_p );
\end{lstlisting}
\noindent La routine assembleur serait :
\begin{AsmCodeListing}
%define      y_offset  4
_zero_y:
      enter  0,0
      mov    eax, [ebp + 8]      ; r�cup�re s_p depuis la pile
      mov    dword [eax + y_offset], 0
      leave
      ret
\end{AsmCodeListing}

Le C permet de passer une structure par valeur � une fonction ; cependant,
c'est une mauvaise id�e la plupart du temps. Toutes les donn�es de la structure
doivent �tre copi�es sur sur la pile puis r�cup�r�es par la routine. Il est
beaucoup plus efficace de passer un pointeur vers la structure � la place.

Le C permet aussi qu'une fonction renvoie une structure. Evidemment, une structure
ne peut pas �tre retourn�e dans le registre {\code EAX}. Des compilateurs diff�rents
g�rent cette situation de fa�on diff�rente. Une situation courante que les compilateurs
utilisent est de r��crire la fonction en interne de fa�on � ce qu'elle prenne un pointeur
sur la structure en param�tre. Le pointeur est utilis� pour placer la valeur de retour
dans une structure d�finie en dehors de la routine appel�e.

La plupart des assembleur (y compris NASM) ont un support int�gr� pour d�finir
des structures dans votre code assembleur. Reportez vous � votre documentation
pour plus de d�tails.

% add section on structure return values for functions

\index{structures|)}

\section{Assembleur et C++\index{C++|(}}

Le langage de programmation C++ est une extension du langage C. Beaucoup des
r�gles valables pour interfacer le C et l'assembleur s'appliquent �galement au C++.
Cependant, certaines r�gles doivent �tre modifi�es. De plus, certaines
extension du C++ sont plus faciles � comprendre en connaissant le langage
assembleur. Cette section suppose une connaissance basique du C++.

\subsection{Surcharge et D�coration de Noms\index{C++!d�coration de noms|(}}
\label{subsec:mangling}
\begin{figure}
\centering
\begin{lstlisting}[frame=tlrb]{}
#include <stdio.h>

void f( int x )
{
  printf("%d\n", x);
}

void f( double x )
{
  printf("%g\n", x);
}
\end{lstlisting}
\caption{Deux fonctions {\code f()}\label{fig:twof}}
\end{figure}

Le C++ permet de d�finir des fonctions (et des fonctions membres) diff�rentes
avec le m�me nom. Lorsque plus d'une fonction partagent le m�me nom,
les fonctions sont dites \emph{surcharg�es}. Si deux fonctions sont
d�finies avec le m�me nom en C, l'�diteur de liens produira une erreur
car il trouvera deux d�finitions pour le m�me symbole dans les fichiers
objets qu'il est en train de lier. Par exemple, prenons le code de la
Figure~\ref{fig:twof}. Le code assembleur �quivalent d�finirait deux
�tiquettes appel�es {\code \_f} ce qui serait bien s�r une erreur.

Le C++ utilise le m�me proc�d� d'�dition de liens que le C mais �vite
cette erreur en effectuant une \emph{d�coration de nom} (name mangling)
ou en modifiant le symbole utilis� pour nommer une fonction. D'une
certaine fa�on, le C utilise d�j� la d�coration de nom. Il ajoute un
caract�re de soulignement au nom de la fonction C lorsqu'il cr�e l'�tiquette
pour la fonction. Cependant, il d�corera le nom des deux fonctions de
la Figure~\ref{fig:twof} de la m�me fa�on et produira une erreur. Le
C++ utilise un proc�d� de d�coration plus sophistiqu� qui produit deux
�tiquettes diff�rentes pour les fonctions. Par exemple, la premi�re fonction
de la Figure~\ref{fig:twof} recevrait l'�tiquette {\code \_f\_\_Fi} 
et la seconde, {\code \_f\_\_Fd}, sous DJGPP. Cela �vite toute
erreur d'�dition de liens.
% check to make sure that DJGPP does still but an _ at beginning for C++

Malheureusement, il n'y a pas de standard sur la gestion des noms en C++ et
des compilateurs diff�rents d�corent les noms de fa�on diff�rente. Par
exemple, Borland C++ utiliserait les �tiquettes {\code @f\$qi} et {\code @f\$qd} 
pour les deux fonctions de la Figure~\ref{fig:twof}. Cependant, les r�gles
ne sont pas totalement arbitraires. Le nom d�cor� encode la \emph{signature}
de la fonction. La signature d'une fonction est donn�e par l'ordre et le type
de ses param�tres. Notez que la fonction qui ne prend qu'un argument {\code int} 
a un \emph{i} � la fin de son nom d�cor� (� la fois sous DJGPP et Borland) et
que celle qui prend un argument {\code double} a un \emph{d} � la fin de son
nom d�cor�. S'il y avait une fonction appel�e {\code f} avec le prototype
suivant :
\begin{lstlisting}[stepnumber=0]{}
  void f( int x, int y, double z);
\end{lstlisting}
\noindent DJGPP d�corerait son nom en {\code \_f\_\_Fiid} et Borland en
{\code @f\$qiid}.

Le type de la fonction ne fait \emph{pas} partie de la signature d'une
fonction et n'est pas encod� dans nom d�cor�. Ce fait explique une r�gle
de la surcharge en C++. Seules les fonctions dont les signatures sont
uniques peuvent �tre surcharg�es. Comme on le voit, si deux fonctions
avec le m�me nom et la m�me signature sont d�finies en C++, elle
donneront le m�me nom d�cor� et cr�eront une erreur lors de l'�dition de
liens. Par d�faut, toutes les fonctions C++ sont d�cor�es, m�me celles
qui ne sont pas surcharg�es. Lorsqu'il compile un fichier, le compilateur
n'a aucun moyen de savoir si une fonction particuli�re est surcharg�e
ou non, il d�core donc touts les noms. En fait, il d�core �galement les
noms des variables globales en encodant le type de la variable d'une
fa�on similaire � celle utilis�e pour les signatures de fonctions.
Donc, si l'on d�finit une variable globale dans un fichier avec un
certain type puis que l'on essaie de l'utiliser dans un autre fichier
avec le mauvais type, l'�diteur de liens produira une erreur. Cette
caract�ristique du C++  est connue sous le nom de \emph{typesafe
linking} (�dition de liens avec respect des types)
\index{C++!�dition de liens avec respect des types}. Cela cr�e un autre type d'erreurs,
les prototypes inconsistants. Cela arrive lorsque la d�finition d'une
fonction dans un module ne correspond pas avec le prototype utilis�
par un autre module. En C, cela peut �tre un probl�me tr�s difficile
� corriger. Le C ne d�tecte pas cette erreur. Le programme compilera
et sera li� mais aura un comportement impr�visible car le code appelant
placera sur la pile des types diff�rents de ceux que la fonction attend.
En C++ cela produira une erreur lors de l'�dition de liens.

Lorsque le compilateur C++ analyse un appel de fonction, il recherche
la fonction correspondante en observant les arguments qui lui sont
pass�s\footnote{La correspondance n'a pas � �tre exacte, le compilateur prendra
en compte les correspondances trouv�es en transtypant les arguments.
Les r�gles de ce proc�d� sont en dehors de la port�e de ce livre. Consultez
un livre sur le C++ pour plus de d�tails.}. S'il trouve une correspondance,
il cr�e un {\code CALL} vers la fonction ad�quate en utilisant les r�gles de
d�coration du compilateur.

Comme des compilateurs diff�rents utilisent diff�rentes r�gles de
d�coration de nom, il est possible que des codes C++ compil�s par
des compilateurs diff�rents ne puissent pas �tre li�s ensemble.
C'est important lorsque l'on a l'intention d'utiliser une biblioth�que
C++ pr�compil�e ! Si l'on veut �crire une fonction en assembleur qui
sera utilis�e avec du code C++, il faut conna�tre les r�gles de
d�coration de nom du compilateur C++ utilis� (ou utiliser la technique
expliqu�e plus bas).

L'�tudiant astucieux pourrait se demander si le code de la Figure~\ref{fig:twof}
fonctionnera de la fa�on attendue. Comme le C++ d�core toutes les fonctions,
alors la fonction {\code printf} sera d�cor�e et le compilateur ne
produira pas un {\code CALL} � l'�tiquette {\code \_printf}. C'est une question
pertinente. Si le prototype de {\code printf} �tait simplement plac� au d�but
du fichier, cela arriverait. Son prototype est :
\begin{lstlisting}[stepnumber=0]{}
  int printf( const char *, ...);
\end{lstlisting}
\noindent DJGPP d�corerait ce nom en {\code \_printf\_\_FPCce} ({\code F}
pour \emph{fonction}, {\code P} pour \emph{pointeur}, {\code C} pour \emph{const},
{\code c} pour \emph{char} et {\code e} pour ellipse). Cela n'appelerait pas la
fonction {\code printf} de la biblioth�que C standard ! Bien s�r, il doit y avoir
un moyen pour que le C++ puisse appeler du code C. C'est tr�s important car il
existe une \emph{�norme} quantit� de vieux code C utile. En plus de permettre l'acc�s
au code h�rit� de C, le C++ permet �galement d'appeler du code assembleur en
utilisant les conventions de d�coration standards du C.

\index{C++!extern ""C""|(}
Le C++ �tend le mot-cl� {\code extern} pour lui permettre de sp�cifier que la
fonction ou la variable globale qu'il modifie utilise les conventions C normales.
Dans la terminologie C++, la fonction ou la variable globale utilise une
\emph{�dition de liens C}. Par exemple, pour d�clarer la fonction {\code printf}
comme ayant une �dition de liens C, utilisez le prototype :
\begin{lstlisting}[language=C++,stepnumber=0]{}
extern "C" int printf( const char *, ... );
\end{lstlisting}
\noindent Cela impose au compilateur de ne pas utiliser les r�gles de d�coration
de nom du C++ sur la fonction, mais d'utiliser les r�gles C � la place. Cependant,
en faisant cela, la fonction {\code printf} ne peut pas �tre surcharg�e. Cela
constitue la fa�on la plus simple d'interfacer du C++ et de l'assembleur, d�finir
une fonction comme utilisant une �dition de liens C puis utiliser la convention d'appel
C.

Pour plus de facilit�, le C++ permet �galement de d�finir une �dition de liens C
sur un bloc de fonctions et de variables globales. Le bloc est indiqu� en utilisant
les accolades habituelles.
\begin{lstlisting}[stepnumber=0,language=C++]{}
extern "C" {
  /* variables globales et prototypes des fonction ayant une �dition de liens C */
}
\end{lstlisting}

Si l'on examine les fichiers d'en-t�te ANSI C fournis avec les compilateur C/C++
actuels, on trouve ce qui suit vers le d�but de chaque fichier d'en-t�te :
\begin{lstlisting}[stepnumber=0,language=C++]{}
#ifdef __cplusplus
extern "C" {
#endif
\end{lstlisting}
\noindent Et une construction similaire, vers la fin, contenant une accolade
fermante. Les compilateurs C++ d�finissent la macro {\code \_\_cplusplus}
(avec \emph{deux} caract�res de soulignement au d�but). L'extrait ci-dessus
entoure tout le fichier d'en-t�te dans un bloc {\code extern~"C"} si le fichier
d'en-t�te est compil� en C++, mais ne fait rien s'il est compil� en C
(puisqu'un compilateur C g�n�rerait une erreur de syntaxe sur {\code extern~"C"}).
La m�me technique peut �tre utilis�e par n'importe quel programmeur pour
cr�er un fichier d'en-t�te pour des routines assembleur pouvant �tre utilis�es
en C ou en C++.
\index{C++!extern ""C""|)}
\index{C++!d�coration de noms|)}

\begin{figure}
\begin{lstlisting}[language=C++,frame=tlrb]{}
void f( int \& x )     // le \& indique un param�tre par r�f�rence
{ x++; }

int main()
{
  int y = 5;
  f(y);               // une r�f�rence sur y est pass�e, pas de \& ici !
  printf("%d\n", y);  // affiche 6 !
  return 0;
}
\end{lstlisting}
\caption{Exemple de r�f�rence \label{fig:refex}}
\end{figure}

\subsection{R�f�rences\index{C++!r�f�rences|(}}

Les \emph{r�f�rences} sont une autre nouvelle fonctionnalit� du C++.
Elles permettent de passer des param�tres � une fonction sans utiliser
explicitement de pointeur. Par exemple, consid�rons le code de la
Figure~\ref{fig:refex}. En fait, les param�tres par r�f�rence sont plut�t
simples, ce sont des pointeurs. Le compilateur masque simplement ce fait
aux yeux du programmeur (exactement de la m�me fa�on que les compilateurs
Pascal qui impl�mentent les param�tres {\code var} comme des pointeurs).
Lorsque le compilateur g�n�re l'assembleur pour l'appel de fonction
ligne~7, il passe l'\emph{adresse} de {\code y}. Si l'on �crivait la
fonction {\code f} en assembleur, on ferait comme si le prototype �tait
\footnote{Bien s�r, il faudrait d�clarer la fonction avec une �dition
de liens en C, comme nous en avons parl� dans la Section~\ref{subsec:mangling}} :
\begin{lstlisting}[stepnumber=0]{}
  void f( int * xp);
\end{lstlisting}

Les r�f�rences sont juste une facilit� qui est particuli�rement utile
pour la surcharge d'op�rateurs. C'est une autre fonctionnalit� du C++
qui permet de donner une signification aux op�rateurs de base lorsqu'ils
sont appliqu�s � des structures ou des classes. Par exemple, une utilisation
courante est de d�finir l'op�rateur plus ({\code +}) de fa�on � ce qu'il
concat�ne les objets string. Donc, si {\code a} et {\code b} sont des
strings, {\code a~+~b} renverra la concat�nation des cha�nes 
{\code a} et {\code b}. Le C++ appelerait en r�alit� une fonction pour ce faire
(en fait, cette expression pourrait �tre r��crite avec une notation fonction
sous la forme {\code operator~+(a,b)}).  Pour plus d'efficacit�, il est
souhaitable de passer l'adresse des objets string � la place de les passer
par valeur. Sans r�f�rence, cela pourrait �tre fait en �crivant
{\code operator~+(\&a,\&b)}, mais cela imposerait d'�crire l'op�rateur avec
la syntaxe {\code \&a~+~\&b}. Cela serait tr�s maladroit et confus. Par
contre, en utilisant les r�f�rences, il est possible d'�crire {\code a~+~b}, 
ce qui semble tr�s naturel.
\index{C++!r�f�rences|)}

\subsection{Fonctions inline\index{C++!fonctions inline|(}}

Les \emph{fonctions inline} sont encore une autre fonctionnalit� du 
C++\footnote{Les compilateurs supportent souvent cette fonctionnalit�
comme une extension du C ANSI.}. Les fonctions inline sont destin�es
� remplacer les macros du pr�processeur qui prennent des param�tres,
sources d'erreur. Sourvenez vous en C, une macro qui �l�ve un
nombre au carr� ressemble � cela :
\begin{lstlisting}[stepnumber=0]{}
#define SQR(x) ((x)*(x))
\end{lstlisting}
\noindent Comme le pr�processeur ne comprend pas le C et ne fait que de
simples substitutions, les parenth�ses sont requises pour calculer le
r�sultat correct dans la plupart des cas. Cependant, m�me cette version
ne donnerait pas la bonne r�ponse pour {\code SQR(x++)}.

\begin{figure}
\begin{lstlisting}[language=C++,frame=tlrb]{}
inline int inline_f( int x ) 
{ return x*x; }

int f( int x ) 
{ return x*x; }

int main()
{
  int y, x = 5;
  y = f(x);
  y = inline_f(x);
  return 0;
}
\end{lstlisting}
\caption{Exemple d'inlining \label{fig:InlineFun}}
\end{figure}


Les macros sont utilis�es car elles �liminent la surcharge d'un appel
pour une fonction simple. Comme le chapitre sur les sous-programmes l'a
d�montr�, effectuer un appel de fonction implique plusieurs �tapes. Pour
une fonction tr�s simple, le temps pass� � l'appeler peut �tre
plus grand que celui pass� dans la fonction !
Les fonctions inline sont une fa�on beaucoup plus pratique d'�crire du
code qui ressemble � une fonction mais qui n'effectue \emph{pas} de
{\code CALL} � un bloc commun. Au lieu de cela, les appels � des fonctions
inline sont remplac�s par le code de la fonction. Le C++ permet de rendre
une fonction inline en pla�ant le mot-cl� {\code inline} au d�but de
sa d�finition. Par exemple, consid�rons les fonctions d�clar�es dans la
Figure~\ref{fig:InlineFun}. L'appel � la fonction {\code f}, ligne~10,
est un appel de fonction normal (en assembleur, en supposant que
{\code x} est � l'adresse {\code ebp-8} et {\code y} en {\code ebp-4}):
\begin{AsmCodeListing}
      push   dword [ebp-8]
      call   _f
      pop    ecx
      mov    [ebp-4], eax
\end{AsmCodeListing}
Cependant, l'appel � la fonction {\code inline\_f}, ligne~11 ressemblerait �  :
\begin{AsmCodeListing}
      mov    eax, [ebp-8]
      imul   eax, eax
      mov    [ebp-4], eax
\end{AsmCodeListing}

Dans ce cas, il y a deux avantages � inliner. Tout d'abord, la fonction
inline est plus rapide. Aucun param�tre n'est plac� sur la pile, aucun cadre
de pile n'est cr�� puis d�truit, aucun branchement n'est effectu�. Ensuite,
l'appel � la fonction inline utilise moins de code ! Ce dernier point est vrai
pour cet exemple, mais ne reste pas vrai dans tous les cas.

La principal inconv�nient de l'inlining est que le code inline n'est
pas li� et donc le code d'une fonction inline doit �tre disponible pour
\emph{tous} les fichiers qui l'utilisent. L'exemple de code assembleur pr�c�dent
le montre. L'appel de la fonction non-inline ne n�cessite que la connaissance
des param�tres, du type de valeur de retour, de la convention d'appel et
du nom de l'�tiquette de la fonction. Toutes ces informations sont disponibles
par le biais du prototype de la fonction. Cependant, l'utilisation de la
fonction inline n�cessite la connaissance de tout le code de la fonction.
Cela signifie que si \emph{n'importe quelle} partie de la fonction change,
\emph{tous} les fichiers source qui utilisent la fonction doivent �tre recompil�s.
Souvenez vous que pour les fonctions non-inline, si le prototype ne change
pas, souvent les fichiers qui utilisent la fonction n'ont pas besoin
d'�tre recompil�s. Pour toutes ces raisons, le code des fonctions inline
est g�n�ralement plac� dans les fichiers d'en-t�te. Cette pratique est
contraire � la r�gle stricte habituelle du C selon laquelle on ne doit
\emph{jamais} placer de code ex�cutable dans les fichiers d'en-t�te.
\index{C++!fonctions inline|)}

\begin{figure}[t]
\begin{lstlisting}[language=C++,frame=tlrb]{}
class Simple {
public:
  Simple();                // constructeur par d�faut
  ~Simple();               // destructeur
  int get_data() const;    // fonctions membres
  void set_data( int );
private:
  int data;                // donn�es membres
};

Simple::Simple()
{ data = 0; }

Simple::~Simple()
{ /* rien */ }

int Simple::get_data() const
{ return data; }

void Simple::set_data( int x )
{ data = x; }
\end{lstlisting}
\caption{Une classe C++ simple\label{fig:SimpleClass}}
\end{figure}

\subsection{Classes\index{C++!classes|(}}

Une classe C++ d�crit un type d'\emph{objet}. Un objet poss�de � la fois
%********************************* ce sont les membres qui sont appel�s
des membres donn�es et des membres fonctions\footnote{Souvent appel�s
\emph{fonctions membres} en C++ ou plus g�n�ralement \emph{m�thodes}\index{m�thodes}.}. 
En d'autres termes, il s'agit d'une {\code struct} � laquelle sont associ�es des
donn�es et des fonctions. Consid�rons la classe simple d�finie dans la
Figure~\ref{fig:SimpleClass}. Une variable de type {\code Simple} ressemblerait
� une {\code struct} C normale avec un seul membre {\code int}.
\MarginNote{En fait, le C++ utilise le mot cl� {\code this} pour
acc�der au pointeur vers l'objet lorsque l'on se trouve � l'int�rieur
d'une fonction membre.} Les fonctions ne sont \emph{pas} affect�es � la
structure en m�moire. Cependant, les fonctions membres sont diff�rentes
des autres fonctions. On leur passe un param�tre \emph{cach�}. Ce param�tre
est un pointeur vers l'objet sur lequel agit la fonction.

\begin{figure}[t]
\begin{lstlisting}[stepnumber=0]{}
void set_data( Simple * object, int x )
{
  object->data = x;
}
\end{lstlisting}
\caption{Version C de Simple::set\_data()\label{fig:SimpleCVer}}
\end{figure}


\begin{figure}[t]
\begin{AsmCodeListing}
_set_data__6Simplei:           ; nom d�cor�
      push   ebp
      mov    ebp, esp

      mov    eax, [ebp + 8]   ; eax = pointeur vers l'objet (this)
      mov    edx, [ebp + 12]  ; edx = param�tre entier
      mov    [eax], edx       ; data est au d�placement 0

      leave
      ret
\end{AsmCodeListing}
\caption{Traduction de Simple::set\_data( int ) par le compilateur\label{fig:SimpleAsm}}
\end{figure}


Par exemple, consid�rons la m�thode {\code set\_data} de la classe {\code
Simple} de la Figure~\ref{fig:SimpleClass}. Si elle �tait �crite en C, elle
ressemblerait � une fonction � laquelle on passerait explicitement un pointeur
vers l'objet sur lequel elle agit comme le montre le code de la 
Figure~\ref{fig:SimpleCVer}.  L'option {\code -S} du compilateur
\emph{DJGPP} (et des compilateurs \emph{gcc} et Borland �galement)
indique au compilateur de produire un fichier assembleur contenant
l'�quivalent assembleur du code. Pour \emph{DJGPP} et \emph{gcc}
le fichier assembleur poss�de une extension {\code .s} et utilise
malheureusement la syntaxe du langage assembleur AT\&T qui est assez
diff�rente des syntaxes NASM et MASM\footnote{Le compilateur \emph{gcc}
inclut son propre assembleur appel� \emph{gas}\index{gas}. L'assembleur
\emph{gas} utilise la syntaxe AT\&T et donc le compilateur produit un
code au format \emph{gas}. Il y a plusieurs sites sur le web qui expliquent
les diff�rences entre les formats INTEL et AT\&T. Il existe �galement
un programme gratuit appel� {\code a2i} ({http://www.multimania.com/placr/a2i.html}), 
qui passe du format AT\&T au format NASM.}
(les compilateurs Borland et MS g�n�rent un fichier
avec l'extension {\code .asm} utilisant la syntaxe MASM.)
La Figure~\ref{fig:SimpleAsm} montre la sortie de \emph{DJGPP} convertie
en syntaxe NASM avec des commentaires suppl�mentaires pour clarifier
le but des instructions. Sur la toute premi�re ligne, notez
que la m�thode {\code set\_data} re�oit une �tiquette d�cor�e
qui encode le nom de la m�thode, le nom de la classe et les param�tres.
Le nom de la classe est encod� car d'autres classes peuvent avoir une
m�thode appel�e {\code set\_data} et les deux m�thodes \emph{doivent}
recevoir des �tiquettes diff�rentes. Les param�tres sont encod�s afin que
la classe puisse surcharger la m�thode {\code set\_data} afin qu'elle
prenne d'autres param�tres, comme les fonctions C++ normales. Cependant,
comme pr�c�demment, des compilateurs diff�rents encoderont ces informations
diff�remment dans l'�tiquette.

Ensuite, aux lignes~2 et 3 nous retrouvons le prologue habituel.
A la ligne~5, le premier param�tre sur la pile est stock� dans {\code
EAX}. Ce n'est \emph{pas} le param�tre {\code x} ! Il s'agit du
param�tre cach�\footnote{Comme d'habitude, \emph{rien} n'est cach�
dans le code assembleur !} qui pointe vers l'objet sur lequel on agit.
La ligne~6 stocke le param�tre {\code x} dans {\code EDX} et la ligne~7
stocke {\code EDX} dans le double mot sur lequel pointe {\code EAX}. Il
s'agit du membre {\code data} de l'objet {\code Simple} sur lequel on agit,
qui, �tant la seule donn�e de la classe, est stock� au d�placement 0 de la
structure {\code Simple}.

\begin{figure}[tp]
\begin{lstlisting}[frame=tlrb,language=C++]{}
class Big_int {
public:
   /* 
   * Param�tres :
   *   size           - taille de l'entier exprim�e en nombre d'unsigned
   *                    int normaux
   *   initial_value  - valeur initiale du Big_int sous forme d'un
   *                    unsigned int normal
   */
  explicit Big_int( size_t   size,
                    unsigned initial_value = 0);
  /*
   * Param�tres :
   *   size           - taille de l'entier exprim�e en nombre d'unsigned
   *                    int normaux
   *   initial_value  - valeur initiale du Big_int sous forme d'une cha�ne
   *                    contenant une repr�sentation hexad�cimale de la 
   *                    valeur.
   */
  Big_int( size_t       size,
           const char * initial_value);

  Big_int( const Big_int & big_int_to_copy);
  ~Big_int();

  // renvoie la taille du Big\_int (en termes d'unsigned int)
  size_t size() const;

  const Big_int & operator = ( const Big_int & big_int_to_copy);
  friend Big_int operator + ( const Big_int & op1,
                              const Big_int & op2 );
  friend Big_int operator - ( const Big_int & op1,
                              const Big_int & op2);
  friend bool operator == ( const Big_int & op1,
                            const Big_int & op2 );
  friend bool operator < ( const Big_int & op1,
                           const Big_int & op2);
  friend ostream & operator << ( ostream &       os,
                                 const Big_int & op );
private:
  size_t      size_;    // taille du tableau d'unsigned
  unsigned *  number_;  // pointeur vers un tableau d'unsigned contenant
                           la valeur
};
\end{lstlisting}
\caption{D�finition de la classe Big\_int\label{fig:BigIntClass}}
\end{figure}

\begin{figure}[tp]
\begin{lstlisting}[frame=tlrb,language=C++]{}
// prototypes des routines assembleur
extern "C" {
  int add_big_ints( Big_int &       res, 
                    const Big_int & op1, 
                    const Big_int & op2);
  int sub_big_ints( Big_int &       res, 
                    const Big_int & op1, 
                    const Big_int & op2);
}

inline Big_int operator + ( const Big_int & op1, const Big_int & op2)
{
  Big_int result(op1.size());
  int res = add_big_ints(result, op1, op2);
  if (res == 1)
    throw Big_int::Overflow();
  if (res == 2)
    throw Big_int::Size_mismatch();
  return result;
}

inline Big_int operator - ( const Big_int & op1, const Big_int & op2)
{
  Big_int result(op1.size());
  int res = sub_big_ints(result, op1, op2);
  if (res == 1)
    throw Big_int::Overflow();
  if (res == 2)
    throw Big_int::Size_mismatch();
  return result;
}
\end{lstlisting}
\caption{Code de l'Arithm�tique sur la Classe Big\_int\label{fig:BigIntAdd}}
\end{figure}

\subsubsection{Exemple}
\index{C++!exemple Big\_int|(}
Cette section utilise les id�es de ce chapitre pour cr�er une classe C++
qui repr�sente un entier non sign� d'une taille arbitraire. Comme l'entier
peut faire n'importe quelle taille, il sera stock� dans un tableau
d'entiers non sign�s (doubles mots). Il peut faire n'importe quelle taille
en utilisant l'allocation dynamique. Les doubles mots sont stock�s dans
l'ordre inverse\footnote{Pourquoi ? Car les op�rations d'addition commenceront
ainsi toujours par le d�but du tableau et avanceront.}  (\emph{i.e.} 
le double mot le moins significatif est au d�placement 0). 
La Figure~\ref{fig:BigIntClass} montre la d�finition de la classe
{\code Big\_int}\footnote{Voyez le code source d'exemple pour obtenir
le code complet de cet exemple. Le texte ne se r�f�rera qu'� certaines parties
du code.}. La taille d'un {\code Big\_int} est mesur�e par la taille du
tableau d'{\code unsigned} utilis� pour stocker les donn�es. La donn�e
membre {\code size\_} de la classe est affect�e au d�placement 0 et le membre
{\code number\_} est affect� au d�placement 4.

Pour simplifier l'exemple, seuls les objets ayant des tableaux de la m�me
taille peuvent �tre additionn�s entre eux.

La classe a trois constructeurs : le premier (ligne~9) initialise l'instance
de la classe en utilisant un entier non sign� normal ; le second, 
(ligne~19) initalise l'instance en utilisant une cha�ne qui contient une
valeur hexad�cimale. Le troisi�me constructeur (ligne~22) est le 
\emph{constructeur par copie}\index{C++!constructeur par copie}.

Cette explication se concentre sur la fa�on dont fonctionnent les
op�rateurs d'addition et de soustraction car c'est l� que l'on utilise
de l'assembleur. La Figure~\ref{fig:BigIntAdd} montre les parties du
fichier d'en-t�te relatives � ces op�rateurs. Elles montrent comment
les op�rateurs sont param�tr�s pour appeler des routines assembleur.
Comme des compilateurs diff�rents utilisent des r�gles de d�coration
radicalement diff�rentes pour les fonctions op�rateur, des fonctions
op�rateur inline sont utilis�es pour initialiser les appels aux routines
assembleur li�es au format C. Cela les rend relativement simples � porter
sur des compilateurs diff�rents et est aussi rapide qu'un appel direct.
Cette technique �limine �galement le besoin de soulever une exception depuis
l'assembleur !

Pourquoi l'assembleur n'est-il utilis� qu'ici ? Souvenez vous que
pour effectuer de l'arithm�tique en pr�cision multiple, la retenue
doit �tre ajout�e au double mot significatif suivant. Le C++ (et le C)
ne permet pas au programmeur d'acc�der au drapeau de retenue du processeur.
On ne pourrait effectuer l'addition qu'en recalculant ind�pendamment en C++
la valeur du drapeau de retenue et en l'ajoutant de fa�on conditionnelle au
double mot suivant. Il est beaucoup plus efficace d'�crire le code en
assembleur � partir duquel on peut acc�der au drapeau de retenue et utiliser
l'instruction {\code ADC} qui ajoute automatiquement le drapeau de retenue.

Par concision, seule la routine assembleur {\code add\_big\_ints} sera expliqu�e
ici. Voici le code de cette routine (contenu dans {\code big\_math.asm}) :
\begin{AsmCodeListing}[label=big\_math.asm]
segment .text
        global  add_big_ints, sub_big_ints
%define size_offset 0
%define number_offset 4

%define EXIT_OK 0
%define EXIT_OVERFLOW 1
%define EXIT_SIZE_MISMATCH 2

; Param�tres des routines add et sub
%define res ebp+8
%define op1 ebp+12
%define op2 ebp+16

add_big_ints:
        push    ebp
        mov     ebp, esp
        push    ebx
        push    esi
        push    edi
        ;
        ; initialise esi pour pointer vers op1
        ;            edi pour pointer vers op2
        ;            ebx pour pointer vers res
        mov     esi, [op1]
        mov     edi, [op2]
        mov     ebx, [res]
        ;
        ; s'assure que les 3 Big_int ont la m�me taille
        ;
        mov     eax, [esi + size_offset]
        cmp     eax, [edi + size_offset]
        jne     sizes_not_equal                 ; op1.size_ != op2.size_
        cmp     eax, [ebx + size_offset]
        jne     sizes_not_equal                 ; op1.size_ != res.size_

        mov     ecx, eax                        ; ecx = taille des Big_int
        ;
        ; initialise les registres pour qu'ils pointent vers leurs tableaux respectifs
        ;      esi = op1.number_
        ;      edi = op2.number_
        ;      ebx = res.number_
        ;
        mov     ebx, [ebx + number_offset]
        mov     esi, [esi + number_offset]
        mov     edi, [edi + number_offset]
        
        clc                                     ; met le drapeau de retenue � 0
        xor     edx, edx                        ; edx = 0
        ;
        ; boucle d'addition
add_loop:
        mov     eax, [edi+4*edx]
        adc     eax, [esi+4*edx]
        mov     [ebx + 4*edx], eax
        inc     edx                             ; ne modifie pas le drapeau de retenue
        loop    add_loop

        jc      overflow
ok_done:
        xor     eax, eax                        ; valeur de retour = EXIT_OK
        jmp     done
overflow:
        mov     eax, EXIT_OVERFLOW
        jmp     done
sizes_not_equal:
        mov     eax, EXIT_SIZE_MISMATCH
done:
        pop     edi
        pop     esi
        pop     ebx
        leave
        ret
\end{AsmCodeListing}

Heureusement, la majorit� de ce code devrait �tre compr�hensible pour le
lecteur maintenant. Les lignes~25 � 27 stockent les pointeurs vers les 
objets {\code Big\_int} pass�s � la fonction via des registres.
Souvenez vous que les r�f�rences sont des pointeurs. Les lignes~31 �
35 v�rifient que les tailles des trois objets sont les m�mes
(Notez que le d�placement de {\code size\_} est ajout� au pointeur pour
acc�der � la donn�e membre). Les lignes~44 � 46 ajustent les registres
pour qu'ils pointent vers les tableaux utilis�s par leurs objets respectifs
au lieu des objets eux-m�mes (l� encore, le d�placement du membre
{\code number\_} est ajout� au pointeur sur l'objet).

\begin{figure}[tp]
\begin{lstlisting}[language=C++, frame=tlrb]{}
#include "big_int.hpp"
#include <iostream>
using namespace std;

int main()
{
  try {
    Big_int b(5,"8000000000000a00b");
    Big_int a(5,"80000000000010230");
    Big_int c = a + b;
    cout << a << " + " << b << " = " << c << endl;
    for( int i=0; i < 2; i++ ) {
      c = c + a;
      cout << "c = " << c << endl;
    }
    cout << "c-1 = " << c - Big_int(5,1) << endl;
    Big_int d(5, "12345678");
    cout << "d = " << d << endl;
    cout << "c == d " << (c == d) << endl;
    cout << "c > d " << (c > d) << endl;
  }
  catch( const char * str ) {
    cerr << "Caught : " << str << endl;
  }
  catch( Big_int::Overflow ) {
    cerr << "D�passement de capacit� " << endl;
  }
  catch( Big_int::Size_mismatch ) {
    cerr << "Non concordance de taille" << endl;
  }
  return 0;
}
\end{lstlisting}
\caption{ Utilisation Simple de {\code Big\_int} \label{fig:BigIntEx}}
\end{figure}

La boucle des lignes~52 � 57 additionne les entiers stock�s dans les tableaux
en additionnant le double mot le moins significatif en premier, puis les
doubles mots suivants, \emph{etc.} L'addition doit �tre effectu�e dans cet
ordre pour l'arithm�tique en pr�cision �tendue (voir Section~\ref{sec:ExtPrecArith}).
La ligne~59 v�rifie qu'il n'y a pas de d�passement de capacit�, lors d'un
d�passement de capacit�, le drapeau de retenue sera allum� par la derni�re
addition du double mot le plus significatif. Comme les doubles mots du tableau
sont stock�s dans l'ordre little endian, la boucle commence au d�but du tableau
et avance jusqu'� la fin.

La Figure~\ref{fig:BigIntEx} montre un court exemple utilisant la classe
{\code Big\_int}. Notez que les constantes de {\code Big\_int} doivent �tre
d�clar�es explicitement comme � la ligne~16. C'est n�cessaire pour deux raisons.
Tout d'abord, il n'y a pas de constructeur de conversion qui convertisse un
entier non sign� en {\code Big\_int}. Ensuite, seuls les {\code Big\_int}
de m�me taille peuvent �tr additionn�s. Cela rend la conversion probl�matique
puisqu'il serait difficile de savoir vers quelle taille convertir. Une
impl�mentation plus sophistiqu�e de la classe permettrait d'additionner
n'importe quelle taille avec n'importe quelle autre. L'auteur ne voulait
pas compliquer inutilement cet exemple en l'impl�mentant ici (cependant,
le lecteur est encourag� � le faire).
\index{C++!exemple Big\_int|)}

\begin{figure}[tp]
\begin{lstlisting}[language=C++, frame=tlrb]{}
#include <cstddef>
#include <iostream>
using namespace std;

class A {
public:
  void __cdecl m() { cout << "A::m()" << endl; }
  int ad;
};

class B : public A {
public:
  void __cdecl m() { cout << "B::m()" << endl; }
  int bd;
};

void f( A * p )
{
  p->ad = 5;
  p->m();
}

int main()
{
  A a;
  B b;
  cout << "Taille de a : " << sizeof(a)
       << " D�placement de ad : " << offsetof(A,ad) << endl;
  cout << "Taille de b : " << sizeof(b)
       << " D�placement de ad : " << offsetof(B,ad)
       << " D�placement de bd : " << offsetof(B,bd) << endl;
  f(&a);
  f(&b);
  return 0;
}
\end{lstlisting}
\caption{ H�ritage Simple\label{fig:SimpInh}}
\end{figure}


\subsection{H�ritage et Polymorphisme\index{C++!h�ritage|(}}


\begin{figure}[tp]
\begin{AsmCodeListing}
_f__FP1A:                       ; nom de fonction d�cor�
      push   ebp
      mov    ebp, esp
      mov    eax, [ebp+8]       ; eax pointe sur l'objet
      mov    dword [eax], 5     ; utilisation du d�placement 0 pour ad
      mov    eax, [ebp+8]       ; passage de l'adresse de l'objet � A::m()
      push   eax
      call   _m__1A             ; nom d�cor� de la m�thode A::m()
      add    esp, 4
      leave
      ret
\end{AsmCodeListing}
\caption{Code Assembleur pour un H�ritage Simple \label{fig:FAsm1}}
\end{figure}

L'\emph{h�ritage} permet � une classe d'h�riter des donn�es et des m�thodes
d'une autre. Par exemple, consid�rons le code de la Figure~\ref{fig:SimpInh}.
Il montre deux classes, {\code A} et {\code B}, o� la classe {\code B}
h�rite de {\code A}.
La sortie du programme est :
\begin{verbatim}
Taille de a : 4 D�placement de ad : 0
Taille de b : 8 D�placement de ad: 0 D�placement de bd: 4
A::m()
A::m()
\end{verbatim}
Notez que les membres {\code ad} des deux classes ({\code B} l'h�rite
de {\code A}) sont au m�me d�placement. C'est important puisque
l'on peut passer � la fonction {\code f} soit un pointeur vers un
objet {\code A} soit un pointeur vers un objet de n'importe quel type
d�riv� de (\emph{i.e.} qui h�rite de) {\code A}.  La Figure~\ref{fig:FAsm1} 
montre le code assembleur (�dit�) de la fonction (g�n�r� par \emph{gcc}).

\begin{figure}[tp]
\begin{lstlisting}[language=C++, frame=tlrb]{}
class A {
public:
  virtual void __cdecl m() { cout << "A::m()" << endl; }
  int ad;
};

class B : public A {
public:
  virtual void __cdecl m() { cout << "B::m()" << endl; }
  int bd;
};
\end{lstlisting}
\caption{ H�ritage Polymorphique\label{fig:VirtInh}}
\end{figure}

\index{C++!polymorphisme|(}
Notez que la sortie de la m�thode {\code m} de {\code A} a �t� produite
� la fois par l'objet {\code a} et l'objet {\code b}. D'apr�s l'assembleur,
on peut voir que l'appel � {\code A::m()} est cod� en dur dans la fonction.
Dans le cadre d'une vraie programmation orient�e objet, la m�thode appel�e
devrait d�pendre du type d'objet pass� � la fonction. On appelle cela le
\emph{polymorphisme}. Le C++ d�sactive cette fonctionnalit� par d�faut.
On utilise le mot-cl� \emph{virtual} \index{C++!virtual} pour l'activer.
La Figure~\ref{fig:VirtInh} montre comment les deux classes seraient
modifi�es. Rien dans le restet du code n'a besoin d'�tre chang�. Le
polymorphisme peut �tre impl�ment� de beaucoup de mani�res. Malheureusement,
l'impl�mentation de \emph{gcc} est en transition au moment d'�crire ces lignes
et devient beaucoup plus compliqu�e que l'impl�mentation initiale. Afin
de simplifier cette explication, l'auteur ne couvrira que l'impl�mentation
du polymorphisme que les compilateurs Microsoft et Borland utilisent. Cette
impl�mentation n'a pas chang� depuis des ann�es et ne changera probablement
pas dans un futur proche.

Avec ces changements, la sortie du programme change :
\begin{verbatim}
Size of a: 8 D�placement de ad : 4
Size of b: 12 D�placement de ad : 4 D�placement de bd : 8
A::m()
B::m()
\end{verbatim}


\begin{figure}[tp]
\begin{AsmCodeListing}[commentchar=!]
?f@@YAXPAVA@@@Z:
      push   ebp
      mov    ebp, esp

      mov    eax, [ebp+8]
      mov    dword [eax+4], 5  ; p->ad = 5;

      mov    ecx, [ebp + 8]    ; ecx = p
      mov    edx, [ecx]        ; edx = pointeur sur la vtable
      mov    eax, [ebp + 8]    ; eax = p
      push   eax               ; empile le pointeur "this"
      call   dword [edx]       ; appelle la premi�re fonction de la vtable
      add    esp, 4            ; nettoie la pile

      pop    ebp
      ret
\end{AsmCodeListing}
\caption{Code Assembleur de la Fonction {\code f()}\label{fig:FAsm2}}
\end{figure}

\begin{figure}[tp]
\begin{lstlisting}[language=C++, frame=tlrb]{}
class A {
public:
  virtual void __cdecl m1() { cout << "A::m1()" << endl; }
  virtual void __cdecl m2() { cout << "A::m2()" << endl; }
  int ad;
};

class B : public A {    // B h�rite du m2() de A
public:
  virtual void __cdecl m1() { cout << "B::m1()" << endl; }
  int bd;
};
/* affiche la vtable de l'objet fourni */
void print_vtable( A * pa )
{
  // p voit pa comme un tableau de doubles mots
  unsigned * p = reinterpret_cast<unsigned *>(pa);
  // vt voit la vtable comme un tableau de pointeurs
  void ** vt = reinterpret_cast<void **>(p[0]);
  cout << hex << "adresse de la vtable = " << vt << endl;
  for( int i=0; i < 2; i++ )
    cout << "dword " << i << " : " << vt[i] << endl;

  // appelle les fonctions virtuelle d'une fa�on ABSOLUMENT non portable
  void (*m1func_pointer)(A *);   // variable pointeur de fonction
  m1func_pointer = reinterpret_cast<void (*)(A*)>(vt[0]);
  m1func_pointer(pa);            // appelle m1 via le pointeur de fonction

  void (*m2func_pointer)(A *);   // variable pointeur de fonction
  m2func_pointer = reinterpret_cast<void (*)(A*)>(vt[1]);
  m2func_pointer(pa);            // appelle m2 via le pointeur de fonction
}

int main()
{
  A a;   B b1;  B b2;
  cout << "a: " << endl;   print_vtable(&a);
  cout << "b1: " << endl;  print_vtable(&b1);
  cout << "b2: " << endl;  print_vtable(&b2);
  return 0;
}
\end{lstlisting}
\caption{ Exemple plus compliqu� \label{fig:2mEx}}
\end{figure}


\begin{figure}[tp]
\centering
%\epsfig{file=vtable}
\input{vtable.latex}
\caption{Repr�sentation interne de {\code b1}\label{fig:vtable}}
\end{figure}

Maintenant, le second appel � {\code f} appelle la m�thode {\code B::m()}
car on lui passe un objet {\code B}. Ce n'est pas le seul changement cependant.
La taille d'un {\code A} vaut maintenant 8 (et 12 pour {\code B}). De plus,
le d�placement de {\code ad} vaut maintenant 4, plus 0. Qu'y a-t-il au
d�placement~0 ? La r�ponse � cette question est li�e � la fa�on dont
est impl�ment� le polymorphisme. 

\index{C++!vtable|(} Une classe C++ qui a une (ou plusieurs) m�thode(s)
virutelle(s) a un champ cach� qui est un pointeur vers un tableau de
pointeurs sur des m�thodes\footnote{Pour les classes sans m�thode virtuelle,
les compilateurs C++ rendent toujours la classe compatible avec une structure
C normale qui aurait les m�mes donn�es membres.}. Cette table est souvent
appel�e la \emph{vtable}. Pour les classes {\code A} et {\code B} 
ce pointeur est stock� au d�placement 0. Les compilateurs Windows placent
toujours ce pointeur au d�but de la classe au sommet de l'arbre d'h�ritage.
En regardant le code assembleur (Figure~\ref{fig:FAsm2}) g�n�r� pour la
fonction {\code f} (de la Figure~\ref{fig:SimpInh}) dans la
version du programme avec les m�thodes virtuelles, on peut voir que
l'appel � la m�thode {\code m} ne se fait pas via une �tiquette. La
ligne~9 trouve l'adresse de la vtable de l'objet. L'adresse de l'objet
est plac�e sur la pile ligne~11. La ligne~12 appelle la m�thode virtuelle
en se branchant � la premi�re adresse dans la vtable\footnote{Bien s�r,
la valeur est d�j� dans le registre {\code ECX}. Elle y a �t� plac�e
� la ligne~8 et le ligne~10 pourrait �tre supprim�e et la ligne suivante
chang�e de fa�on � empiler {\code ECX}. Le code n'est pas tr�s efficace
car il a �t� g�n�r� en d�sactivant les optimisations du compilateur.}.
Cet appel n'utilise pas d'�tiquette, il se branche � l'adresse du code
sur lequel pointe {\code EDX}. Ce type d'appel est un exemple de
\emph{liaison tardive}\index{C++!liaison tardive} (late binding). 
La liaison tardive repousse le choix de la m�thode � appeler au moment
de l'ex�cution du code. Cela permet d'appeler la m�thode correspondant
� l'objet. Le cas normal (Figure~\ref{fig:FAsm1}) code en dur un appel
� une certaine m�thode et est appel� \emph{liaison pr�coce}\index{C++!liaison pr�coce}
(early binding), car la m�thode est li�e au moment de la compilation.

Le lecteur attentif se demandera pourquoi les m�thodes de classe de la
Figure~\ref{fig:VirtInh} sont explicitement d�clar�es pour utiliser la
convention d'appel C en utilisant le mot-cl� {\code \_\_cdecl}.
Par d�faut, Microsoft utilise une convention diff�rente de la convention
C standard pour les m�thodes de classe C++. Il passe le pointeur
sur l'objet sur lequel agit la m�thode via le registre {\code ECX} 
au lieu d'utiliser la pile. La pile est toujours utilis�e pour les autres
param�tres explicites de la m�thode. Le modificateur
{\code \_\_cdecl} demande l'utilisation de la convention d'appel C standard.
Borland~C++ utilise la convention d'appel C par d�faut.

\begin{figure}[tp]
\fbox{ \parbox{\textwidth}{\code
a: \\
Adresse de la vtable = 004120E8\\
dword 0: 00401320\\
dword 1: 00401350\\
A::m1()\\
A::m2()\\
b1:\\
Adresse de la vtable = 004120F0\\
dword 0: 004013A0\\
dword 1: 00401350\\
B::m1()\\
A::m2()\\
b2:\\
Adresse de la vtable = 004120F0\\
dword 0: 004013A0\\
dword 1: 00401350\\
B::m1()\\
A::m2()\\
} }
\caption{Sortie du programme de la Figure~\ref{fig:2mEx} \label{fig:2mExOut}}
\end{figure}

Observons maintenant un exemple l�g�rement plus compliqu�
(Figure~\ref{fig:2mEx}). Dans celui-l�, les classes {\code A} et {\code B}
ont chacune deux m�thodes : {\code m1} et {\code m2}. Souvenez vous que
comme la classe {\code B} ne d�finit pas sa propre m�thode {\code m2},
elle h�rite de la m�thode de classe de {\code A}. La Figure~\ref{fig:vtable}
montre comment l'objet {\code b} appara�t en m�moire. La Figure~\ref{fig:2mExOut}
montre la sortie du programme. Tout d'abord, regardons l'adresse de la vtable
de chaque objet. Les adresses des deux objets {\code B} sont les m�mes, ils
partagent donc la m�me vtable. Une vtable appartient � une classe pas � un objet
(comme une donn�e membre {\code static}). Ensuite, regardons les adresses
dans les vtables. En regardant la sortie assembleur, on peut d�terminer que le
pointeur sur la m�thode {\code m1} est au d�placement~0 (ou dword~0) et
celui sur {\code m2} est au d�placement~4 (dword~1). Les pointeurs sur la
m�thode {\code m2} sont les m�mes pour les vtables des classes {\code A}
et {\code B} car la classe {\code B} h�rite de la m�thode {\code m2} de la
classe {\code A}.

Les lignes~25 � 32 montrent comment l'on pourrait appeler une fonction virtuelle
en lisant son adresse depuis la vtable de l'objet\footnote{Souvenez vous, ce
code ne fonctionne qu'avec les compilateurs MS et Borland, pas \emph{gcc}.}.
L'adresse de la m�thode est stock�e dans un pointeur de fonction de type C
avec un pointeur \emph{this} explicite.  D'apr�s la sortie de la 
Figure~\ref{fig:2mExOut}, on peut voir comment cela fonctionne. Cependant,
s'il vous pla�t, n'�crivez \emph{pas} de code de ce genre ! Il n'est utilis�
que pour illustrer le fait que les m�thodes virtuelles utilisent la vtable.

%Looking at the output of Figure~\ref{fig:2mExOut} does demonstrate several
%features of the implementation of polymorphism.  The {\code b1} and {\code b2}
%variables have the same vtable address; however the {\code a} variable
%has a different vtable address. The vtable is a property of the class not
%a variable of the class. All class variables share a common vtable. The two
%{\code dword} values in the table are the pointers to the virtual methods.
%The first one (number 0) is for {\code m1}. Note that it is different for the
%{\code A} and {\code B} classes. This makes sense since the A and B classes
%have different {\code m1} methods. However, the second method pointer is 
%the same for both classes, since class {\code B} inherits the {\code m2}
%method from its base class, {\code A}.

Il y a plusieurs le�ons pratiques � tirer de cela. Un fait important est
qu'il faut �tre tr�s attentif lorsque l'on �crit ou que l'on lit des
variables de type classe depuis un fichier binaire. On ne peut pas utiliser
simplement une lecture ou une �criture binaire car cela lirait ou �crirait
le pointeur vtable depuis le fichier ! C'est un pointeur sur l'endroit o�
la vtable r�side dans la m�moire du programme et il varie d'un programme �
l'autre. La m�me chose peut arriver avec les structures en C, mais en C,
les structures n'ont des pointeurs que si le programmeur en d�finit 
explicitement. Il n'y a pas de pointeurs explicites d�clar�s dans les
classes {\code A} ou {\code B}.


Une fois encore, il est n�cessaire de r�aliser que des compilateurs
diff�rents impl�mentent les m�thodes virtuelles diff�remment. Sous Windows,
les objets de la classe COM (Component Object Model)
\index{COM} utilisent des vtables pour impl�menter les interfaces 
COM\footnote{Les classes COM utilisent �galement la convention d'appel
{\code \_\_stdcall}\index{convention d'appel!stdcall}, pas la convention
C standard.}. Seuls les compilateurs qui impl�mentent les vtables des
m�thodes virtuelles comme le fait Microsoft peuvent cr�er des classes 
COM. C'est pourquoi Borland utilise la m�me impl�mentation que Microsoft
et une des raisons pour lesquelles \emph{gcc} ne peut pas �tre utilis�
pour cr�er des classes COM.

Le code des m�thodes virtuelles est identique � celui des m�thodes non-virtuelles.
Seul le code d'appel est diff�rent. Si le compilateur peut �tre absolument
s�r de la m�thode virtuelle qui sera appel�e, il peut ignorer la vtable
et appeler la m�thode directement (\emph{p.e.}, en utilisant la liaison pr�coce).
\index{C++!vtable|)}
\index{C++!polymorphisme|)}
\index{C++!h�ritage|)}
\index{C++!classes|)}
\index{C++|)}

\subsection{Autres fonctionnalit�s C++}

Le fonctionnement des autres fonctionnalit�s du C++ (\emph{p.e.},
le RunTime Type Information, la gestion des exceptions et l'h�ritage
multiple) d�passe le cadre de ce livre. Si le lecteur veut aller
plus loin, \emph{The Annotated C++ Reference Manual} de Ellis
et Stroustrup et \emph{The Design and Evolution of C++} de
Stroustrup constituent un bon point de d�part.


%%-*- latex -*-
\chapter{Dynamic Link Libraries}

\section{Using the Window's API and Dynamic Link Libraries}

UNIX systems provide a simple C based Application Programming
Interface (API).  In contrast, Microsoft Windows\texttrademark
\hspace{0.5em} packages its API in Dynamic Link Libraries that load in
each executables address space before its use. From the C/C++
programmers perspective, it appears as a normal C function call;
however, at the assembly level, it is different.

\subsection{Standard Call Calling Convention}
The Windows API uses the \emph{Standard Call}\index{calling
convention!standard call} calling convention. As stated eariler, this
convention pushes the arguments in reverse order just as the standard
C calling convention. However, there are two important
differences. First, the subroutine is responsible for clearing the
parameters from the stack. Secondly, the label of the function is
generated differently. An underscore is prepended the name as before,
but in addition the \emph{@} character is added to the end of the
function name along a number equal to the number of bytes on the stack
for the parameters of the function (in decimal).

\index{CloseHandle|(}
For example, consider the {\code CloseHandle} Windows API function. It's prototype
looks like:
\begin{lstlisting}[stepnumber=0]{}
BOOL WINAPI CloseHandle( HANDLE hObject );
\end{lstlisting}
Since a {\code HANDLE} is a double word in 32-bit Windows, the label
for this function would be {\code \_CloseHandle@4}. The {\code WINAPI}\index{WINAPI}
in the prototype is a C macro defined to be {\code \_\_stdcall}. Below
is a sample call to the function assuming that the {\code hObject}
value is in {\code EBX}.
\begin{AsmCodeListing}[frame=single]
  push  ebx             ; Push hObject on stack
  call  _CloseHandle@4
  mov   esi, eax        ; Save return value in ESI
\end{AsmCodeListing}
The stack does not have to adjusted after the function call, {\code CloseHandle}
fixes the stack automatically.
\index{CloseHandle|)}

\subsection{Static and Shared Libraries}

\index{static library|(}
A \emph{static library} is a collection of object files that can be
linked to an executable when it is created. The object code is
inserted directly into the executable just like an ordinary object
file. The library file is just a convenience. It allows a single file
to be included in the link step of the build process. All the object
code is probably not stored in the final executable. The linker will
look at which object modules are needed and only include the required
ones. Static libraries are created using the {\code LIB} program under
Windows or the {\code ar} program under UNIX. Windows libraries end in
a {\code .lib} extension and UNIX libraries end in a {\code .a}
extension.
\index{static library|)}

\index{shared library|(}
A shared library (or DLL in Windows) as its name implies shares code
among executables. When the executable is run, the OS finds all the 
shared libraries it requires and loads them into the processes memory
so the executable can use the code in them. Using the virtual memory
mapping capabilities on a protected mode operating system, shared
library code used by two or more concurrently running processes is
only loaded into physical memory once. 

There are advantages and disadvantages to shared libraries. The first
advantage is that executable sizes can be greatly reduced. If a large
library is used by many executables. Only one copy of the code (in the
shared library) is on the system (and in physical memory). If the
library was included statically, each executable would include a copy
of the code.

Another advantage is that if a bug is found in the shared library,
then the library can be replaced with one with the bug fixed. The
executables will automatically use the new fixed code without
recompiling the executable. However, this only works if the interface
of the functions in the shared library remain unchanged.

Shared libraries are also used to allow code written in different
languages to interoperate. For example, C++ can be interfaced to
Visual Basic\index{Visual Basic}, DotNet\index{DotNet} and
Java\index{Java} using shared libraries under Windows.

The disadvantages of shared libraries are that they can be more complicated
to maintain. One of the most common problems is that a new version of the
library breaks older code because it behaves slightly differently. Then
some executables need one version and others need a different one. In Windows,
this condition is known as \emph{DLL hell}\index{DLL hell}.
\index{shared library|)}

\index{DLL|(}
\subsection{Windows DLLs}

A Windows DLL is constructed much like an executable program. Object files are linked
together to create a DLL file. Unlike an executable, a DLL can have many entry points.
An entry point is just a function that can be called externally to the DLL. Only
functions that have been \emph{exported}\index{export} can be called externally.

A function (or global variable) may be exported by either entering its
name in the definition file\index{DLL!definition file} for the DLL or
by using Microsoft specific keywords.

\begin{figure}[t]
\begin{Verbatim}[frame=single,commandchars=\\\{\}]
LIBRARY \textit{library root name}
DESCRIPTION '\textit{short text description}'
EXPORTS
\textit{list of exported functions (one per line)}
\end{Verbatim}
\caption{Windows DLL definition file\label{fig:DefFile}}
\end{figure}

\subsubsection{Definition file}
This is a text file with a {\code .def} extension that lists all the
functions that the DLL exports. This file is used during the link step
of the DLL creation process. Figure~\ref{fig:DefFile} shows the
general format of a definition file.


\index{DLL|)}

\begin{appendix}
%appendix
\chapter{80x86 Instructions}
\section{Non-floating Point Instructions}
This section lists and describes the actions and formats of the 
non-floating point instructions of the Intel 80x86 CPU family.

The formats use the following abbreviations:
\begin{center}
\begin{tabular}{|l|l|}
\hline
R   & general register \\
R8  & 8-bit register \\
R16 & 16-bit register \\
R32 & 32-bit register \\
SR  & segment register \\
M   & memory \\
M8  & byte \\
M16 & word \\
M32 & double word \\
I   & immediate value \\
\hline
\end{tabular}
\end{center}
These can be combined for the multiple operand instructions. For example,
the format \emph{R, R} means that the instruction takes two register operands.
Many of the two operand instructions allow the same operands. The abbreviation
\emph{O2} is used to represent these operands: \emph{R,R R,M R,I M,R M,I}. If
a 8-bit register or memory can be used for an operand, the abbreviation,
\emph{R/M8} is used.

The table also shows how various bits of the FLAGS register are affected by
each instruction. If the column is blank, the corresponding bit is not
affected at all. If the bit is always changed to a particular value, a 1 or
0 is shown in the column. If the bit is changed to a value that depends on
the operands of the instruction, a \emph{C} is placed in the column. Finally,
if the bit is modified in some undefined way a \emph{?} appears in the
column. Because the only instructions that change the direction flag are 
{\code CLD} and {\code STD}, it is not listed under the FLAGS columns.

\begin{longtable}{||l|p{1.5in}|p{0.75in}|c|c|c|c|c|c||}
\hline \hline
\multicolumn{1}{||c}{} & 
   \multicolumn{1}{c}{} &
   \multicolumn{1}{c}{} &
  \multicolumn{6}{c||}{\textbf{Flags}} \\ \cline{4-9}
\multicolumn{1}{||c}{\textbf{Name}} & 
   \multicolumn{1}{c}{\textbf{Description}} &
   \multicolumn{1}{c}{\textbf{Formats}} &
   \multicolumn{1}{c}{\textbf{O}} &
   \multicolumn{1}{c}{\textbf{S}} &
   \multicolumn{1}{c}{\textbf{Z}} &
   \multicolumn{1}{c}{\textbf{A}} &
   \multicolumn{1}{c}{\textbf{P}} &
   \multicolumn{1}{c||}{\textbf{C}} \\ \hline \endhead
\hline \hline \endfoot
%                                              O   S   Z   A   P   C
{\code ADC} & Add with Carry & O2            & C & C & C & C & C & C \\
{\code ADD} & Add Integers   & O2            & C & C & C & C & C & C \\
{\code AND} & Bitwise AND    & O2            & 0 & C & C & ? & C & 0 \\
{\code BSWAP} & Byte Swap    & R32           &   &   &   &   &   &  \\
{\code CALL} & Call Routine  & R M I         &   &   &   &   &   &   \\
{\code CBW} & Convert Byte to Word &         &   &   &   &   &   & \\
{\code CDQ} & Convert Dword to Qword &       &   &   &   &   &   & \\
{\code CLC} & Clear Carry &                  &   &   &   &   &   & 0 \\
{\code CLD} & Clear Direction Flag &         &   &   &   &   &   & \\
{\code CMC} & Complement Carry &             &   &   &   &   &   & C \\
{\code CMP} & Compare Integers & O2          & C & C & C & C & C & C \\
{\code CMPSB} & Compare Bytes &              & C & C & C & C & C & C \\
{\code CMPSW} & Compare Words &              & C & C & C & C & C & C \\
{\code CMPSD} & Compare Dwords &             & C & C & C & C & C & C \\
{\code CWD} & Convert Word to Dword into DX:AX & &   &   &   &   &   & \\
{\code CWDE} & Convert Word to Dword into EAX & &   &   &   &   &   & \\
{\code DEC} & Decrement Integer & R M        & C & C & C & C & C & \\
{\code DIV} & Unsigned Divide & R M          & ? & ? & ? & ? & ? & ? \\
{\code ENTER} & Make stack frame & I,0       &   &   &   &   &   & \\
{\code IDIV} & Signed Divide & R M           & ? & ? & ? & ? & ? & ? \\
{\code IMUL} & Signed Multiply & R M R16,R/M16 R32,R/M32 R16,I R32,I 
                                       {\small R16,R/M16,I R32,R/M32,I}
                                             & C & ? & ? & ? & ? & C \\
{\code INC} & Increment Integer & R M        & C & C & C & C & C & \\
{\code INT} & Generate Interrupt & I         &   &   &   &   &   & \\
{\code JA } & Jump Above & I                 &   &   &   &   &   & \\
{\code JAE } & Jump Above or Equal & I       &   &   &   &   &   & \\
{\code JB } & Jump Below & I                 &   &   &   &   &   & \\
{\code JBE } & Jump Below or Equal  & I      &   &   &   &   &   & \\
{\code JC } & Jump Carry & I                 &   &   &   &   &   & \\
{\code JCXZ } & Jump if CX = 0 & I           &   &   &   &   &   & \\
{\code JE } & Jump Equal & I                 &   &   &   &   &   & \\
{\code JG } & Jump Greater & I               &   &   &   &   &   & \\
{\code JGE } & Jump Greater or Equal & I     &   &   &   &   &   & \\
{\code JL } & Jump Less & I                  &   &   &   &   &   & \\
{\code JLE } & Jump Less or Equal & I        &   &   &   &   &   & \\
{\code JMP } & Unconditional Jump & R M I    &   &   &   &   &   & \\
{\code JNA } & Jump Not Above & I            &   &   &   &   &   & \\
{\code JNAE } & Jump Not Above or Equal& I   &   &   &   &   &   & \\
{\code JNB } & Jump Not Below & I            &   &   &   &   &   & \\
{\code JNBE } & Jump Not Below or Equal & I  &   &   &   &   &   & \\
{\code JNC } & Jump No Carry & I             &   &   &   &   &   & \\
{\code JNE } & Jump Not Equal & I            &   &   &   &   &   & \\
{\code JNG } & Jump Not Greater & I          &   &   &   &   &   & \\
{\code JNGE } & Jump Not Greater or Equal & I&   &   &   &   &   & \\
{\code JNL } & Jump Not Less & I             &   &   &   &   &   & \\
{\code JNLE } & Jump Not Less or Equal & I   &   &   &   &   &   & \\
{\code JNO } & Jump No Overflow & I          &   &   &   &   &   & \\
{\code JNS } & Jump No Sign & I              &   &   &   &   &   & \\
{\code JNZ } & Jump Not Zero & I             &   &   &   &   &   & \\
{\code JO } & Jump Overflow & I              &   &   &   &   &   & \\
{\code JPE } & Jump Parity Even & I          &   &   &   &   &   & \\
{\code JPO } & Jump Parity Odd & I           &   &   &   &   &   & \\
{\code JS } & Jump Sign & I                  &   &   &   &   &   & \\
{\code JZ } & Jump Zero & I                  &   &   &   &   &   & \\
{\code LAHF} & Load FLAGS into AH &          &   &   &   &   &   & \\
{\code LEA} & Load Effective Address & R32,M &   &   &   &   &   & \\
{\code LEAVE} & Leave Stack Frame &          &   &   &   &   &   & \\
{\code LODSB} & Load Byte &                  &   &   &   &   &   & \\
{\code LODSW} & Load Word &                  &   &   &   &   &   & \\
{\code LODSD} & Load Dword &                 &   &   &   &   &   & \\
{\code LOOP}  & Loop       & I               &   &   &   &   &   & \\
{\code LOOPE/LOOPZ} & Loop If Equal & I     &   &   &   &   &   & \\
{\code LOOPNE/LOOPNZ} & Loop If Not Equal & I  &   &   &   &   &   & \\
{\code MOV} & Move Data & O2 \mbox{SR,R/M16} R/M16,SR
                                             &   &   &   &   &   & \\
{\code MOVSB} & Move Byte &                  &   &   &   &   &   & \\
{\code MOVSW} & Move Word &                  &   &   &   &   &   & \\
{\code MOVSD} & Move Dword &                 &   &   &   &   &   & \\
{\code MOVSX} & Move Signed & R16,R/M8 R32,R/M8 R32,R/M16
                                             &   &   &   &   &   & \\
{\code MOVZX} & Move Unsigned & R16,R/M8 R32,R/M8 R32,R/M16
                                             &   &   &   &   &   & \\
{\code MUL} & Unsigned Multiply & R M        & C & ? & ? & ? & ? & C \\
{\code NEG} & Negate & R M                   & C & C & C & C & C & C \\
{\code NOP} & No Operation &                 &   &   &   &   &   & \\
{\code NOT} & 1's Complement & R M           &   &   &   &   &   & \\
{\code OR} & Bitwise OR    & O2              & 0 & C & C & ? & C & 0 \\
{\code POP} & Pop From Stack & R/M16 R/M32   &   &   &   &   &   & \\
{\code POPA} & Pop All &                     &   &   &   &   &   & \\
{\code POPF} & Pop FLAGS &                   & C & C & C & C & C & C \\
{\code PUSH} & Push to Stack & R/M16 R/M32 I &   &   &   &   &   & \\
{\code PUSHA} & Push All &                   &   &   &   &   &   & \\
{\code PUSHF} & Push FLAGS &                 &   &   &   &   &   & \\
{\code RCL} & Rotate Left with Carry & R/M,I R/M,CL
                                             & C &   &   &   &   & C \\
{\code RCR} & Rotate Right with Carry & R/M,I R/M,CL
                                             & C &   &   &   &   & C \\
{\code REP} & Repeat &                       &   &   &   &   &   & \\
{\code REPE/REPZ} & Repeat If Equal&        &   &   &   &   &   & \\
{\code REPNE/REPNZ} & Repeat If Not Equal&  &   &   &   &   &   & \\
{\code RET} & Return &                       &   &   &   &   &   & \\
{\code ROL} & Rotate Left & R/M,I R/M,CL     & C &   &   &   &   & C \\
{\code ROR} & Rotate Right & R/M,I R/M,CL    & C &   &   &   &   & C \\
{\code SAHF} & Copies AH into FLAGS &        &   & C & C & C & C & C \\
{\code SAL} & Shifts to Left & R/M,I R/M, CL &   &   &   &   &   & C \\
{\code SBB}  & Subtract with Borrow & O2     & C & C & C & C & C & C \\
{\code SCASB} & Scan for Byte &              & C & C & C & C & C & C \\
{\code SCASW} & Scan for Word &              & C & C & C & C & C & C \\
{\code SCASD} & Scan for Dword &             & C & C & C & C & C & C \\
{\code SETA } & Set Above & R/M8                 &   &   &   &   &   & \\
{\code SETAE } & Set Above or Equal & R/M8       &   &   &   &   &   & \\
{\code SETB } & Set Below & R/M8                 &   &   &   &   &   & \\
{\code SETBE } & Set Below or Equal  & R/M8      &   &   &   &   &   & \\
{\code SETC } & Set Carry & R/M8                 &   &   &   &   &   & \\
{\code SETE } & Set Equal & R/M8                 &   &   &   &   &   & \\
{\code SETG } & Set Greater & R/M8               &   &   &   &   &   & \\
{\code SETGE } & Set Greater or Equal & R/M8     &   &   &   &   &   & \\
{\code SETL } & Set Less & R/M8                  &   &   &   &   &   & \\
{\code SETLE } & Set Less or Equal & R/M8        &   &   &   &   &   & \\
{\code SETNA } & Set Not Above & R/M8            &   &   &   &   &   & \\
{\code SETNAE } & Set Not Above or Equal& R/M8   &   &   &   &   &   & \\
{\code SETNB } & Set Not Below & R/M8            &   &   &   &   &   & \\
{\code SETNBE } & Set Not Below or Equal & R/M8  &   &   &   &   &   & \\
{\code SETNC } & Set No Carry & R/M8             &   &   &   &   &   & \\
{\code SETNE } & Set Not Equal & R/M8            &   &   &   &   &   & \\
{\code SETNG } & Set Not Greater & R/M8          &   &   &   &   &   & \\
{\code SETNGE } & Set Not Greater or Equal & R/M8&   &   &   &   &   & \\
{\code SETNL } & Set Not Less & R/M8             &   &   &   &   &   & \\
{\code SETNLE } & Set Not LEss or Equal & R/M8   &   &   &   &   &   & \\
{\code SETNO } & Set No Overflow & R/M8          &   &   &   &   &   & \\
{\code SETNS } & Set No Sign & R/M8              &   &   &   &   &   & \\
{\code SETNZ } & Set Not Zero & R/M8             &   &   &   &   &   & \\
{\code SETO } & Set Overflow & R/M8              &   &   &   &   &   & \\
{\code SETPE } & Set Parity Even & R/M8          &   &   &   &   &   & \\
{\code SETPO } & Set Parity Odd & R/M8           &   &   &   &   &   & \\
{\code SETS } & Set Sign & R/M8                  &   &   &   &   &   & \\
{\code SETZ } & Set Zero & R/M8                  &   &   &   &   &   & \\

{\code SAR} & Arithmetic Shift to Right & R/M,I R/M, CL 
                                             &   &   &   &   &   & C \\
{\code SHR} & Logical Shift to Right & R/M,I R/M, CL 
                                             &   &   &   &   &   & C \\
{\code SHL} & Logical Shift to Left & R/M,I R/M, CL 
                                             &   &   &   &   &   & C \\
{\code STC} & Set Carry &                    &   &   &   &   &   & 1 \\
{\code STD} & Set Direction Flag &           &   &   &   &   &   & \\
{\code STOSB} & Store Btye &                 &   &   &   &   &   & \\
{\code STOSW} & Store Word &                 &   &   &   &   &   & \\
{\code STOSD} & Store Dword &                &   &   &   &   &   & \\
{\code SUB} & Subtract & O2                  & C & C & C & C & C & C\\
{\code TEST} & Logical Compare & R/M,R R/M,I & 0 & C & C & ? & C & 0\\
{\code XCHG} & Exchange & R/M,R R,R/M        &   &   &   &   &   & \\
{\code XOR} & Bitwise XOR    & O2            & 0 & C & C & ? & C & 0 \\

\end{longtable}

\newpage
\section{Floating Point Instructions}

\renewcommand{\thefootnote}{\fnsymbol{footnote}} In this section, many
of the 80x86 math coprocessor instructions are described. The
description section briefly describes the operation of the
instruction. To save space, information about whether the instruction
pops the stack is not given in the description. 

The format column shows what type of operands can be used with each
instruction. The following abbreviations are used:
\begin{center}
\begin{tabular}{|l|l|}
\hline
ST\emph{n} & A coprocessor register \\
F          & Single precision number in memory \\
D          & Double precision number in memory \\
E          & Extended precision number in memory \\
I16        & Integer word in memory \\
I32        & Integer double word in memory \\
I64        & Integer quad word in memory \\
\hline
\end{tabular}
\end{center}

Instructions requiring a Pentium Pro or better are marked with an 
asterisk(\footnotemark[1]).

\begin{longtable}{||l|l|l||}
\hline \hline
\multicolumn{1}{||c}{\textbf{Instruction}} & 
  \multicolumn{1}{c}{\textbf{Description}} &
\multicolumn{1}{c||}{\textbf{Format}} \\
\hline
\endhead
\hline \hline \endfoot
{\code FABS} & $\mathtt{ST0} = |\mathtt{ST0}|$ & \\
{\code FADD \emph{src}} & {\code ST0 += \emph{src}} & ST\emph{n} F D \\
{\code FADD \emph{dest}, ST0} & {\code \emph{dest} += STO} & ST\emph{n} \\
{\code FADDP \emph{dest}[,ST0]} & {\code \emph{dest} += ST0} & ST\emph{n} \\
{\code FCHS} & $\mathtt{ST0} = - \mathtt{ST0}$ & \\
{\code FCOM \emph{src}} & Compare {\code ST0} and {\code \emph{src}} &
ST\emph{n} F D \\
{\code FCOMP \emph{src}} & Compare {\code ST0} and {\code \emph{src}} &
ST\emph{n} F D \\
{\code FCOMPP \emph{src}} & Compares {\code ST0} and {\code ST1} & \\
{\code FCOMI\footnotemark[1] \emph{src}} & Compares into FLAGS 
& ST\emph{n} \\
{\code FCOMIP\footnotemark[1] \emph{src}} & Compares into FLAGS 
& ST\emph{n} \\
{\code FDIV \emph{src}} & {\code ST0 /= \emph{src}} & ST\emph{n} F D \\
{\code FDIV \emph{dest}, ST0} & {\code \emph{dest} /= STO} & ST\emph{n} \\
{\code FDIVP \emph{dest}[,ST0]} & {\code \emph{dest} /= ST0} & ST\emph{n} \\
{\code FDIVR \emph{src}} & {\code ST0 = \emph{src}/ST0} & ST\emph{n} F D \\
{\code FDIVR \emph{dest}, ST0} & {\code \emph{dest} = ST0/\emph{dest}} 
& ST\emph{n} \\
{\code FDIVRP \emph{dest}[,ST0]} & {\code \emph{dest} = ST0/\emph{dest}} 
& ST\emph{n} \\
{\code FFREE \emph{dest}} & Marks as empty & ST\emph{n} \\
{\code FIADD \emph{src}} & {\code ST0 += \emph{src}} & I16 I32 \\
{\code FICOM \emph{src}} & Compare {\code ST0} and {\code \emph{src}} &
I16 I32 \\
{\code FICOMP \emph{src}} & Compare {\code ST0} and {\code \emph{src}} &
I16 I32 \\
{\code FIDIV \emph{src}} & {\code STO /= \emph{src}} & I16 I32 \\
{\code FIDIVR \emph{src}} & {\code STO = \emph{src}/ST0} & I16 I32 \\
{\code FILD \emph{src}} & Push \emph{src} on Stack & I16 I32 I64 \\
{\code FIMUL \emph{src}} & {\code ST0 *= \emph{src}} & I16 I32 \\
{\code FINIT} & Initialize Coprocessor & \\
{\code FIST \emph{dest}} & Store {\code ST0} & I16 I32 \\
{\code FISTP \emph{dest}} & Store {\code ST0} & I16 I32 I64\\
{\code FISUB \emph{src}} & {\code ST0 -= \emph{src}} & I16 I32 \\
{\code FISUBR \emph{src}} & {\code ST0 = \emph{src} - ST0} & I16 I32 \\
{\code FLD \emph{src}} & Push \emph{src} on Stack & ST\emph{n} F D E \\
{\code FLD1} & Push 1.0 on Stack & \\
{\code FLDCW \emph{src}} & Load Control Word Register & I16 \\
{\code FLDPI} & Push $\pi$ on Stack & \\
{\code FLDZ} & Push 0.0 on Stack & \\
{\code FMUL \emph{src}} & {\code ST0 *= \emph{src}} & ST\emph{n} F D \\
{\code FMUL \emph{dest}, ST0} & {\code \emph{dest} *= STO} & ST\emph{n} \\
{\code FMULP \emph{dest}[,ST0]} & {\code \emph{dest} *= ST0} & ST\emph{n} \\
{\code FRNDINT} & Round {\code ST0} & \\
{\code FSCALE} & $\mathtt{ST0} = \mathtt{ST0} \times 2^{\lfloor \mathtt{ST1} \rfloor}$ & \\
{\code FSQRT} & $\mathtt{ST0} = \sqrt{\mathtt{STO}}$ & \\
{\code FST \emph{dest}} & Store {\code ST0} & ST\emph{n} F D \\
{\code FSTP \emph{dest}} & Store {\code ST0} & ST\emph{n} F D E \\
{\code FSTCW \emph{dest}} & Store Control Word Register & I16 \\
{\code FSTSW \emph{dest}} & Store Status Word Register & I16 AX \\
{\code FSUB \emph{src}} & {\code ST0 -= \emph{src}} & ST\emph{n} F D \\
{\code FSUB \emph{dest}, ST0} & {\code \emph{dest} -= STO} & ST\emph{n} \\
{\code FSUBP \emph{dest}[,ST0]} & {\code \emph{dest} -= ST0} & ST\emph{n} \\
{\code FSUBR \emph{src}} & {\code ST0 = \emph{src}-ST0} & ST\emph{n} F D \\
{\code FSUBR \emph{dest}, ST0} & {\code \emph{dest} = ST0-\emph{dest}} 
& ST\emph{n} \\
{\code FSUBP \emph{dest}[,ST0]} & {\code \emph{dest} = ST0-\emph{dest}} 
& ST\emph{n} \\
{\code FTST} & Compare {\code ST0} with 0.0 & \\
{\code FXCH \emph{dest}} & Exchange {\code ST0} and {\code \emph{dest}} 
& ST\emph{n} \\
\end{longtable}

\renewcommand{\thefootnote}{\arabic{footnote}}



\end{appendix}
\clearpage
\ifmypdf
\phantomsection % fixes link anchor
\fi
\addcontentsline{toc}{chapter}{Index}
\printindex
\end{document}

