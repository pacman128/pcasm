%-*- latex -*-
\chapter{서브프로그램}

이 장에서는 서브프로그램(subprogram) 을 이용하여 모듈화 된 프로그램
(modular program) 및 C 와 같은 고급 언어들과 함께 작업하는 방법에 대해서
다루어 보고자 한다. 고급언어에서의 서브프로그램으로는 함수와 프로시져들이 있다. 

서브프로그램을 호출하는 코드와 서브프로그램 그 자체는 반드시 이들 
사이에 데이터를 주고 받는 방법이 같아야 한다. 이렇게, 이들 사이에 
어떠한 방식으로 데이터를 주고받는 방법을 정해 놓은 것을 \emph{호출 규약(calling convention)}
이라고 부른다. 이 장은 C 프로그램들과 소통(interface)할 수 있는 어셈블리
서브프로그램 제작을 위해서, C 호출 규약에 대해 많은 부분을 할애할 
것이다. 이 호출 규약은 (그 외 많은 호출 규약 들도) 데이터의 주소(\emph{i.e.} 포인터) 를 전달해
서브프로그램이 메모리의 데이터에 접근 할 수 있게 하는 방식으로 이루어진다. 

\section{간접 주소 지정\index{간접 주소 지정|(}}

간접적인 주소 지정은 레지스터를 포인터 변수들 처럼 사용할
수 있게 한다. 레지스터가 포인터로 사용됨을 알려주기 위해서 대괄호([])로 묶어
주어야 한다. 예를 들어 

\begin{AsmCodeListing}[frame=none]
      mov    ax, [Data]     ; 워드의 직접 메모리 주소 지정 
      mov    ebx, Data      ; ebx = & Data
      mov    ax, [ebx]      ; ax = *ebx
\end{AsmCodeListing}

AX 가 워드를 보관하고 있기 때문에 ~3행은 EBX 에 저장된 주소에서 시작되는 워드
를 읽어 들이게 된다. 만약 AX 대신 AL 을 이용한다면 오직 한 개의 바이트만 읽어
들이게 될 것이다. C 에서의 변수들과는 달리 레지스터들은 형(type) 이 없다는 사실
을 명심해야 한다. EBX 에 무엇을 가리키냐는 어떠한 명령이 사용되냐에 따라 완전히
달라진다. 심지어 명령에 의해서도 EBX 가 포인터라는
사실도 정해진다. 만약 EBX 가 잘못 사용 되어도 어셈블러는 아무런 오류를 내놓지 
않는다. 그러나, 프로그램은 올바르게 작동하지 않는다. 이 때문에 어셈블리 프로그래밍은
고급 언어 프로그래밍 보다 훨씬 오류가 많게 된다. 

모든 32 비트 범용 레지스터 (EAX, EBX, EDX, EDX) 와 인덱스 (ESI, EDI) 레지스터들은
모두 간접 주소 지정으로 사용될 수 있다. 통상적으로 16 비트와 8 비트는 사용될 수 없다, 

\index{간접 주소 지정|)}

\section{간단한 서브프로그램 예제\index{서브프로그램|(}}

서브프로그램은 프로그램의 다른 영역에서 쓰이는 독립적인 코드라고 볼 수 있다. 즉,
서브프로그램은 C 언어에서의 함수라고 생각하면 된다. 분기 명령을 통해 서브프로그램을
호출 할 수 있지만 리턴하는 부분에서 문제가 생긴다. 만약 서브프로그램이 프로그램의 각기
다른 영역에서 호출되었다면 서브프로그램은 호출 된 각각의 위치로 돌아가기 위해
엄청난 수의 라벨을 사용하게 되므로 매우 복잡해 질 것이다. 반면에 아래의 코드는 이를 간접 형태의 {\code JMP} 
명령을 사용하여 구현한 것을 볼 수 있다. 
이 형태의 명령은 레지스터에 저장된 값을 이용하여 어디로 분기할 것인지 알 게 된다.
(따라서, 레지스터들은 C 에서의 \emph{함수 포인터(function pointer)} 과 같이 쓰인다)
아래는 ~1 장의 첫 번째 프로그램을 서브프로그램을 이용한 형태로 다시 작성한 것이다. 

\begin{AsmCodeListing}[label=sub1.asm]
; 파일: sub1.asm
; 서브프로그램 예제 프로그램
%include "asm_io.inc"

segment .data
prompt1 db    "Enter a number: ", 0       ; 널 종료 문자를 잊지 마시오
prompt2 db    "Enter another number: ", 0
outmsg1 db    "You entered ", 0
outmsg2 db    " and ", 0
outmsg3 db    ", the sum of these is ", 0

segment .bss
input1  resd 1
input2  resd 1

segment .text
        global  _asm_main
_asm_main:
        enter   0,0               ; 셋업 루틴
        pusha

        mov     eax, prompt1      ; prompt 를 출력
        call    print_string

        mov     ebx, input1       ; input1 의 주소를 ebx 에 저장 
        mov     ecx, ret1         ; ecx 에 리턴 주소를 저장 
        jmp     short get_int     ; 정수를 읽는다 
ret1:
        mov     eax, prompt2      ; prompt 출력 
        call    print_string

        mov     ebx, input2
        mov     ecx, $ + 7        ; ecx = 현재주소 + 7
        jmp     short get_int
  
        mov     eax, [input1]     ; eax = input1 의 dword
        add     eax, [input2]     ; eax += input2 의 dword
        mov     ebx, eax          ; ebx = eax

        mov     eax, outmsg1
        call    print_string      ; 첫 번째 메세지 출력 
        mov     eax, [input1]     
        call    print_int         ; input1 을 출력 
        mov     eax, outmsg2
        call    print_string      ; 두 번째 메세지 출력 
        mov     eax, [input2]
        call    print_int         ; input2 를 출력 
        mov     eax, outmsg3
        call    print_string      ; 세 번째 메세지 출력 
        mov     eax, ebx
        call    print_int         ; 합 (ebx) 출력 
        call    print_nl          ; 개행 문자 출력 
        
        popa
        mov     eax, 0            ; C 로 리턴
        leave                     
        ret
; 서브프로그램 get_int
; 인자:
;   ebx - 정수를 저장한 dword 의 주소 
;   ecx - 리턴할 명령의 주소 
; 참고 사항:
;    eax 에 저장된 값은 사라진다. 
get_int:
        call    read_int
        mov     [ebx], eax         ; input 을 메모리에 저장 
        jmp     ecx                ; 호출한 곳으로 리턴 
\end{AsmCodeListing}

{\code get\_int} 서브프로그램은 단순한 레지스터 기반 호출 규약을 사용한다.
이는 EBX 레지스터가 입력값을 저장할 DWORD 의 주소를 갖고 있고, ECX 레지스터가
되돌아갈 코드 주소값을 가지고 있다고 생각한다. ~25 에서 28 행까지 {\code ret1} 라벨은
리턴 주소를 계산하기 위해 사용되었다. ~32 에서 34 행까지 {\code \$} 연산자는 리턴
주소를 계산하기 위해 사용되었다. {\code \$} 는 현재 행의 주소를 리턴한다. {\code \$ + 7}
은 ~36 행의 {\code MOV} 명령의 주소를 계산하기 위해 사용되었다. 

위 두 가지 방식의 리턴 주소 계산은 매우 복잡하다. 첫 번째 방법은 매 서브프로그램 호출
마다 라벨을 필요로 한다. 두 번째 방법은 라벨을 필요로 하진 않지만 오류가 나지 않기
위해 심사숙고 할 필요가 있다. 예를 들어 near 분기 대신에 short 분기가 사용되었다면 
{\code \$} 에 더해진 값은 7 이 아니게 된다. 다행이도, 서브프로그램을 호출 하기 위한
더 쉬운 방법이 있다. 이 방법은 \emph{스택(stack)} 을 이용한다. 

\section{스택\index{스택|(}}

대다수의 CPU 는 스택의 사용을 지원한다. 스택은 후입 선출(Last-In First-Out,
\emph{LIFO} 라고 부르며, 나중에 들어간 데이터가 먼저 나오게 된다) 리스트 이다.
스택은 이와 같은 규칙으로 구성된 메모리의 일부분 이다. {\code PUSH} 명령을 통해
스택에 데이터를 추가하고 {\code POP} 명령을 통해 데이터를 빼낼 수 있다. 
언제나 빼내진 데이터는 마지막으로 추가된 데이터 이다. (이 때문에 스택을 
후입 선출 리스트라 부른다)

SS 세그먼트 레지스터는 스택을 보관한 세그먼트를 정의한다. (보통 이는 
데이터가 저장된 세그먼트와 동일하다) ESP 레지스터는 스택으로 부터 빼내질
데이터의 주소를 보관한다. 이 데이터는 흔히 스택의 \emph{최상위(top)} 에 있다고
말한다. 데이터는 오직 더블워드의 형태로만 저장된다. 따라서, 스택에 단일
바이트를 집어 넣을 수 없다. 

{\code PUSH} 명령은 ESP 레지스터의 값을 4 감소 시킨 후, 더블워드\footnote{사실, 워드들도 스택에
들어 갈 수 있지만 32 비트 보호 모드에서는 오직 더블워드들만 가지고 다루는
것이 낫다.} 를 {\code [ESP]} 에 위치한 더블워드에 집어 넣는다. {\code POP} 명령은
{\code [ESP]} 에 위치한 더블워드를 읽어 들이고, ESP 에 4 를 더한다. 아래의 코드는 
ESP 가 {\code 1000H} 이였다고 할 때 명령에 따라 값이 어떻게 변화하는지 보여준다. 

\begin{AsmCodeListing}[frame=none]
      push   dword 1    ; 1 이 0FFCh 에 저장됨, ESP = 0FFCh
      push   dword 2    ; 2 가 0FF8h 에 저장됨, ESP = 0FF8h
      push   dword 3    ; 3 이 0FF4h 에 저장됨, ESP = 0FF4h
      pop    eax        ; EAX = 3, ESP = 0FF8h
      pop    ebx        ; EBX = 2, ESP = 0FFCh
      pop    ecx        ; ECX = 1, ESP = 1000h
\end{AsmCodeListing}

스택은 임시적으로 데이터를 저장하는데 요긴하게 사용할 수 있다. 이를 통해서
인자(parameter)들과 지역 변수(local variable) 을 전달하고 
서브프로그램 호출을 할 수 있다. 

80x86 은 또한 {\code PUSHA} 명령을 이용하여 EAX, EBX, ECX, EDX, ESI, EDI, EBP 
레지스터의 값들을 모두 스택에 집어 넣을(푸시) 수 있게 한다. (이 순서로 집어 넣는 것은 아니다) 
{\code POPA} 명령을 통해 이들을 다시 빼낼(팝) 수 있게 한다.

\index{스택|)}

\section{CALL 과 RET 명령\index{서브프로그램!호출|(}}
\index{CALL|(}
\index{RET|(}
80x86 은 서브프로그램을 빠르고 간편하게 호출하기 위해서 스택을 이용한 2 가지의
명령을 지원한다. CALL 명령은 서브프로그램으로의 무조건 분기를 한 후, 그 다음에 
실행될 명령의 주소를 스택에 \emph{푸시(push)}한다. RET 명령은 그 주소를 \emph{팝(pop)} 한 후
그 주소로 점프를 한다. 이 명령을 이용할 때, 스택을 정확하게 관리하여 RET 명령에 의해
정확한 주소 값이 팝 될 수 있도록 해야 한다. 

따라서 이전의 프로그램은 새로운 명령을 이용하여 아래와 같이 ~25 에서 34 행을 바꾸어 
쓸 수 있다. 

\begin{AsmCodeListing}[numbers=none]
      mov    ebx, input1
      call   get_int

      mov    ebx, input2
      call   get_int
\end{AsmCodeListing}
또한 서브프로그램 {\code get\_int} 를 아래와 같이 바꾼다:
\begin{AsmCodeListing}[numbers=none]
get_int:
      call   read_int
      mov    [ebx], eax
      ret
\end{AsmCodeListing}

CALL 과 RET 에는 몇 가지 장점들이 있다. 

\begin{itemize}
\item 이는 단순하다! 
\item 서브프로그램 호출이 쉽게 이루어지게 한다. {\code get\_int} 가 {\code read\_int} 를
호출 하는 부분을 보아라. 이 호출은 스택에 또다른 주소를 집어 넣는다. {\code read\_int}
코드의 마지막 부분에는 RET 가 리턴 주소를 빼내고, {\code get\_int} 의 코드로 
다시 분기 된다. 그리고 {\code get\_int} 의 RET 이 이루어진다면 리턴 주소를 
빼내어 {\code asm\_main} 으로 분기된다. 이는 스택이 LIFO 구조 이기에 올바르게 일어난다. 

\end{itemize}

스택에 푸시 된 모든 데이터는 나중에 \emph{반드시} 다시 팝 해야 한다.
예를 들어, 아래와 같은 소스 코드를 생각해 보자:

\begin{AsmCodeListing}[frame=none]
get_int:
      call   read_int
      mov    [ebx], eax
      push   eax
      ret                  ; 리턴 주소 대신 EAX 값을 팝!
\end{AsmCodeListing}
이 코드는 올바르게 리턴되지 않는다. 

\index{RET|)}
\index{CALL|)}

\section{호출 규약\index{호출 규약|(}}

서브프로그램이 호출 되었을 때, 호출하는 코드와 서브프로그램 (이를 
\emph{피호출자(callee)} 은 서로의 데이터 전송 방식이 같아야 한다. 
고급 언어에서는 \emph{호출 규약(calling conventions)} 이라는 기본적인 
데이터 전송 방법이 있다. 고급 언어 코드와 어셈블리가 함께 작업하기 위해서는
어셈블리 코드들도 반드시 고급 언어에서의 규약과 일치해야 한다. 이 호출 규약은
컴파일러 마다 다를 수 있으며, 코드가 어떠한 방식으로 컴파일 되느냐에 따라서
(\emph{e.g} 최적화의 유무)도 다를 수 있다. 하나의 공통된 점이라 하면
코드는 {\code CALL} 을 통해 호출하고 {\code RET} 을 통해 리턴된다는 점이다. 

모든 PC C 컴파일러는 이 장의 나머지 부분에서 다루게 될 호출 규약을 지원한다. 
이 규약을 통해 \emph{재진입(reentrant)} 서브프로그램을 만들 수 있다. 재진입
서브프로그램은 프로그램의 어떤 부분에서도 자유롭게 호출 될 수 있다. 
(심지어 서브프로그램 그 자체 내부에서도) 

\subsection{스택에 인자들을 전달하기}\index{스택|(}\index{스택!인자|(}

서브프로그램으로 스택을 통해 인자들이 전달 될 수 있다. 이 인자들은 {\code CALL}
명령을 하기 전에 스택으로 푸시되어진다. C 에서처럼 서브프로그램에서 인자의 
값이 바뀌기 위해서는 인자의 \emph{값}이 아닌, 인자의 \emph{주소}가 반드시 
전달 되어야만 한다. 만약 인자의 크기가 더블워드보다 작다면, 반드시 푸시 되기
전에 더블워드로 변환되어야만 한다. 

스택에 저장된 인자들은 서브프로그램에 의해 빼내어(Pop) 지지 않는다. 그 대신에, 이
들은 스택 자체에서 접근이 가능하다. 왜냐하면, 

\begin{itemize}
\item 이들은 {\code CALL} 명령 전에 스택에 푸시 되어야만 하기 때문에 
리턴 주소는 가장 먼저 팝 되어야만 한다 

\item 인자들은 종종 서브프로그램의 여러 부분에서 사용된다. 보통, 서브프로그램
실행 내내 레지스터에 인자들을 저장할 수 없으므로, 메모리에 저장되어야 한다.
따라서, 인자들을 스택에 보관함으로써 서브프로그램의 어떠한 부분에서도 인자에
접근 할 수 있게 된다. 

\end{itemize}

\begin{figure}
\centering
\begin{tabular}{l|c|}
\cline{2-2}
&  \\ \cline{2-2}
ESP + 4 & 인자 \\ \cline{2-2}
ESP     & 리턴 주소 \\ \cline{2-2}
 & \\ \cline{2-2}
\end{tabular}
\caption{}
\label{fig:stack1}
\end{figure}
스택에 하나의 인자를 저장한 서브프로그램을 생각하자 \MarginNote{간접
주소지정을 사용시, 80x86 프로세서는 간접 주소지정 표현시 어떠한 레지스터가
사용되느냐에 따라서 각기 다른 세그먼트에 접근한다. ESP (또한 EBP) 는 
스택 세그먼트를 사용하며, EAX, EBX, ECX, EDX 는 데이타 세그먼트를 사용한다.
그러나, 보호 모드 프로그램에서는 별로 중요한 문제는 아니다. 왜냐하면
데이터 세그먼트와 스택 세그먼트는 동일하기 때문이다. }
서브프로그램이 호출 되었을 때, 스택은 ~\ref{fig:stack1} 과 같이 나타난다.
간접 주소 지정을 사용하여 인자에 접근({\code [ESP+4]} \footnote{간접 주소지정을 사용시 레지스터에 상수를 더하는 것이 가능하다. 
뿐만 아니라 더 복잡한 표현들도 가능하다. 이것에 대한 자세한 내용은 다음 장에서
다룬다}) 할 수 있다. 

\begin{figure}
\centering
\begin{tabular}{l|c|}
\cline{2-2}
&  \\ \cline{2-2}
ESP + 8 & 인자 \\ \cline{2-2}
ESP + 4 & 리턴 주소 \\ \cline{2-2}
ESP     & 서브프로그램 데이터 \\ \cline{2-2}
\end{tabular}
\caption{}
\label{fig:stack2}
\end{figure}

\begin{figure}[t]
\begin{AsmCodeListing}[frame=single]
subprogram_label:
      push   ebp           ; 원래의 EBP 값을 스택에 저장
      mov    ebp, esp      ; 새로운 EBP = ESP
; subprogram code
      pop    ebp           ; 이전의 EBP 값을 불러온다. 
      ret
\end{AsmCodeListing}
\caption{통상적인 서브프로그램의 형태 \label{fig:subskel1}}
\end{figure}

만일, 서브프로그램 내에서도 데이터를 저장하기 위해 스택을 사용한다면, 
ESP 에 더해저야할 값이 달라지게 된다. 예를 들어 그림 ~\ref{fig:stack2} 에서
더블워드가 스택에 푸시되었을 때 어떻게 나타나는지 보여준다. 이 때, 
인자는 {\code ESP + 4} 가 아닌 {\code ESP + 8} 에 위치하게 된다. 따라서,
ESP 를 이용하여 인자에 접근시 오류를 낼 확률이 크다. 이 문제를 해결하기 위해
80386 은 EBP 라는 또다른 레지스터를 사용 가능하게 했다. 이 레지스터의 목적은 
스택의 데이터를 가리키는 것 이다. C 호출 규약에선 반드시 서브프로그램이 먼저 EBP 
의 값을 스택에 저장하고 EBP 가 ESP 와 같게 만들게 해야 한다. 이를 통해 EBP 의 값을
변경하지 않고도 스택에 데이터가 푸시되고 팝 됨에 따라 ESP 가 변경될 수 있게 
한다. 서브프로그램의 끝 부분에선 원래의 EBP 값이 반드시 불러와져야 한다
(이 때문에 우리가 서브프로그램 처음에 EBP 값을 저장했다) 그림 ~\ref{fig:subskel1}
은 위 규약들을 만족하는 서브프로그램의 일반적인 모습을 보여준다. 

\begin{figure}[t]
\centering
\begin{tabular}{ll|c|}
\cline{3-3}
&  & \\ \cline{3-3}
ESP + 8 & EBP + 8 & 인자 \\ \cline{3-3}
ESP + 4 & EBP + 4 & 리턴 주소\\ \cline{3-3}
ESP     & EBP     & 저장된 EBP \\ \cline{3-3}
\end{tabular}
\caption{}
\label{fig:stack3}
\end{figure}

그림 ~\ref{fig:subskel1} 의 2 와 3 행에서 보통의 서브프로그램의 
\emph{시작 부분}을 보여주고 있다. 5 와 6 행에서는 \emph{마지막 부분}을 
보여준다. 그림 ~\ref{fig:stack3} 에서 시작 부분을 막 통과한 후의 스택의
모습을 보여준다. 이제, 인자들은 {\code [EBP + 8]} 을 통해
서브프로그램에 의해 스택에 데이터가 푸시되고 팝 되는 것에 무관하게 
어떠한 부분에서도 인자에 접근할 수 있게 되었다. 

서브프로그램이 종료되면, 스택에 푸시되었던 인자들은 반드시 제거되어야만
한다. C 호출 규약은 호출자의 코드가 이를 하게 명시했다. \index{호출 규약!C}
모든 언어의 호출 규약이 동일한 것이 아니다. 예를 들어 파스칼의 호출 규약 \index{호출 규약!파스칼}
은 서브프로그램이 이를 제거하라 명시했다. (이를 쉽게 하기 위한 RET 명령\index{RET}의 다른 형태가 있다)
일부 C 컴파일러들은 이 규약을 지원하기도 한다. {\code pascal} 키워드를 
함수의 원형과 정의 부분에서 사용하여 컴파일러로 하여금 파스칼의 호출 규약을 
이용하게 만들 수 있다. 사실, 마이크로소프트 윈도우즈 API C 함수들이 
이용하는 {\code stdcall} 규약 \index{호출 규약!stdcall} 도 파스칼의 경우와 같다.
이 방법을 이용하면 어떠한 장점이 있을까? 이는 C 호출 규약보다 조금 더 능률적이다. 
그렇다면 왜 C 함수들은 이 호출규약을 이용하지 않는 것일까? 
보통 C 는 함수가 가변 개수의 인자를 가짐을 지원한다 (\emph{e.g.}, {\code printf}
와 {\code scanf} 가 대표적). 이러한 형식의 함수에선 스택에서 인자들을 
지우는 명령들의 횟수가 달라지게 된다. 
이러한 형식의 함수들은, 스택으로 부터 인자들을 제거하는 명령은 함수의 호출마다 
달라지게 된다. C 호출 규약에선 이러한 연산들을 손쉽게 할 수 있는 명령들을 지원한다. 그러나 
파스칼이나 stdcall 호출 규약에선 이러한 연산들은 매우 복잡하다. 
따라서 파스칼 호출 규약은 이러한 형식의 함수를 지원하지 않는다.
마이크로소프트 윈도우즈의 경우 어떠한 API
함수도 가변 인자를 가지지 않기 때문에 파스칼 형태의 호출 규약을 이용할 수 있었다. 


\begin{figure}[t]
\begin{AsmCodeListing}[frame=single]
      push   dword 1        ; 1 을 인자로 전달
      call   fun
      add    esp, 4         ; 스택에서 인자를 제거
\end{AsmCodeListing}
\caption{서브프로그램 호출 예제\label{fig:subcall}}
\end{figure}

그림 ~\ref{fig:subcall} 는 C 호출 규약을 이용하는 서브프로그램을
어떻게 호출하는지 보여준다. ~3행은 스택 포인터의 값을 조정함으로써 인자를 스택에서
지워버린다. {\code POP} 명령을 통해 이를 할 수 도 있으나, 필요 없는 값이 레지스터에
저장되어야 한다. 사실, 이 특별한 경우에 대해선 많은 컴파일러들은 
{\code POP ECX} 명령을 이용하여 인자를 제거한다. 왜냐하면 {\code ADD} 명령이 더 많은 바이트를
사용하기 때문이다. 그러나 {\code POP} 또한 ECX 의 값을 바꾼다. 다음의 또 다른 예제는 
위에서 설명한 C 호출 규약을 사용한 프로그램들이다. ~54 행 (그리고 다른 행들에서) 에서 
하나의 소스파일에 여러개의 데이타와 텍스트 세그먼트들이 정의되었음을
볼 수 있다. 이들은 링크 과정에서 하나의 데이터와 텍스트 세그먼트들로 합쳐질 것이다. 
데이터와 코드를 여러개의 세그먼트로 쪼갬을 통해 서브프로그램이 사용하는 데이터가
서브프로그램 근처에 정의되게 할 수 있다. 

\index{스택!인자|)}

\begin{AsmCodeListing}[label=sub3.asm]
%include "asm_io.inc"

segment .data
sum     dd   0

segment .bss
input   resd 1

;
; 의사 코드 알고리즘
; i = 1;
; sum = 0;
; while( get_int(i, &input), input != 0 ) {
;   sum += input;
;   i++;
; }
; print_sum(num);
segment .text
        global  _asm_main
_asm_main:
        enter   0,0               ; 셋업 루틴
        pusha

        mov     edx, 1            ; 의사 코드에서의 i 가 edx 이다. 
while_loop:
        push    edx               ; i 를 스택에 저장 
        push    dword input       ; 입력값의 주소를 스택에 저장 % 정확한가?
        call    get_int
        add     esp, 8            ; i 와 &input 을 스택에서 제거 

        mov     eax, [input]
        cmp     eax, 0
        je      end_while

        add     [sum], eax        ; sum += input

        inc     edx
        jmp     short while_loop

end_while:
        push    dword [sum]       ; 합을 스택에 푸시 
        call    print_sum
        pop     ecx               ; 스택에서 [sum] 을 제거 

        popa
        leave                     
        ret

; 서브프로그램 get_int
; 인자들 (스택에 푸시된 순서로 나타남)
;   입력값의 수 ([ebp + 12] 에 위치)
;   입력값을 저장할 워드의 주소 ([ebp + 8] 에 위치)
; 참고:
;   eax 와 ebx 의 값들은 파괴된다. 
segment .data
prompt  db      ") Enter an integer number (0 to quit): ", 0

segment .text
get_int:
        push    ebp
        mov     ebp, esp

        mov     eax, [ebp + 12]
        call    print_int

        mov     eax, prompt
        call    print_string
        
        call    read_int
        mov     ebx, [ebp + 8]
        mov     [ebx], eax         ; 입력값을 메모리에 저장

        pop     ebp
        ret                        ; 호출자로 분기  

; 서브프로그램 print_sum
; 합을 출력한다 
; 인자:
;   출력하기 위한 합([ebp+8] 에 위치)
; 참고: eax 의 값이 파괴된다.
;
segment .data
result  db      "The sum is ", 0

segment .text
print_sum:
        push    ebp
        mov     ebp, esp

        mov     eax, result
        call    print_string

        mov     eax, [ebp+8]
        call    print_int
        call    print_nl

        pop     ebp
        ret
\end{AsmCodeListing}


\subsection{스택에서의 지역변수}\index{스택!지역변수|(}

스택은 지역변수들의 편리한 저장 창고로 사용될 수 있다. 이는 C 가 보통의
변수들 (또는C 용어로 \emph{automatic})을 저장하는 곳이다. 변수들을
위해 스택을 사용하는 것은 재진입 서브프로그램을 만드는데 유용하다. 
재진입 서브프로그램은 서브프로그램 그 자체를 포함한 어떠한 곳에서도 호출되어도
잘 작동하는 서브프로그램을 말한다. 다시 말해, 재진입 서브프로그램은 
\emph{재귀적(recursively)}으로 호출 가능하다. 변수들을 위해 스택을 사용하는 것을 통해
메모리도 아낄 수 있다. 스택에 저장되지 않는 데이터들은 프로그램의 
시작 부터 종료 될 때 까지 메모리 상에 상주하게 된다. (C 는 이러한 형식의 변수들을
\emph{전역(global)} 이나 \emph{정적(static)} 변수라 부른다) 스택에 저장된 
데이터들의 경우 오직 서브프로그램이 실행 될 때 에만 메모리 상에 상주한다. 

\begin{figure}[t]
\begin{AsmCodeListing}[frame=single]
subprogram_label:
      push   ebp                ; 기존의 EBP 값을 스택에 저장한다. 
      mov    ebp, esp           ; 새로운 EBP = ESP
      sub    esp, LOCAL_BYTES   ; 지역 변수에 의해 = # 바이트 필요
; 서브프로그램 코드
      mov    esp, ebp           ; 지역 변수를 해제(deallocate)
      pop    ebp                ; 기존의 EBP 값을 불러온다. 
      ret
\end{AsmCodeListing}
\caption{지역 변수를 이용한 일반적인 서브프로그램 모습\label{fig:subskel2}}
\end{figure}

\begin{figure}[t]
\begin{lstlisting}[frame=tlrb]{}
void calc_sum( int n, int * sump )
{
  int i, sum = 0;

  for( i=1; i <= n; i++ )
    sum += i;
  *sump = sum;
}
\end{lstlisting}
\caption{합 프로그램의 C 언어 버전 \label{fig:Csum}}
\end{figure}

\begin{figure}[t]
\begin{AsmCodeListing}[frame=single]
cal_sum:
      push   ebp
      mov    ebp, esp
      sub    esp, 4               ; 지역 변수 sum 의 공간을 할당 

      mov    dword [ebp - 4], 0   ; sum = 0
      mov    ebx, 1               ; ebx (i) = 1
for_loop:
      cmp    ebx, [ebp+8]         ; i <= n 인가?
      jnle   end_for

      add    [ebp-4], ebx         ; sum += i
      inc    ebx
      jmp    short for_loop

end_for:
      mov    ebx, [ebp+12]        ; ebx = sump
      mov    eax, [ebp-4]         ; eax = sum
      mov    [ebx], eax           ; *sump = sum;

      mov    esp, ebp
      pop    ebp
      ret
\end{AsmCodeListing}
\caption{합 프로그램의 어셈블리 버전 \label{fig:Asmsum}}
\end{figure}

지역 변수들은 스택에 저장된 EBP 값 바로 다음부터 저장된다. 이들은 프로그램 서두에서,
ESP 로 부터 필요한 만큼의 바이트를 뺌으로써 할당 될 수 있다. 그림 ~\ref{fig:subskel2} 는
새로운 서브프로그램의 구조를 보여준다. EBP 레지스터는 지역 변수에 접근하기 위해 사용된다.
그림 ~\ref{fig:Csum} 의 C 함수를 보자. 그림 ~\ref{fig:Asmsum} 은 이와 동일한 서브프로그램이
어셈블리어로 어떻게 쓰일 수 있는지 보여준다. 

\begin{figure}[t]
\centering
\begin{tabular}{ll|c|}
\cline{3-3}
ESP + 16 & EBP + 12 & {\code sump} \\ \cline{3-3}
ESP + 12 & EBP + 8  & {\code n} \\ \cline{3-3}
ESP + 8  & EBP + 4  & 리턴 주소 \\ \cline{3-3}
ESP + 4  & EBP      & 저장된 EBP \\ \cline{3-3}
ESP      & EBP - 4  & {\code sum} \\ \cline{3-3}
\end{tabular}
\caption{}
\label{fig:SumStack}
\end{figure}

그림 ~\ref{fig:SumStack} 은 그림 ~\ref{fig:Asmsum} 의 프로그램 상단 부분을 
실행한 후의 스택의 모습을 보여준다. 인자, 리턴 정보, 지역 변수를 포함한
위와 같은 스택의 영역을 \emph{스택 프레임(stack frame)} 이라 부른다. C 함수를 
호출 할 때 마다 스택에 새로운 스택 프레임을 생성하게 된다. 


\begin{figure}[t]
\begin{AsmCodeListing}[frame=single]
subprogram_label:
      enter  LOCAL_BYTES, 0   ; 지역 변수에 의해 필요한 = # 바이트
; subprogram code
      leave
      ret
\end{AsmCodeListing}
\caption{{\code ENTER} 와 {\code LEAVE} 명령, 그리고 지역 변수를 사용한
일반적인 서브프로그램의 모습 \label{fig:subskel3}}
\end{figure}

\MarginNote{비록 {\code ENTER} 와 {\code LEAVE} 명령들이 서브프로그램의 시작 부분과
끝 부분을 간단하게 해주지만 잘 쓰이지는 않는다. 왜냐하면, 이들은 동일한 간단한
명령들의 모임보다 더 느리기 때문이다. 이는 우리가 언제나 하나의 명령이 여러개의 명령들
보다 빠르다고 확신할 수 없음을 보여주는 하나의 예이다. }

프로그램의 시작 부분과 끝 부분은 오직 이러한 목적을 위해 만들어진 두 명령
{\code ENTER} 와 {\code LEAVE} 명령이 사용되고 있다. {\code ENTER} 명령은 
시작 부분의 명령을 처리하고 {\code LEAVE} 명령은 마지막 부분의 명령을 처리한다.
{\code ENTER} 명령은 두 개의 즉시 피연산자를 가진다. C 호출 규약에선 
두 번째 피연산자는 언제나 0 이다. 첫 번째 인자는 지역 변수가 필요로 하는 
바이트의 크기를 나타낸다. {\code LEAVE} 명령은 피연산자를 갖지 않는다.
그림 ~\ref{fig:subskel3} 은 이러한 명령들이 어떻게 쓰이는지 보여주고 있다. 이 때,
골격 프로그램 (그림 ~\ref{fig:skel}) 또한 {\code ENTER} 와 {\code LEAVE}를
사용하고 있음을 주목해라. 

\index{스택!지역 변수|)}
\index{스택|)}
\index{호출 규약|)}
\index{서브프로그램!호출|)}

\section{다중 모듈 프로그램}\index{다중 모듈 프로그램|(}

\emph{다중 모듈 프로그램(multi-module program)}은 한 개보다 많은 수의
목적 파일 (object file)들로 구성된 프로그램을 말한다. 여태까지 소개한 프로그램들은
모두 다중 모듈 프로그램이였다. 이들은 C 드라이버 목적 파일과 어셈블리 목적 파일
(C 라이브러리 목적 파일도)들로 이루어져 있다. 링커는 이러한 여러 개의 목적 파일들을
하나의 실행 가능한(executable) 프로그램으로 통합해 주는 것을 상기해라. 링커는
각 모듈(\emph{i.e.} 목적 파일)들에서의 라벨들을 각각의 정의에 맞게 대응시켜 주어야만 한다.
모듈 B 에서 정의된 라벨을 모듈 A 에서 사용하기 위해선, {\code extern} 지시어가 사용되어만
한다. {\code extern} \index{지시어!extern} 지시어 옆에는 반점으로 구분은 라벨들의 목록들이
나열된다. 이 지시어는 어셈블러에 이 라벨들이 모듈이 \emph{외부(external)}에서 
정의되었음을 알려준다. 따라서, 이러한 라벨들은 다른 모듈에서 정의되었음에도 불구하고
사용될 수 있다. {\code asm\_io.inc} 파일은 {\code read\_int}, 루틴 \emph{등등}을 모두 외부로
선언한다. 

어셈블리에서는 기본적으로 라벨들은 외부에서 접근이 불가능하도록 되어있다. 만약 라벨이 다른
모듈에서 접근이 가능하다면 이 라벨은 정의된 모듈에서 반드시 \emph{전역}으로 선언되어야
한다. {\code global} \index{지시어!전역} 지시어가 이 일을 한다. 아래 그림 ~\ref{fig:skel} 의
뼈대 프로그램의 ~13 행을 보면, {\code \_asm\_main} 라벨이 전역으로 정의되어있음을 볼 수 
있다. 이 정의 없이는 링커 오류가 발생할 것이다. 왜냐하면 C 코드는 \emph{내부(internal)}로 선언된
{\code \_asm\_main} 라벨을 참조할 수 없기 때문이다. 

다음 코드는 이전의 예제를 두 개의 모듈을 이용하여 사용한 것이다. 이 두 개의 서브프로그램
({\code get\_int} 와 {\code print\_sum}) 은 {\code \_asm\_main} 루틴과는 달리 2 개의 개개의
소스 파일에 작성되어 있다. 

\begin{AsmCodeListing}[label=main4.asm,commandchars=\\\{\}]
%include "asm_io.inc"

segment .data
sum     dd   0

segment .bss
input   resd 1

segment .text
        global  _asm_main
\textit{        extern  get_int, print_sum}
_asm_main:
        enter   0,0               ; 셋업 루틴
        pusha

        mov     edx, 1            ; edx 는 의사코드에서 'i' 와 같다. 
while_loop:
        push    edx               ; i 를 스택에 저장한다. 
        push    dword input       ; 입력값의 주소를 스택에 푸시 
        call    get_int
        add     esp, 8            ; i 와 &input 을 스택에서 제거 

        mov     eax, [input]
        cmp     eax, 0
        je      end_while

        add     [sum], eax        ; sum += input

        inc     edx
        jmp     short while_loop

end_while:
        push    dword [sum]       ; 합을 스택에 푸시 
        call    print_sum
        pop     ecx               ; [sum] 을 스택에서 제거 

        popa
        leave                     
        ret
\end{AsmCodeListing}

\begin{AsmCodeListing}[label=sub4.asm,commandchars=\\\{\}]
%include "asm_io.inc"

segment .data
prompt  db      ") Enter an integer number (0 to quit): ", 0

segment .text
\textit{        global  get_int, print_sum}
get_int:
        enter   0,0

        mov     eax, [ebp + 12]
        call    print_int

        mov     eax, prompt
        call    print_string
        
        call    read_int
        mov     ebx, [ebp + 8]
        mov     [ebx], eax         ; 입력값을 메모리에 저장

        leave
        ret                        ; 호출자로 분기

segment .data
result  db      "The sum is ", 0

segment .text
print_sum:
        enter   0,0

        mov     eax, result
        call    print_string

        mov     eax, [ebp+8]
        call    print_int
        call    print_nl

        leave
        ret
\end{AsmCodeListing}

이전의 코드는 오직 전역 \index{지시어!전역}코드 라벨들만 가지고 있었다.
그러나, 전역 데이터 라벨들은 똑같은 방법으로 작동된다. 

\index{다중 모듈 프로그램|)}

\section{C 와 소통하기}\index{C 와 소통하기|(}\index{호출 규약!C|(}

오늘날, 매우 적은 수의 프로그램들만 완전히 어셈블리어로 작성된다. 컴파일러는
고급 레벨 코드를 효율적인 기계어 코드로 변환하는 일을 매우 잘한다. 고급 언어로
코드를 작성하는 것이 쉽기에, 이는 훨씬 대중적이다. 게다가, 고급 언어는 어셈블리어에
비해 \emph{훨씬} 이식성이 높다!

만일 어셈블리어가 직접적으로 사용된다면 오직 코드의 작은 부분들에서만 사용된다. 
이는 두 가지 방법으로 가능하다: C 에서 어셈블리 서브루틴을 호출하던지, 인라인(inline)
어셈블리를 이용하는 것이다. 인라인 어셈블리는 프로그래머로 하여금 C 코드에 어셈블리
문장을 직접적으로 넣을 수 있게 해주는 것이다. 이는 매우 편리하다. 그러나, 인라인 어셈블리에도
몇 가지 단점이 있다. 어셈블리 코드는 반드시 컴파일러가 사용하는 형식으로 작성되어야만
한다. 그러나, 현재 NASM 의 형식을 지원하는 컴파일러가 없다. 다른 컴파일러들은
다른 형식을 사용한다. 볼랜드와 마이크로소프트사의 경우 MASM 형식을 이용한다. 
DJGPP 와 리눅스의 gcc 는 GAS \footnote{GAS 는 모든 GNU 컴파일러가 사용하는 어셈블러
이다. 이는 AT\&T 문법을 이용하는데, AT\&T 의 문법은 문법이 대략 비슷한 MASM, TASM, NASM 들과는 
달리 매우 다르다.} 형식을 이용한다. PC 에선 어셈블리 서브루틴을 호출하는 방법이 
훨씬 표준화 되어 있다. 

어셈블리 루틴들은 다음과 같은 이유로 C 코드와 함게 사용된다:

\begin{itemize}
\item C 로 하기에 어렵거나 매우 힘든 컴퓨터 하드웨어의 
      직접적인 접근 
\item 매우 빠르게 작동 되어야만 루틴, 컴파일러가 할 수 있는 것  
      보다도 빠르게 프로그래머가 직접 손으로 최적화를 해야 되는 부분 
\end{itemize}

마지막 이유는 예전과는 달리 거의 사실이 아니게 되었다. 컴파일러 기술은 해를
거듭하여 진보하였고, 컴파일러들은 매우 효율적인 코드(특히, 컴파일러 최적화가
진행된다면)를 만들어 낼 수 있다. 어셈블리 루틴의 단점은 : 이식성이 낮고,
가독성이 떨어진다. 

대부분의 C 호출 규약은 이미 이야기 하였다. 그러나, 몇 개의 설명이 필요한
부분들이 남아있다. 

\subsection{레지스터 저장하기}\index{호출 규약!C!레지스터|(}
먼저, C 는 서브루틴이 다음의 레지스터의 값을 보관하다고 생각한다:
EBX, ESI, EDI, EBP, CS, DS, SS, ES. 
\MarginNote{
{\code register} 키워드는 컴파일러에게 C 변수를 메모리 공간에 저장하지
않고 레지스터에 저장하라고 명령하는 것이다. 현대의 컴파일러들은 이를
특별히 명시하지 않고도 자동적으로 해준다.} 이 말은, 서브루틴이 내부적으로
이들을 바꿀 수 없다는 것이 아니다. 그 대신에, 이이것말은 이 레지스터들의 값을
바꾸더라도 서브루틴이 리턴할 경우 이전의 값들을 복원할 수 있어야만 한다는 
것이다. EBX, ESI, EDI 의 값을은 반드시 변경되면 안되는데 왜냐하면 C 는 이 
레지스터들을 \emph{레지스터 변수(register variable)} 로 사용하기 때문이다.
보통 스택은 위 레지스터들의 원래 값들을 저장하는데 사용된다. 

\begin{figure}[t]
\begin{AsmCodeListing}[frame=single]
segment .data
x            dd     0
format       db     "x = %d\n", 0

segment .text
...
      push   dword [x]     ; x 의 값을 푸시
      push   dword format  ; 형식 문자열의 주소를 푸시
      call   _printf       ; _ 에 주목!
      add    esp, 8        ; 스택에서 인자들을 제거 
\end{AsmCodeListing}
\caption{{\code printf} 로 호출 \label{fig:Cprintf}}
\end{figure}

\begin{figure}[t]
\centering
\begin{tabular}{l|c|}
\cline{2-2}
EBP + 12 & {\code x} 의 값 \\ \cline{2-2}
EBP + 8  & 형식 문자열의 주소\\ \cline{2-2}
EBP + 4  & 주소를 리턴 \\ \cline{2-2}
EBP      & EBP 를 저장 \\ \cline{2-2}
\end{tabular}
\caption{{\code printf} 내부의 스택 모습\label{fig:CprintfStack}}
\end{figure}
\index{호출 규약!C!레지스터|)}

\subsection{함수들의 라벨}\index{호출 규약!C!라벨|(}

대부분의 C 컴파일러들은 함수나, 전역/정적 변수들의 이름 앞에 
{\code \_} 한 개를 붙인다. 예를 들어서 {\code f} 라는 이름의 함수는
{\code \_f} 라는 이름의 라벨로 대응된다. 따라서, 이것이 어셈블리
루틴이 되기 위해선 반드시 {\code f} 가 아닌 {\code \_f} 로
라벨이 붙어야만 한다. 리눅스 gcc 컴파일러는 어떠한 문자도 붙이지
\emph{않는다}. 리눅스 ELF 실행 가능 파일들의 경우, C 함수 {\code f}
에 대해 그냥 {\code f} 라는 이름의 라벨을 사용한다. 그러나
DJGPP 의 gcc 는 {\code \_} 를 붙인다. 어셈블리 골격 프로그램(그림 ~\ref{fig:skel})
을 보면 메인 루틴의 라벨은 {\code \_asm\_main} 로 되어 있음을 주목해라. 

\index{호출 규약!C!라벨|)}

\subsection{인자 전달하기}\index{호출 규약!C!인자|(}

C 호출 규약의 경우, 스택에 푸시된 함수들의 인자들은 함수 호출시 \emph{역순}으로
나타나게 된다.

예를 들어 다음과 같은 C 문장을 생각해 보자:\verb|printf("x = %d\n",x);|
그림 ~\ref{fig:Cprintf} 는 이것이 어떻게 컴파일 될 지 보여준다. (동일한 NASM 형식으로 나타남)
그림 ~\ref{fig:CprintfStack} 은 {\code printf} 의 처음 부분을 실행한 후의 스택의
모습을 보여준다. {\code printf} 함수는 C 라이브러리의 함수 중 하나로, 가변 개수의
인자들을 가질 수 있다. C 호출 규약의 경우 이러한 형식의 함수들을 작성할 수 있다. 

\MarginNote{C 에서 어셈블리를 이용하여 임의의 개수의 인자를 받는 과정을 만들어 줄 필요는 없다.
{\code stdarg.h} 헤더파일은 위 과정을 손쉽게 할 수 있는 메크로를 제공해 준다. 다른 
C 언어 책을 참조하면 된다.}

형식 문자열의 주소가 마지막에 푸시되므로, 이것의 스택에서의 위치는 \emph{언제나} 얼마나
많은 수의 인자들이 푸시되느냐에 상관 없이 {\code EBP+8} 이 된다. {\code printf} 코드는 
형식 문자열을 보아서 알마나 많은 수의 인자들이 전해 졌는지 알아 내어, 스택을 찾아 볼 수 있다. 

당연하게도, \verb|printf("x = %d\n")| 와 같이 실수를 한다면 {\code printf} 코드는 {\code [EBP+12]}
에 위치한 더블워드 값을 출력할 것이다. 그러나, 이 값은 {\code x} 의 값이 아니다! 

\index{호출 규약!C!인자|)}

\subsection{지역 변수의 주소 계산하기}\index{스택!지역 변수|(}

{\code data} 나 {\code bss} 세그먼트에 정의된 라벨의 주소를 찾는 것은 
간단하다. 기본적으로 링커가 이러한 작업을 한다. 그러나, 직관적으로 스택에
저장된 지역 변수(혹은 인자) 의 주소를 계산하는 것은 어렵다. 그러나, 우리가 
서브루틴들을 호출시에 이 작업을 자주 하게 된다. 아래와 같이 변수 ({\code x} 라 하자) 
의 주소를 함수(이를 {\code foo} 라 한다)에 전달하는 경우를 살펴보자. 
만일 {\code x} 가 EBP $-$ 8 의 스택에 위치해 있다면, 우리는 아래와 같이
그냥 사용할 수 없다. 

\begin{AsmCodeListing}[numbers=none,frame=none]
      mov    eax, ebp - 8
\end{AsmCodeListing}

왜냐하면 {\code MOV} 명령을 통해 EAX 로 저장되는 값이 계산되어야만 하기 
때문이다. (이 명령의 피연산자는 반드시 상수여야 한다.) 그러나, 위와 같은 연산을 하는
명령이 있다. 이는 \index{LEA|(} {\code LEA} (\emph{Load Effective Address} 의 약자)
이다. 아래의 코드는 EAX 로 {\code x} 의 주소를 계산하여 집어 넣는다. 

\begin{AsmCodeListing}[numbers=none,frame=none]
      lea    eax, [ebp - 8]
\end{AsmCodeListing}

이제 EAX 는 {\code x} 의 주소를 가지고, {\code foo} 를 호출시에 스택에 푸시될 수 있다.
종종 위 명령을 [EBP\nolinebreak$-$\nolinebreak8] 에 있는 데이터를 읽어 들이는 것으로 착각하는데,
이것은 사실이 \emph{아니다}. {\code LEA} 명령은 \emph{절대로} 메모리를 읽어들이지 않는다.
이는 오직 다른 명령이 읽어들일 주소값을 계산하고, 이를 첫 번째 레지스터 피연산자에 저장할
뿐 이다. 이것이 메모리를 읽어들이지 않기 때문에 메모리 크기를 지정(\emph{e.g} 
{\code dword}) 할 필요가 없다. 

\index{LEA|)}
\index{스택!지역 변수|)}

\subsection{리턴값}\index{호출 규약!C!리턴값|(}

void C 함수가 아닌 것들은 모두 값을 반환한다. C 호출 규약은 이를 어떻게 해야할 
지 정해 놓았다. 리턴값은 레지스터를 통해서 전달된다. 모든 정수형 ({\code char}, {\code int}, {\code enum}, \emph{etc.})
리턴값들은 EAX 레지스터에 저장되어 리턴된다. 만일 리턴값이 32 비트 보다 작을 경우 32 비트로 확장되어
EAX 에 저장된다. 64 비트 값들은 EDX:EAX \index{레지스터!EDX:EAX} 레지스터 쌍에 저장된다. 
포인터 값 또한 EAX 에 저장된다. 부동 소수점 값들의 경우, 수치 부프로세서의 ST0 레지스터에 저장된다.
(이 레지스터에 대해선, 부동 소수점 장에서 다루겠다.)

\index{호출 규약!C!리턴값|)}
\index{호출 규약!C|)}

\subsection{다른 호출 규약}\index{호출 규약|(}

위에 설명한 것들은 모두 80x86 C 컴파일러에 의해 지원되는 표준
C 호출 규약이였다. 보통 컴파일러들은 다른 형식의 호출 규약들도 
지원한다. 어셈블리 언어와 다른 언어를 같이 이용한다면, 컴파일러가 무슨 호출 규약을
이용하는지 아는 것이 매우 중요하다. 많은 경우 기본적으로는 표준 호출 규약을
이용한다. 그러나, 모든 컴파일러가 그러는 것은 아니다.
\footnote{Watcom C 컴파일러는 \index{컴파일러!Watcom} 표준 호출 규약을 
기본값으로 이용하지 \emph{않는} 예이다. 자세한 내용은 Watcom 을 위한
예제 코드를 참조해라.}
여러 종류의 호출 규약을 사용할 수 있는 컴파일러들의 경우, 커맨드 라인에서 
스위치를 이용하여 기본값으로 무슨 호출 규약을 이용 할지 설정 할 수 있다. 
또한, C의 문법을 확장하여 개개의 함수를 다른 호출 규약을 이용하여 컴파일
할 수도 있다. 그러나, 이러한 확장은 표준화 되어 있지 않으며, 컴파일러 마다
다를 수 있다. 

GCC 컴파일러는 여러 종류의 호출 규약을 지원한다. 함수의 호출 규약은 
{\code \_\_attribute\_\_} \index{컴파일러!gcc!\_\_attribute\_\_}. 확장을 이용하여 개별로 지정할 수 있다. 
예를 들어서, 표준 호출 규약을 이용하는 {\code f} 란 이름의 1 개의 {\code int} 인자를 가지는 void 함수를 선언하려면,
\index{호출 규약!C} 함수의 원형에 대해 아래와 같이 이용하면 된다. 

\begin{lstlisting}[stepnumber=0]{}
void f( int ) __attribute__((cdecl));
\end{lstlisting}

GCC 는 또한 \emph{표준 호출(standard call)} \index{호출 규약!stdcall} 규약
을 지원한다. {\code stdcall} 로 정의하려면 {\code cdecl} 을 {\code stdcall} 로 바꾸면
된다. {\code stdcall} 과 {\code cdecl} 의 차이점은 {\code stdcall} 의 경우
서브루틴이 스택으로 부터 인자를 제거해야 된다는 점이다. (이는 Pascal 호출 규약과 유사하다) 
따라서, {\code stdcall} 호출 규약은 오직 고정된 수의 인자를 가지는 함수들에게서만
사용될 수 있다. (\emph{i.e} 이는 {\code printf} 나 {\code scanf} 와는 다르다) 

GCC 는 또한 {\code regparm} 라는 attribute 를 지원하여 
\index{호출 규약!레지스터} 컴파일러로 하여금 스택을 이용하지 않고
레지스터를 통해 최대 3 개의 정수 인자를 전달하게 한다. 이러한 형식의 최적화는
많은 컴파일러들이 이용하고 있다. 

볼랜드와 마이크로소프트는 호출 규약을 선언할 때 동일한 문법을 사용한다.
이 들은 C 에 {\code \_\_cdecl}\index{호출 규약!\_\_cdecl} 와
{\code \_\_stdcall}\index{호출 규약!\_\_stdcall} 키워드를 추가하였다. 
이 키워드들은 함수의 원형의 이름 바로 앞에 써 주면 된다. 예를 들어 위의 함수 {\code f} 는
볼랜드나 마이크로소프트사의 경우 아래와 같이 정의된다. 

\begin{lstlisting}[stepnumber=0]{}
void __cdecl f( int );
\end{lstlisting}

각각의 호출 규약에는 장점과 단점이 있다. {\code cdecl} 
\index{호출 규약!C} 의 가장 큰 장점으로는 이는 매우 단순하고 다루기 쉽다.
이는 어떠한 형식의 C 함수나 C 컴파일러에서 든지 사용될 수 있다. 다른 형식의
호출 규약의 경우 서브루틴의 이식성을 떨어뜨린다. 하지만, 이 규약의 가장 큰
단점은 바로 다른 호출 규약들에 비해 느릴 수 있고, 더 많은 메모리를 차지한다는
것이다. (왜냐하면 함수가 호출 될 때마다, 코드에 저장된 인자들을 제거하는 명령을
수행해야 되기 때문이다) 

{\code stdcall} 호출 규약의 장점\index{호출 규약!표준 호출}은 {\code cdecl}
보다 메모리를 적게 소모한다는 점이다. {\code CALL} 명령 이후에 스택을 정리할 필요가 없다.
다만, 이 규약의 가장 큰 단점으로는 이 규약이 가변 개수의 인자들을 가지는 
함수들에게 사용 될 수 없다는 점이다. 

레지스터를 이용하는 호출 규약의 장점으로는 인자를 전달하는 속도가 매우 빠르다는
점이다. 그러나, 이러한 규약들은 좀 더 복잡하다. 인자들 중 일부는 레지스터에, 
나머지는 스택에 보관되어야 하기 때문이다. 

\index{호출 규약|)}

\subsection{예제}

다음 예제는 어떻게 어셈블리 루틴이 C 프로그램에서 작동할 수 있는지 보여준다.
(이 프로그램은 그림 ~\ref{fig:skel} 과 같은 어셈블리 뼈대 프로그램이나, driver.c 
모듈과 같은 것을 사용하지 않는다.) 

\LabelLine{main5.c}
\begin{lstlisting}[escapeinside=~~]{}
#include <stdio.h>
/* ~ 어셈블리 루틴의 원형~ */
void calc_sum( int, int * ) __attribute__((cdecl));

int main( void )
{
  int n, sum;

  printf("Sum integers up to: ");
  scanf("%d", &n);
  calc_sum(n, &sum);
  printf("Sum is %d\n", sum);
  return 0;
}
\end{lstlisting}
\LabelLine{main5.c}

\begin{AsmCodeListing}[label=sub5.asm, commandchars=\\\%|]
; 서브루틴 _calc_sum
; 1 부터 n 까지 정수들의 합을 구한다. 
; 인자:
;   n    - 어디까지 합할 지 저장 ([ebp + 8] 에 위치)
;   sump - sum 을 저장할 공간에 대한 int 포인터([ebp + 12] 에 위치)
; 의사 C 코드:
; void calc_sum( int n, int * sump )
; {
;   int i, sum = 0;
;   for( i=1; i <= n; i++ )
;     sum += i;
;   *sump = sum;
; }

segment .text
        global  _calc_sum
;
; 지역변수:
;   [ebp-4] 에 위치한 합 
_calc_sum:
        enter   4,0               ; sum 을 위한 스택 공간을 할당
        push    ebx               ; 매우 중요하다! \label%line:pushebx|

        mov     dword [ebp-4],0   ; sum = 0
        dump_stack 1, 2, 4        ; ebp-8 에서 ebp+16 까지 스택을 출력\label%line:dumpstack|
        mov     ecx, 1            ; ecx 는 의사코드에서의 i 
for_loop:
        cmp     ecx, [ebp+8]      ; cmp i 와 n
        jnle    end_for           ; if i <= n 가 아니면, 종료

        add     [ebp-4], ecx      ; sum += i
        inc     ecx
        jmp     short for_loop

end_for:
        mov     ebx, [ebp+12]     ; ebx = sump
        mov     eax, [ebp-4]      ; eax = sum
        mov     [ebx], eax

        pop     ebx               ; restore ebx
        leave
        ret
\end{AsmCodeListing}

\begin{figure}[t]
\begin{Verbatim}[frame=single]
Sum integers up to: 10
Stack Dump # 1
EBP = BFFFFB70 ESP = BFFFFB68
 +16  BFFFFB80  080499EC
 +12  BFFFFB7C  BFFFFB80
  +8  BFFFFB78  0000000A
  +4  BFFFFB74  08048501
  +0  BFFFFB70  BFFFFB88
  -4  BFFFFB6C  00000000
  -8  BFFFFB68  4010648C
Sum is 55
\end{Verbatim}
\caption{sub5 프로그램의 실행 예제 \label{fig:dumpstack}}
\end{figure}

왜 {\code sub5.asm} 의 ~\ref{line:pushebx} 행이 중요할까? 왜냐하면 C 호출 규약은
EBX 의 값이 바뀌면 안되기 때문이다. 만일, 이 처리를 해주지
않는다면 프로그램이 제대로 동작하지 않을 가능성이 매우 크다. 

~\ref{line:dumpstack} 행은 {\code dump\_stack} 매크로가 어떻게 작동하는지 보여주고 있다.
첫 번째 인자는 단지 정수 라벨, 두 번째와 세번째 인자는 출력할 EBP 의 범위를 설정하는 것임을
기억하자. 그림 ~\ref{fig:dumpstack} 은 프로그램을 실행한 모습을 보여준다. 이 덤프에서, 합을
저장한 곳의 주소가 BFFFFB80 (EBP~+~12 에 위치) 이고, 어디까지 합할지 기억한 값은 
0000000A (EBP~+~8 에 위치), 루틴의 리턴 주소는 08048501(EBP~+~4 에 위치), 
저장된 EBP 의 값은 BFFFFB88(EBP 에 위치), 지역 변수의 값은 0 (EBP~-~4 에 위치), 
마지막으로 저장된 EBX 의 값은 4010648C(EBP ~-~8 에 위치) 임을 알 수 있다. 

{\code calc\_sum} 함수는 합을 저장하는 것을 포인터 인자로 받아들이는 대신에 
합을 리턴값으로 출력하게 수정할 수 있다. 합이 정수값이기 때문에 그 합은 EAX 에 저장되어야 한다. 
{\code main5.c} 의 ~11 행은 아래와 같이 바뀐다:

\begin{lstlisting}[stepnumber=0]{}
  sum = calc_sum(n);
\end{lstlisting}
또한, {\code calc\_sum} 의 원형 또한 변경되어야 한다. 아래는 수정된
어셈블리 코드 이다:
\begin{AsmCodeListing}[label=sub6.asm]
; 서브루틴 _calc_sum
; 1 부터 n 까지 정수들의 합을 구한다. 
; 인자:
;   n    - 어디까지 합할 지 저장 ([ebp + 8] 에 위치)
; 리턴값:
;   합한 값
; 의사 C 코드:
; int calc_sum( int n )
; {
;   int i, sum = 0;
;   for( i=1; i <= n; i++ )
;     sum += i;
;   return sum;
; }
segment .text
        global  _calc_sum
;
; local variable:
;   sum at [ebp-4]
_calc_sum:
        enter   4,0               ; 합을 저장할 공간을 스택에 할당

        mov     dword [ebp-4],0   ; sum = 0
        mov     ecx, 1            ; ecx 는 의사코드에서의 i
for_loop:
        cmp     ecx, [ebp+8]      ; cmp i 와 n
        jnle    end_for           ; if i <= n 가 아니면, 종료

        add     [ebp-4], ecx      ; sum += i
        inc     ecx
        jmp     short for_loop

end_for:
        mov     eax, [ebp-4]      ; eax = sum

        leave
        ret
\end{AsmCodeListing}

\subsection{어셈블리에서 C 함수를 호출하기}

\begin{figure}[t]
\begin{AsmCodeListing}[frame=single]
segment .data
format       db "%d", 0

segment .text
...
      lea    eax, [ebp-16]
      push   eax
      push   dword format
      call   _scanf
      add    esp, 8
...
\end{AsmCodeListing}
\caption{어셈블리에서 {\code scanf} 호출하기\label{fig:scanf}}
\end{figure}

C 와 어셈블리를 같이 이용하는데 있어서 가장 큰 장점은 바로 어셈블리 코드를 
통해서 C 라이브러리와 유저가 작성한 함수들에 접근이 가능하다는 점이다.
예를 들어, 만약 {\code scanf} 함수를 호출하여 키보드로 부터 정수를 
입력 받고 싶다고 하자. 그림 ~\ref{fig:scanf} 는 위 작업을 하는 코드를
보여준다. {\code scanf} 는 C 호출 규약을 지킨다는 점을 잊지 말아야 한다. 
이 말은 즉슨, 이는 EBX, ESI, EDI 레지스터의 값을 저장하지만, EAX, ECX, EDX
레지스터의 값은 변경 될 수 있다는 점이다. 사실, EAX 는 {\code scanf} 호출의
리턴값을 저장하기 때문에 값이 반드시 바뀌게 된다. C 코드를 함께 이용하는 어셈블리
코드들의 예를 보고 싶다면 {\code asm\_io.obj} 를 만들기 이용하였던
{\code asm\_io.asm} 의 코드를 한 번 봐보면 좋다. 

\index{C 와 함께 작업하기|)}

\section{재진입 및 재귀 서브프로그램}\index{재귀|(}

\index{서브프로그램!재진입|(}
재진입 서브프로그램(reetrant) 는 다음의 조건들을 반드시 만족해야 한다. 

\begin{itemize}
\item 
이는 어떠한 코드 명령도 수정하면 안된다. 고급 언어에선 조금 어렵겠지만,
어셈블리에서는 자기 자신의 코드를 수정하지 않는 일은 어렵지 않다. 예를 들어:

\begin{AsmCodeListing}[frame=none, numbers=none]
      mov    word [cs:$+7], 5      ; 5 를 7 바이트 앞에 있는 워드에 복사 
      add    ax, 2                 ; 이전의 문장은 2 에서 5 로 바뀐다.
\end{AsmCodeListing}

위 코드는 실제모드에서 작동한다. 그러나 보호모드에서는 코드 세그먼트가 오직
읽기 전용(Read-Only)이다. 따라서, 위 첫번째 행이 실행된다면 보호모드 시스템에선 프로그램이
중단될 것이다. 이러한 형식의 프로그래밍은 여러 이유에서 좋지 않다. 이는 혼란스러울
뿐더러, 유지, 보수가 어렵고, 코드 공유(code sharing)를 할 수 없다. (아래 참고)

\item 이는 전역 데이터를 수정하면 안된다. (예를 들어 {\code data} 나
{\code bss} 세그먼트에 들어있는 데이터) 모든 변수들은 스택에 보관된다. 

\end{itemize}

재진입 코드를 쓰면 몇 가지 좋은 점들이 있다. 

\begin{itemize}
\item 재진입 프로그램은 재귀적으로 호출 가능하다.
\item
재진입 프로그램은 다수의 프로세스들에 의해 공유 가능하다. 많은 수의 
멀티태스킹 운영체제에선, 하나의 프로그램에서 작동되는 여러 개의 인스턴스(instance)가
있을 때, 오직 코드의 \emph{한} 부분만이 메모리에 상주한다. 공유되는 
라이브러리와 DLL (\emph{동적 링크 라이브러리(Dynamic Link Libraries)}) 또한
이 아이디어를 이용한다. 
\item 재진입 프로그램은 \emph{다중 쓰레드(multi-threaded)} 에서 더 잘 작동된다. 
\footnote{다중 쓰레드 프로그램은 실행시 여러 개의 쓰레드를 가진다. 이 말은, 
프로그램 자체가 멀티태스킹을 한다는 뜻이다.} pro\-grams. Windows 9x/NT 그리고
대부분의 UNIX 꼴의 운영체제 (Solaris, Linux, \emph{etc.}) 가 다중 쓰레드 프로그램을
지원한다. 


\end{itemize}
\index{서브프로그램!재진입|)}

\subsection{재귀 서브프로그램}

이러한 형식의 서브프로그램들은 자기 자신을 호출한다. 재귀 호출은 
\emph{직접(direct)} 혹은 \emph{간접(indirect)} 호출이 될 수 있다. 직접
재귀 호출은 {\code foo} 라는 서브프로그램이 {\code foo} 내부에서
스스로 호출했을 때를 일컫는다. 간접 재귀 호출은 서브프로그램이 자기
자신으로 부터 직접적으로 호출되지 않았지만 다른 서브프로그램이 호출하기에 
호출된 것을 의미한다. 예를 들면 {\code foo} 는 {\code bar} 을 호출 할 수 있고,
{\code bar} 은 {\code foo} 를 호출 할 수 있는 경우 이다. 

재귀 서브프로그램들은 반드시 \emph{종료 조건(termination condition)} 
이 있어야 한다. 만약 위 조건이 참이라면 더이상의 재귀 호출은 만들어
지지 않는다. 만약 재귀 루틴이 종료 조건이 없거나, 조건이 절대로 참이 
될 수 없다면 이는 스스로 재귀를 무한히 반복할 것이다. (마치 무한 루프 처럼)

\begin{figure}
\begin{AsmCodeListing}[frame=single]
; finds n!
segment .text
      global _fact
_fact:
      enter  0,0

      mov    eax, [ebp+8]    ; eax = n
      cmp    eax, 1
      jbe    term_cond       ; 만일 n <= 1 이면 종료
      dec    eax
      push   eax
      call   _fact           ; eax = fact(n-1)
      pop    ecx             ; eax 에 답 저장
      mul    dword [ebp+8]   ; edx:eax = eax * [ebp+8]
      jmp    short end_fact
term_cond:
      mov    eax, 1
end_fact:
      leave
      ret
\end{AsmCodeListing}
\caption{재귀 팩토리얼 함수\label{fig:factorial}}
\end{figure}

\begin{figure}
\centering
%\includegraphics{factStack.eps}
\input{factStack.latex}
\caption{팩토리얼 함수의 스택 프레임\label{fig:factStack}}
\end{figure}

그림 ~\ref{fig:factorial} 은 재귀적으로 팩토리얼을 계산하는 함수를 보여준다.
이는 C 를 통해 아래와 같이 호출될 수 있다:

\begin{lstlisting}[stepnumber=0]{}
x = fact(3);         /* find 3! */
\end{lstlisting}

그림 ~\ref{fig:factStack} 은 마지막으로 호출 될 때의 스택의 모습을 보여준다.

\begin{figure}[t]
\begin{lstlisting}[frame=tlrb]{}
void f( int x )
{
  int i;
  for( i=0; i < x; i++ ) {
    printf("%d\n", i);
    f(i);
  }
}
\end{lstlisting}
\caption{또다른 예제 (C 버전)\label{fig:rec2C}}
\end{figure}

\begin{figure}
\begin{AsmCodeListing}[frame=single]
%define i ebp-4
%define x ebp+8          ; 유용한 매크로
segment .data
format       db "%d", 10, 0     ; 10 = '\n'
segment .text
      global _f
      extern _printf
_f:
      enter  4,0           ; i 를 위한 스택 공간 할당

      mov    dword [i], 0  ; i = 0
lp:
      mov    eax, [i]      ; i < x 인가?
      cmp    eax, [x]
      jnl    quit

      push   eax           ; printf 호출
      push   format
      call   _printf
      add    esp, 8

      push   dword [i]     ; f 호출
      call   _f
      pop    eax

      inc    dword [i]     ; i++
      jmp    short lp
quit:
      leave
      ret
\end{AsmCodeListing}
\caption{또다른 예제 (어셈블리 버전)\label{fig:rec2Asm}}
\end{figure}

그림 ~\ref{fig:rec2C} 와 \ref{fig:rec2Asm} 은 각각 C 와 어셈블리를 이용한
좀 더 복잡한 예제를 보여준다. {\code f(3)} 의 결과는 무엇일까? {\code ENTER} 명령이
새로운 {\code i} 를 각 재귀 호출 마다 스택에 생성한다는 점을 주목해라. 따라서, 각각의 
{\code f} 의 재귀 인스턴스는 독립적인 변수 {\code i} 를 가지게 된다. 이는 {\code i} 를 
{\code 데이터} 세그먼트에 더블워드로 정의하는 것과 다르다. 

\index{재귀|)}

\subsection{C 변수 저장 형식에 대해서}

C 는 변수를 저장하기 위한 방법으로 몇 가지 형식들을 지원한다. 
\begin{description}
\item[전역(global)] 
\index{저장 형식!전역}
이러한 변수들은 모든 함수의 외부에서 정의되며, 고정된 메모리 위치에
저장되어 있다 ({\code data} 나 {\code bss} 세그먼트에) 또한, 프로그램
시작 부터 끝날 때 까지 계속 존재한다. 기본적으로 이들은 프로그램의 어떠한
함수에서든지 접근이 가능하다. 그러나, 이들이 {\code static} 으로 선언되어
있으면, 오직 같은 모듈에서의 함수들만이 이들에 접근가능하다 (\emph{i.e.}
어셈블리 형식으로 말하자면, 라벨이 외부(external) 이 아닌 내부(internal) 이다).

\item[정적(static)] 
\index{저장 형식!정적}

이들은 함수의 \emph{지역(local)} 변수로, {\code static} 으로 선언된다.
(불행이도, C 언어는 {\code static} 을 두 가지 용도로 사용한다!) 이러한 변수들
또한 고정된 메모리 위치에 상주한나, ({\code data} 나 {\code bss}) 오직
이들이 정의되어 있는 함수에서만 직접적으로 접근이 가능하다. 

\item[자동(automatic)] 
\index{저장 형식!자동}

이는 C 변수가 함수 내에서 정의될 때 기본적으로 지정되는 형식이다. 이러한 
변수들은 함수가 호출 될 때 스택에 할당되고, 함수가 리턴될 때 스택에서 
사라진다. 따라서 이들은 고정된 메모리 위치를 갖고 있지 않는다. 

\item[레지스터(register)] 
\index{저장 형식!레지스터}

이 키워드는 컴파일러로 하여금 이 변수의 데이터를 위해 레지스터를 사용하도록 한다.
이는 단순히 \emph{요청} 이다. 즉, 컴파일러는 이를 반드시 지켜야 할 필요가 \emph{없다}.
이 변수의 주소가 프로그램에서 사용된다면 이 요청은 무시된다 (왜냐하면 레지스터들은
주소가 없기 때문이다). 또한, 오직 단순한 정수 형식들 만이 레지스터의 값이 될 수 있다. 
구조형은 될 수 없다; 왜냐하면 이들은 레지스터 크기에 맞지 않기 때문이다! C 컴파일러는 종종 보통의
자동 변수들을 프로그래머에 알리지 않고 레지스터 변수로 바꾸어 주기도 한다. 

\item[휘발성(volatile)] 
\index{저장 형식!휘발성}

이 키워드는 컴파일러에게 이 변수의 값이 어무 때나 바뀔 수 있음을 알려준다. 
이 말은 컴파일러가 이 변수가 언제 수정될지에 대한 정보를 알 수 없다는 
뜻이다. 종종 컴파일러는 변수의 값을 레지스터에 잠시 보관해 놓고, 변수가 
사용되는 코드에서 이를 대신 사용한다. 그러나, {\code volatile} 형식의 변수들에겐
이러한 형태의 최적화를 할 수 없다. 가장 흔한 volatile 예제로는 다중 쓰레드 프로그램에서
두 개의 쓰레드에 의해 값이 변경 될 수 있는 변수 이다. 아래의 코드를 고려하자
\begin{lstlisting}{}
x = 10;
y = 20;
z = x;
\end{lstlisting}

{\code x} 는 다른 쓰레드에 의해 변경 될 수 있고, ~1 과 3 행에서 다른 쓰레드가
{\code x} 의 값을 변경하여 {\code z} 가 0 이 안될 수 있다. 그러나, {\code x} 가
휘발성으로 선언되지 않으면 컴파일러는 {\code x} 가 변경되지 않았다고 생각하고
{\code z} 를 10 으로 바꾸어 준다. 

{\code volatile} 은 컴파일러가 변수를 위해 레지스터를 사용하는 것을
막을 수 있다. 

\end{description}
\index{서브프로그램|)}
