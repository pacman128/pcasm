% -*-LaTex-*-
%front matter of pcasm book

\chapter*{Preface}
\markboth{\MakeUppercase{Preface}}{} % Clear running headers
\addcontentsline{toc}{chapter}{Preface}

\section*{Purpose}

The purpose of this book is to give the reader a better understanding
of how computers really work at a lower level than in programming
languages like Pascal. By gaining a deeper understanding of how
computers work, the reader can often be much more productive
developing software in higher level languages such as C and
C++. Learning to program in assembly language is an excellent way to
achieve this goal. Other PC assembly language books still teach how to
program the 8086 processor that the original PC used in 1981!  The
8086 processor only supported \emph{real} mode. In this mode, any
program may address any memory or device in the computer. This mode is
not suitable for a secure, multitasking operating system.  This book
instead discusses how to program the 80386 and later processors in
\emph{protected} mode (the mode that Windows and Linux runs in).
This mode supports the features that modern operating systems expect,
such as virtual memory and memory protection.
There are several reasons to use protected mode:
\begin{enumerate}
\item It is easier to program in protected mode than in the 8086 real mode
      that other books use.
\item All modern PC operating systems run in protected mode.
\item There is free software available that runs in this mode.
\end{enumerate}
The lack of textbooks for protected mode PC assembly programming is the
main reason that the author wrote this book.

As alluded to above, this text makes use of Free/Open Source software: namely,
the NASM assembler and the DJGPP C/C++ compiler. Both of these are available
to download from the Internet. The text also discusses how to use NASM 
assembly code under the Linux operating system and with Borland's and
Microsoft's C/C++ compilers under Windows. Examples for all of these
platforms can be found on my web site: 
{\code http://pacman128.github.io/pcasm/}.
You \emph{must} download the example code if you wish to assemble
and run many of the examples in this tutorial.

Be aware that this text does not attempt to cover every aspect of assembly
programming. The author has tried to cover the most important topics that
\emph{all} programmers should be acquainted with.

\section*{Acknowledgements}

The author would like to thank the many programmers around the world
that have contributed to the Free/Open Source movement. All the
programs and even this book itself were produced using free
software. Specifically, the author would like to thank John~S.~Fine,
Simon~Tatham, Julian~Hall and others for developing the NASM assembler
that all the examples in this book are based on; DJ Delorie for
developing the DJGPP C/C++ compiler used; the numerous people who have
contributed to the GNU gcc compiler on which DJGPP is based on; Donald
Knuth and others for developing the \TeX\ and \LaTeXe\ typesetting
languages that were used to produce the book; Richard Stallman
(founder of the Free Software Foundation), Linus Torvalds (creator of
the Linux kernel) and others who produced the underlying software the
author used to produce this work.

Thanks to the following people for corrections:
\begin{itemize}
\item John S. Fine
\item Marcelo Henrique Pinto de Almeida
\item Sam Hopkins
\item Nick D'Imperio
\item Jeremiah Lawrence
\item Ed Beroset
\item Jerry Gembarowski
\item Ziqiang Peng
\item Eno Compton
\item Josh I Cates
\item Mik Mifflin
\item Luke Wallis
\item Gaku Ueda
\item Brian Heward
\item Chad Gorshing
\item F. Gotti
\item Bob Wilkinson
\item Markus Koegel
\item Louis Taber
\item Dave Kiddell
\item Eduardo Horowitz
\item S\'{e}bastien Le Ray
\item Nehal Mistry
\item Jianyue Wang
\item Jeremias Kleer
\item Marc Janicki
\item Trevor Hansen
\item Giacomo Bruschi
\item Leonardo Rodr\'{i}guez M\'{u}jica
\item Ulrich Bicheler
\item Wu Xing
\item Oleksandr Baranyuk
\end{itemize}


\section*{Resources on the Internet}
\begin{center}
\begin{tabular}{|ll|}
\hline
Author's page & {\code http://pacman128.github.io/} \\
NASM SourceForge page & {\code http://www.nasm.us/} \\
DJGPP  & {\code http://www.delorie.com/djgpp} \\
The Art of Assembly & {\code http://webster.cs.ucr.edu/} \\
USENET & {\code comp.lang.asm.x86 } \\
\hline
\end{tabular}
\end{center}


\section*{Feedback}

The author welcomes any feedback on this work.
\begin{center}
\begin{tabular}{ll}
\textbf{E-mail:} & {\code pacman128@gmail.com} \\
\textbf{WWW:}    & {\code http://pacman128.github.io/pcasm/} \\
\end{tabular}
\end{center}



