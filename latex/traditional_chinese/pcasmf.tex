% -*-LaTex-*-
%front matter of pcasm book

\chapter{前言}

\section*{目的}

這本書的目的是為了讓讀者更好地理解電腦在相比於編程語言如Pascal的更底層如何工作。通過更深刻地瞭解電腦如何工作,讀者通常可以更有能力用高階語言如C和C++來開發軟體。學習用組合語言來編程是達到這個目的的一個極好的方法。其他的PC組合語言程式的書仍然在講授著如何在1981年使用在初始的PC機上的8086處理器上進行編程!那時的8086處理器只支援\emph{實}模式。在這種模式下,任何程式都可以定址任意記憶體或訪問電腦裏的任意設備。這種模式不適合於安全,多工作業系統。這本書改為敍述在80386和後來的處理器如何在\emph{保護}模式(也就是Windows和Linux運行的模式)下進行編程。這種模式支援現在作業系統所期望的特徵,比如:虛擬記憶體和記憶體保護。使用保護模式有以下幾個原因:
\begin{enumerate}
\item 在保護模式下編程比在其他書使用的8086實模式下要容易。
\item 所有的現代的PC作業系統都運行在保護模式下。
\item 可以獲得運行在此模式下的免費軟體。
\end{enumerate}
關於保護模式下的PC彙編編程的書籍的缺乏是作者書寫這本書的主要原因。

就像上面所提到的,這本書使用了免費/開源的軟體:也就是,NASM彙編器和DJGPP C/C++編譯器。它們都可以在網際網路上下載到。本書同樣討論如何在Linux作業系統下和在Windows下的Borland和Microsoft的C/C++編譯器中如何使用NASM彙編代碼。所有這些平臺上的例子都可以在我的網頁上找到:
{\code http://www.drpaulcarter.com/pcasm}.
如何你想彙編和運行這本教程上的例子,你\emph{必須}下載這些樣例代碼。

注意這本書並不打算涵蓋彙編編程的各個方面。作者盡可能地涉及了\emph{所有}程式師都應該熟悉的最重要的方面。

\section*{致謝}

作者要感謝世界上許多為免費/開源運動作出貢獻的程式師。這本書的程式甚至是這本書本身都是使用免費軟體來書寫的。作者特別要感謝John~S.~Fine,
Simon~Tatham,Julian~Hall和其他開發NASM彙編器的人,這本書的所有例子都基於這個彙編器;開發DJGPP
C/C++DJ編譯器的Delorie;眾多對DJGPP所基於的GNU
gcc編譯器有貢獻的人們;Donald Knuth和其他開發\TeX\ 和 \LaTeXe\
排版語言的人,也正是用於書寫這本書的語言;Richard Stallman
(免費軟體基金會的創立者),Linus Torvalds
(Linux內核的創建者)和其他作者用來書寫這本書的底層軟體的創建者。

謝謝以下的人提出的修改意見:
\begin{itemize}
\item John S. Fine
\item Marcelo Henrique Pinto de Almeida
\item Sam Hopkins
\item Nick D'Imperio
\item Jeremiah Lawrence
\item Ed Beroset
\item Jerry Gembarowski
\item Ziqiang Peng
\item Eno Compton
\item Josh I Cates
\item Mik Mifflin
\item Luke Wallis
\item Gaku Ueda
\item Brian Heward
\item Chad Gorshing
\item F. Gotti
\item Bob Wilkinson
\item Markus Koegel
\item Louis Taber
\item Dave Kiddell
\item Eduardo Horowitz
\item S\'{e}bastien Le Ray
\item Nehal Mistry
\item Jianyue Wang
\item Jeremias Kleer
\item Marc Janicki
\end{itemize}


\section*{網際網路上的資源}
\begin{center}
\begin{tabular}{|ll|}
\hline
作者的網頁 & {\code http://www.drpaulcarter.com/} \\
%NASM   & {\code http://nasm.2y.net/} \\
NASM源代碼的網頁 & {\code http://sourceforge.net/projects/nasm/} \\
DJGPP  & {\code http://www.delorie.com/djgpp} \\
Linux彙編器 & {\code http://www.linuxassembly.org/} \\
彙編的藝術(The Art of Assembly) & {\code http://webster.cs.ucr.edu/} \\
USENET & {\code comp.lang.asm.x86 } \\
Intel文件 & {\code http://developer.intel.com/design} \\
       &  {\code /Pentium4/documentation.htm} \\
\hline
\end{tabular}
\end{center}


\section*{回饋}

作者歡迎任意關於這本書的回饋資訊。.
\begin{center}
\begin{tabular}{ll}
\textbf{E-mail:} & {\code pacman128@gmail.com} \\
\textbf{WWW:}    & {\code http://www.drpaulcarter.com/pcasm} \\
\end{tabular}
\end{center}

